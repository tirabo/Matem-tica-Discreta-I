% PDFLaTeX
\documentclass[a4paper,12pt,twoside,spanish,reqno]{amsbook}
%%%---------------------------------------------------

%\renewcommand{\familydefault}{\sfdefault} % la font por default es sans serif
%\usepackage[T1]{fontenc}

% Para hacer el  indice en linea de comando hacer 
% makeindex main
%% En http://www.tug.org/pracjourn/2006-1/hartke/hartke.pdf hay ejemplos de packages de fonts libres, como los siguientes:
%\usepackage{cmbright}
%\usepackage{pxfonts}
%\usepackage[varg]{txfonts}
%\usepackage{ccfonts}
%\usepackage[math]{iwona}
\usepackage[math]{kurier}

\usepackage{etex}
\usepackage{t1enc}
\usepackage{latexsym}
\usepackage[utf8]{inputenc}
\usepackage{verbatim}
\usepackage{multicol}
\usepackage{amsgen,amsmath,amstext,amsbsy,amsopn,amsfonts,amssymb}
\usepackage{amsthm}
\usepackage{calc}         % From LaTeX distribution
\usepackage{graphicx}     % From LaTeX distribution
\usepackage{ifthen}
\input{random.tex}        % From CTAN/macros/generic
\usepackage{subfigure} 
\usepackage{tikz}
\usetikzlibrary{arrows}
\usetikzlibrary{matrix}
\usepackage{mathtools}
\usepackage{stackrel}
\usepackage{enumitem}
\usepackage{tkz-graph}
%\usepackage{makeidx}
\usepackage{hyperref}
\hypersetup{
    colorlinks=true,
    linkcolor=blue,
    filecolor=magenta,      
    urlcolor=cyan,
}
\usepackage{hypcap}
\numberwithin{equation}{section}
% http://www.texnia.com/archive/enumitem.pdf (para las labels de los enumerate)
\renewcommand\labelitemi{$\circ$}
\setlist[enumerate, 1]{label={(\arabic*)}}
\setlist[enumerate, 2]{label=\emph{\alph*)}}


%%% FORMATOS %%%%%%%%%%%%%%%%%%%%%%%%%%%%%%%%%%%%%%%%%%%%%%%%%%%%%%%%%%%%%%%%%%%%%
\tolerance=10000
\renewcommand{\baselinestretch}{1.3}
\usepackage[a4paper, top=3cm, left=3cm, right=2cm, bottom=2.5cm]{geometry}
\usepackage{setspace}
%\setlength{\parindent}{0,7cm}% tamaño de sangria.
\setlength{\parskip}{0,4cm} % separación entre parrafos.
\renewcommand{\baselinestretch}{0.90}% separacion del interlineado
\setlist[1]{topsep=8pt,itemsep=.4cm,partopsep=4pt, parsep=4pt} %espacios nivel 1 listas
\setlist[2]{itemsep=.15cm}  %espacios nivel 2 listas
%%%%%%%%%%%%%%%%%%%%%%%%%%%%%%%%%%%%%%%%%%%%%%%%%%%%%%%%%%%%%%%%%%%%%%%%%%%%%%%%%%%
%\end{comment}
%%% FIN FORMATOS  %%%%%%%%%%%%%%%%%%%%%%%%%%%%%%%%%%%%%%%%%%%%%%%%%%%%%%%%%%%%%%%%%

\newcommand{\rta}{\noindent\textit{Rta: }} 

\begin{document}
    \baselineskip=0.55truecm %original    

{\bf \begin{center} Práctico 3 - Repaso \\ Matemática Discreta I -- Año 2021/1 \\ FAMAF\end{center}}




\begin{enumerate}
\setlength\itemsep{1.1em}

\item Sea $p$ primo positivo. Probar que $(p,(p-1)!)=1$.



\item Demostrar que $\forall n\in{\mathbb Z}$, $n>2$, existe $p$ primo tal que $n<p<n!$. (Ayuda: pensar qu\'e primos dividen a $n! - 1$.)





\item Dado un entero $a>0$ fijo, caracterizar aquellos n\'umeros que al dividirlos por $a$ tienen cociente igual al resto.




\item Probar que si $(a,4)=2$ y $(b,4)=2$ entonces $(a+b,4)=4$.


\item Probar que si $a,b$ son coprimos entonces $(a+b,a-b)=1 \text{
\'o } 2 $.







\item Completar y demostrar:

a) Si $a \in {\mathbb Z}$, entonces $[a,a]=\dots$

b) Si $a$, $b \in {\mathbb Z}$, $[a,b]=b$ si y s\'olo si $\ldots$

c) $(a,b)=[a,b]$ si y s\'olo si $\ldots$




\item Probar que si $d$ es un divisor com\'un de $a$ y $b$, entonces $\dfrac{[a,b]}{d} = \left[\dfrac{a}{d},\dfrac{b}{d}\right]$.





\item Probar que $(a+b,[a,b])=(a,b)$. %En particular, si dos n\'umeros son coprimos, tambi\'en lo son su suma y su producto.



\item Probar que si $(a,b)=1$ y $n+2$ es un n\'umero primo, entonces $(a+b, a^2 + b^2 - nab) = 1$ \'o $n+2$.





\item Si $a\cdot b$ es un cuadrado y $a$ y $b$ son coprimos, probar que $a$ y $b$ son cuadrados.



\medskip

\item Probar que $\sqrt 6$ es irracional.

%\medskip

%\item Probar que $2^{3n+4} + 7^{3n+1}$ es divisible por $9$, para todo $n \in {\mathbb N}$, $n$ impar.





\medskip

\item Hallar el menor m\'ultiplo de 168 que es un cuadrado.



\medskip

\item Probar que el producto de dos enteros consecutivos no nulos no es un cuadrado. (Ayuda: usar el Teorema Fundamental de la Aritm\'etica).

\medskip



\item ¿Existen enteros $m$ y $n$ tales que:

a) $m^4=27$? \qquad \qquad b) $m^2 = 12n^2$? \qquad \qquad c) $m^3 = 47n^3$?





\item Sean $a$ y $b$ enteros coprimos. Probar que
\begin{enumerate}
  \item $(a\cdot c, b)=(b,c)$, para todo entero $c$.
  \item $a^m$ y $b^n$ son coprimos, para todo $m,n\in \mathbb N$.
  \item $a+b$ y $a\cdot b$ son coprimos.
\end{enumerate}






\item ¿Cu\'al es la mayor potencia de $3$ que divide a $100!$? ¿En cu\'antos ceros termina el de\-sa\-rro\-llo decimal de $100!$?



\item Determinar todos los $p\in\mathbb N$ tales que
\[ p,\, p+2,\, p+6,\, p+8,\, p+12,\, p+14 \]
sean todos primos.




\item  Sea $\{f_n\}_{n\in\mathbb N}$ la sucesi\'on de Fibonacci, definida recursivamente por: $f_1=1,\, f_2=1$, $f_{n+1}=f_{n}+f_{n-1},\, n\geq 2$. Probar que:
\begin{enumerate}
\item $f_{3n}$ es par $\forall\, n\in\mathbb N$.
\item $f_{3n+1}$ y $f_{3n+2}$ son impares $\forall\, n\in\mathbb N$.
\item $f_{n+m}=f_mf_{n+1}+f_{m-1}f_n\; \forall n,m\in\mathbb N,\, m\geq 2$.
\item $f_n\mid f_{nk}\;  \forall k\in\mathbb N$.
\item $f_{n+1}f_{n-1}-f_n^2=(-1)^n\; \forall n\geq 2$.
\item $(f_{n+1},f_n)=1\; \forall\, n\in\mathbb N$.
\end{enumerate}



\end{enumerate}

\end{document}

