% PDFLaTeX
\documentclass[a4paper,12pt,twoside,spanish,reqno]{amsbook}
%%%---------------------------------------------------

%\renewcommand{\familydefault}{\sfdefault} % la font por default es sans serif
%\usepackage[T1]{fontenc}

% Para hacer el  indice en linea de comando hacer 
% makeindex main
%% En http://www.tug.org/pracjourn/2006-1/hartke/hartke.pdf hay ejemplos de packages de fonts libres, como los siguientes:
%\usepackage{cmbright}
%\usepackage{pxfonts}
%\usepackage[varg]{txfonts}
%\usepackage{ccfonts}
%\usepackage[math]{iwona}
\usepackage[math]{kurier}

\usepackage{etex}
\usepackage{t1enc}
\usepackage{latexsym}
\usepackage[utf8]{inputenc}
\usepackage{verbatim}
\usepackage{multicol}
\usepackage{amsgen,amsmath,amstext,amsbsy,amsopn,amsfonts,amssymb}
\usepackage{amsthm}
\usepackage{calc}         % From LaTeX distribution
\usepackage{graphicx}     % From LaTeX distribution
\usepackage{ifthen}
\input{random.tex}        % From CTAN/macros/generic
\usepackage{subfigure} 
\usepackage{tikz}
\usetikzlibrary{arrows}
\usetikzlibrary{matrix}
\usepackage{mathtools}
\usepackage{stackrel}
\usepackage{enumitem}
\usepackage{tkz-graph}
%\usepackage{makeidx}
\usepackage{hyperref}
\hypersetup{
    colorlinks=true,
    linkcolor=blue,
    filecolor=magenta,      
    urlcolor=cyan,
}
\usepackage{hypcap}
\numberwithin{equation}{section}
% http://www.texnia.com/archive/enumitem.pdf (para las labels de los enumerate)
\renewcommand\labelitemi{$\circ$}
\setlist[enumerate, 1]{label={(\arabic*)}}
\setlist[enumerate, 2]{label=\emph{\alph*)}}


%%% FORMATOS %%%%%%%%%%%%%%%%%%%%%%%%%%%%%%%%%%%%%%%%%%%%%%%%%%%%%%%%%%%%%%%%%%%%%
\tolerance=10000
\renewcommand{\baselinestretch}{1.3}
\usepackage[a4paper, top=3cm, left=3cm, right=2cm, bottom=2.5cm]{geometry}
\usepackage{setspace}
%\setlength{\parindent}{0,7cm}% tamaño de sangria.
\setlength{\parskip}{0,4cm} % separación entre parrafos.
\renewcommand{\baselinestretch}{0.90}% separacion del interlineado
\setlist[1]{topsep=8pt,itemsep=.4cm,partopsep=4pt, parsep=4pt} %espacios nivel 1 listas
\setlist[2]{itemsep=.15cm}  %espacios nivel 2 listas
%%%%%%%%%%%%%%%%%%%%%%%%%%%%%%%%%%%%%%%%%%%%%%%%%%%%%%%%%%%%%%%%%%%%%%%%%%%%%%%%%%%
%\end{comment}
%%% FIN FORMATOS  %%%%%%%%%%%%%%%%%%%%%%%%%%%%%%%%%%%%%%%%%%%%%%%%%%%%%%%%%%%%%%%%%

\newcommand{\rta}{\noindent\textit{Rta: }} 

\begin{document}
    \baselineskip=0.55truecm %original    
    
{\bf \begin{center} Práctico 2 \\ Matemática Discreta I -- Año 2021/1 \\ FAMAF \end{center}}

{\bf \begin{center} Ejercicios resueltos \end{center}}

\begin{enumerate}
\setlength\itemsep{1.1em}

\item  La cantidad de dígitos o cifras de un número se cuenta a partir del primer dígito
distinto de cero. Por ejemplo, 0035010 es un número de 5 dígitos.
\begin{enumerate}
\item ¿Cuántos números de 5 dígitos hay?

\textit{Rta:} Tenemos 9 posibilidades para el primero (todos salvo el cero) y diez para cada uno de los restantes cuatro lugares. Por lo tanto hay 90000 números de 5 dígitos.

\item ¿Cuántos números pares de 5 dígitos hay?

\textit{Rta:} En este caso el último dígito sólo puede ser 0,2,4,6 u 8. Por lo tanto hay $9\times 10^3\times 5= 45000$.

\item ¿Cuántos números de 5 dígitos existen con sólo un 3?

\noindent\textit{Rta:} El 3 puede estar en cada uno de los 5 lugares. Si está en el lugar más significativo en los restantes 4 lugares sólo pueden tomar 9 valores, ya que el 3 está excluido. Tenemos así $\times 9^4.$ A esto hay que sumarle los números que comienzan con un dígito distinto de 0 y 3 y que tienen exactamente un 3 entre sus cuatro últimos dígitos. Estos son $4.8.9^3$ números. La suma da $41.9^3=29889$.

\item ¿Cuántos números capicúas de 5 dígitos existen?

\textit{Rta:} Un capicúa de 5 dígitos está determinado por los primeros 3 dígitos. El primero tiene 9 posibilidades y los dos restantes 10 cada uno. Luego tenemos $9\times 10^2=900 $ capicúas de 5 dígitos.

\item ¿Cuántos números capicúas de a lo sumo 5 dígitos hay?

\textit{Rta:} A los anteriores 900 debemos sumarle 90 de 4 dígitos 90 de 3 dígitos 9 de 2 dígitos y 9 de un dígito. En total tenemos 1098.
 
\end{enumerate}

\medskip

\item¿Cuántos números de 6 cifras pueden formarse con los dígitos de 112200?

\noindent\textit{Rta:} Si no consideramos que la primera cifra no debe ser cero tenemos que llenar 6 casillas con dichos dígitos.
Tenemos que seleccionar 2 casillas entre las 6 para poner los unos y 2 entre las cuatro que quedan para poner los 2. En las dos restantes van los ceros. Así quedan $  \frac{6!}{2!4!}\frac{4!}{2!2!}\frac{2!}{2!0!}$. A estos debemos restar los que comienzan con cero, es decir $\frac{5!}{2!3!}\frac{3!}{2!1!}\frac{1!}{1!0!}$. Queda $90 -30=60$.

Otra forma de pensarlo es que tenemos que armar todas las permutaciones de seis objetos de los cuales se repiten tres de ellos dos veces cada uno (tenemos tres pares de objetos). Esto da $\frac{6!}{2!2!2!}=90$. A esto debemos retarle como antes todas las permutaciones que comienzan con un 0. esto es $\frac{5!}{2!2!1!}=30$.

Otra forma de pensarlo sería notar que la cantidad de números que empiezan con 1 es igual a la cantidad de números que empiezan con 2, porque en cada caso tengo que ordenar 5 objetos donde hay dos repeticiones de dos de esos objetos. Entonces el total de combinaciones sería $2.(5!/(2*2))$ .

\medskip


\item ¿Cuántos números impares de cuatro cifras hay?

\noindent\textit{Rta:} La primera cifra tiene 9 posibilidades, la última tiene 5 posibilidades y las dos restantes 10. Por lo tanto, se tienen $9\times 5\times 10^2=4500$.

\medskip

\item ¿Cuántos números múltiplos de 5 y menores que 4999 hay?

\noindent\textit{Rta:} Como $5000/5=1000$, cada múltiplo de cinco buscado, al dividirlo por 5, me dará un número entre 1 y 999. Recíprocamente cada número entre 1 y 999 al multiplicarlo por 5 me da un múltiplo de 5 menor que 4999.
Luego hay 999 múltiplos de 5.

 Otra forma de pensarlo es que tengo 5 elecciones posibles para el primer dígito {0,...,4}, 10 elecciones posibles para el segundo y tercer dígito, y dos para el cuarto {0,5}. Eso da $5.10.10.2 =1000$ posibilidades, pero estoy contando también el 0000. Si descuento ese caso, obtengo 1000-1=999 números

\medskip

\item En los viejos boletos de ómnibus, aparecía un número de 5 cifras (en este caso
podían empezar con 0), y uno tenía un boleto capicúa si el número lo era.


\begin{enumerate}
\item ¿Cuántos boletos capicúas había? \noindent\textit{Rta:} 1000.

\item ¿Cuántos boletos había en los cuales no hubiera ningún dígito repetido? \noindent\textit{Rta:} $10\times9\times8\times 7\times6$.
\end{enumerate}

\medskip

\item Las antiguas patentes de auto tenían una letra indicativa de la provincia y luego 6 dígitos. (En algunas provincias, Bs. As. y Capital, tenían 7 dígitos, pero ignoremos eso por el momento). Luego  vinieron patentes que tienen 3 letras y luego 3 dígitos. Finalmente, ahora, las patentes tienen 2 letras, luego 3 dígitos y a continuación dos letras más ¿Cuántas patentes pueden hacerse con cada uno de los sistemas?

\noindent\textit{Rta:} Con el primer sistema teníamos 24 letras contando Tierra del Fuego y ciudad de Buenos Aires. Entonces con el primer sistema podíamos hacer $24\times 10^6$ patentes (24 millones). Con el segundo sistema  $27^3\times 10^3$ (menos de 20 millones). Con el tercer sistema hay $27^2\times 10^3 \times 27^2$ posibilidades (alrededor de 530 millones).

\medskip

\item Si uno tiene 8 CD distintos de Rock, 7 CD distintos de música clásica y 5 CD
distintos de cuartetos,

\begin{enumerate}
\item 
¿Cuántas formas distintas hay de seleccionar un CD?

\noindent\textit{Rta:}  $20= 8+7+5$

\item ¿Cuántas formas hay de seleccionar tres CD, uno de cada tipo?

\noindent\textit{Rta:} Si los ordenamos según estilo RoClCu tenemos 8 posibles para el primero, 7 para el segundo y 5 para el tercero. En total 280=8.7.5. Si no están ordenados por estilo hay que multiplicar por 6.

\item Un sonidista en una fiesta de casamientos planea poner 3 CD, uno a continuación
de otro. ¿Cuántas formas distintas tiene de hacerlo si le han dicho que no
mezcle más de dos estilos?

\noindent\textit{Rta:} Si pudiera mezclar estilos  sin restricciones tendría  $20\times 19\times 18$. Para cumplir con la restricción impuesta debemos restar todos los que usan los tres estilos es decir uno de cada uno, que es lo que calculamos en el  apartado anterior cuando no están ordenados por estilo: $280\times 6$. Queda entonces $20.19.18-280.6=43.120=5160$. En el caso que importe el orden y ABC sea lo mismo que CBA, etc., se tendría $\frac{20.19.18}{6}-280=3140$.
\end{enumerate}

\medskip

\item Mostrar que si uno arroja un dado n veces y suma todos los resultados obtenidos,
hay $\frac{6^n}{2}$
formas distintas de obtener una suma par.

\noindent\textit{Rta:} Probemos por inducción en $n$. Si $n=1$ como el dado tiene tres caras pares (2,4,6), es cierto.
Suponiendo que fuese cierto para $k$ dados cuando tiramos $k+1$ la suma total será par si la suma de los primeros $k$ dados era par y el último fue par o si la suma de los primeros $k$dados fue impar y el último fue impar. Notemos que si había $\frac{6^k}{2}$ posibilidades de que la suma de los primeros $k$ fuese par, entonces las posibilidades de que fuese impar son $6^k-\frac{6^k}{2}=\frac{6^k}{2}$. Por lo tanto las posibilidades para $k+1$ serán:${ \frac{6^k}{2}\times 3+\frac{6^k}{2}\times 3=\frac{6^{k+1}}{2}}$.

\medskip

\item ¿Cuántos enteros entre 1 y 10000 tienen exactamente un 7 y exactamente un 5
entre sus cifras?

\noindent\textit{Rta:} Podemos suponer que es un número de 4 cifras ya que 10000 no tiene una cifra 7.
De los de cuatro cifras elegimos dos de ellas donde irán ubicados el 5 y el 7. Las restantes podrán llevar cualquiera de de los otro ocho dígitos. Tenemos entonces $6\times 8^2\times 2$ posibilidades. El último 2 es porque podemos poner 5 y 7 o 7 y 5 en los dos casilleros elegidos al comienzo.

\medskip

\item ¿Cuántos subconjuntos de $\{0,1,2,\dots,8,9\}$ contienen al menos un impar?

\noindent\textit{Rta:} Es más fácil contar todos los subconjuntos ($2^{10}$) y luego eliminar aquellos que no contienen ningún impar, es decir aquellos que están formados únicamente por los elementos de $\{0,2,4,6,8\}$ que son $2^5$. Luego la solución es $2^{10}-2^5$.

Otra forma sería pensar que para armar un subconjunto de $\{0,...,9\}$ tengo que elegir primero un subconjunto de $\{0,2,4,6,8\}$, después elegir un subconjunto de $\{1,3,5,7,9\}$ y por último unirlos. Lo primero se puede hacer de $2^5$ formas. Lo segundo de $2^5-1$ formas, descontando el conjunto vacío porque quiero que haya al menos un impar. Son entonces $2^5 (2^5 - 1)$ formas.

\medskip

\item El truco se juega con un mazo de 40 cartas, y se reparten 3 cartas a cada jugador. Obtener el 1 de espadas (el {\it macho}) es muy bueno. También lo es, por otros motivos, obtener un 7 y un 6 del mismo palo ({\it tener 33}). ¿Qué es más probable: obtener el macho, o tener 33?

\noindent\textit{Rta:} Si contamos todos los subconjuntos de 3 cartas una de las cuales es el macho, tendremos que son todos los subconjuntos de dos cartas entre las 39 restantes esto es $\frac{39!}{2!37!}$.
Ahora, para cada palo, la cantidad de subconjuntos de 3 cartas, de entre las cuales 2 de ellas son el 7 y el 6 correspondientes, son exactamente 38 (las posibles cartas restantes). Esto para cada palo entonces tenemos $4\times 38$ posibilidades. Ahora bien, es más probable el macho pues $4\times 38< 39\times 19$. 

\medskip

\item ¿Cuántos comités pueden formarse de un conjunto de 6 mujeres y 4 hombres, si el comité debe estar compuesto por 3 mujeres y 2 hombres?

\noindent\textit{Rta:} Debemos elegir 3 mujeres entre 6 y combinarlo con 2 hombres elegidos entre 4. Esto da $\frac{6!}{2!4!}\times \frac{4!}{2!2!}=90$.

\medskip

\item ¿De cuántas formas puede formarse un comité de 5 personas tomadas de un grupo
de 11 personas entre las cuales hay 4 profesores y 7 estudiantes, si:
\begin{enumerate}
\item 
No hay restricciones en la selección?
\noindent\textit{Rta:}  $\binom{11}{5}$

\item El comité debe tener exactamente 2 profesores?
\noindent\textit{Rta:}  $\binom{4}{2}\binom{7}{3}$.

\item El comité debe tener al menos 3 profesores?

\noindent\textit{Rta:} Si tuviera exactamente 3 sería:  $\binom{4}{3}\binom{7}{2}=84$. A estos debemos agregar los que tienen 4 profesores:  $\binom{4}{4}\binom{7}{1}=7$. En total quedan 91.


\item El profesor $X$ y el estudiante $Y$ no pueden estar juntos en el comité?

\noindent\textit{Rta:} Sin restricción tenemos  $\binom{11}{5}$. A esto le restamos los que tienen a $X$ e $Y$:  $\binom{9}{3}$.
En total quedan $462-84=378$. 
\end{enumerate}

\medskip

\item Si en un torneo de fútbol participan $2n$ equipos, probar que el número total de opciones posibles para la primera fecha es $1\cdot 3\cdot 5 \cdots (2n - 1)$. sugerencia: use un argumento por inducción. 

\noindent\textit{Rta:} Si son 2 equipos hay un único partido posible o sea $2\cdot 1-1$. suponiendo que vale para $2k$ equipos consideremos el caso de $2k+2$ equipos. Tomamos uno de los nuevos equipos y tenemos $2k+1$ rivales posibles y una vez definido este partido por hipótesis inductiva tenemos $1\cdot 3\cdots (2k-1)$ posibilidades para completar la fecha. Es decir en total tenemos $1\cdot 3\cdots(2k-1)\cdot(2k+1)$ posibilidades para la primera fecha.

 Otra forma es elegir el primer partido, $\binom{2n}{2}$ formas, el segundo partido,$\binom{2n-2}{2}$ formas, y asi sucesivamente hasta $\binom{4}{2}$ y $\binom{2}{2}$ formas para los últimos dos partidos. Al hacer eso se están contando más de una vez la misma elección de partidos pero en diferentes ordenes, así que se deben descontar todas las maneras posibles de ordenar los $n$ partidos para los $2n$ equipos. El resultado sería $(1 / n!) \binom{2n}{2}\binom{2n-2}{2}\dots \binom{4}{2}\binom{2}{2}= (2n)! / ( 2^n n!)$.

Viendo esa última expresión se puede deducir también otra forma: Ordeno a los $2n$ equipos en fila, y yendo de dos en dos voy determinando quienes juegan entre si. Como ahí estoy contando de más, tengo que corregir descontando todas las maneras de ordenar los partidos, $n!$, y todas las maneras de ordenar los equipos en cada partido, $2^n$.

\medskip

\item En una clase hay $n$ chicas y $n$ chicos. Dar el número de maneras de ubicarlos en una fila de modo que todas las chicas estén juntas.

\noindent\textit{Rta:} Podríamos pensar a las chicas como un único bloque y este puede ir en $n+1$ posiciones distintas. Si ahora individualizamos cada chico tenemos $n!$ posibilidades de ordenarlos e idénticamente $n!$ posibilidades individualizando cada chica. Luego tenemos $(n+1)(n!)n!=(n+1)!n!$ formas posibles de armar la fila.

\medskip

\item ¿De cuántas maneras distintas pueden sentarse 8 personas en una mesa circular?

\noindent\textit{Rta:} Si numeramos las sillas del 1 al 8 tendríamos $8!$ maneras distintas de ubicar a las personas a partir de la número uno. Rotando las sillas podemos hacer que la número 1 corresponda a una de las 8 personas en particular. Esto hace que dividamos el número de posibilidades por 8 y tenemos $7!$ maneras distintas de distribuir.

\medskip

\item 
\begin{enumerate}
\item 
¿De cuántas maneras distintas pueden sentarse 6 hombres y 6 mujeres en una
mesa circular si nunca deben quedar dos mujeres juntas?

\noindent\textit{Rta:} Como siempre hay al menos un hombre entre cada par de mujeres y hay sólo seis hombres, debe haber exactamente uno separando dos mujeres cercanas. Entonces tenemos que alternar hombre y mujer. Por lo cual tenemos $6!6!$ posibilidades para ponerlos en fila comenzando por un hombre y dividimos por 6 ya que, por ser circular, no importa que hombre comienza. 

Alternativamente si tenemos doce sillas numeradas del 1 al 12 alrededor de la mesa y ubicamos a los hombres en las sillas impares, esto se puede hacer de $5!$ formas distintas (ver ejercicio anterior). Luego tenemos que llenar las sillas pares y cualquiera de las $6!$ permutaciones da una distribución distinta. Así quedan $5!6!$.

\item Ídem, pero con 10 hombres y 7 mujeres.

\noindent\textit{Rta:}  Ponemos 20 sillas y sentamos al primer  hombre en la silla 2 y  a los restantes en las otras sillas pares. Tenemos $9!$ formas de hacerlo. Ahora ubicamos a las mujeres en las $10$ sillas vacías (las sillas sobrantes se retiran). Tenemos $\binom{10}{7}$ formas de elegir los lugares por $7!$ formas de ordenarlas. Así tenemos: $\displaystyle 9! \frac{10!}{7!3!} 7!=\frac{10!9!}{3!}$.

\end{enumerate}

\medskip

\item 
\begin{enumerate}
\item 
 ¿De cuántas formas distintas pueden ordenarse las letras de la palabra MATEMATICA

\noindent\textit{Rta:} Tenemos 10 letras, de las cuales hay 3 A, 2 M, 2 T, 1 E, 1 I, 1 C. Al total de las $10!$ permutaciones debemos dividirlo por las permutaciones de las repeticiones. Entonces tendremos $\frac{10!}{3!2!2!}$ formas distintas.

\item Ídem con las palabras ALGEBRA, GEOMETRIA.

\noindent\textit{Rta:}  $\frac{7!}{2!}$ y $\frac{9!}{2!}$.

\item ¿De cuántas formas distintas pueden ordenarse las letras de la palabra MATEMATICA
si se pide que las consonantes y las vocales se alternen?

\noindent\textit{Rta:} Tenemos 5 vocales y 5 consonantes, al alternarlos tendremos $5!5!$A esto hay que multiplicarlo por 2 ya que se puede empezar con vocal o consonante y dividirlo por las repeticiones y nos quedan $\frac{5!5!2}{3!2!2!}=1200$.
\end{enumerate}

\medskip

\item ¿Cuántas diagonales tiene un polígono regular de $n$ lados?

\noindent\textit{Rta:} Da cada vértice del polígono salen $n-3$ diagonales si sumamos $n$ veces $n-3$, resulta $n(n-3)$,  lo cual hay dividir por dos ya que cada diagonal fue contada dos veces. Así nos quedan: $ \frac{n(n-3)}{2}$ diagonales.

Otra forma es considerar todos los segmentos que tienen como extremos a dos vértices del polígono. Como cada segmento está determinado por dos vértices sin importar el orden, hay $\binom{n}{2}$ segmentos. Ahora, en esa cuenta están incluidos los lados del polígono, que no queremos contar. Esos son $n$, así que la cantidad de diagonales sería $\binom{n}{2}-n$.

\medskip

\item Dados $m$, $n$ y $k$ naturales tales que $m \le k \le n$, probar que se verifica
\begin{equation*}
\binom{n}{k}\binom{k}{m} = \binom{n}{m}\binom{n-m}{k-m}.
\end{equation*}

\noindent\textit{Rta:} Tenemos: $$\binom{n}{k}\binom{k}{m}=\frac{n!k!}{k!(n-k)!m!(k-m)!}=\frac{n!}{(n-k)!m!(k-m)!}.$$

Y por otra parte: $$\binom{n}{m}\binom{n-m}{k-m}=\frac{n!(n-m)!}{m!(n-m)!(k-m)!(n-m-(k-m))!}=\frac{n!}{m!(k-m)!(n-k)!}$$ de donde se desprende que se verifica la igualdad deseada.


\medskip

\item Probar que para todo $i$, $j$, $k \in {\mathbb N}_0$ vale
\begin{equation*}
\binom{i + j + k}{i}\binom{j+k}{j} = \frac{(i+j+k)!}{i!j!k!}
\end{equation*}

\noindent\textit{Rta:} Por definición: 
\begin{equation*}
\binom{i + j + k}{i}\binom{j+k}{j}=\frac{(i + j + k)!}{i!(i+j+k-i)!}\frac{(j+k)!}{j!k!(j+k-k)!}=\frac{(i + j + k)!}{i!j!k!}.
\end{equation*}


\medskip

\item Demostrar que para todo $n\in \mathbb{N}$  vale:
\vskip .3cm
\begin{enumerate}
\item{}
$\displaystyle\binom{n}{0}+\binom{n}{1}+\dots+\binom{n}{n}=2^n$
\vskip .3cm
\noindent\textit{Rta:}  $2^n$ es la cantidad de subconjuntos de $\{1,2,\dots,n\}$. Si agrupamos los subconjuntos de acuerdo a su cardinal y usamos que $\binom{n}{k}$ es la cantidad de subconjuntos de $k$ elementos de  $\{1,2,\dots,n\}$
Se tiene la igualdad.

Alternativamente si conocemos la fórmula del binomio de Newton $(a+b)^n=\sum_{i=0}^n\binom{n}{i}a^ib^{n-i}$ podemos aplicarla al caso particular $a=1, b=1$ y obtener la igualdad del ejercicio.
\vskip .3cm
\item{} $\displaystyle\binom{n}{0}-\binom{n}{1}+\dots+(-1)^n\binom{n}{n}=0$.
\vskip .3cm
\noindent\textit{Rta:} Aquí podemos usar la fórmula del binomio al caso particular $a=1, b=-1$. Podemos interpretar la fórmula como la igualdad de la cantidad de subconjuntos de cardinal par con los de cardinal impar.
\end{enumerate}

\medskip

\item Probar que para todo natural $n$ vale que 
\begin{equation*}
\binom{2n}{2} = 2 \binom{n}{2} + n^2.
\end{equation*}

\noindent\textit{Rta:} Tenemos: $$2\binom{n}{2}+n^2=2\frac{n(n-1)}{2}+n^2={n(n-1+n)}=\frac{2n(2n-1)}{2}=\binom{2n}{2}.$$

\medskip


\begin{center}
    \textbf{Ejercicios adicionales (para repasar)}
\end{center}


\medskip

\item Con 20 socios de un club se desea formar 5 listas electorales (disjuntas). Cada lista
consta de 1 Presidente, 1 Tesorero y 2 vocales. ¿De cuántas formas puede hacerse?

\noindent\textit{Rta:} Dado un subconjunto de cuatro miembros, tenemos 4 posibilidades para presidente y cada una de ellas tiene tres para tesorero, siendo los restantes miembros vocales. Es decir cada subconjunto de 4 tiene doce posibilidades para listas. Para elegir un subconjunto de 4 tenemos $\binom{20}{4}$ posibilidades. A cada una de ellas le corresponde $\binom{16}{4}$ para un segundo subconjunto y luego $\binom{12}{4}$ para el tercero y $\binom{8}{4}$ para el cuarto, quedando determinado el quinto. A esto tenemos que dividirlo por $5!$ ya que la elección $ABCDE$ es equivalente a  $BCDEA$ y cualquier otra permutación. 
Finalmente quedan $$\frac{12^5}{5!}\binom{20}{4}\binom{16}{4}\binom{12^5}{4}\binom{8}{4}=\frac{20!12^5}{5!4!4!4!4!4!}=\frac{20!}{5!2^5}$$ formas distintas de elegir listas.

Otra forma es elegir a los presidentes primero, eso se puede de $\binom{20}{5}$ formas. Ahora para cada presidente elijo un tesorero, eso son 15!/10! formas. Por último, tengo que dividir los últimos 10 en 5 parejas, eso se puede hacer de ${10 \choose 2} {8 \choose 2} {6 \choose 2} {4 \choose 2} = 10! / ( 2^5)$ formas. Alternativamente, recordando el ejercicio 14, armar 5 parejas con 10 personas se puede hacer de $10! / ( 2^5 5!)$ formas si no importa el orden. Como en este caso si importa porque los presidentes ya fueron elegidos, serían $10! / ( 2^5)$ formas. La cantidad total es entonces $\binom{20}{5} (15! / 10!) ( 10! / ( 2^5) ) = 20! / (2^5 5!)$.

Y también, viendo la última expresión se puede deducir otra forma: Ordeno a las 20 personas en fila y yendo de cinco en cinco determino las listas. Al hacer eso estoy contando de más, corrijo descontando todas las maneras de ordenar las listas, $5!$, y todas las maneras de ordenar los vocales en cada lista, $2^5$.

\medskip

\item ¿De cuántas formas se pueden fotografiar 7 matrimonios en una hilera, de tal forma
que cada hombre aparezca al lado de su esposa?

\noindent\textit{Rta:} Cada pareja tiene dos formas de posar y podemos permutar a las 7 parejas. Luego tenemos $7!2=1680$ formas.

\medskip

\item ¿De cuántas formas pueden distribuirse 14 libros distintos entre dos personas de
manera tal que cada persona reciba al menos 3 libros?

\noindent\textit{Rta:} Para cada libro tenemos dos posibles destinatarios, entonces hay $2^{14}$ formas de repartir 14 libros entre dos personas.  Las que dan menos de 3 libros a alguno de los dos son $\binom{14}{0}+\binom{14}{1}+\binom{14}{2}+\binom{14}{12}+\binom{14}{13}+\binom{14}{14}=1+14+91+91+14+1=212$. Si las restamos se tiene el resultado buscado: $2^{14}-212$.



\item
Cecilia ha olvidado la contraseña de su correo electrónico, la cual est\'a formada por $11$ caracteres: $4$ d\'igitos y $7$ letras todos mezclados. ?`Cu\'al es el m\'aximo n\'umero de intentos que deber\'ia probar para entrar a su correo si:
\begin{enumerate}
    \item  tiene en cuenta que hay 26 letras?
    
    \textit{Rta:} Pensemos que tenemos 11 posiciones donde poner los caracteres de la contraseña. Hay $\binom{11}{7}$ formas de elegir 7 posiciones y ahí ponemos las letras de $26^7$ formas. En  las  posiciones que quedan ubicamos los 4 dígitos de $10^4$ formas. En total tenemos $\binom{11}{7}26^710^4$   contraseñas posibles de 7 letras y 4 dígitos.
    
    \item la letra B y el d\'igito 2 no pueden estar ambos en la contraseña?
    
    \textit{Rta:} Aquí debemos pensar en  tres casos mutuamente excluyentes: 1) la contraseña no tiene ni B ni 2, 2) la contraseña tiene B, pero no 2 y 3) la contraseña no tiene B pero tiene 2. Estos casos cubren todas las posibilidades,  encontrando  la solución en cada caso y sumándolas obtenemos la respuesta al ejercicio.
    
    1) En  este caso es como  el problema en \textit{a)} pero sin poder usar ni B, ni 2,  es decir teniendo 25 letras y 9 dígitos. La respuesta es $\binom{11}{7}25^79^4$.
    
    2) Tenemos 11 posiciones donde ubicar B. Una vez ubicado  B,  quedan 26 letras (todas) y 9 dígitos (no puede estar 2) y debemos poner 6 letras y 4 dígitos,  es decir tenemos $\binom{10}{6}26^69^4$ posibilidades. Luego el total es $11\binom{10}{6}26^69^4$
    
    3) Se hace parecido a 2): tenemos 11 posiciones donde ubicar 2. Una vez ubicado  2,  quedan 25 letras (no puede esta B) y 10 dígitos (todos) y debemos poner 7 letras y 3 dígitos,  es decir tenemos $\binom{10}{6}25^710^3$ posibilidades. Luego el total es $11\binom{10}{6}25^710^4$.
    
    Finalmente la  solución es :  $\binom{11}{7}25^79^4+11\binom{10}{6}26^69^4+11\binom{10}{6}25^710^4$.
    
    \item  recuerda que ha formado la clave de la siguiente manera: los primeros siete caracteres son una mezcla de las letras de su nombre (CECILIA), y los \'ultimos cuatro caracteres son una permutaci\'on de las cuatro cifras de su a\~no de nacimiento (1998)?
    
    \textit{Rta:} Como CECILIA es una palabra de longitud 7 formada por  las letras A, C, C,  E, I, I, L las permutaciones posibles son $\frac{7!}{2!2!}$. Las permutaciones de 1998  son $\frac{4!}{2!}$. Luego el toral de contraseñas posibles es   $\frac{7!}{2!2!}\frac{4!}{2!}$.
    
\end{enumerate}


%Forma errada: Si seleccionamos 3 libros para X y 3 libros para Y tenemos $\binom{14}{3}\binom{11}{3}$ formas de hacerlo. Para los restantes 8 libros tenemos $2^8$ formas de repartirlos entre dos (tomando un subconjunto de 8 y su complemento). Luego hay $\binom{14}{3}\binom{11}{3}2^8$ formas de distribuir.
\end{enumerate}
\end{document}
1.e La suma no da 999. Sería 900+90+90+9+9 = 1098 . La segunda forma de pensarlo justamente tiene el problema de que no se están contando algunos números. Por ejemplo, en la lista {1,...,999} está el 3, que se correspondería con dos números capicuas, el mismo 3 y el 33.

2 Otra forma de pensarlo sería notar que la cantidad de números que empiezan con 1 es igual a la cantidad de números que empiezan con 2, porque en cada caso tengo que ordenar 5 objetos donde hay dos repeticiones de dos de esos objetos. Entonces el total de combinaciones sería 2* (5!/(2*2)) .

4. Otra forma de pensarlo es que tengo 5 elecciones posibles para el primer dígito {0,...,4}, 10 elecciones posibles para el segundo y tercer dígito, y dos para el cuarto {0,5}. Eso da 5*10*10*2 =1000 posibilidades, pero estoy contando también el 0000. Si descuento ese caso, obtengo 1000-1=999 números

10. Error de tipeo, supongo que debería ser 2^{10} en lugar de 2^10

Otra forma sería pensar que para armar un subconjunto de {0,...,9} tengo que elegir primero un subconjunto de {0,2,4,6,8}, despues elegir un subconjunto de {1,3,5,7,9} y por último unirlos. Lo primero se puede hacer de 2^5 formas. Lo segundo de 2^5-1 formas, descontando el conjunto vacío porque quiero que haya al menos un impar. Son entonces 2^5 * (2^5 - 1) formas.

14. Otra forma es elegir el primer partido, 2n \choose 2 formas, el segundo partido, 2n-2 \choose 2 formas, y asi sucesivamente hasta 4 \choose 2 formas y 2 \choose 2 formas para los últimos dos partidos. Al hacer eso se están contando más de una vez la misma elección de partidos pero en diferentes ordenes, así que se deben descontar todas las maneras posibles de ordenar los n partidos para los 2n equipos. El resutlado sería (1 / n!) {2n \choose 2} {2n-2 \choose 2} ... {4 \choose 2} {2 \choose 2} = (2n)! / ( 2^n n!).

Viendo esa última expresión se puede deducir también otra forma: Ordeno a los 2n equipos en fila, y yendo de dos en dos voy determinando quienes juegan entre si. Como ahí estoy contando de más, tengo que corregir descontando todas las maneras de ordenar los partidos, n!, y todas las maneras de ordenar los equipos en cada partido, 2^n.

18.c Falta multiplicar por 2, ya que la palabra puede empezar con vocal o con consonante.

19. Otra forma es considerar todos los segmentos que tienen como extremos a dos vértices del polígono. Como cada segmento está determinado por dos vértices sin importar el orden, hay n \choose 2 segmentos. Ahora, en esa cuenta están incluídos los lados del polígono, que no queremos contar. Esos son n, así que la cantidad de diagonales sería {n \choose 2} - n.

24. El 12 del principio me parece que tiene que estar elevado a la 5, porque esas posibilidades son para cada lista. O sea, 12^5.

Otra forma es elegir a los presidentes primero, eso se puede de 20 \choose 5 formas. Ahora para cada presidente elijo un tesorero, eso son 15!/10! formas. Por último, tengo que dividir los último 10 en 5 parejas, eso se puede hacer de {10 \choose 2} {8 \choose 2} {6 \choose 2} {4 \choose 2} = 10! / ( 2^5) formas. Alternativamente, recordando el ejercicio 14, armar 5 parejas con 10 personas se puede hacer de 10! / ( 2^5 5!) formas si no importa el orden. Como en este caso si importa porque los presidentes ya fueron elegidos, serían 10! / ( 2^5) formas. La cantidad total es entonces {20 \choose 5} (15! / 10!) ( 10! / ( 2^5) ) = 20! / (2^5 5!).

Y también, viendo la última expresión se puede deducir otra forma: Ordeno a las 20 personas en fila y yendo de cinco en cinco determino las listas. Al hacer eso estoy contando de más, corrijo descontando todas las maneras de ordenar las listas, 5!, y todas las maneras de ordenar los vocales en cada lista, 2^5.

26. La que mandaste en el último correo es la misma que tengo yo, no se me ocurre ninguna otra manera de hacerlo.


