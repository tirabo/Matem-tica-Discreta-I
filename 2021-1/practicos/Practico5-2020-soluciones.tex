\documentclass[12pt,spanish,makeidx]{amsbook}
\tolerance=10000
\renewcommand{\baselinestretch}{1.3}



\usepackage{t1enc}
\usepackage[spanish]{babel}
\usepackage{latexsym}
\usepackage[utf8]{inputenc}
\usepackage{verbatim}
\usepackage{multicol}
\usepackage{amsgen,amsmath,amstext,amsbsy,amsopn,amsfonts,amssymb}
\usepackage{calc}         % From LaTeX distribution
\usepackage{graphicx}     % From LaTeX distribution
\usepackage{ifthen}       % From LaTeX distribution
\usepackage{subfigure}    % From CTAN/macros/latex/contrib/supported/subfigure
\usepackage{tikz}
\usepackage{tkz-graph}
\usepackage{tikz-3dplot} %for tikz-3dplot functionality
\usepackage{pgfplots}
\usetikzlibrary{graphs}
\usetikzlibrary{graphs.standard}
\usetikzlibrary{matrix}
\usetikzlibrary{arrows}
\input{random.tex}        % From CTAN/macros/generic


%% \theoremstyle{plain} %% This is the default
\oddsidemargin 0.0in \evensidemargin -1.0cm \topmargin 0in
\headheight .3in \headsep .2in \footskip .2in
\setlength{\textwidth}{16cm} %ancho para apunte
\setlength{\textheight}{21cm} %largo para apunte
%\leftmargin 2.5cm
%\rightmargin 2.5cm
\topmargin 0.5 cm



\usepackage{hyperref}
\hypersetup{
    colorlinks=true,
    linkcolor=blue,
    filecolor=magenta,      
    urlcolor=cyan,
}
\usepackage{hypcap}

\newcommand\itemitem[1]{\item[#1]}


\newcommand{\ExampleSubFigure}[2][0.3333]{%
\subfigure[Example #2]{%
  \begin{minipage}[t]{#1\textwidth}
    \parbox[b]{\textwidth}{%
      \centering
      \input{lgc/#2.inl}}
  \end{minipage}}}


%% \theoremstyle{plain} %% This is the default
\oddsidemargin 0.0in \evensidemargin -1.0cm \topmargin 0in
\headheight .3in \headsep .2in \footskip .2in
\setlength{\textwidth}{16cm} %ancho para apunte
\setlength{\textheight}{21cm} %largo para apunte
%\leftmargin 2.5cm
%\rightmargin 2.5cm
\topmargin 0.5 cm



%% \theoremstyle{plain} %% This is the default



\renewcommand{\thesection}{\thechapter.\arabic{section}}
\renewcommand{\thesubsection}{\thesection.\arabic{subsection}}




\newtheorem{teorema}{Teorema}[section]
\newtheorem{proposicion}[teorema]{Proposici\'on}
\newtheorem{corolario}[teorema]{Corolario}
\newtheorem{lema}[teorema]{Lema}

\theoremstyle{definition}

\newtheorem{definicion}{Definici\'on}[section]
\newtheorem{ejemplo}{Ejemplo}[section]
\newtheorem{problema}{Problema}[section]
\newtheorem{ejercicio}{Ejercicio}[section]
\newtheorem{ejerciciof}{}[section]

\theoremstyle{remark}
\newtheorem{observacion}{Observaci\'on}[section]
\newtheorem{nota}{Nota}[section]



\renewcommand{\abstractname}{Resumen}
\renewcommand{\partname }{Parte }
\renewcommand{\indexname}{Indice }
\renewcommand{\figurename }{Figura }
\renewcommand{\tablename }{Tabla }
\renewcommand{\proofname}{Demostraci\'on}
\renewcommand{\refname }{Referencias }
\renewcommand{\appendixname }{Ap\'endice }
\renewcommand{\contentsname }{Contenidos }
\renewcommand{\chaptername }{Cap\'\i tulo }
\renewcommand{\bibname }{Bibliograf\'\i a }





\def\conc{+\hspace{-1.5ex}+\hspace{0.5ex}}
\def\con{{\rm con}}
\def\hn{\hspace{-0.2cm}}
\def\h4n{\hspace{-0.4cm}}
\def\impli{\Rightarrow}
\def\ssi{\equiv}
\def\disc{\not\ssi}
\def\cons{\Leftarrow}
\def\la{\leftarrow}
\def\lt{\triangleleft}
\def\lv{[\;\,]}
\def\Max{ {\rm Max} }
\def\Min{ {\rm Min} }
\def\N{I\hspace{-0.8ex} N}
\def\noi{\noindent}
\def\R{I\hspace{-0.8ex} R}
\def\ra{\rightarrow}
\def\Rn{\R^{n}}
\def\rt{\triangleright}
%\def\tomar{\uparrow}
\def\tomar{\hspace{-0.6ex}\uparrow\hspace{-0.6ex}}
%\def\tirar{\downarrow}
\def\tirar{\hspace{-0.6ex}\downarrow\hspace{-0.6ex}}
\def\udo{ {\rm\bf\underline{do}} }
\def\ufi{ {\rm\bf\underline{fi}} }
\def\uif{ {\rm\bf\underline{if}} }
\def\uod{ {\rm\bf\underline{od}} }
\def\v3{\vspace{0.3cm}}
\def\var{{\rm var}}
\def\[{|\hspace{-0.2ex} [}
\def\]{]\hspace{-0.2ex} |}
\def\true{\mbox{\it true\ }}
\def\false{\mbox{\it false\ }}

\newcommand{\bag}[1]{ [\hspace{-0.4ex}[ #1 ]\hspace{-0.4ex}] }
\newcommand{\abs}[1]{ [\hspace{-0.4ex}[ #1 ]\hspace{-0.4ex}] }
%\newcommand{\binom}[2]{ \left(\hspace{-1.2ex}\begin{array}{c} #1 \\ #2 \end{array} \hspace{-1.2ex}\right) }
\newcommand{\cau}[2]{\noi #1 \hspace{0.5cm}\{{\sl #2}\} }
\newcommand{\causa}[2]{\vspace{0.15cm} \noi #1 \hspace{0.5cm} \{{\sl #2}\} \vspace{0.15cm}}
\newcommand \RR{{\mathbb R}}
\newcommand \ZZ{{\mathbb Z}}
\newcommand \NN{{\mathbb N}}
%\newcommand \supr{\displaystyle{\ \,{\mbox{\footnotesize S}}\hspace{-1.7ex}\bigcirc\, }}
\newcommand \supr{\displaystyle{\ \,\vee \hspace{-1.9ex}\bigcirc\,}}
\newcommand \infi{\displaystyle{\ \,\wedge\hspace{-1.9ex}\bigcirc\,}}
%\newcommand \infi{\displaystyle{\ \,{\mbox{\footnotesize I}}\hspace{-1.5ex}\bigcirc\, }}
\newcommand \mcd{\operatorname{mcd}}
\newcommand \mcm{\operatorname{mcm}}
\newcommand \sisolosi{\Leftrightarrow}
\newcommand{\cskip}{\vskip .4cm}
\newcommand{\rskip}{\vskip .2cm}




\newcommand\flecha{\Rightarrow}

\newcommand{\programa}[1]{  \vskip .5cm \noindent Programa: {\it #1} }
\newcommand{\principio}[1]{\vskip .3 cm \centerline{\sc #1} \vskip .3cm}



\usepackage{fancyhdr}
\pagestyle{fancy}
\fancyhf{}
\fancyhead[LE,RO]{FAMAF}
\fancyhead[RE,LO]{Matemática Discreta I}
\fancyfoot[LE,RO]{\leftmark}
\fancyfoot[RE,LO]{\thepage}
 
\renewcommand{\headrulewidth}{0.5pt}
%\renewcommand{\footrulewidth}{0.5pt}
 \newcommand{\rta}{\vskip .15cm \noindent\textit{Rta: }}
 \newcommand{\alt}{\noindent\textit{Rta alternativa: }}


\begin{document}
	
	{\bf \begin{center} Práctico 5 \\ Matemática Discreta I -- Año 2020/1 \\ FAMAF \end{center}}
	
	{\bf \begin{center} Ejercicios resueltos \end{center}}
	
	\cskip
		
\begin{enumerate}
		
	\item \label{ej-kn} ¿Cuántas aristas tiene el  grafo completo $K_n$? ¿Para cuáles valores de $n$ se puede encontrar un dibujo de $K_n$ con la propiedad que las líneas representan las aristas sin cruzarse?
	
	\rta En un grafo  completo de $n$ vértices podemos contar las aristas de la siguiente manera: del primer vértice salen $n-1$ aristas, del segundo salen $n-2$ pues no debemos contar la arista al primer vértice, del tercer vértice salen $n-3$ aristas (no contamos la que van al primero y al segundo), y así sucesivamente. Concluyendo, la cantidad de aristas es 
	\begin{equation*}
		(n-1)+(n-2)+(n-3)+\cdots + 2 +1 +0 = \sum_{i=0}^{n-1} i = \frac{n(n-1)}{2}.
	\end{equation*}
	Otra forma de contar es la siguiente: de cada arista salen $n-1$ aristas. Si las sumamos a todas tenemos $n(n-1)$ aristas, pero cada arista está contada dos veces, por lo tanto la cantidad total de aristas es $n(n-1)/2$.
	
	Con respecto a la segunda pregunta, es claro  que podemos dibujar $K_1$, $K_2$ y $K_3$ sin cruces de aristas. Aunque $K_4$ solemos dibujarlo como un cuadrado con sus diagonales, también podemos dibujar $K_4$ sin cruces:
	\vskip .3cm 
	\begin{center}
		\begin{tikzpicture}[scale=1]
		\SetVertexSimple[Shape=circle,FillColor=white]
		\tikzset{EdgeStyle/.append style = {bend left = 00}}
		\Vertex[x=0,y=0]{A}
		\Vertex[x=2,y=0]{B}
		\Vertex[x=2,y=-2]{C}
		\Vertex[x=0,y=-2]{D}
		\Edges(A,B,C,D,A,B,D)
		\draw[thick] (0.15,0.15) .. controls (1,1) and (4,1) .. (2.15,-1.85);
		\end{tikzpicture}
	\end{center}
	Sin embargo, no hay forma de dibujar $K_5$ sin cruces entre las aristas. La demostración de este hecho no es sencilla y debemos convencernos tratando (y fracasando) de hacer un dibujo de $K_5$ sin cruces entre aristas.
	
	\cskip
		
	\item Encuentre un isomorfismo entre los grafos por las siguientes listas. (Ambas listas especifican versiones de un famoso grafo conocida como {\it grafo de Petersen}.)
	$$
	\begin{matrix}
	a&b&c&d&e&f&g&h&i&j\\ \hline
	b&a&b&c&d&a&b&c&d&e\\
	e&c&d&e&a&h&i&j&f&g\\
	f&g&h&i&j&i&j&f&g&h
	\end{matrix}
	\qquad \begin{matrix}
	0&1&2&3&4&5&6&7&8&9\\ \hline
	1&2&3&4&5&0&1&0&2&6\\
	5&0&1&2&3&4&4&3&5&7\\
	7&6&8&7&6&8&9&9&9&8
	\end{matrix}
	$$
	
	\rta
	
	Podemos dibujar el primer grafo y  el segundo  grafo de la siguiente forma
	\vskip .3cm
	\begin{tabular}{ccc}
		\begin{tikzpicture}[every node/.style={draw,circle},scale=0.5]
		\SetVertexSimple[Shape=circle,FillColor=white,MinSize=8 pt]
		\graph[clockwise, radius=3cm] {subgraph C_n [n=5,name=A,V={$a$,$b$,$c$,$d$,$e$}]};
		\graph[clockwise, radius=1.5cm] {subgraph I_n [n=5,name=B,V={$f$,$g$,$h$,$i$,$j$}]};
		\Edge(A $a$)(B $f$)
		\Edge(A $e$)(B $j$)
		\Edge(A $d$)(B $i$)
		\Edge(A $c$)(B $h$)
		\Edge(A $b$)(B $g$)
		\Edge(B $f$)(B $i$)
		\Edge(B $f$)(B $h$)
		\Edge(B $j$)(B $g$)
		\Edge(B $j$)(B $h$)
		\Edge(B $g$)(B $i$)
		\end{tikzpicture}
		&
		&
		\begin{tikzpicture}[every node/.style={draw,circle},scale=0.5]
		\graph[clockwise, radius=3cm] {subgraph I_n [n=10,V={0,1,2,3,4,5,6,7,8,9}]};
		\Edge(0)(1)
		\Edge(0)(5)
		\Edge(0)(7)
		\Edge(1)(2)
		\Edge(1)(6)
		\Edge(2)(3)
		\Edge(2)(8)
		\Edge(3)(4)
		\Edge(3)(7)
		\Edge(4)(5)
		\Edge(4)(6)
		\Edge(5)(8)
		\Edge(6)(9)
		\Edge(7)(9)
		\Edge(8)(9)
		\draw[fill=blue](0);
		\end{tikzpicture}
	\end{tabular}
	No es fácil darse cuenta que estos grafos son isomorfos. Sin embargo, reordenando el segundo  grafo, con un poco de suerte,  obtenemos
	\begin{center}
		\begin{tikzpicture}[every node/.style={draw,circle},scale=0.5]
		\SetVertexSimple[Shape=circle,FillColor=white,MinSize=8 pt]
		\graph[clockwise, radius=3cm] {subgraph C_n [n=5,name=A,V={0,1,6,9,7}]};
		\graph[clockwise, radius=1.5cm] {subgraph I_n [n=5,name=B,V={5,2,4,8,3}]};
		\Edge(A 0)(B 5)
		\Edge(A 7)(B $j$)
		\Edge(A 9)(B 8)
		\Edge(A 6)(B 4)
		\Edge(A 1)(B 2)
		\Edge(B 5)(B 8)
		\Edge(B 5)(B 4)
		\Edge(B 3)(B 2)
		\Edge(B 3)(B 4)
		\Edge(B 2)(B 8)
		\end{tikzpicture}
	\end{center}
	Luego  el isomorfismo viene dado por
	\begin{equation*}
	\begin{matrix}
	a \to 0 && c\to 6 && e \to 7 && g \to 2 && i \to 8 \\
	b \to 1 && d \to 9 && f \to 5 && h \to 4 && j \to 3
	\end{matrix}
	\end{equation*}
	
	\cskip
	
	\item
	\begin{enumerate}
		\item Encuentre todos los grafos de 5 vértices y 2 aristas no isomorfos entre sí.
		
		\rta Hay 2 posibilidades 1) las dos aristas no se tocan, 2)  las dos aristas se tocan en un punto. Es decir las aristas son 1)  $\{v,v'\},\{w,w'\} $ o 2) $\{v,v'\},\{v',w'\}$, donde $v,v',w,w'$ son todos distintos. Luego  hay solo dos grafos no isomorfos entre sí. Dibujándolos son:
		\vskip .3cm
		
		\begin{center}
			\begin{tikzpicture}[scale=1]
			\SetVertexSimple[Shape=circle, FillColor=white,MinSize=8 pt]
			\SetVertexNoLabel
			\Vertex[x=0,y=0]{A}
			\Vertex[x=1,y=0]{B}
			\Vertex[x=2,y=0]{C}
			\Vertex[x=3,y=0]{D}
			\Vertex[x=4,y=0]{E}
			%
			\Edges(A,B)
			\Edges(C,D)

			
			\Vertex[x=7,y=0]{A}
			\Vertex[x=8,y=0]{B}
			\Vertex[x=9,y=0]{C}
			\Vertex[x=10,y=0]{D}
			\Vertex[x=11,y=0]{E}
			%
			\Edges(A,B,C)
			\end{tikzpicture}
		\end{center}
		
			\vskip .3cm
			
		\item ¿Cuál es el máximo número de aristas que puede tener un grafo de 5 vértices?
		
		\rta El grafo  completo $K_5$ es el que tiene todas las aristas posibles y estas son $5 \cdot 4 / 2 = 10$ (ver ejercicio \ref{ej-kn}). 
	\end{enumerate}
	
	\cskip
	
	\item Para cada una de las siguientes secuencias, encuentre un grafo que tenga exactamente las valencias indicadas o demuestre que tal grafo no existe:
	\begin{enumerate}
		\item $3,3,1,1$
		
		\rta Como la suma de valencias $3+3+1+1 = 8$, el grafo debe tener 4 aristas. Como hay 4 vértices y dos de ellos tienen valencia 3, digamos $a$ y $b$, de esos vértices salen aristas que llegan a todos los otros vértices. Es decir tenemos la siguiente situación:
		
		\begin{center}
			\begin{tikzpicture}[scale=1]
			%\SetVertexSimple[Shape=circle, FillColor=white,MinSize=8 pt]
			%\SetVertexNoLabel
			\Vertex[x=0,y=0]{$a$}
			\Vertex[x=2,y=0]{$b$}
			\Vertex[x=2,y=-2]{$c$}
			\Vertex[x=0,y=-2]{$d$}
			
			%
			\Edges($a$,$b$,$a$,$c$,$a$,$d$)
			\Edges($b$,$c$,$b$,$d$)
			
			\end{tikzpicture}
		\end{center}
		
		Por lo tanto, los dos vértices restantes tiene valencia $\ge 2$, lo cual nos dice que no puede haber un grafo con valencias $3,3,1,1$.
		
		\item $3,2,2,1$
		
		\rta  Un grafo que cumple con esa secuencia es:
		
		\begin{center}
			\begin{tikzpicture}[scale=1]
			\SetVertexSimple[Shape=circle, FillColor=white]
			\SetVertexNoLabel
			\Vertex[x=0,y=0]{$a$}
			\Vertex[x=2,y=0]{$b$}
			\Vertex[x=2,y=-2]{$c$}
			\Vertex[x=0,y=-2]{$d$}
			
			%
			\Edges($a$,$b$,$a$,$c$,$a$,$d$)
			\Edges($b$,$d$)
			
			\end{tikzpicture}
		\end{center}
		
		\item $3,3,2,2,1,1$
		
		\rta El grafo  tiene 6 vértices y $12/2 = 6$ aristas:
		\vskip .3cm
		\begin{center}
			\begin{tikzpicture}[scale=0.65]
			\SetVertexSimple[Shape=circle,FillColor=white]
			%
			\Vertex[x=3.00, y=0.00]{1}
			\Vertex[x=1.50, y=2.60]{2}
			\Vertex[x=-1.50, y=2.60]{3}
			\Vertex[x=-3.00, y=0.00]{4}
			\Vertex[x=-1.50, y=-2.60]{5}
			\Vertex[x=1.50, y=-2.60]{6}
			\Edges(1,2,3,4,1)
			\Edges(1,4) \Edges(3,5) \Edges(2,6)
			\end{tikzpicture}
		\end{center}
		
		\item $4,1,1,1,1$
		
		\rta  El grafo  tiene 5 vértices y $8/2 = 4$ aristas. El único grafo posible es el cual desde un vértice salen 4 aristas a los otros vértices.
		
		\item $7,3,3,3,2,2$ 
		
		\rta El grafo tendría 6 vértices, por lo tanto la valencia máxima podría ser 5 y es imposible que uno de los vértices tenga valencia 7.
		
		\item $4,1,1,1$
		
		\rta Como la suma de las valencias $4+1+1+1 =7$ es impar, no existe un grafo con esas valencias.  
	\end{enumerate}

	\cskip

	\item Demuestre que los siguientes pares de grafos son isomorfos (encuentre un isomorfismo):
	
	\begin{tabular}{ll}
		${}^{}$ \qquad &
		\begin{tikzpicture}[scale=1]
		\draw (-1,2) node {(a)};
		\SetVertexSimple[Shape=circle, FillColor=white,MinSize=8 pt]
		\SetVertexNoLabel
		\Vertex[]{A}
		\Vertex[x=1.5,y=0]{B}
		\Vertex[x=3,y=0]{C}
		\Vertex[x=1.5,y=1.5]{D}
		\Vertex[x=1.5,y=-1.5]{E}
		%
		\Edges(A,D,C,E,A)
		\Edges(A,B,C)
		\Edges(D,B)
		
		\Vertex[x=4.5,y=0.5]{2}
		\Vertex[x=6,y=0.5]{3}
		\Vertex[x=7.5,y=0.5]{4}
		\Vertex[x=4.5,y=-1]{5}
		\Vertex[x=6,y=-1]{6}
		\Edge[style={bend left}](2)(4)
		\Edges(2,3,4,6,5,2)
		\Edges(4,3,6)
		\end{tikzpicture}
	\end{tabular}
	
	\rta Si mandamos, en orden, los puntos del rombo a los puntos del cuadrado y el punto del centro al punto exterior al cuadrado obtenemos un isomorfismo. 
	
	\begin{tabular}{ll}
		${}^{}$ \qquad &
		\begin{tikzpicture}[scale=1]
		\draw (-1,1) node {(b)};
		\SetVertexSimple[Shape=circle, FillColor=white,MinSize=8 pt]
		%\SetVertexNoLabel
		\Vertex[x=0,y=0]{A}
		\Vertex[x=1.5,y=0.8]{B}
		\Vertex[x=3,y=0]{C}
		\Vertex[x=1.5,y=-0.8]{D}
		\Vertex[x=0,y=-0.8]{E}
		\Vertex[x=1.5,y=0]{F}
		\Vertex[x=3,y=-0.8]{G}
		\Vertex[x=1.5,y=-1.6]{H}
		%
		\Edges(A,B,C,D,A)
		\Edges(E,F,G,H,E)
		\Edges(A,E)
		\Edges(B,F)
		\Edges(C,G)
		\Edges(D,H)
		
		
		\Vertex[x=4.5,y=0]{1}
		\Vertex[x=5.5,y=0]{2}
		\Vertex[x=6.5,y=0]{3}
		\Vertex[x=7.5,y=0]{4}
		\Vertex[x=4.5,y=-1]{5}
		\Vertex[x=5.5,y=-1]{6}
		\Vertex[x=6.5,y=-1]{7}
		\Vertex[x=7.5,y=-1]{8}
		\Edge[style={bend left}](1)(4)
		\Edges(1,2,3,4,8,7,6,5,1)
		\Edges(2,6,7,3)
		\Edge[style={bend right}](5)(8)
		\end{tikzpicture}
	\end{tabular}
	
	\rta Observar que el primer grafo es un cubo. Si en el segundo  grafo ``levantamos'' el cuadrado del  centro, podemos deformar ese grafo a un cubo.


	\cskip

	\item Sean $G=(V,E)$ y $G^{\prime}=(V^{\prime},E^{\prime})$ dos grafos y sea $\alpha :V \mapsto V^{\prime}$ una función biyectiva  tal que $\delta (v)=\delta (\alpha (v)) \;\;\forall\,\; v \in V$.
	\begin{enumerate}
		\item ¿Puede afirmar que $\alpha $ es un isomorfismo?.
		
		\rta No. Los grafos 
		\begin{center}
			\begin{tabular}{llll}
				&
				\begin{tikzpicture}[scale=1]
				\SetVertexSimple[Shape=circle,FillColor=white,MinSize=8 pt]
				\Vertex[x=0.00, y=2.00]{a}
				\Vertex[x=2., y=-1.50]{b}
				\Vertex[x=-2., y=-1.50]{c}
				\Edges(a,b,c,a)
				\Vertex[x=0.00, y=0.85]{1}
				\Vertex[x=1., y=-0.9]{2}
				\Vertex[x=-1., y=-0.9]{3}
				\Edges(1,2,3,1)
				\Edges(a,1,3,c,b,2)
				\end{tikzpicture}
				&
				\qquad
				& 
				\begin{tikzpicture}[scale=0.65]
				\SetVertexSimple[Shape=circle,FillColor=white,MinSize=8 pt]
				%
				\Vertex[x=3.00, y=0.00]{1}
				\Vertex[x=1.50, y=2.60]{2}
				\Vertex[x=-1.50, y=2.60]{3}
				\Vertex[x=-3.00, y=0.00]{4}
				\Vertex[x=-1.50, y=-2.60]{5}
				\Vertex[x=1.50, y=-2.60]{6}
				\Edges(1,2,3,4,5,6,1)
				\Edges(1,4) \Edges(3,6) \Edges(2,5)
				\end{tikzpicture}
			\end{tabular}
		\end{center}
		son regulares de valencia 3 y no son isomorfos: el primero tiene 3-ciclos y el segundo no. 
		\item ¿Puede afirmarlo si $|V|=3$ ó 4?.
		
		\rta Si $|V|=3$, los únicos grafos posibles son 
		\begin{center}
			\begin{tabular}{lllllll}
				\begin{tikzpicture}[scale=0.8]
				\SetVertexSimple[Shape=circle,FillColor=white,MinSize=8 pt]
				\Vertex[x=0, y=0]{a}
				\Vertex[x=1, y=0]{b}
				\Vertex[x=2, y=0]{c}
				\end{tikzpicture}
				&\qquad\qquad
				&
				\begin{tikzpicture}[scale=0.8]
				\SetVertexSimple[Shape=circle,FillColor=white,MinSize=8 pt]
				\Vertex[x=0, y=0]{a}
				\Vertex[x=1, y=0]{b}
				\Vertex[x=2, y=0]{c}
				\Edges(a,b)
				\end{tikzpicture}
				&\qquad\qquad
				&
				\begin{tikzpicture}[scale=0.8]
				\SetVertexSimple[Shape=circle,FillColor=white,MinSize=8 pt]
				\Vertex[x=0, y=0]{a}
				\Vertex[x=1, y=0]{b}
				\Vertex[x=2, y=0]{c}
				\Edges(a,b,c)
				\end{tikzpicture}
				&\qquad\qquad
				& 
				\SetVertexSimple[Shape=circle,FillColor=white,MinSize=8 pt]
				\begin{tikzpicture}[scale=0.8]
				\Vertex[x=0, y=0]{a}
				\Vertex[x=1, y=0]{b}
				\Vertex[x=2, y=0]{c}
				\Edges(a,b,c)
				\Edge[style={bend left}](a)(c)
				\end{tikzpicture}
			\end{tabular}
		\end{center}
		La lista de valencias de estos grafos es $[0,0,0]$, $[1,1,0]$, $[1,1,2]$ y $[2,2,2]$, respectivamente. Por lo tanto, la lista de valencias determinan el grafo.
		
		\vskip .3cm
		
		En  el caso que  $|V|=4$ tenemos un solo grafo sin aristas, un solo grafo con una arista y  si seguimos agregando aristas podemos ver que hay 2 grafos con 2 aristas, 3 grafos con 3 aristas, 2 grafos con 4 aristas, un grafo con 5 aristas y un grafo con 6 aristas. Estos grafos son: 
		\begin{center}
			\begin{tabular}{l}
				$0 \to $ \hskip 0.5cm
				\begin{tikzpicture}[scale=0.8]
				\SetVertexSimple[Shape=circle,FillColor=white,MinSize=8 pt]
				\Vertex[x=0, y=0]{a}
				\Vertex[x=1, y=0]{b}
				\Vertex[x=1, y=1]{c}
				\Vertex[x=0, y=1]{d}
				\end{tikzpicture}
				\hskip 0.5cm$1 \to $ \hskip 0.5cm
				\begin{tikzpicture}[scale=0.8]
				\SetVertexSimple[Shape=circle,FillColor=white,MinSize=8 pt]
				\Vertex[x=0, y=0]{a}
				\Vertex[x=1, y=0]{b}
				\Vertex[x=1, y=1]{c}
				\Vertex[x=0, y=1]{d}
				\Edges(a,b)
				\end{tikzpicture}
				\\
				\\
				$2 \to $	\hskip 0.5cm
				\begin{tikzpicture}[scale=0.8]
				\SetVertexSimple[Shape=circle,FillColor=white,MinSize=8 pt]
				\Vertex[x=0, y=0]{a}
				\Vertex[x=1, y=0]{b}
				\Vertex[x=1, y=1]{c}
				\Vertex[x=0, y=1]{d}
				\Edges(a,b,c)
				\end{tikzpicture}
				\hskip 0.5cm\begin{tikzpicture}[scale=0.8]
				\SetVertexSimple[Shape=circle,FillColor=white,MinSize=8 pt]
				\Vertex[x=0, y=0]{a}
				\Vertex[x=1, y=0]{b}
				\Vertex[x=1, y=1]{c}
				\Vertex[x=0, y=1]{d}
				\Edges(a,b)
				\Edges(c,d)
				\end{tikzpicture}
				\\
				\\
				$3 \to $	\hskip 0.5cm
				\begin{tikzpicture}[scale=0.8]
				\SetVertexSimple[Shape=circle,FillColor=white,MinSize=8 pt]
				\Vertex[x=0, y=0]{a}
				\Vertex[x=1, y=0]{b}
				\Vertex[x=1, y=1]{c}
				\Vertex[x=0, y=1]{d}
				\Edges(a,b,c,a)
				\end{tikzpicture}
				\hskip 0.5cm\begin{tikzpicture}[scale=0.8]
				\SetVertexSimple[Shape=circle,FillColor=white,MinSize=8 pt]
				\Vertex[x=0, y=0]{a}
				\Vertex[x=1, y=0]{b}
				\Vertex[x=1, y=1]{c}
				\Vertex[x=0, y=1]{d}
				\Edges(a,b,c,b,d)
				\end{tikzpicture}
				\hskip 0.5cm\begin{tikzpicture}[scale=0.8]
				\SetVertexSimple[Shape=circle,FillColor=white,MinSize=8 pt]
				\Vertex[x=0, y=0]{a}
				\Vertex[x=1, y=0]{b}
				\Vertex[x=1, y=1]{c}
				\Vertex[x=0, y=1]{d}
				\Edges(a,b,c,d)
				\end{tikzpicture}
				\\
				\\
				$4 \to $	\hskip 0.5cm
				\begin{tikzpicture}[scale=0.8]
				\SetVertexSimple[Shape=circle,FillColor=white,MinSize=8 pt]
				\Vertex[x=0, y=0]{a}
				\Vertex[x=1, y=0]{b}
				\Vertex[x=1, y=1]{c}
				\Vertex[x=0, y=1]{d}
				\Edges(a,b,c,a,d)
				\end{tikzpicture}
				\hskip 0.5cm\begin{tikzpicture}[scale=0.8]
				\SetVertexSimple[Shape=circle,FillColor=white,MinSize=8 pt]
				\Vertex[x=0, y=0]{a}
				\Vertex[x=1, y=0]{b}
				\Vertex[x=1, y=1]{c}
				\Vertex[x=0, y=1]{d}
				\Edges(a,b,c,d,a)
				\end{tikzpicture}
				\\
				\\
				$5 \to $ \hskip 0.5cm
				\begin{tikzpicture}[scale=0.8]
				\SetVertexSimple[Shape=circle,FillColor=white,MinSize=8 pt]
				\Vertex[x=0, y=0]{a}
				\Vertex[x=1, y=0]{b}
				\Vertex[x=1, y=1]{c}
				\Vertex[x=0, y=1]{d}
				\Edges(a,b,c,d,a,c)
				\end{tikzpicture}
				\hskip 0.5cm$6 \to $ \hskip 0.5cm
				\begin{tikzpicture}[scale=0.8]
				\SetVertexSimple[Shape=circle,FillColor=white,MinSize=8 pt]
				\Vertex[x=0, y=0]{a}
				\Vertex[x=1, y=0]{b}
				\Vertex[x=1, y=1]{c}
				\Vertex[x=0, y=1]{d}
				\Edges(a,b,c,d,a,c,b,d)
				\end{tikzpicture}
			\end{tabular}
		\end{center}
	\end{enumerate}
	Es fácil verificar que las  valencias determinan el grafo.

	\cskip

	\item Encuentre una función del grafo $A$ al $B$ que preserve valencias. ¿Es un isomorfismo?.
	
	\begin{tabular}{llll}
		$A:$ & &\qquad$B:$& \\
		&
		\begin{tikzpicture}[scale=1]
		\SetVertexSimple[Shape=circle,FillColor=white,MinSize=8 pt]
		\Vertex[x=0,y=0]{A}
		\Vertex[x=3,y=0]{B}
		\Vertex[x=3,y=-3]{C}
		\Vertex[x=0,y=-3]{D}
		\Vertex[x=1,y=-1]{E}
		\Vertex[x=2,y=-1]{F}
		\Vertex[x=2,y=-2]{G}
		\Vertex[x=1,y=-2]{H}
		\Edges(A,B,C,D,A)
		\Edges(E,F)
		\Edges(G,H)
		\Edges(A,E,G,C)
		\Edges(B,F,H,D)
		\end{tikzpicture}
		&
		& \begin{tikzpicture}[scale=1]
		\SetVertexSimple[Shape=circle,FillColor=white,MinSize=8 pt]
		\Vertex[x=0,y=0]{A}
		\Vertex[x=3,y=0]{B}
		\Vertex[x=3,y=-3]{C}
		\Vertex[x=0,y=-3]{D}
		\Vertex[x=1,y=-1]{E}
		\Vertex[x=2,y=-1]{F}
		\Vertex[x=2,y=-2]{G}
		\Vertex[x=1,y=-2]{H}
		\Edges(A,B,C,D,A)
		\Edges(E,F,G,H,E)
		\Edges(A,E)
		\Edges(B,F)
		\Edges(C,G)
		\Edges(H,D)
		\end{tikzpicture}
	\end{tabular}
	
	\rta Todos los vértices tienen valencia 3 en ambos grafos, así que cualquier función biyectiva entre los vértices del primer grafo y los vértices del segundo preserva valencias. Sin embargo, ninguna de estas funciones es un isomorfismo de grafos, pues el primer grafo tiene un 5-ciclo y el segundo no lo tiene.

	\cskip

	\item Pruebe que si $G$ es un grafo con más de un vértice, entonces existen dos vértices con la misma valencia.
	
	\rta Sea $G$ un grafo tiene $n$ vértices con $n >1$, entonces como de cada vértice pueden salir como máximo $n-1$ aristas, las  valencias pueden ser $0, 1, 2,\ldots,n-1$. Supongamos primero que en el grafo  no hay vértices con valencia 0, por lo tanto, las valencias posibles son $1, 2,\ldots,n-1$ que son $n-1$ números y como tenemos $n$ vértices, por el principio de las casillas hay una valencia que se repite. Si $G$ tiene dos vértices con valencia 0, entonces se repite una valencia y listo. Nos queda el caso  de que hay un solo vértice de valencia 0. Si consideramos el subgrafo que se obtiene eliminando el vértice de valencia 0, obtenemos un grafo de $n-1$ vértices y cuyas valencias están en el rango $1, 2,\ldots,n-2$, es decir $n-2$ valores posibles. Por lo tanto, uno de los valores de las valencias se repite. 
	
	\cskip
	
	\item Si $G=(V,E)$ grafo,  el \textit{grafo complemento}  es $G' = (V,E')$, donde $E'$ son todos los 2-subconjuntos de $V$ que no están en $E$. Es decir, el grafo complemento tiene los mismos vértices que el grafo original y todas las aristas que le faltan a $G$ para ser grafo completo. 
	\begin{enumerate}
		\item  Halle el complemento de los siguientes grafos:
		
		\vskip .3cm
		
		\begin{tikzpicture}[scale=1]
		\SetVertexSimple[Shape=circle,FillColor=white,MinSize=8 pt]
		
		\draw (-0.8,0) node {(a)};
		\Vertex[x=0,y=0]{A}
		\Vertex[x=0,y=-1]{B}
		\Edges(A,B)
		
		
		\draw (1.2,0) node {(b)};
		\Vertex[x=2,y=0]{A}
		\Vertex[x=3,y=-0.5]{B}
		\Vertex[x=2,y=-1]{C}
		\Edges(A,B,C,A)
		
		\draw (4.2,0) node {(b)};
		\Vertex[x=5,y=0]{A}
		\Vertex[x=6.5,y=0]{B}
		\Vertex[x=6.5,y=-1.5]{C}
		\Vertex[x=5,y=-1.5]{D}
		\Edges(A,B,C,D,A,C)
		
		\draw (7.2,0) node {(b)};
		\Vertex[x=8,y=-1]{A}
		\Vertex[x=9.5,y=0]{B}
		\Vertex[x=11,y=0]{C}
		\Vertex[x=12.5,y=-1]{D}
		\Vertex[x=11,y=-2]{E}
		\Vertex[x=9.5,y=-2]{F}
		\Edges(A,B,C,D,E,F,A)
		\Edges(B,C,E,F,B)
		\Edges(A,C)
		\end{tikzpicture}
		
		\rta
		
		\vskip .3cm
		
		\begin{tikzpicture}[scale=1]
		\SetVertexSimple[Shape=circle,FillColor=white,MinSize=8 pt]
		
		\draw (-0.8,0) node {(a)};
		\Vertex[x=0,y=0]{A}
		\Vertex[x=0,y=-1]{B}
		
		\draw (1.2,0) node {(b)};
		\Vertex[x=2,y=0]{A}
		\Vertex[x=3,y=-0.5]{B}
		\Vertex[x=2,y=-1]{C}
		
		\draw (4.2,0) node {(b)};
		\Vertex[x=5,y=0]{A}
		\Vertex[x=6.5,y=0]{B}
		\Vertex[x=6.5,y=-1.5]{C}
		\Vertex[x=5,y=-1.5]{D}
		\Edges(B,D)
		
		\draw (7.2,0) node {(b)};
		\Vertex[x=8,y=-1]{A}
		\Vertex[x=9.5,y=0]{B}
		\Vertex[x=11,y=0]{C}
		\Vertex[x=12.5,y=-1]{D}
		\Vertex[x=11,y=-2]{E}
		\Vertex[x=9.5,y=-2]{F}
		\Edges(A,D,A,E)
		\Edges(B,E,B,D,F)
		\Edges(F,C)
		
		\end{tikzpicture}
		
		\cskip
		
		\item  Si $V=\{ v_1 \dots v_n \} $ y $\delta (v_i)=d_i \;\;\forall\,\; i=1,\dots , n\,$, calcule las valencias de el grafo complemento.
		
		\rta Si $v$ vértice del grafo con valencia $\delta(v)=k$, por definición de grafo complemento la valencia de $v$ en el grafo complemento es $n-1 -k$. Por lo tanto si denotamos $\delta'(v)$ la valencia de $v$ en el grafo complemento, la respuesta del ejercicio es $\delta' (v_i)=n-1-d_i \;\;\forall\,\; i=1,\dots , n\,$.
	\end{enumerate}
	
	\cskip

	\item Pruebe que los siguientes grafos no son isomorfos:

	\vskip .3cm 
	
	\begin{tabular}{llll}
		&
		\begin{tikzpicture}[scale=1]
		\SetVertexSimple[Shape=circle,FillColor=white,MinSize=8 pt]
		\Vertex[x=0.00, y=2.00]{a}
		\Vertex[x=2., y=-1.50]{b}
		\Vertex[x=-2., y=-1.50]{c}
		\Edges(a,b,c,a)
		\Vertex[x=0.00, y=0.85]{1}
		\Vertex[x=1., y=-0.9]{2}
		\Vertex[x=-1., y=-0.9]{3}
		\Edges(1,2,3,1)
		\Edges(a,1,3,c,b,2)
		\end{tikzpicture}
		&
		\qquad
		& 
		\begin{tikzpicture}[scale=0.65]
		\SetVertexSimple[Shape=circle,FillColor=white,MinSize=8 pt]
		%
		\Vertex[x=3.00, y=0.00]{1}
		\Vertex[x=1.50, y=2.60]{2}
		\Vertex[x=-1.50, y=2.60]{3}
		\Vertex[x=-3.00, y=0.00]{4}
		\Vertex[x=-1.50, y=-2.60]{5}
		\Vertex[x=1.50, y=-2.60]{6}
		\Edges(1,2,3,4,5,6,1)
		\Edges(1,4) \Edges(3,6) \Edges(2,5)
		\end{tikzpicture}
	\end{tabular}
	
	\rta Los grafos no son isomorfos pues el primero tiene un 3-ciclo y el segundo no lo tiene. 

	\cskip

	\item Dados los siguientes grafos:
	
	\begin{tikzpicture}[scale=0.8]
	%\SetVertexSimple[Shape=circle,FillColor=white,MinSize=8 pt]
	
	\draw (-0.8,0.2) node {(1)};
	\Vertex[x=0,y=-1.5, L=$a$]{A} %\draw (-0.50,-1.5) node {$a$};
	\Vertex[x=1.5,y=0, L=$b$]{B}  %\draw (1,0) node {$b$};
	\Vertex[x=3,y=-1.5, L=$c$]{C}
	\Vertex[x=1.5,y=-3, L=$d$]{D}
	\Vertex[x=1.5,y=-1.5, L=$e$]{E}
	\Edges(A,B,C,D,A)
	\Edges(A,E,C)
	\Edges(B,E,D)
	
	\draw (4.2,0.2) node {(2)};
	\Vertex[x=5,y=-1.5]{1}
	\Vertex[x=6.5,y=0]{2}
	\Vertex[x=8.5,y=-1.5]{3}
	\Vertex[x=5.7,y=-2.5]{4}
	\Vertex[x=7.7,y=-2.7]{5}
	\Edges(1,2,3,4,1,5)
	\Edges(1,3)
	\Edges(2,4)
	
	\draw (10,0.2) node {(3)};
	\Vertex[x=12.8,y=0]{1}
	\Vertex[x=11.8,y=-1.2]{2}
	\Vertex[x=3+10.8,y=-1.2]{3}
	\Vertex[x=0+10.8,y=-2.4]{4}
	\Vertex[x=4+10.8,y=-2.4]{5}
	\Vertex[x=1+10.8,y=-3.6]{6}
	\Vertex[x=3+10.8,y=-3.6]{7}
	\Vertex[x=2+10.8,y=-3.8]{8}
	\Edges(1,6,8,7,1)
	\Edges(3,2,7,8,6,4,5)
	\end{tikzpicture} 
	
	\begin{tikzpicture}[scale=0.8]
	\draw (-0.8,-5.5) node {(4)};
	\Vertex[x=0+0.2,y=0-5.9]{1}
	\Vertex[x=3+0.2,y=0-5.9]{2}
	\Vertex[x=3+0.2,y=-3-5.9]{3}
	\Vertex[x=1.5+0.2,y=-3-5.9]{4}
	\Vertex[x=0+0.2,y=-3-5.9]{5}
	\Edges(1,2,3,4,5,1)
	\Edges(2,4,1,3)
	
	\draw (4.2,-5.3) node {(5)};
	\Vertex[x=0+5,y=-1.5-5.9, L=$a$]{A}
	\Vertex[x=1.5+5,y=0-5.9, L=$b$]{B}
	\Vertex[x=3+5,y=-1.5-5.9, L=$c$]{C}
	\Vertex[x=1.5+5,y=-3-5.9, L=$d$]{D}
	\Vertex[x=1.5+5,y=-1.5-5.9, L=$e$]{E}
	\Edges(A,B,C,D,A)
	\Edges(A,E,C)
	
	
	\draw (10,-5.3) node {(6)};
	\Vertex[x=0+5+5.8,y=-1.5-5.9, L=$a$]{A}
	\Vertex[x=1.5+5+5.8,y=0-5.9, L=$b$]{B}
	\Vertex[x=3+5+5.8,y=-1.5-5.9, L=$c$]{C}
	\Vertex[x=1.5+5+5.8,y=-3-5.9, L=$d$]{D}
	\Vertex[x=1.5+5+5.8,y=-1.5-5.9, L=$e$]{E}
	\Edges(A,B,C,E,A,D,E)
	\end{tikzpicture} 
	
	\begin{tikzpicture}[scale=0.7]
	\draw (-0.8,-10) node {(7)};
	\Vertex[x=2,y=-11]{1}
	\Vertex[x=4,y=-11]{2}
	\Vertex[x=6,y=-11]{3}
	\Vertex[x=0,y=-13]{4}
	\Vertex[x=5,y=-14]{5}
	\Vertex[x=6.3,y=-13]{6}
	\Vertex[x=2,y=-15]{7}
	\Vertex[x=8,y=-15]{8}
	\Vertex[x=3,y=-16]{9}
	\Vertex[x=7,y=-16]{10}
	\Vertex[x=3,y=-18]{11}
	\Vertex[x=5,y=-18]{12}
	\Edges(1,6,12,9,1)
	\Edges(4,2,5)
	\Edges(3,7,11,10,8,3)
	\end{tikzpicture} 
	
	
	
	\begin{enumerate}
		\item Determine en cada caso si existen subgrafos completos de más de 2 vértices.
		
		\rta En (1) hay varios $K_3$: $abe$, $bec$, $aed$, $ecd$. En (2) El subgrafo con vértices $1,2,3,4$  es un $K_4$. En (3) no hay grafos completos. En (4) hay varios subrafos isomorfos s $K_3$, por ejemplo el formado por los vértices $1,4,5$. En (5) no hay subgrafos completos. En (6) hay un $K_3$ formado por los vértices $a,e,d$. En (7) no hay grafos completos. 
		
		\item Para el grafo (1), dé todos los caminos que unen $a$ con $b$.
		
		\rta $ab$, $adcb$, $adeb$, $adceb$, $adecb$, $aeb$, $aecb$, $aedcb$.  
		
		
		\item Dé caminatas eulerianas en los grafos (4), (5) y (6).
		
		\rta  En  el  grafo (4), $\delta(2)=3$, $\delta(3)=3$ y todas las demás valencias son pares. Luego la caminata debes salir de $2$ y llegar a $3$ o viceversa. Una caminata posible es $234514213$.    
		
		En  el grafo (5) la caminata euleriana debe salir de $a$ y llegar a $c$ o viceversa. Una caminata posible es: $abcdaec$.
		
		En  el grafo (6) la caminata euleriana debe salir de $a$ y llegar a $e$ o viceversa. Una caminata posible es: $abcedae$.
		
		\item Para (2) y (3), decir si existen ciclos hamiltonianos.
		
		\rta En (2) el  ciclo $512341$ es hamiltoniano. 
		
		
		En (3) no hay ciclos hamiltonianos: observemos  que un ciclo hamiltoniano puede empezar de cualquier vértice. Supongamos que (3) tiene un ciclo hamiltoniano y lo empezamos del vértice $3$. El comienzo del ciclo es $327$, luego podría ser  a) $3271$ o  b) $3278$. 
		
		En  el caso a) solo  puede continuar como $32716$ y  de ahí  como $327168$ o $327164$. El $327168$ solo puede continuar en $3271687$ repitiendo vértice sin ser ciclo. El  $327164$ solo puede continuar en $3271645$ y ahí terminar  sin ser ciclo. 
		
		En  el caso b) solo puede continuar como $32786$ y luego  como $3278645$ terminando sin ser ciclo,  o $3278617$ repitiendo vértice sin ser ciclo. 
		
		En  definitiva no hay ciclos hamiltonianos en (3). 
		
		\item Determinar cuales de los siguientes pares de grafos son isomorfos:
		
		(i) (4) y (2),\quad 
		
		\rta (2) tiene un vértice de  valencia 1, mientras que en  (4) todas las valencias son mayores que 1, por lo tanto (2) y (4) no pueden ser isomorfos. 
		
		(ii) (5) y (6), \quad 
		
		\rta En este caso  la lista de valencias de ambos grafos es 2, 2, 2, 3, 3 y por lo tanto no podemos usar valencias para distinguir estos grafos. Ahora bien, en (6) tenemos el 3-ciclo $aed$, mientras que en (5) no hay 3-ciclos, por lo tanto (5) y (6) no son isomorfos.   
		
		(iii) (5) y (1). 
		
		\rta 
		En (1) hay un vértice de valencia 4 mientras que  en (5) no lo hay, por lo tanto los grafos no son isomorfos. 
		
		
		
		\item Halle las componentes conexas del grafo (7).
		
		\rta Los subgrafos con vértices $\{1,6,12,9 \}$, $\{4,2,5 \}$ y $\{3,7,11,10,8 \}$ son las componentes conexas de (7).
		
	
	\end{enumerate}

	\cskip

	\item\label{ej-ciclo-hamiltoniano} Dado el siguiente grafo
	$$
	\begin{matrix}
	0&1&2&3&4&5&6&7&8\\ \hline
	1&0&1&0&3&0&1&0&1\\
	3&2&3&2&5&4&5&2&3\\
	5&6&7&4& &6&7&6&5\\
	7&8& &8& &8& &8&7.
	\end{matrix}
	$$
	encuentre un ciclo hamiltoniano (si existe). Determine si existe una caminata euleriana y en caso de ser así encuentre una. 
	
	\rta En  general el  problema de encontrar ciclos hamiltonianos es difícil de resolver, pero en este caso hay una pista que nos puede ayudar: observar que los vértices pares se relacionan solamente con vértices impares y recíprocamente los vértices impares solo son adyacentes a vértices pares. Podemos graficar el grafo  poniendo todos los vértices impares en una columna a la izquierda y los pares a la derecha. Vemos como hay aristas que van de un costado al otro pero  na hay aristas dentro de la misma columna, estos grafos se llaman  \textit{bipartitos}.  
	
	\begin{center}
		\begin{tikzpicture}[scale=0.8]
		\Vertex[x=0,y=0]{1}
		\Vertex[x=0,y=-1]{3}
		\Vertex[x=0,y=-2]{5}
		\Vertex[x=0,y=-3]{7}
		\Vertex[x=5,y=1]{0}
		\Vertex[x=5,y=0]{2}
		\Vertex[x=5,y=-1]{4}
		\Vertex[x=5,y=-2]{6}
		\Vertex[x=5,y=-3]{8}
		\Edge(0)(1) \Edge(0)(3) \Edge(0)(5) \Edge(0)(7) 
		\Edge(1)(2) \Edge(1)(6) \Edge(1)(8)
		\Edge(2)(3) \Edge(2)(7)
		\Edge(3)(4) \Edge(3)(8)
		\Edge(4)(5)
		\Edge(5)(6) \Edge(5)(8)
		\Edge(6)(7)
		\Edge(7)(8)
		\end{tikzpicture}
	\end{center}
	Ahora bien, más allá del dibujo, un ciclo hamiltoniano pasa por los 9 vértices y por lo tanto debe tener 9 aristas. Si partimos de la columna de impares (por ejemplo de 1), la primera arista va a la columna de pares, la segunda vuelve a la columna de impares, y  así sucesivamente hasta que nos damos cuenta que una arista impar debe terminar en la columna de pares. Si  partimos de la columna de pares el razonamiento es análogo y podemos concluir \textit{cualquier ciclo impar termina en la otra columna.} Por lo tanto no hay ciclos hamiltonianos. 
	
	
	Con respecto a las caminata eulerianas, como solo hay dos vértices de valencia impar (2 y 6),  hay caminatas eulerianas que van de 2 a 6 o viceversa. La cantidad de aristas total es $\frac{6\cdot 4 + 2 \cdot 3 + 2}{2} = 16$,  que debe ser el largo de la caminata. Una posible caminata es $21034567850723816$.  
	
	\cskip
	
	\cskip

	\item Un ratón intenta comer un $3\times 3\times 3$ cubo de queso. Él comienza en una esquina y come un subcubo de $1\times 1\times 1$, para luego pasar a un subcubo adyacente. ¿Podrá el ratón terminar de comer el queso en el centro?
	
	\rskip
	
	 \rta Podemos modelar este problema con un cubo subdividido en 2 en cada dimensión, o también como un grafo en el espacio con vértices $(x,y,z)$ con $0 \le x,y,z \le 2$ y $x,y,x$ enteros. El grafo sería como en el siguiente dibujo, donde el vértice de partida está en gris y el vértice de llegada en negro. 
	 \begin{center}
	 	\begin{tikzpicture}[scale=2]
	 	\foreach \x in{0,...,2}
	 	{   \draw (0,\x ,2) -- (2,\x ,2);
	 		\draw (\x ,0,2) -- (\x ,2,2);
	 		\draw (2,\x ,2) -- (2,\x ,0);
	 		\draw (\x ,2,2) -- (\x ,2,0);
	 		\draw (2,0,\x ) -- (2,2,\x );
	 		\draw (0,2,\x ) -- (2,2,\x );
	 	}
	 	%\draw (1,0,0) node {(100)};
	 	%\draw (0,1,0) node {(010)};
	 	%\draw (0,0,1) node {(001)};
	 	
	 	\draw[gray] (0,0 ,1) -- (2,0,1);
	 	\draw[gray] (0,0 ,0) -- (2,0,0);
	 	\draw[gray] (0,1,0) -- (2,1,0);
	 	\draw[gray] (0,1,0) -- (0,1,2);
	 	\draw[gray] (0,1 ,1) -- (2,1,1);
	 	\draw[gray] (0,0 ,1) -- (0,2,1);
	 	\draw[gray] (1,0 ,1) -- (1,2,1); 
	 	\draw[gray] (0,0 ,0) -- (0,2,0);
	 	\draw[gray] (1,0 ,0) -- (1,2,0);
	 	\draw[gray] (0,0 ,0) -- (0,0,2);
	 	\draw[gray] (1,0 ,0) -- (1,0,2);
	 	\draw[gray] (1,1 ,0) -- (1,1,2);
	 	\foreach \x in{0,...,2}
	 	{
	 		\foreach \y in{0,...,2}
	 		{
	 			\foreach \z in{0,...,2}
	 			{\draw[fill=white] (\x,\y,\z) circle (0.07cm);}
	 		}
	 	}
	 	\draw[fill=gray] (0,0,2) circle (0.07cm);
	 	\draw[fill=black] (1,1,1) circle (0.07cm);
	 	
	 	\end{tikzpicture}  
	 	
	 \end{center}
	 Má allá de la representación gráfica, lo que se pregunta en el ejercicio es sobre la existencia de un \textit{camino hamiltoniano} desde $(0,0,0)$ hasta $(1,1,1)$, es decir un camino  que comience en $(0,0,0)$ pase por todos los vértices y  termine en $(1,1,1)$. Veremos que no podremos hacer el camino requerido por el ejercicio y que la forma de resolver el problema es similar a la del ejercicio \ref{ej-ciclo-hamiltoniano}.    
	 
	 Primero definiremos un concepto muy sencillo: diremos  que una 3-upla de enteros $(x,y,z)$ es  \textit{par} si $x+y+z$ es un número par, en caso contrario diremos que es impar. 
	 Ahora bien, podemos ver el pasaje de un vértice a otro como una operación algebraica, por ejemplo, la arista que une $(0,0,2)$ con $(0,1,2)$ se obtiene sumando $(0,1,0)$ a $(0,0,2)$.  En general  para pasar de un nodo a otro debemos sumar o restar  $(1,0,0)$ o $(0,1,0)$ o $(0,0,1)$ y es claro que al hacer esta operación cambiamos la paridad del vértice. Si pasamos de $v_0$ a $v_1$ y de $v_1$ a $v_2$ ($v_0,v_1,v_2$ vértices) entonces la parida de $v_0$ es igual a la paridad de $v_2$.  A partir de esta observación podemos concluir que si partimos de un vértice par y hacemos una caminata con un número par de aristas,  entonces terminaremos en un vértice par. Ahora bien, el  ejercicio nos pregunta sobre un camino hamiltoniano desde $(0,0,0)$ hasta $(1,1,1)$. En este caso el camino hamiltoniano debería tener 26 aristas, pues pasa por 27 vértices y por lo tanto,  al partir de un vértice par,  debería terminar  en otro vértice par, sin embargo $(1,1,1)$ es un vértice impar y concluimos que  este camino hamiltoniano  no  puede existir.  
 
	\cskip 
 
	\item Dé todos los árboles de 5 vértices no isomorfos.
	 
	\rta Sabemos que un árbol (conexo por definición) con 5 vértices debe tener 4 aristas. Y que cualquier grafo con 4 aristas tendría un grado total (suma de todas las valencias) de 8.
	Así que nuestro problema se reduce en encontrar árboles con 5 vértices cuyo grado total sea 8, y donde, cada uno de los vértice tiene valencia $\ge 1$. Las posible configuraciones de valencias son 
	\begin{align*}
		&1, 1, 1, 1, 4 \\
		&1, 1, 1, 2, 3 \\
		&1, 1, 2, 2, 2
	\end{align*}
	Veamos caso  por caso.
	
	$1, 1, 1, 1, 4$: es claro que en este caso hay un solo grafo posible,
	\begin{center}
		\begin{tikzpicture}[scale=0.71]
		\SetVertexSimple[Shape=circle,FillColor=white,MinSize=8 pt]
		\Vertex[x=0.00, y=0.00]{1}
		\Vertex[x=-1., y=-2]{2}
		\Vertex[x=-3., y=-2]{3}
		\Vertex[x=1., y=-2]{4}
		\Vertex[x=3., y=-2]{5}
		\Edges(1,2,1,3,1,4,1,5)
		\end{tikzpicture}
	\end{center}
	
	$1, 1, 1, 2, 3$: se puede construir a partir del anterior ``cambiando'' una arista de lugar
	\begin{center}
		\begin{tikzpicture}[scale=0.71]
		\SetVertexSimple[Shape=circle,FillColor=white,MinSize=8 pt]
		\Vertex[x=0.00, y=0.00]{1}
		\Vertex[x=-1., y=-2]{2}
		\Vertex[x=-3., y=-2]{3}
		\Vertex[x=1., y=-2]{4}
		\Vertex[x=3., y=-2]{5}
		\Edges(1,2,1,3,1,4,5)
		\end{tikzpicture}
	\end{center}
	y vemos que no hay más.
	
	$1, 1, 2, 2, 2$: en este caso solo podemos tener
	\begin{center}
		\begin{tikzpicture}[scale=0.71]
		\SetVertexSimple[Shape=circle,FillColor=white,MinSize=8 pt]
		\Vertex[x=5, y=-2]{1}
		\Vertex[x=-1., y=-2]{2}
		\Vertex[x=-3., y=-2]{3}
		\Vertex[x=1., y=-2]{4}
		\Vertex[x=3., y=-2]{5}
		\Edges(3,2,4,5,1)
		\end{tikzpicture}
	\end{center}

\end{enumerate}
\end{document}