%\documentclass{beamer} 
\documentclass[handout]{beamer} % sin pausas
\usetheme{CambridgeUS}

\usepackage{etex}
\usepackage{t1enc}
\usepackage[spanish,es-nodecimaldot]{babel}
\usepackage{latexsym}
\usepackage[utf8]{inputenc}
\usepackage{verbatim}
\usepackage{multicol}
\usepackage{amsgen,amsmath,amstext,amsbsy,amsopn,amsfonts,amssymb}
\usepackage{amsthm}
\usepackage{calc}         % From LaTeX distribution
\usepackage{graphicx}     % From LaTeX distribution
\usepackage{ifthen}
%\usepackage{makeidx}
\input{random.tex}        % From CTAN/macros/generic
\usepackage{subfigure} 
\usepackage{tikz}
\usepackage[customcolors]{hf-tikz}
\usetikzlibrary{arrows}
\usetikzlibrary{matrix}
\tikzset{
	every picture/.append style={
		execute at begin picture={\deactivatequoting},
		execute at end picture={\activatequoting}
	}
}
\usetikzlibrary{decorations.pathreplacing,angles,quotes}
\usetikzlibrary{shapes.geometric}
\usepackage{mathtools}
\usepackage{stackrel}
%\usepackage{enumerate}
\usepackage{enumitem}
\usepackage{tkz-graph}
\usepackage{polynom}
\polyset{%
	style=B,
	delims={(}{)},
	div=:
}
\renewcommand\labelitemi{$\circ$}
\setlist[enumerate]{label={(\arabic*)}}
%\setbeamertemplate{background}[grid][step=8 ] % cuadriculado
\setbeamertemplate{itemize item}{$\circ$}
\setbeamertemplate{enumerate items}[default]
\definecolor{links}{HTML}{2A1B81}
\hypersetup{colorlinks,linkcolor=,urlcolor=links}


\newcommand{\Id}{\operatorname{Id}}
\newcommand{\img}{\operatorname{Im}}
\newcommand{\nuc}{\operatorname{Nu}}
\newcommand{\im}{\operatorname{Im}}
\renewcommand\nu{\operatorname{Nu}}
\newcommand{\la}{\langle}
\newcommand{\ra}{\rangle}
\renewcommand{\t}{{\operatorname{t}}}
\renewcommand{\sin}{{\,\operatorname{sen}}}
\newcommand{\Q}{\mathbb Q}
\newcommand{\R}{\mathbb R}
\newcommand{\C}{\mathbb C}
\newcommand{\K}{\mathbb K}
\newcommand{\F}{\mathbb F}
\newcommand{\Z}{\mathbb Z}
\newcommand{\N}{\mathbb N}
\newcommand\sgn{\operatorname{sgn}}
\renewcommand{\t}{{\operatorname{t}}}
\renewcommand{\figurename }{Figura}

%
% Ver http://joshua.smcvt.edu/latex2e/_005cnewenvironment-_0026-_005crenewenvironment.html
%

\renewenvironment{block}[1]% environment name
{% begin code
	\par\vskip .2cm%
	{\color{blue}#1}%
	\vskip .2cm
}%
{%
	\vskip .2cm}% end code


\renewenvironment{alertblock}[1]% environment name
{% begin code
	\par\vskip .2cm%
	{\color{red!80!black}#1}%
	\vskip .2cm
}%
{%
	\vskip .2cm}% end code


\renewenvironment{exampleblock}[1]% environment name
{% begin code
	\par\vskip .2cm%
	{\color{blue}#1}%
	\vskip .2cm
}%
{%
	\vskip .2cm}% end code




\newenvironment{exercise}[1]% environment name
{% begin code
	\par\vspace{\baselineskip}\noindent
	\textbf{Ejercicio (#1)}\begin{itshape}%
		\par\vspace{\baselineskip}\noindent\ignorespaces
	}%
	{% end code
	\end{itshape}\ignorespacesafterend
}


\newenvironment{definicion}[1][]% environment name
{% begin code
	\par\vskip .2cm%
	{\color{blue}Definición #1}%
	\vskip .2cm
}%
{%
	\vskip .2cm}% end code

    \newenvironment{notacion}[1][]% environment name
    {% begin code
        \par\vskip .2cm%
        {\color{blue}Notación #1}%
        \vskip .2cm
    }%
    {%
        \vskip .2cm}% end code

\newenvironment{observacion}[1][]% environment name
{% begin code
	\par\vskip .2cm%
	{\color{blue}Observación #1}%
	\vskip .2cm
}%
{%
	\vskip .2cm}% end code

\newenvironment{ejemplo}[1][]% environment name
{% begin code
	\par\vskip .2cm%
	{\color{blue}Ejemplo #1}%
	\vskip .2cm
}%
{%
	\vskip .2cm}% end code


\newenvironment{preguntas}[1][]% environment name
{% begin code
    \par\vskip .2cm%
    {\color{blue}Preguntas #1}%
    \vskip .2cm
}%
{%
    \vskip .2cm}% end code

\newenvironment{ejercicio}[1][]% environment name
{% begin code
	\par\vskip .2cm%
	{\color{blue}Ejercicio #1}%
	\vskip .2cm
}%
{%
	\vskip .2cm}% end code


\renewenvironment{proof}% environment name
{% begin code
	\par\vskip .2cm%
	{\color{blue}Demostración}%
	\vskip .2cm
}%
{%
	\vskip .2cm}% end code



\newenvironment{demostracion}% environment name
{% begin code
	\par\vskip .2cm%
	{\color{blue}Demostración}%
	\vskip .2cm
}%
{%
	\vskip .2cm}% end code

\newenvironment{idea}% environment name
{% begin code
	\par\vskip .2cm%
	{\color{blue}Idea de la demostración}%
	\vskip .2cm
}%
{%
	\vskip .2cm}% end code

\newenvironment{solucion}% environment name
{% begin code
	\par\vskip .2cm%
	{\color{blue}Solución}%
	\vskip .2cm
}%
{%
	\vskip .2cm}% end code



\newenvironment{lema}[1][]% environment name
{% begin code
	\par\vskip .2cm%
	{\color{blue}Lema #1}\begin{itshape}%
		\par\vskip .2cm
	}%
	{% end code
	\end{itshape}\vskip .2cm\ignorespacesafterend
}

\newenvironment{proposicion}[1][]% environment name
{% begin code
	\par\vskip .2cm%
	{\color{blue}Proposición #1}\begin{itshape}%
		\par\vskip .2cm
	}%
	{% end code
	\end{itshape}\vskip .2cm\ignorespacesafterend
}

\newenvironment{teorema}[1][]% environment name
{% begin code
	\par\vskip .2cm%
	{\color{blue}Teorema #1}\begin{itshape}%
		\par\vskip .2cm
	}%
	{% end code
	\end{itshape}\vskip .2cm\ignorespacesafterend
}


\newenvironment{corolario}[1][]% environment name
{% begin code
	\par\vskip .2cm%
	{\color{blue}Corolario #1}\begin{itshape}%
		\par\vskip .2cm
	}%
	{% end code
	\end{itshape}\vskip .2cm\ignorespacesafterend
}

\newenvironment{propiedad}% environment name
{% begin code
	\par\vskip .2cm%
	{\color{blue}Propiedad}\begin{itshape}%
		\par\vskip .2cm
	}%
	{% end code
	\end{itshape}\vskip .2cm\ignorespacesafterend
}

\newenvironment{conclusion}% environment name
{% begin code
	\par\vskip .2cm%
	{\color{blue}Conclusión}\begin{itshape}%
		\par\vskip .2cm
	}%
	{% end code
	\end{itshape}\vskip .2cm\ignorespacesafterend
}


\newenvironment{definicion*}% environment name
{% begin code
	\par\vskip .2cm%
	{\color{blue}Definición}%
	\vskip .2cm
}%
{%
	\vskip .2cm}% end code

\newenvironment{observacion*}% environment name
{% begin code
	\par\vskip .2cm%
	{\color{blue}Observación}%
	\vskip .2cm
}%
{%
	\vskip .2cm}% end code


\newenvironment{obs*}% environment name
	{% begin code
		\par\vskip .2cm%
		{\color{blue}Observación}%
		\vskip .2cm
	}%
	{%
		\vskip .2cm}% end code

\newenvironment{ejemplo*}% environment name
{% begin code
	\par\vskip .2cm%
	{\color{blue}Ejemplo}%
	\vskip .2cm
}%
{%
	\vskip .2cm}% end code

\newenvironment{ejercicio*}% environment name
{% begin code
	\par\vskip .2cm%
	{\color{blue}Ejercicio}%
	\vskip .2cm
}%
{%
	\vskip .2cm}% end code

\newenvironment{propiedad*}% environment name
{% begin code
	\par\vskip .2cm%
	{\color{blue}Propiedad}\begin{itshape}%
		\par\vskip .2cm
	}%
	{% end code
	\end{itshape}\vskip .2cm\ignorespacesafterend
}

\newenvironment{conclusion*}% environment name
{% begin code
	\par\vskip .2cm%
	{\color{blue}Conclusión}\begin{itshape}%
		\par\vskip .2cm
	}%
	{% end code
	\end{itshape}\vskip .2cm\ignorespacesafterend
}






\newcommand{\nc}{\newcommand}

%%%%%%%%%%%%%%%%%%%%%%%%%LETRAS

\nc{\FF}{{\mathbb F}} \nc{\NN}{{\mathbb N}} \nc{\QQ}{{\mathbb Q}}
\nc{\PP}{{\mathbb P}} \nc{\DD}{{\mathbb D}} \nc{\Sn}{{\mathbb S}}
\nc{\uno}{\mathbb{1}} \nc{\BB}{{\mathbb B}} \nc{\An}{{\mathbb A}}

\nc{\ba}{\mathbf{a}} \nc{\bb}{\mathbf{b}} \nc{\bt}{\mathbf{t}}
\nc{\bB}{\mathbf{B}}

\nc{\cP}{\mathcal{P}} \nc{\cU}{\mathcal{U}} \nc{\cX}{\mathcal{X}}
\nc{\cE}{\mathcal{E}} \nc{\cS}{\mathcal{S}} \nc{\cA}{\mathcal{A}}
\nc{\cC}{\mathcal{C}} \nc{\cO}{\mathcal{O}} \nc{\cQ}{\mathcal{Q}}
\nc{\cB}{\mathcal{B}} \nc{\cJ}{\mathcal{J}} \nc{\cI}{\mathcal{I}}
\nc{\cM}{\mathcal{M}} \nc{\cK}{\mathcal{K}}

\nc{\fD}{\mathfrak{D}} \nc{\fI}{\mathfrak{I}} \nc{\fJ}{\mathfrak{J}}
\nc{\fS}{\mathfrak{S}} \nc{\gA}{\mathfrak{A}}
%%%%%%%%%%%%%%%%%%%%%%%%%LETRAS


\title[Clase 15 - Factorización en primos]{Matemática Discreta I \\ Clase 15 - Factorización en primos 2}
%\author[C. Olmos / A. Tiraboschi]{Carlos Olmos / Alejandro Tiraboschi}
\institute[]{\normalsize FAMAF / UNC
	\\[\baselineskip] ${}^{}$
	\\[\baselineskip]
}
\date[07/05/2020]{7 de mayo de 2020}




\begin{document}
	
	\frame{\titlepage} 
	
	\begin{frame}
		\begin{proposicion} Existen infinitos números primos. 
		\end{proposicion} \pause
		\begin{proof}\pause Haremos la demostración por el absurdo. 
			\vskip .3cm
			Sean  $p_1,p_2,\ldots, p_r$ todos los números primos. 
			\vskip .3cm
			Sea $n =  p_1p_2\ldots p_r+1$. 
			\vskip .3cm
			
			Sea $p$ primo tal que $p|n$ $\Rightarrow$ existe $i$  tal que $p=p_i$. 
			\vskip .3cm
			Ahora bien $p_i| n$ y $p_i|p_1p_2\ldots p_r$, luego $p_i|n-p_1p_2\ldots p_r =1$. Absurdo.  
			
			\qed
		\end{proof}
		
	\end{frame}
	
	\begin{frame}
		
		\begin{ejemplo} Probemos que si $m$ y $n$ son enteros tales que
			$m\ge 2$ y $n\ge 2$, entonces $m^2 \not=2n^2$.
		\end{ejemplo}\pause
		\begin{proof} \vskip -.6cm\pause
			$$ n = 2^{x}p_2^{e_2}\ldots p_r^{e_r}\text{\quad ($p_i$ todos primos diferentes a 2.)} $$
			$$ n^2 = 2^{2x}p_2^{2e_2}\ldots p_r^{2e_r}$$
			\begin{equation}
				\colorbox{lightgray}{$2n^2= 2^{2x+1}p_2^{2e_2}\ldots p_r^{2e_r}.$ }\tag{*}
			\end{equation}
			
			$$
			m = 2^{y}q_2^{f_2}\ldots q_s^{f_s}\text{\quad ($q_i$ todos primos diferentes a 2.)}
			$$
			\begin{equation}
				\colorbox{lightgray}{$m^2 = 2^{2y}q_2^{2f_2}\ldots q_s^{2f_s}$} \tag{**}
			\end{equation}
			
			Por unicidad de la descomposición, (*) $\ne$ (**), es decir $m^2 \not=2n^2$.\qed
			
		\end{proof}
		
	\end{frame}
	
	
	\begin{frame}
		\begin{observacion}
			El  ejemplo anterior nos dice que 
			$$
			m^2 \not=2n^2\quad \Rightarrow \quad \frac{m^2}{n^2}\not=2 \quad \Rightarrow \quad \left(\frac{m}{n}\right)^2\not=2 \quad \Rightarrow \quad \frac{m}{n}\not=\sqrt{2}. 
			$$
			
			Es decir $\sqrt{2}$ \textit{no es un número racional.}
		\end{observacion}
	\end{frame}
	
	\begin{frame}\frametitle{Notación}
		
		Sean $m$ y $n$ dos enteros positivos, a veces es conveniente escribir la factorización prima de ambos números usando los mismos primos.
		
		Los primos que usamos son los que se encuentran en la factorización prima de ambos:
		$$
		m=p_1^{e_1}p_2^{e_2}\ldots p_r^{e_r},\qquad
		n=p_1^{f_1}p_2^{f_2}\ldots p_r^{f_r}.
		$$
		con $e_i,f_i \ge 0$ para $i=1,\ldots,r$ y $e_i$ o $f_i$ distinto de cero. 
		\pause
		\begin{ejemplo}
			168 y 495. \pause Tenemos que
			\begin{equation*}
				168 = 2^3 \cdot 3^1 \cdot 7^1, \qquad 495 = 3^2 \cdot 5^1 \cdot 11^1
			\end{equation*}
			Luego 
			\begin{align*}
				168 &= 2^3 \cdot 3^1 \cdot 5^0\cdot 7^1\cdot 11^0, \\
				495 &= 2^0 \cdot3^2 \cdot 5^1 \cdot 7^0 \cdot 11^1
			\end{align*}
		\end{ejemplo}
		
	\end{frame}
	
	
	\begin{frame}
		Veremos ahora un resultado que se puede deducir fácilmente del Teorema Fundamental de la Aritmética (TFA).\pause 
		
		\begin{proposicion} Sean $m,n \ge2$ con
			$$
			m=p_1^{e_1}p_2^{e_2}\ldots p_r^{e_r},\qquad
			n=p_1^{f_1}p_2^{f_2}\ldots p_r^{f_r}.
			$$
			donde $p_i$ primo y $e_i,f_i \ge 0$ para $i=1,\ldots,r$. 
			
			Entonces $m|n$ si y sólo si $e_i \le f_i$ para todo $i$.
		\end{proposicion}\pause 
		\begin{proof}\pause 
			\noindent($\Rightarrow$) Por la descomposición de $m$ es claro que $p^{e_i}|m$. Como $m|n$ entonces   $p^{e_i}|n$. Es decir $n =  p^{e_i}u$. Es claro por TFA entonces que $e_i \le f_i$.
			\vskip .2cm
			
			\noindent($\Leftarrow$) Como $e_i \le f_i$, tenemos que $p^{e_i}|p^{f_i}$, para $1 \le i \le r$.  Luego  $$p_1^{e_1}p_2^{e_2}\ldots p_r^{e_r}| p_1^{f_1}p_2^{f_2}\ldots p_r^{f_r}.$$ Es decir $m|n$.
		\end{proof}
		
	\end{frame}
	
	
	\begin{frame}
		
		Ahora veremos que es posible calcular el $\operatorname{mcd}$ y el $\operatorname{mcm}$ de un par de números sabiendo sus descomposiciones primas.
		
		\begin{proposicion}
			Sean $m$ y $n$ enteros positivos cuyas factorizaciones primas son
			$$
			m=p_1^{e_1}p_2^{e_2}\ldots p_r^{e_r},\qquad
			n=p_1^{f_1}p_2^{f_2}\ldots p_r^{f_r}.
			$$
			\begin{enumerate}
				\item[a)] El mcd de $m$ y $n$ es $d=p_1^{k_1}p_2^{k_2}\ldots
				p_r^{k_r}$ donde, para cada $i$ en el rango $1\le i \le r$, $k_i$
				es el mínimo entre $e_i$ y $f_i$.\vskip .2cm
				\item[b)] El mcm de $m$ y $n$ es $u=p_1^{h_1}p_2^{h_2}\ldots
				p_r^{h_r}$ donde, para cada $i$ en el rango $1\le i \le r$, $h_i$
				es el máximo entre $e_i$ y $f_i$.
			\end{enumerate}
		\end{proposicion}
	\end{frame}
	
	
	\begin{frame}
		\begin{proof}
			\
			
			\noindent(a)  Es claro  que  $d = p_1^{k_1}p_2^{k_2}\ldots p_r^{k_r}$ divide a $m$ y $n$.
			
			\vskip .3cm
			Sea $c$ tal que $c|n$ y $c|m$, entonces los primos que intervienen en la factorización de $c$ son $p_1,\ldots,p_r$ y por lo tanto 
			$$
			c =  p_1^{t_1}p_2^{t_2}\ldots p_r^{t_r}.
			$$
			
			Además, como $c|n$ y $c|m$ tenemos que $t_i \le e_i,f_i$ y por lo tanto $t_i \le k_i = \min(e_i,f_i)$. 
			
			\vskip .3cm
			
			De esto se deduce que $c|p_1^{k_1}p_2^{k_2}\ldots p_r^{k_r}=d$. 
			
			\vskip .6cm
			\noindent(b) Se deja como ejercicio. 
			
			\qed
		\end{proof}
		
	\end{frame}
	
	
	\begin{frame}
		
		\begin{ejemplo}
			Encontremos el mcd y el mcm  de $168$ y $495$.
			\vskip .2cm
			Ya habíamos visto que 
			\begin{align*}
				168 &= 2^3 \cdot 3^1 \cdot 5^0\cdot 7^1\cdot 11^0, \\
				495 &= 2^0 \cdot3^2 \cdot 5^1 \cdot 7^0 \cdot 11^1
			\end{align*}
			
			Luego 
			
			\begin{align*}
				\text{mcd}(168,495) &= 2^0 \cdot 3^1 \cdot 5^0\cdot 7^0\cdot 11^0 = 3, \\
				\text{mcm}(168,495) &= 2^3 \cdot3^2 \cdot 5^1 \cdot 7^1 \cdot 11^1.
			\end{align*}
			
		\end{ejemplo}
		
	\end{frame}
	
	
\end{document}

