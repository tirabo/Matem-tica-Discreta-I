%\documentclass{beamer} 
\documentclass[handout]{beamer} % sin pausas
\usetheme{CambridgeUS}

\usepackage{etex}
\usepackage{t1enc}
\usepackage[spanish,es-nodecimaldot]{babel}
\usepackage{latexsym}
\usepackage[utf8]{inputenc}
\usepackage{verbatim}
\usepackage{multicol}
\usepackage{amsgen,amsmath,amstext,amsbsy,amsopn,amsfonts,amssymb}
\usepackage{amsthm}
\usepackage{calc}         % From LaTeX distribution
\usepackage{graphicx}     % From LaTeX distribution
\usepackage{ifthen}
%\usepackage{makeidx}
\input{random.tex}        % From CTAN/macros/generic
\usepackage{subfigure} 
\usepackage{tikz}
\usepackage[customcolors]{hf-tikz}
\usetikzlibrary{arrows}
\usetikzlibrary{matrix}
\tikzset{
	every picture/.append style={
		execute at begin picture={\deactivatequoting},
		execute at end picture={\activatequoting}
	}
}
\usetikzlibrary{decorations.pathreplacing,angles,quotes}
\usetikzlibrary{shapes.geometric}
\usepackage{mathtools}
\usepackage{stackrel}
%\usepackage{enumerate}
\usepackage{enumitem}
\usepackage{tkz-graph}
\usepackage{polynom}
\polyset{%
	style=B,
	delims={(}{)},
	div=:
}
\renewcommand\labelitemi{$\circ$}
\setlist[enumerate]{label={(\arabic*)}}
%\setbeamertemplate{background}[grid][step=8 ] % cuadriculado
\setbeamertemplate{itemize item}{$\circ$}
\setbeamertemplate{enumerate items}[default]
\definecolor{links}{HTML}{2A1B81}
\hypersetup{colorlinks,linkcolor=,urlcolor=links}


\newcommand{\Id}{\operatorname{Id}}
\newcommand{\img}{\operatorname{Im}}
\newcommand{\nuc}{\operatorname{Nu}}
\newcommand{\im}{\operatorname{Im}}
\renewcommand\nu{\operatorname{Nu}}
\newcommand{\la}{\langle}
\newcommand{\ra}{\rangle}
\renewcommand{\t}{{\operatorname{t}}}
\renewcommand{\sin}{{\,\operatorname{sen}}}
\newcommand{\Q}{\mathbb Q}
\newcommand{\R}{\mathbb R}
\newcommand{\C}{\mathbb C}
\newcommand{\K}{\mathbb K}
\newcommand{\F}{\mathbb F}
\newcommand{\Z}{\mathbb Z}
\newcommand{\N}{\mathbb N}
\newcommand\sgn{\operatorname{sgn}}
\renewcommand{\t}{{\operatorname{t}}}
\renewcommand{\figurename }{Figura}

%
% Ver http://joshua.smcvt.edu/latex2e/_005cnewenvironment-_0026-_005crenewenvironment.html
%

\renewenvironment{block}[1]% environment name
{% begin code
	\par\vskip .2cm%
	{\color{blue}#1}%
	\vskip .2cm
}%
{%
	\vskip .2cm}% end code


\renewenvironment{alertblock}[1]% environment name
{% begin code
	\par\vskip .2cm%
	{\color{red!80!black}#1}%
	\vskip .2cm
}%
{%
	\vskip .2cm}% end code


\renewenvironment{exampleblock}[1]% environment name
{% begin code
	\par\vskip .2cm%
	{\color{blue}#1}%
	\vskip .2cm
}%
{%
	\vskip .2cm}% end code




\newenvironment{exercise}[1]% environment name
{% begin code
	\par\vspace{\baselineskip}\noindent
	\textbf{Ejercicio (#1)}\begin{itshape}%
		\par\vspace{\baselineskip}\noindent\ignorespaces
	}%
	{% end code
	\end{itshape}\ignorespacesafterend
}


\newenvironment{definicion}[1][]% environment name
{% begin code
	\par\vskip .2cm%
	{\color{blue}Definición #1}%
	\vskip .2cm
}%
{%
	\vskip .2cm}% end code

    \newenvironment{notacion}[1][]% environment name
    {% begin code
        \par\vskip .2cm%
        {\color{blue}Notación #1}%
        \vskip .2cm
    }%
    {%
        \vskip .2cm}% end code

\newenvironment{observacion}[1][]% environment name
{% begin code
	\par\vskip .2cm%
	{\color{blue}Observación #1}%
	\vskip .2cm
}%
{%
	\vskip .2cm}% end code

\newenvironment{ejemplo}[1][]% environment name
{% begin code
	\par\vskip .2cm%
	{\color{blue}Ejemplo #1}%
	\vskip .2cm
}%
{%
	\vskip .2cm}% end code


\newenvironment{preguntas}[1][]% environment name
{% begin code
    \par\vskip .2cm%
    {\color{blue}Preguntas #1}%
    \vskip .2cm
}%
{%
    \vskip .2cm}% end code

\newenvironment{ejercicio}[1][]% environment name
{% begin code
	\par\vskip .2cm%
	{\color{blue}Ejercicio #1}%
	\vskip .2cm
}%
{%
	\vskip .2cm}% end code


\renewenvironment{proof}% environment name
{% begin code
	\par\vskip .2cm%
	{\color{blue}Demostración}%
	\vskip .2cm
}%
{%
	\vskip .2cm}% end code



\newenvironment{demostracion}% environment name
{% begin code
	\par\vskip .2cm%
	{\color{blue}Demostración}%
	\vskip .2cm
}%
{%
	\vskip .2cm}% end code

\newenvironment{idea}% environment name
{% begin code
	\par\vskip .2cm%
	{\color{blue}Idea de la demostración}%
	\vskip .2cm
}%
{%
	\vskip .2cm}% end code

\newenvironment{solucion}% environment name
{% begin code
	\par\vskip .2cm%
	{\color{blue}Solución}%
	\vskip .2cm
}%
{%
	\vskip .2cm}% end code



\newenvironment{lema}[1][]% environment name
{% begin code
	\par\vskip .2cm%
	{\color{blue}Lema #1}\begin{itshape}%
		\par\vskip .2cm
	}%
	{% end code
	\end{itshape}\vskip .2cm\ignorespacesafterend
}

\newenvironment{proposicion}[1][]% environment name
{% begin code
	\par\vskip .2cm%
	{\color{blue}Proposición #1}\begin{itshape}%
		\par\vskip .2cm
	}%
	{% end code
	\end{itshape}\vskip .2cm\ignorespacesafterend
}

\newenvironment{teorema}[1][]% environment name
{% begin code
	\par\vskip .2cm%
	{\color{blue}Teorema #1}\begin{itshape}%
		\par\vskip .2cm
	}%
	{% end code
	\end{itshape}\vskip .2cm\ignorespacesafterend
}


\newenvironment{corolario}[1][]% environment name
{% begin code
	\par\vskip .2cm%
	{\color{blue}Corolario #1}\begin{itshape}%
		\par\vskip .2cm
	}%
	{% end code
	\end{itshape}\vskip .2cm\ignorespacesafterend
}

\newenvironment{propiedad}% environment name
{% begin code
	\par\vskip .2cm%
	{\color{blue}Propiedad}\begin{itshape}%
		\par\vskip .2cm
	}%
	{% end code
	\end{itshape}\vskip .2cm\ignorespacesafterend
}

\newenvironment{conclusion}% environment name
{% begin code
	\par\vskip .2cm%
	{\color{blue}Conclusión}\begin{itshape}%
		\par\vskip .2cm
	}%
	{% end code
	\end{itshape}\vskip .2cm\ignorespacesafterend
}


\newenvironment{definicion*}% environment name
{% begin code
	\par\vskip .2cm%
	{\color{blue}Definición}%
	\vskip .2cm
}%
{%
	\vskip .2cm}% end code

\newenvironment{observacion*}% environment name
{% begin code
	\par\vskip .2cm%
	{\color{blue}Observación}%
	\vskip .2cm
}%
{%
	\vskip .2cm}% end code


\newenvironment{obs*}% environment name
	{% begin code
		\par\vskip .2cm%
		{\color{blue}Observación}%
		\vskip .2cm
	}%
	{%
		\vskip .2cm}% end code

\newenvironment{ejemplo*}% environment name
{% begin code
	\par\vskip .2cm%
	{\color{blue}Ejemplo}%
	\vskip .2cm
}%
{%
	\vskip .2cm}% end code

\newenvironment{ejercicio*}% environment name
{% begin code
	\par\vskip .2cm%
	{\color{blue}Ejercicio}%
	\vskip .2cm
}%
{%
	\vskip .2cm}% end code

\newenvironment{propiedad*}% environment name
{% begin code
	\par\vskip .2cm%
	{\color{blue}Propiedad}\begin{itshape}%
		\par\vskip .2cm
	}%
	{% end code
	\end{itshape}\vskip .2cm\ignorespacesafterend
}

\newenvironment{conclusion*}% environment name
{% begin code
	\par\vskip .2cm%
	{\color{blue}Conclusión}\begin{itshape}%
		\par\vskip .2cm
	}%
	{% end code
	\end{itshape}\vskip .2cm\ignorespacesafterend
}






\newcommand{\nc}{\newcommand}

%%%%%%%%%%%%%%%%%%%%%%%%%LETRAS

\nc{\FF}{{\mathbb F}} \nc{\NN}{{\mathbb N}} \nc{\QQ}{{\mathbb Q}}
\nc{\PP}{{\mathbb P}} \nc{\DD}{{\mathbb D}} \nc{\Sn}{{\mathbb S}}
\nc{\uno}{\mathbb{1}} \nc{\BB}{{\mathbb B}} \nc{\An}{{\mathbb A}}

\nc{\ba}{\mathbf{a}} \nc{\bb}{\mathbf{b}} \nc{\bt}{\mathbf{t}}
\nc{\bB}{\mathbf{B}}

\nc{\cP}{\mathcal{P}} \nc{\cU}{\mathcal{U}} \nc{\cX}{\mathcal{X}}
\nc{\cE}{\mathcal{E}} \nc{\cS}{\mathcal{S}} \nc{\cA}{\mathcal{A}}
\nc{\cC}{\mathcal{C}} \nc{\cO}{\mathcal{O}} \nc{\cQ}{\mathcal{Q}}
\nc{\cB}{\mathcal{B}} \nc{\cJ}{\mathcal{J}} \nc{\cI}{\mathcal{I}}
\nc{\cM}{\mathcal{M}} \nc{\cK}{\mathcal{K}}

\nc{\fD}{\mathfrak{D}} \nc{\fI}{\mathfrak{I}} \nc{\fJ}{\mathfrak{J}}
\nc{\fS}{\mathfrak{S}} \nc{\gA}{\mathfrak{A}}
%%%%%%%%%%%%%%%%%%%%%%%%%LETRAS


\title[Clase 17 - Congruencia ]{Matemática Discreta I \\ Clase 17 - Congruencia 2}
%\author[C. Olmos / A. Tiraboschi]{Carlos Olmos / Alejandro Tiraboschi}
\institute[]{\normalsize FAMAF / UNC
	\\[\baselineskip] ${}^{}$
	\\[\baselineskip]
}
\date[14/05/2020]{14 de mayo de 2020}




\begin{document}
	
	\frame{\titlepage} 
	
	\begin{frame}\frametitle{Ecuación lineal de congruencia}
		Estudiaremos el problema de encontrar los $x \in \mathbb Z$ tal que
		
		\begin{equation}\label{ecuacionlineal}
			ax \equiv b \pmod{m}.
		\end{equation}
		
		\pause
		
		Una ecuación como 	(\ref{ecuacionlineal}) es llamada una \textit{ecuación lineal de congruencia.}
		\vskip .4cm\pause
		El problema no siempre admite solución. 
		\vskip .4cm
		\pause
		Por  ejemplo, \,$2x\equiv 3 \pmod{2}$\, no posee ninguna solución en
		$\mathbb Z$, pues cualquiera se $k \in \mathbb Z$, $2k-3$ es
		impar, luego no es divisible por $2$.
		\vskip .4cm\pause
		Notemos además que si $x_0$ es solución de la ecuación
		(\ref{ecuacionlineal}), también lo es $x_0+km$ de manera que si la
		ecuación posee una solución, posee infinitas soluciones. 
		
		
	\end{frame}
	
	
	\begin{frame}
		
		\begin{ejemplo} La solución general de la ecuación $3x\equiv 7
			\pmod{11}$ es $6+k11$ con $k \in \mathbb Z$.
		\end{ejemplo}\vskip .2cm\pause
		\begin{proof} Si probamos con los enteros $x$ tal que  $0\le x < 11$, veremos que la ecuación admite una única solución, a saber $x=6$. 
			\vskip .2cm\pause
			
			Otras soluciones se obtienen tomando $6+11k$. \pause 
			\vskip .2cm
			Por otra parte si $u$ es también solución de la ecuación, se tiene 
			$$
			3u \equiv 7 \hskip -.3cm\pmod{11},\quad 3 \cdot 6 \equiv 7\hskip -.3cm \pmod{11} \quad \Rightarrow \quad 3u \equiv 3\cdot 6\hskip -.3cm \pmod{11}.
			$$
			\pause Por lo tanto $3(u-6)$ es múltiplo de $11$. 
			\vskip .2cm\pause
			
			Como $11$ no divide a $3$ se tiene que $11|(u-6)$, o sea $u=6+11k$. \qed
		\end{proof}
	\end{frame}
	
	
	\begin{frame}
		\begin{ejemplo} Encontrar $0 \le x < 109$ solución  de la ecuación $74x\equiv 5 \pmod{109}$.
		\end{ejemplo}\pause
		\begin{proof} \pause
			
			\begin{itemize}
				\item[$\circ$] $1 =(74,109)$, por lo tanto, existen $s,t \in \mathbb Z$ tal que
				\begin{equation}\label{eq:elc1}
					1  = s \cdot 74 + t\cdot 109
				\end{equation}
				\vskip .2cm\pause
				\item[$\circ$] Luego,
				\begin{equation}\label{eq:elc2}
					1  \equiv  s \cdot 74\pmod{109}.
				\end{equation} \vskip .2cm\pause
				\item[$\circ$] Multiplicando  por $5$ la ecuación (\ref{eq:elc2}), obtenemos
				\begin{equation}\label{eq:elc3}
					5 \equiv  (5s) \cdot 74 \pmod{109}.
				\end{equation}
				Eso  implica que $5s$ es solución de  $74x\equiv 5 \pmod{109}$.
			\end{itemize}
			
			
		\end{proof}
	\end{frame}
	
	
	\begin{frame}
		Obtengamos entonces la ecuación (\ref{eq:elc1}).\pause
		\begin{alignat}3
			109 &= 74 \cdot 1 + 35& \qquad&\Rightarrow& \qquad 35 &= 109 - 74 \label{eq:cle1}\\ 
			74 &=  35\cdot 2 +4 & \qquad&\Rightarrow& \qquad 4 &= 74 - 35\cdot 2 \label{eq:cle2}\qquad\\
			35 &= 4 \cdot 8 + 3& \qquad&\Rightarrow& \qquad  3 &= 35 - 4 \cdot 8 \label{eq:cle3}  \\
			4 &=  3\cdot 1 + 1& \qquad&\Rightarrow& \qquad 1 &= 4 - 3\label{eq:cle4}\\
			3 &=  1\cdot 3  +0 & \qquad&& \qquad  &
		\end{alignat}
		\pause	Luego,\pause
		\begin{align*}
			1 & \overset{(\ref{eq:cle4})}{=} 4 -3 \overset{(\ref{eq:cle3})}{=} 4 - ( 35 - 4 \cdot 8) \\
			&=  9 \cdot 4 - 35 \\
			& \overset{(\ref{eq:cle2})}{=} 9 \cdot (74 - 35\cdot 2) - 35  = 9 \cdot  74 + (-18) \cdot 35 - 35 \\
			&= 9 \cdot  74 + (-19) \cdot 35 \\
			& \overset{(\ref{eq:cle1})}{=}  9 \cdot  74 + (-19) \cdot (109 - 74) \\ %= 9 \cdot  74 + (-19) \cdot 109 + 19 \cdot 74 \\
			&= 28 \cdot  74 + (-19) \cdot 109 .
		\end{align*}
	\end{frame}
	
	
	\begin{frame}
		Recordemos que debemos resolver la ecuación
		\begin{equation*}
			74x\equiv 5 \pmod{109}.
		\end{equation*}
		\pause Como 
		\begin{equation*}
			1 = 28 \cdot  74 + (-19) \cdot 109,
		\end{equation*}
		\pause tenemos
		\begin{equation*}
			1  \equiv 28 \cdot  74\pmod{109}.
		\end{equation*}
		\pause Por lo tanto, 
		\begin{equation*}
			5 \equiv (5 \cdot28)  74\pmod{109}.
		\end{equation*}
		\vskip .3cm
		\pause Es decir, $140$ es solución y $31 = 140 -109$ también lo es, pues \linebreak $109 \equiv 0\pmod{109}$. 
		\vskip .3cm
		Luego 
		$$5 \equiv 31 \cdot  74\pmod{109} \qquad \text{y} \qquad0 \le 31 < 109.\qed $$
	\end{frame}
	
	
	\begin{frame}
		Analicemos ahora la situación general de la ecuación $ ax\equiv b\pmod{m}$. 
		\vskip .3cm 
		\begin{teorema}\label{elc} Sean $a,b$ números enteros y $m$ un entero positivo y denotemos $d =\operatorname{mcd}(a,m)$. La  ecuación 
			\begin{equation}\label{elc}
				ax \equiv b \pmod{m}
			\end{equation}
			admite solución si y sólo si $d|b$, y en este caso dada $x_0$ una solución, todas las soluciones son de la forma 
			$$x = x_0 + k n,\qquad \mbox{ con } k \in \mathbb Z \mbox{ y } n = {\frac{m}{d}} $$
		\end{teorema}
	\end{frame}
	
	
	\begin{frame}
		\begin{proof}
			($\Leftarrow$) Muy  similar al ejemplo anterior. \pause Como 
			\begin{equation*}
				d = s \cdot  a +t \cdot m,\qquad \text{para algunos $s,t \in \mathbb Z$,}
			\end{equation*}
			\pause tenemos
			\begin{equation*}
				d  \equiv s \cdot  a\pmod{m}.
			\end{equation*}
			\pause Como  $d|b$, tenemos $b = d q$. Por lo tanto, 
			\begin{equation*}
				b = d q \equiv qs \cdot  a\pmod{m}.
			\end{equation*}
			Es decir 
			\begin{equation*}
				a(qs)  \equiv b\pmod{m}.
			\end{equation*}
			\vskip .3cm
			\pause Es decir, $x_0 = qs$ es solución de $ax \equiv b \pmod{m}$. 
		\end{proof}
	\end{frame}
	
	
	\begin{frame}
		Veamos ahora que si $n = {\frac{m}{d}}$ y $k \in \mathbb Z$, entonces $x_0 + kn$ también es solución.
		\vskip .3cm
		Es decir
		\begin{equation*}
			a(x_0 + kn) \equiv b \pmod{m}.
		\end{equation*} \pause
		
		Como $d|a$ $\Rightarrow$ $a = dr$, luego $akn = dr\frac{m}{d} = rm \equiv 0 \pmod{m}$. 
		\vskip .3cm\pause
		Luego
		$$
		a(x_0 + kn) \equiv 	ax_0 + akn \equiv	ax_0 \equiv b \pmod{m}.
		$$
		\vskip .6cm\pause
		($\Rightarrow$) Ver apunte. 
		
	\end{frame}
	
	
	\begin{frame}
		\begin{proposicion}
			Sean $a$, $b$, $m \in {\mathbb Z}$, $d>0$ tales que  \,$d\mid a$,\,  \,$d\mid b$\, y \,$d\mid m$. Entonces, 
			\begin{equation*}
				ax \equiv b\,({m}) \quad \Leftrightarrow\quad\frac{a}{d} x \equiv \frac{b}{d}\,\left(\frac{m}{d}\right).
			\end{equation*}
			\vskip -.4cm
		\end{proposicion} \pause
		\begin{proof} \pause
			\vskip -.8cm
			\begin{align*}
				a x \equiv b\,({m}) &\Leftrightarrow m| ax -b \\&\Leftrightarrow  a x -b = m \cdot q   \\
				&\Leftrightarrow \frac{a}{d} x -\frac{b}{d} =\frac{m}{d}\cdot q   \\
				&\Leftrightarrow \frac{a}{d} x \equiv \frac{b}{d}\,\left(\frac{m}{d}\right)\qquad  
			\end{align*}
			\qed
		\end{proof}
	\end{frame}
	
	
	\begin{frame}
		La proposición anterior nos permite reducir las ecuaciones al caso $	ax \equiv b\,({m})$ con $(a,m)=1$. \pause
		
		\vskip .4cm
		
		\begin{ejemplo}
			Encontrar todos los $x \in \mathbb Z$ tales que
			\begin{equation}
				6x \equiv 30 \pmod{42}. \tag{*}
			\end{equation}
		\end{ejemplo} \pause \vskip -.4cm
		\begin{solucion} \pause
			Como $6 =(6,42)$ y $6|30$, la ecuación (*) tiene solución y tenemos que 
			\begin{equation*}
				6x \equiv 30 \pmod{42} \quad \Leftrightarrow \quad 	x \equiv 5 \pmod{7}.
			\end{equation*}
			Una solución obvia es $x_0 =5$, luego todas las soluciones son $x= 5 + 7k$, $k \in \mathbb Z$.\qed
		\end{solucion}
	\end{frame}
	
	
\end{document}

