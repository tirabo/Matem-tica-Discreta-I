% PDFLaTeX
\documentclass[a4paper,12pt,twoside,spanish]{amsbook}
%\documentclass[a4paper,11pt,twoside]{book}
%\documentclass[a4paper,11pt,twoside,spanish]{amsbook}

%%%---------------------------------------------------


\usepackage{etex}
\tolerance=10000
\renewcommand{\baselinestretch}{1.3}

\renewcommand{\familydefault}{\sfdefault} % la font por default es sans serif

% Para hacer el  indice en linea de comando hacer 
% makeindex main
%% En http://www.tug.org/pracjourn/2006-1/hartke/hartke.pdf hay ejemplos de packages de fonts libres, como los siguientes:
%\usepackage{cmbright}
%\usepackage{pxfonts}
%\usepackage[varg]{txfonts}
%\usepackage{ccfonts}
%\usepackage[math]{iwona}
%\usepackage[math]{kurier}


\usepackage{t1enc}
%\usepackage[spanish]{babel}
\usepackage{latexsym}
\usepackage[utf8]{inputenc}
\usepackage{verbatim}
\usepackage{multicol}
\usepackage{amsgen,amsmath,amstext,amsbsy,amsopn,amsfonts,amssymb}
\usepackage{amsthm}
\usepackage{calc}         % From LaTeX distribution
\usepackage{graphicx}     % From LaTeX distribution
\usepackage{ifthen}
\input{random.tex}        % From CTAN/macros/generic
\usepackage{subfigure} 
\usepackage{tikz}
\usetikzlibrary{arrows}
\usetikzlibrary{matrix}
%\usetikzlibrary{graphs}
%\usepackage{tikz-3dplot} %for tikz-3dplot functionality
%\usepackage{pgfplots}
\usepackage{mathtools}
\usepackage{stackrel}
\usepackage{enumerate}
\usepackage{tkz-graph}
%\usepackage{makeidx}
\makeindex

%%%----------------------------------------------------------------------------
\usepackage[a4paper, top=3cm, left=3cm, right=2cm, bottom=2.5cm]{geometry}
%% CONTROLADORES DE.
% Tamaño de la hoja de impresión.
% Tamaños de los laterales del documento. 
%%%%%%%%%%%%%%%%%%%%%%%%%%%%%%%%%%%%%%%%%%%%%%%%%%%%%%%%%%%%%%%%%%%%%%%%%%%%%%%%%
%%% \theoremstyle{plain} %% This is the default
%\oddsidemargin 0.0in \evensidemargin -1.0cm \topmargin 0in
%\headheight .3in \headsep .2in \footskip .2in
%\setlength{\textwidth}{16cm} %ancho para apunte
%\setlength{\textheight}{21cm} %largo para apunte
%%%%\leftmargin 2.5cm
%%%%\rightmargin 2.5cm
%\topmargin 0.5 cm
%%%%%%%%%%%%%%%%%%%%%%%%%%%%%%%%%%%%%%%%%%%%%%%%%%%%%%%%%%%%%%%%%%%%%%%%%%%%%%%%%%%

\usepackage{hyperref}
\hypersetup{
	colorlinks=true,
	linkcolor=blue,
	filecolor=magenta,      
	urlcolor=cyan,
}
\usepackage{hypcap}


\renewcommand{\thesection}{\thechapter.\arabic{section}}
\renewcommand{\thesubsection}{\thesection.\arabic{subsection}}

\newtheorem{teorema}{Teorema}[section]
\newtheorem{proposicion}[teorema]{Proposici\'on}
\newtheorem{corolario}[teorema]{Corolario}
\newtheorem{lema}[teorema]{Lema}
\newtheorem{propiedad}[teorema]{Propiedad}

\theoremstyle{definition}

\newtheorem{definicion}{Definici\'on}[section]
\newtheorem{ejemplo}{Ejemplo}[section]
\newtheorem{problema}{Problema}[section]
\newtheorem{ejercicio}{Ejercicio}[section]
\newtheorem{ejerciciof}{}[section]

\theoremstyle{remark}
\newtheorem{observacion}{Observaci\'on}[section]
\newtheorem{nota}{Nota}[section]

\renewcommand{\partname }{Parte }
\renewcommand{\indexname}{Indice }
\renewcommand{\figurename }{Figura }
\renewcommand{\tablename }{Tabla }
\renewcommand{\proofname}{Demostraci\'on}
\renewcommand{\appendixname }{}
\renewcommand{\contentsname }{Contenidos }
\renewcommand{\chaptername }{}
\renewcommand{\bibname }{Bibliograf\'\i a }



\newcommand{\tarea}[1]{
	\begin{center}
		{\Large Matemática Discreta I - 2020/1} \vskip.4cm
		{\Large Tarea #1}\vskip .4cm
\end{center}}

\renewenvironment{ejercicio}% environment name
{% begin code
	\par\vskip .5cm%
	{\noindent\color{blue}Ejercicio}%
	\vskip .2cm
}%
{%
	\vskip .2cm}% end code


\newenvironment{solucion}% environment name
{% begin code
	\par\vskip .2cm%
	{\noindent\color{blue}Solución}%
	\vskip .2cm
}%
{%
	\vskip .2cm}% end code


\begin{document}

%\frame{\titlepage} 


\tarea{6}

\begin{ejercicio}
	\begin{enumerate}
		\item[1.](30 pts) Probar usando  la definición de ''divide a'' que si $n | m$,  entonces $n | m^3$.
		%\item[2.] (30 pts) Probar que para $n \ge 0$ se satisface $3 | 2^{3n+2} + 5^{n+3}$.  
		\item[2.] 
		\begin{enumerate}
			\item[a)] (40 pts) Encontrar usando el algoritmo de Euclides $d = \operatorname{mcd}( 162, 138)$. 
			\item[b)] (30 pts) Expresar $d$  como combinación lineal entera entre  $162$ y $138$.
		\end{enumerate}	 
	\end{enumerate}
\end{ejercicio}
	
\begin{comment}
\begin{solucion}
	1. Como  $n | m$,  entonces existe $q$, número entero, tal que $m = n \cdot q$.
	
	Ahora bien, 
	$$
	m^3 = (n \cdot q)^3 = n^3 \cdot q^3 = n ( n^2 q^3).
	$$
	Por  lo tanto, $n|m^3$.
	
\vskip .4cm

	2 (a).  Primero hallamos el mcd entre  162 y 138.
	\begin{align}	
		162 &= 138  \cdot 1 + 24\\		
		138 &= 24  \cdot 5+ 18\\		
		24 &= 18  \cdot 1+ 6\\		
		18 &= 6  \cdot 3 + 0 
	\end{align}
	Como el último resto no nulo es  6, tenemos que \colorbox{lightgray}{$\operatorname{mcd}(162, 138) =6$}.
	
		\vskip .4cm
	2 (b). Ahora escribamos 6 como combinación lineal entera de   162 y 138.
		\begin{alignat*}2
		6 &= 24 -18& \qquad\qquad&(\text{por (3)}) \\
		&=  24 -(138 -24  \cdot 5 ) & \qquad\qquad&(\text{por (2)}) \\
		&=  24 +(-1) \cdot 138 +5 \cdot 24& & \\
		&= (-1) \cdot 138 +6 \cdot 24& & \\
		&= (-1) \cdot 138 +6 \cdot (162-138)& \qquad\qquad&(\text{por (1)}) \\
		&= (-1) \cdot 138 +6 \cdot 162 + (-6)\cdot 138& & \\
		&= (-7) \cdot 138 +6 \cdot 162& \qquad\qquad&
	\end{alignat*}
	
	Es decir \colorbox{lightgray}{$6 = 6 \cdot 162+ (-7) \cdot 138$}.
\end{solucion}

\end{comment}
\end{document}




	2. Lo haremos por inducción.
\vskip .3cm 
Caso base $n=0$. 

En  este caso  $2^{3n+2} + 5^{n+3} =  2^{2} + 5^{3} = 4 +125  = 129 = 3 \cdot 43$,  es decir $3| 2^{2} + 5^{3}$.

\vskip .3cm 

Paso  inductivo. Supongamos que $3 | 2^{3k+2} + 5^{k+3}$ para un $k \ge 0$ (HI),  debemos probar que 
\begin{equation}\label{eq-q}
3 | 2^{3(k+1)+2} + 5^{(k+1)+3}. \tag{*}
\end{equation}
Ahora bien, 
\begin{align*}
2^{3(k+1)+2} + 5^{(k+1)+3} &= 2^{3k+3+2} + 5^{k+1+3}\\
&= 2^{3k+2}\cdot 2^{3} + 5^{k+3}\cdot 5^{1} \\
&= 8 \cdot 2^{3k+2} + 5\cdot 5^{k+3}\\
&= 5( 2^{3k+2} + 5^{k+3})  +3 \cdot 2^{3k+2}.
\end{align*}
Es decir
\begin{equation}
2^{3(k+1)+2} + 5^{(k+1)+3} = 5( 2^{3k+2} + 5^{k+3})  +3 \cdot 2^{3k+2}. \tag{**}
\end{equation}
Por hipótesis inductiva (HI), $3 | 2^{3k+2} + 5^{k+3}$ y por lo tanto $ 3|5( 2^{3k+2} + 5^{k+3})$. Por otro lado es claro  que $3 | 3 \cdot 2^{3k+2}$. Entonces,  3 divide a la suma de  $5( 2^{3k+2} + 5^{k+3})$ y $3 \cdot 2^{3k+2}$:
\begin{equation*}
3|5( 2^{3k+2} + 5^{k+3}) + 3 \cdot 2^{3k+2} \overset{(**)}{=}2^{3(k+1)+2} + 5^{(k+1)+3},
\end{equation*}
y esto  prueba (*), y por lo tanto prueba el ejercicio.
\vskip .4cm 