% PDFLaTeX
\documentclass[a4paper,12pt,twoside,spanish]{amsbook}
%\documentclass[a4paper,11pt,twoside]{book}
%\documentclass[a4paper,11pt,twoside,spanish]{amsbook}

%%%---------------------------------------------------


\usepackage{etex}
\tolerance=10000
\renewcommand{\baselinestretch}{1.3}

\renewcommand{\familydefault}{\sfdefault} % la font por default es sans serif

% Para hacer el  indice en linea de comando hacer
% makeindex main
%% En http://www.tug.org/pracjourn/2006-1/hartke/hartke.pdf hay ejemplos de packages de fonts libres, como los siguientes:
%\usepackage{cmbright}
%\usepackage{pxfonts}
%\usepackage[varg]{txfonts}
%\usepackage{ccfonts}
%\usepackage[math]{iwona}
%\usepackage[math]{kurier}


\usepackage{t1enc}
%\usepackage[spanish]{babel}
\usepackage{latexsym}
\usepackage[utf8]{inputenc}
\usepackage{verbatim}
\usepackage{multicol}
\usepackage{amsgen,amsmath,amstext,amsbsy,amsopn,amsfonts,amssymb}
\usepackage{amsthm}
\usepackage{calc}         % From LaTeX distribution
\usepackage{graphicx}     % From LaTeX distribution
\usepackage{ifthen}
\input{random.tex}        % From CTAN/macros/generic
\usepackage{subfigure}
\usepackage{tikz}
\usetikzlibrary{arrows}
\usetikzlibrary{matrix}
%\usetikzlibrary{graphs}
%\usepackage{tikz-3dplot} %for tikz-3dplot functionality
%\usepackage{pgfplots}
\usepackage{mathtools}
\usepackage{stackrel}
\usepackage{enumerate}
\usepackage{tkz-graph}
%\usepackage{makeidx}
\makeindex

%%%----------------------------------------------------------------------------
\usepackage[a4paper, top=3cm, left=3cm, right=2cm, bottom=2.5cm]{geometry}
%% CONTROLADORES DE.
% Tamaño de la hoja de impresión.
% Tamaños de los laterales del documento.
%%%%%%%%%%%%%%%%%%%%%%%%%%%%%%%%%%%%%%%%%%%%%%%%%%%%%%%%%%%%%%%%%%%%%%%%%%%%%%%%%
%%% \theoremstyle{plain} %% This is the default
%\oddsidemargin 0.0in \evensidemargin -1.0cm \topmargin 0in
%\headheight .3in \headsep .2in \footskip .2in
%\setlength{\textwidth}{16cm} %ancho para apunte
%\setlength{\textheight}{21cm} %largo para apunte
%%%%\leftmargin 2.5cm
%%%%\rightmargin 2.5cm
%\topmargin 0.5 cm
%%%%%%%%%%%%%%%%%%%%%%%%%%%%%%%%%%%%%%%%%%%%%%%%%%%%%%%%%%%%%%%%%%%%%%%%%%%%%%%%%%%

\usepackage{hyperref}
\hypersetup{
	colorlinks=true,
	linkcolor=blue,
	filecolor=magenta,
	urlcolor=cyan,
}
\usepackage{hypcap}


\renewcommand{\thesection}{\thechapter.\arabic{section}}
\renewcommand{\thesubsection}{\thesection.\arabic{subsection}}

\newtheorem{teorema}{Teorema}[section]
\newtheorem{proposicion}[teorema]{Proposici\'on}
\newtheorem{corolario}[teorema]{Corolario}
\newtheorem{lema}[teorema]{Lema}
\newtheorem{propiedad}[teorema]{Propiedad}

\theoremstyle{definition}

\newtheorem{definicion}{Definici\'on}[section]
\newtheorem{ejemplo}{Ejemplo}[section]
\newtheorem{problema}{Problema}[section]
\newtheorem{ejercicio}{Ejercicio}[section]
\newtheorem{ejerciciof}{}[section]

\theoremstyle{remark}
\newtheorem{observacion}{Observaci\'on}[section]
\newtheorem{nota}{Nota}[section]

\renewcommand{\partname }{Parte }
\renewcommand{\indexname}{Indice }
\renewcommand{\figurename }{Figura }
\renewcommand{\tablename }{Tabla }
\renewcommand{\proofname}{Demostraci\'on}
\renewcommand{\appendixname }{}
\renewcommand{\contentsname }{Contenidos }
\renewcommand{\chaptername }{}
\renewcommand{\bibname }{Bibliograf\'\i a }



\newcommand{\tarea}[1]{
	\begin{center}
		{\Large Matemática Discreta I - 2021/1} \vskip.4cm
		{\Large Tarea #1}\vskip .4cm
\end{center}}

\renewenvironment{ejercicio}% environment name
{% begin code
	\par\vskip .5cm%
	{\noindent\color{blue}Ejercicios}%
	\vskip .2cm
}%
{%
	\vskip .2cm}% end code


\newenvironment{solucion}% environment name
{% begin code
	\par\vskip .2cm%
	{\noindent\color{blue}Solución}%
	\vskip .2cm
}%
{%
	\vskip .2cm}% end code


\begin{document}

%\frame{\titlepage}


\tarea{5}

\begin{ejercicio}
	\begin{enumerate}
		\item[1.] (25 pts) Convertir a base $2$ el número $345$. 
		\item[2.] (25 pts) Convertir a base $10$ el número $(203112)_4$. 
		\item[3.] (50 pts) Calcular  la resta $(4351)_8 -(2310)_4$ y expresarla en base $5$. 
	\end{enumerate}
\end{ejercicio}

\begin{comment}
\begin{solucion}
	\noindent\text{{1.}} Primero calculamos en base 10 el número  $(20332)_5$:
	\begin{align*}
	(20332)_5 &= 2 \cdot 5^4+0 \cdot 5^3+3\cdot 5^2+ 3 \cdot 5^1+ 2 \cdot 5^0\\
	&=  2 \cdot 625+0 \cdot 125+3\cdot 25+ 3 \cdot 5+ 2\\
	& = 1250 + 0 + 75  + 15 +2  \\
	&= 1342.
	\end{align*}
	Ahora lo convertimos a  base 8:
	\begin{equation*}
	\begin{matrix*}[r]
	1342 &=& 8 \cdot 167& + &6\\
	167 &=&  8 \cdot 20& + &7\\
	20 &=&  8 \cdot 2& + &4\\
	2 &=&  8 \cdot 0& + &2
	\end{matrix*}
	\end{equation*}
	Luego, la respuesta es $(20332)_5 = (2476)_8$.
	
	\vskip .4cm	
	\noindent{2.} Calculamos primero en base 10 los números $(4321)_5$ y $(302)_5$.
	\vskip .2cm
	$(4321)_5 = 4 \cdot 5^3+3\cdot 5^2+ 2\cdot 5^1+ 1 \cdot 5^0 =  4 \cdot 125+3\cdot 25+ 2\cdot 5+ 1 = 586$.
	\vskip .2cm
	$(302)_5 =3\cdot 5^2+ 0\cdot 5^1+ 2 \cdot 5^0 =  3\cdot 25+ 0\cdot 5+ 2 = 77$.
	\vskip .2cm
	Ahora restamos $586 -77= 509$. Al resultado  obtenido lo convertimos a base 5:
	\begin{equation*}
	\begin{matrix*}[r]
	509 &=& 5 \cdot 101& + &4\\
	101 &=& 5 \cdot 20& + &1\\
	20 &=&  5 \cdot 4& + &0\\
	4 &=&  5 \cdot 0& + &4.
	\end{matrix*}
	\end{equation*}
	Luego $509 = (4014)_5$ y por lo tanto,
	\begin{equation*}
	(4321)_5 - (302)_5 = (4014)_5.
	\end{equation*}
\end{solucion}
\end{comment}
\end{document}

