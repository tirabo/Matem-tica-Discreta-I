% PDFLaTeX
\documentclass[a4paper,12pt,twoside,spanish]{amsbook}
%\documentclass[a4paper,11pt,twoside]{book}
%\documentclass[a4paper,11pt,twoside,spanish]{amsbook}

%%%---------------------------------------------------


\usepackage{etex}
\tolerance=10000
\renewcommand{\baselinestretch}{1.3}

\renewcommand{\familydefault}{\sfdefault} % la font por default es sans serif

% Para hacer el  indice en linea de comando hacer 
% makeindex main
%% En http://www.tug.org/pracjourn/2006-1/hartke/hartke.pdf hay ejemplos de packages de fonts libres, como los siguientes:
%\usepackage{cmbright}
%\usepackage{pxfonts}
%\usepackage[varg]{txfonts}
%\usepackage{ccfonts}
%\usepackage[math]{iwona}
%\usepackage[math]{kurier}


\usepackage{t1enc}
%\usepackage[spanish]{babel}
\usepackage{latexsym}
\usepackage[utf8]{inputenc}
\usepackage{verbatim}
\usepackage{multicol}
\usepackage{amsgen,amsmath,amstext,amsbsy,amsopn,amsfonts,amssymb}
\usepackage{amsthm}
\usepackage{calc}         % From LaTeX distribution
\usepackage{graphicx}     % From LaTeX distribution
\usepackage{ifthen}
\input{random.tex}        % From CTAN/macros/generic
\usepackage{subfigure} 
\usepackage{tikz}
\usetikzlibrary{arrows}
\usetikzlibrary{matrix}
%\usetikzlibrary{graphs}
%\usepackage{tikz-3dplot} %for tikz-3dplot functionality
%\usepackage{pgfplots}
\usepackage{mathtools}
\usepackage{stackrel}
\usepackage{enumerate}
\usepackage{tkz-graph}
%\usepackage{makeidx}
\makeindex

%%%----------------------------------------------------------------------------
\usepackage[a4paper, top=3cm, left=3cm, right=2cm, bottom=2.5cm]{geometry}
%% CONTROLADORES DE.
% Tamaño de la hoja de impresión.
% Tamaños de los laterales del documento. 
%%%%%%%%%%%%%%%%%%%%%%%%%%%%%%%%%%%%%%%%%%%%%%%%%%%%%%%%%%%%%%%%%%%%%%%%%%%%%%%%%
%%% \theoremstyle{plain} %% This is the default
%\oddsidemargin 0.0in \evensidemargin -1.0cm \topmargin 0in
%\headheight .3in \headsep .2in \footskip .2in
%\setlength{\textwidth}{16cm} %ancho para apunte
%\setlength{\textheight}{21cm} %largo para apunte
%%%%\leftmargin 2.5cm
%%%%\rightmargin 2.5cm
%\topmargin 0.5 cm
%%%%%%%%%%%%%%%%%%%%%%%%%%%%%%%%%%%%%%%%%%%%%%%%%%%%%%%%%%%%%%%%%%%%%%%%%%%%%%%%%%%

\usepackage{hyperref}
\hypersetup{
	colorlinks=true,
	linkcolor=blue,
	filecolor=magenta,      
	urlcolor=cyan,
}
\usepackage{hypcap}


\renewcommand{\thesection}{\thechapter.\arabic{section}}
\renewcommand{\thesubsection}{\thesection.\arabic{subsection}}

\newtheorem{teorema}{Teorema}[section]
\newtheorem{proposicion}[teorema]{Proposici\'on}
\newtheorem{corolario}[teorema]{Corolario}
\newtheorem{lema}[teorema]{Lema}
\newtheorem{propiedad}[teorema]{Propiedad}

\theoremstyle{definition}

\newtheorem{definicion}{Definici\'on}[section]
\newtheorem{ejemplo}{Ejemplo}[section]
\newtheorem{problema}{Problema}[section]
\newtheorem{ejercicio}{Ejercicio}[section]
\newtheorem{ejerciciof}{}[section]

\theoremstyle{remark}
\newtheorem{observacion}{Observaci\'on}[section]
\newtheorem{nota}{Nota}[section]

\renewcommand{\partname }{Parte }
\renewcommand{\indexname}{Indice }
\renewcommand{\figurename }{Figura }
\renewcommand{\tablename }{Tabla }
\renewcommand{\proofname}{Demostraci\'on}
\renewcommand{\appendixname }{}
\renewcommand{\contentsname }{Contenidos }
\renewcommand{\chaptername }{}
\renewcommand{\bibname }{Bibliograf\'\i a }


\newcommand{\tarea}[1]{
	\begin{center}
		{\Large Matemática Discreta I - 2021/1} \vskip.4cm
		{\Large Tarea #1}\vskip .4cm
\end{center}}

\renewenvironment{ejercicio}[1][]% environment name
{% begin code
	\par\vskip .2cm%
	{\noindent\color{blue}Ejercicio #1.}%
	\vskip .2cm
}%
{%
	\vskip .2cm}% end code


\newenvironment{solucion}% environment name
{% begin code
	\par\vskip .2cm%
	{\noindent\color{blue}Solución}%
	\vskip .2cm
}%
{%
	\vskip .2cm}% end code







\begin{document}

%\frame{\titlepage}


\tarea{10}

\begin{ejercicio}[1] (40 pts) Determinar si el grafo $G=(V, E)$ tiene caminatas o circuitos eulerianos y  en caso de que la respuesta sea positiva encontrar una caminata o circuito euleriano. 

\begin{align*}
    V &= \{a, b, c, d, e, f, g, x, y, z, v, w\},\\
    E &= \{\{a, d\},
    \{b, d\},
    \{c, d\}, 
    \{d, e\},
    \{e, y\}, 
    \{f, w\},
    \{f, v\},
    \{g, x\}, 
    \\
    &\qquad\qquad\qquad\qquad\qquad\qquad\qquad\qquad
    \{g, y\},
    \{v, w\}, 
    \{v, y\}, 
    \{v, z\}, 
    \{y, z\}\}.
\end{align*}
\end{ejercicio}

\begin{ejercicio}[2]
    Dados los siguientes grafos:
\vskip .2cm

\begin{center}
    (1)\begin{tikzpicture}[scale=0.7]
        %\SetVertexSimple[Shape=circle,FillColor=white]
        \def\rvar{1.2}
        \Vertex[x=0.00, y=-0.00, L=$c$]{$u$} % 3
        \Vertex[x=\rvar*1.90, y=-0.62, L=$e$]{$t$} % 2
        \Vertex[x=\rvar*1.18, y=1.62, L=$b$]{$q$} % 1
        \Vertex[x=-1.18*\rvar, y=1.62, L=$a$]{$p$} % 0
        \Vertex[x=-1.90*\rvar, y=-0.62, L=$d$]{$r$} % 4
        \Vertex[x=0.00, y=-2.00, L=$f$]{$s$} % 5
        \Edges($u$,$t$,$q$,$p$)
        \Edges($r$,$u$)
        \Edges($s$,$t$)
        \Edges($r$,$s$,$q$,$r$)
        \Edges($p$,$t$,$s$)
        \Edges($s$,$p$,$r$)
    \end{tikzpicture}\quad\quad
    (2)\begin{tikzpicture}[scale=0.7]
        %\SetVertexSimple[Shape=circle,FillColor=white]
        \Vertex[x=-1.2, y=2, L=$p$]{$p$} % 0
        \Vertex[x= 1.2, y=2, L=$q$]{$q$} % 1
        \Vertex[x=-2, y=-0.0, L=$r$]{$r$} % 2
        \Vertex[x=-1.2, y=-2, L=$s$]{$s$} % 3
        \Vertex[x=2, y=-0.0, L=$t$]{$t$} % 4
        \Vertex[x= 1.2, y=-2, L=$u$]{$u$} % 5
        \Edges($s$,$r$,$p$,$q$,$r$,$u$,$t$,$p$,$s$,$q$)
    \end{tikzpicture}\quad\quad
    (3)\begin{tikzpicture}[scale=0.7]
        %\SetVertexSimple[Shape=circle,FillColor=white]
        \def\rvar{1.2}
        \Vertex[x=-1.18*\rvar, y=1.62, L=$4$]{$p$} % 0
        \Vertex[x=\rvar*1.18, y=1.62, L=$3$]{$q$} % 1
        \Vertex[x=-1.90*\rvar, y=-0.62, L=$5$]{$r$} % 2
        \Vertex[x=0, y=0, L=$6$]{$s$} % 3
        \Vertex[x=\rvar*1.90, y=-0.62, L=$2$]{$t$} % 4
        \Vertex[x=0.00, y=-2.00, L=$1$]{$u$} % 5
        
        
        
        \Edges($s$,$r$,$p$,$q$,$r$,$u$,$t$,$p$,$s$,$q$)
    \end{tikzpicture}
\end{center}
\vskip .3cm
\begin{enumerate}
        \item[(a)]  (20 pts) Dé un {ciclo hamiltoniano} en el grafo (1).
        \item[(b)]  (40 pts) Determinar cuales de los siguientes pares de grafos son isomorfos. En el caso de ser isomorfos, especifique un isomorfismo;
        en caso contrario, justificar por que no son isomorfos.

        (i) (1) y (2).\quad % no son isomorfos,(2) tiene una valencia 3, (1) no 

        (ii) (2) y (3). % Son isomorfos: p->4, q->3, r->5, s->6, t->2, u->1  
        % 
	\end{enumerate}
\end{ejercicio}
\vskip .2cm
\begin{comment}

\noindent{\color{blue}Solución.}




\end{comment}
\end{document}

