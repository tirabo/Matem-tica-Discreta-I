%\documentclass{beamer} 
\documentclass[handout]{beamer} % sin pausas
\usetheme{CambridgeUS}
%\setbeamertemplate{background}[grid][step=8 ] % cuadriculado


\usepackage{etex}
\usepackage{t1enc}
\usepackage[spanish,es-nodecimaldot]{babel}
\usepackage{latexsym}
\usepackage[utf8]{inputenc}
\usepackage{verbatim}
\usepackage{multicol}
\usepackage{amsgen,amsmath,amstext,amsbsy,amsopn,amsfonts,amssymb}
\usepackage{amsthm}
\usepackage{calc}         % From LaTeX distribution
\usepackage{graphicx}     % From LaTeX distribution
\usepackage{ifthen}
%\usepackage{makeidx}
\input{random.tex}        % From CTAN/macros/generic
\usepackage{subfigure} 
\usepackage{tikz}
\usepackage[customcolors]{hf-tikz}
\usetikzlibrary{arrows}
\usetikzlibrary{matrix}
\tikzset{
    every picture/.append style={
        execute at begin picture={\deactivatequoting},
        execute at end picture={\activatequoting}
    }
}
\usetikzlibrary{decorations.pathreplacing,angles,quotes}
\usetikzlibrary{shapes.geometric}
\usepackage{mathtools}
\usepackage{stackrel}
%\usepackage{enumerate}
\usepackage{enumitem}
\usepackage{tkz-graph}
\usepackage{polynom}
\polyset{%
    style=B,
    delims={(}{)},
    div=:
}
\renewcommand\labelitemi{$\circ$}
\setlist[enumerate]{label={(\arabic*)}}
%\setbeamertemplate{background}[grid][step=8 ] % cuadriculado
\setbeamertemplate{itemize item}{$\circ$}
\setbeamertemplate{enumerate items}[default]
\definecolor{links}{HTML}{2A1B81}
\hypersetup{colorlinks,linkcolor=,urlcolor=links}


\newcommand{\Id}{\operatorname{Id}}
\newcommand{\img}{\operatorname{Im}}
\newcommand{\nuc}{\operatorname{Nu}}
\newcommand{\im}{\operatorname{Im}}
\renewcommand\nu{\operatorname{Nu}}
\newcommand{\la}{\langle}
\newcommand{\ra}{\rangle}
\renewcommand{\t}{{\operatorname{t}}}
\renewcommand{\sin}{{\,\operatorname{sen}}}
\newcommand{\Q}{\mathbb Q}
\newcommand{\R}{\mathbb R}
\newcommand{\C}{\mathbb C}
\newcommand{\K}{\mathbb K}
\newcommand{\F}{\mathbb F}
\newcommand{\Z}{\mathbb Z}
\newcommand{\N}{\mathbb N}
\newcommand\sgn{\operatorname{sgn}}
\renewcommand{\t}{{\operatorname{t}}}
\renewcommand{\figurename }{Figura}

%
% Ver http://joshua.smcvt.edu/latex2e/_005cnewenvironment-_0026-_005crenewenvironment.html
%

\renewenvironment{block}[1]% environment name
{% begin code
	\par\vskip .2cm%
	{\color{blue}#1}%
	\vskip .2cm
}%
{%
	\vskip .2cm}% end code


\renewenvironment{alertblock}[1]% environment name
{% begin code
	\par\vskip .2cm%
	{\color{red!80!black}#1}%
	\vskip .2cm
}%
{%
	\vskip .2cm}% end code


\renewenvironment{exampleblock}[1]% environment name
{% begin code
	\par\vskip .2cm%
	{\color{blue}#1}%
	\vskip .2cm
}%
{%
	\vskip .2cm}% end code




\newenvironment{exercise}[1]% environment name
{% begin code
	\par\vspace{\baselineskip}\noindent
	\textbf{Ejercicio (#1)}\begin{itshape}%
		\par\vspace{\baselineskip}\noindent\ignorespaces
	}%
	{% end code
	\end{itshape}\ignorespacesafterend
}


\newenvironment{definicion}[1][]% environment name
{% begin code
	\par\vskip .2cm%
	{\color{blue}Definición #1}%
	\vskip .2cm
}%
{%
	\vskip .2cm}% end code

    \newenvironment{notacion}[1][]% environment name
    {% begin code
        \par\vskip .2cm%
        {\color{blue}Notación #1}%
        \vskip .2cm
    }%
    {%
        \vskip .2cm}% end code

\newenvironment{observacion}[1][]% environment name
{% begin code
	\par\vskip .2cm%
	{\color{blue}Observación #1}%
	\vskip .2cm
}%
{%
	\vskip .2cm}% end code

\newenvironment{ejemplo}[1][]% environment name
{% begin code
	\par\vskip .2cm%
	{\color{blue}Ejemplo #1}%
	\vskip .2cm
}%
{%
	\vskip .2cm}% end code


\newenvironment{preguntas}[1][]% environment name
{% begin code
    \par\vskip .2cm%
    {\color{blue}Preguntas #1}%
    \vskip .2cm
}%
{%
    \vskip .2cm}% end code

\newenvironment{ejercicio}[1][]% environment name
{% begin code
	\par\vskip .2cm%
	{\color{blue}Ejercicio #1}%
	\vskip .2cm
}%
{%
	\vskip .2cm}% end code


\renewenvironment{proof}% environment name
{% begin code
	\par\vskip .2cm%
	{\color{blue}Demostración}%
	\vskip .2cm
}%
{%
	\vskip .2cm}% end code



\newenvironment{demostracion}% environment name
{% begin code
	\par\vskip .2cm%
	{\color{blue}Demostración}%
	\vskip .2cm
}%
{%
	\vskip .2cm}% end code

\newenvironment{idea}% environment name
{% begin code
	\par\vskip .2cm%
	{\color{blue}Idea de la demostración}%
	\vskip .2cm
}%
{%
	\vskip .2cm}% end code

\newenvironment{solucion}% environment name
{% begin code
	\par\vskip .2cm%
	{\color{blue}Solución}%
	\vskip .2cm
}%
{%
	\vskip .2cm}% end code



\newenvironment{lema}[1][]% environment name
{% begin code
	\par\vskip .2cm%
	{\color{blue}Lema #1}\begin{itshape}%
		\par\vskip .2cm
	}%
	{% end code
	\end{itshape}\vskip .2cm\ignorespacesafterend
}

\newenvironment{proposicion}[1][]% environment name
{% begin code
	\par\vskip .2cm%
	{\color{blue}Proposición #1}\begin{itshape}%
		\par\vskip .2cm
	}%
	{% end code
	\end{itshape}\vskip .2cm\ignorespacesafterend
}

\newenvironment{teorema}[1][]% environment name
{% begin code
	\par\vskip .2cm%
	{\color{blue}Teorema #1}\begin{itshape}%
		\par\vskip .2cm
	}%
	{% end code
	\end{itshape}\vskip .2cm\ignorespacesafterend
}


\newenvironment{corolario}[1][]% environment name
{% begin code
	\par\vskip .2cm%
	{\color{blue}Corolario #1}\begin{itshape}%
		\par\vskip .2cm
	}%
	{% end code
	\end{itshape}\vskip .2cm\ignorespacesafterend
}

\newenvironment{propiedad}% environment name
{% begin code
	\par\vskip .2cm%
	{\color{blue}Propiedad}\begin{itshape}%
		\par\vskip .2cm
	}%
	{% end code
	\end{itshape}\vskip .2cm\ignorespacesafterend
}

\newenvironment{conclusion}% environment name
{% begin code
	\par\vskip .2cm%
	{\color{blue}Conclusión}\begin{itshape}%
		\par\vskip .2cm
	}%
	{% end code
	\end{itshape}\vskip .2cm\ignorespacesafterend
}


\newenvironment{definicion*}% environment name
{% begin code
	\par\vskip .2cm%
	{\color{blue}Definición}%
	\vskip .2cm
}%
{%
	\vskip .2cm}% end code

\newenvironment{observacion*}% environment name
{% begin code
	\par\vskip .2cm%
	{\color{blue}Observación}%
	\vskip .2cm
}%
{%
	\vskip .2cm}% end code


\newenvironment{obs*}% environment name
	{% begin code
		\par\vskip .2cm%
		{\color{blue}Observación}%
		\vskip .2cm
	}%
	{%
		\vskip .2cm}% end code

\newenvironment{ejemplo*}% environment name
{% begin code
	\par\vskip .2cm%
	{\color{blue}Ejemplo}%
	\vskip .2cm
}%
{%
	\vskip .2cm}% end code

\newenvironment{ejercicio*}% environment name
{% begin code
	\par\vskip .2cm%
	{\color{blue}Ejercicio}%
	\vskip .2cm
}%
{%
	\vskip .2cm}% end code

\newenvironment{propiedad*}% environment name
{% begin code
	\par\vskip .2cm%
	{\color{blue}Propiedad}\begin{itshape}%
		\par\vskip .2cm
	}%
	{% end code
	\end{itshape}\vskip .2cm\ignorespacesafterend
}

\newenvironment{conclusion*}% environment name
{% begin code
	\par\vskip .2cm%
	{\color{blue}Conclusión}\begin{itshape}%
		\par\vskip .2cm
	}%
	{% end code
	\end{itshape}\vskip .2cm\ignorespacesafterend
}






\newcommand{\nc}{\newcommand}

%%%%%%%%%%%%%%%%%%%%%%%%%LETRAS

\nc{\FF}{{\mathbb F}} \nc{\NN}{{\mathbb N}} \nc{\QQ}{{\mathbb Q}}
\nc{\PP}{{\mathbb P}} \nc{\DD}{{\mathbb D}} \nc{\Sn}{{\mathbb S}}
\nc{\uno}{\mathbb{1}} \nc{\BB}{{\mathbb B}} \nc{\An}{{\mathbb A}}

\nc{\ba}{\mathbf{a}} \nc{\bb}{\mathbf{b}} \nc{\bt}{\mathbf{t}}
\nc{\bB}{\mathbf{B}}

\nc{\cP}{\mathcal{P}} \nc{\cU}{\mathcal{U}} \nc{\cX}{\mathcal{X}}
\nc{\cE}{\mathcal{E}} \nc{\cS}{\mathcal{S}} \nc{\cA}{\mathcal{A}}
\nc{\cC}{\mathcal{C}} \nc{\cO}{\mathcal{O}} \nc{\cQ}{\mathcal{Q}}
\nc{\cB}{\mathcal{B}} \nc{\cJ}{\mathcal{J}} \nc{\cI}{\mathcal{I}}
\nc{\cM}{\mathcal{M}} \nc{\cK}{\mathcal{K}}

\nc{\fD}{\mathfrak{D}} \nc{\fI}{\mathfrak{I}} \nc{\fJ}{\mathfrak{J}}
\nc{\fS}{\mathfrak{S}} \nc{\gA}{\mathfrak{A}}
%%%%%%%%%%%%%%%%%%%%%%%%%LETRAS


\title[Clase 22 - Algoritmos greedy]{Matemática Discreta I \\ Clase 22 - Algoritmos greedy en grafos}
%\author[C. Olmos / A. Tiraboschi]{Carlos Olmos / Alejandro Tiraboschi}
\institute[]{\normalsize FAMAF / UNC
    \\[\baselineskip] ${}^{}$
    \\[\baselineskip]
}
\date[15/06/2023]{15 de junio de 2023}




\begin{document}
    
    \frame{\titlepage} 

    \begin{frame}
        \frametitle{Algoritmo greedy para coloración de vértices}
    
        \begin{itemize}
            \item No se conoce ningún algoritmo general para encontrar el número cromático de un grafo que trabaje en ``tiempo polinomial'' \pause
            \item Sin embargo hay un método simple de hacer una coloración usando un  ``razonable'' número de colores. \pause
        \end{itemize}

        \vskip .4cm


        El algoritmo es muy sencillo y se puede describir en una sola línea.  \pause
        \vskip .4cm
        \begin{itemize}
            \item Si hay vértices no coloreados, elegimos un vértice no coloreado  y le otorgamos un color que no tengan sus vecinos.
        \end{itemize} \pause

        \vskip .4cm
        En este algoritmo insistimos en hacer la mejor elección que podemos en cada paso, sin mirar más allá para ver si esta elección nos traerá problemas luego. 
        
        \vskip .4cm \pause
        
        Un algoritmo de esta clase se llama a menudo un \textit{algoritmo greedy (goloso)}.  \index{algoritmo greedy (goloso)}
    
    \end{frame}



    \begin{frame}
        \frametitle{}
    
        El algoritmo greedy para coloración de vértices es fácil de programar.  \pause
        \vskip .8cm
        Supóngase que hemos dado a los vértices algún orden $v_0,v_1,\ldots,v_n$.  \pause
        \vskip .8cm
        \begin{itemize}
            \item Asignemos el color $0$ a $v_0$. \pause
            \item Tomamos $v_i$ el siguiente vértice de la lista y
            \begin{itemize}
                \item $S =$ el conjunto de colores asignados a los vértices $v_j$ ($0\le j <i$) que son adyacentes a $v_i$.
                \item Le damos a $v_i$ el primer color que no está en $S$.  \pause
            \end{itemize}
            \item si $i < n$ volvemos hacemos el procedimiento del paso anterior para $i = i +1$.
            
        \end{itemize}
        
    \end{frame}



    \begin{frame}[fragile]
        \frametitle{}
        Mostramos en pseudocódigo del algoritmo greedy para coloración de vértices.  \pause
        \begin{small}
\begin{verbatim}
# pre: 0,...,n los vértices de un grafo G
# post: devuelve v[0],...,v[n] una coloración de G
color = []  #  color[j] = c dirá que el color de j es c.
for i = 0 to n:
    S = []  # S conjunto de colores asignados a los vértices j
            # (1 <= j <i) que son adyacentes a i (comienza vacío)
    for j = 0 to i-1:
        if j es adyacente a i:
            S.append(color[j])  # agrega el color de j a  S
    k = 0
    while k in S:
        k = k + 1
    color.append(k) # Asigna el color k a i, donde k es el primer 
                    # color que no esta en S. 
\end{verbatim}
\end{small}
    \end{frame}

    \begin{frame}
        \frametitle{}
    
        Debido a que la estrategia greedy es corta de vista, el número de colores que usará será normalmente más grande que le mínimo posible.  \pause

        \begin{ejemplo} Aplicar el algoritmo greedy a 

            \begin{center}
                \begin{tikzpicture}[scale=0.55]
                    %\SetVertexSimple[Shape=circle,FillColor=white]
                    %                
                    \Vertex[x=3.00, y=0.00]{$v_3$}
                    \Vertex[x=1.50, y=2.60]{$v_2$}
                    \Vertex[x=-1.50, y=2.60]{$v_1$}
                    \Vertex[x=-3.00, y=0.00]{$v_6$}
                    \Vertex[x=-1.50, y=-2.60]{$v_5$}
                    \Vertex[x=1.50, y=-2.60]{$v_4$}
                    \Edges($v_2$,$v_1$,$v_6$, $v_5$,$v_4$,$v_1$)
                    \Edges($v_2$,$v_6$)
                    \Edges($v_3$,$v_5$)
                \end{tikzpicture}
                \end{center}

        \end{ejemplo}
    
        \begin{solucion} 
            
            

        \end{solucion}


    \end{frame}


     \begin{frame}
         \frametitle{}

         El orden de los vértices es $v_1$, $v_2$, $v_3$, $v_4$, $v_5$ y $v_6$.  \pause
     
         \vskip .4cm

         El algoritmo es:

         \vskip .4cm
         
         \begin{itemize}
            \item \textbf{Paso 1.} $v_1$ tiene colores vecinos $S = \emptyset$ $\Rightarrow$ $v_1$ color $0$. \pause
            \item \textbf{Paso 2.} $v_2$ tiene colores vecinos $S = \{0\}$ $\Rightarrow$  $v_2$ color $1$. \pause
            \item \textbf{Paso 3.} $v_3$ tiene colores vecinos $S = \emptyset$ $\Rightarrow$  $v_3$ color $0$. \pause
            \item \textbf{Paso 4.} $v_4$ tiene colores vecinos $S = \{0\}$ $\Rightarrow$  $v_4$ color $1$. \pause
            \item  \textbf{Paso 5.} $v_5$ tiene colores vecinos $S = \{0, 1\}$ $\Rightarrow$  $v_5$ color $2$. \pause
            \item \textbf{Paso 6.} $v_6$ tiene colores vecinos $S = \{0, 1, 2\}$ $\Rightarrow$  $v_6$ color $3$. \pause
        \end{itemize}
         
     
     \end{frame}


     \begin{frame}
         \frametitle{}

         Es decir el coloreo queda:
     
         \begin{center}
            \begin{tikzpicture}[scale=0.55]
                %\SetVertexSimple[Shape=circle,FillColor=white]
                %
                \tikzset{VertexStyle/.append style={fill= red!50}}
                \Vertex[x=-1.50, y=2.60]{$v_1$}              
                \Vertex[x=3.00, y=0.00]{$v_3$}
                \tikzset{VertexStyle/.append style={fill= blue!50}}
                \Vertex[x=1.50, y=2.60]{$v_2$}
                \Vertex[x=1.50, y=-2.60]{$v_4$}
                \tikzset{VertexStyle/.append style={fill= green!50}}
                \Vertex[x=-3.00, y=0.00]{$v_6$}
                \tikzset{VertexStyle/.append style={fill= yellow!50}}
                \Vertex[x=-1.50, y=-2.60]{$v_5$}
                \Edges($v_2$,$v_1$,$v_6$, $v_5$,$v_4$,$v_1$)
                \Edges($v_2$,$v_6$)
                \Edges($v_3$,$v_5$)
            \end{tikzpicture}
            \end{center}
            \vskip .4cm

        
            \qed
     
     \end{frame}
        
        \begin{frame}
            \frametitle{}
        
        El orden que se elige inicialmente para los vértices es fundamental para establecer la coloración.  \pause
        \vskip .4cm
        Es bastante fácil ver que si se elige el orden correcto, entonces el algoritmo greedy nos da la mejor coloración posible (ejercicio en el apunte).  \pause
        \vskip .4cm
        Pero hay $n!$ órdenes posibles, y si tuviéramos que controlar cada uno de ellos, el algoritmo requeriría ``tiempo factorial'' (peor aún que tiempo exponencial).
        
        \end{frame}




\begin{frame}
    \frametitle{}

    \begin{ejemplo}
        Aplicar el algorimo greedy al siguiente grafo donde el orden de los vértices es 
        $$
        v_3, v_4, v_6, v_2, v_5, v_1. 
        $$ 
        \vskip .4cm
        \begin{center}
            \begin{tikzpicture}[scale=0.55]
                %\SetVertexSimple[Shape=circle,FillColor=white]
                %                
                \Vertex[x=3.00, y=0.00]{$v_3$}
                \Vertex[x=1.50, y=2.60]{$v_2$}
                \Vertex[x=-1.50, y=2.60]{$v_1$}
                \Vertex[x=-3.00, y=0.00]{$v_6$}
                \Vertex[x=-1.50, y=-2.60]{$v_5$}
                \Vertex[x=1.50, y=-2.60]{$v_4$}
                \Edges($v_2$,$v_1$,$v_6$, $v_5$,$v_4$,$v_1$)
                \Edges($v_2$,$v_6$)
                \Edges($v_3$,$v_5$)
            \end{tikzpicture}
            \end{center}

    \end{ejemplo}

    \begin{solucion} 
            
            

    \end{solucion}


\end{frame}        


\begin{frame}
    \frametitle{}

    El algoritmo es: \pause

    \vskip .4cm
    
    \begin{itemize}
       \item \textbf{Paso 1.} $v_3$ tiene colores vecinos $S = \emptyset$ $\Rightarrow$ $v_3$ color $0$. \pause
       \item \textbf{Paso 2.} $v_4$ tiene colores vecinos $S = \emptyset$ $\Rightarrow$  $v_4$ color $0$. \pause
       \item \textbf{Paso 3.} $v_6$ tiene colores vecinos $S = \emptyset$ $\Rightarrow$  $v_6$ color $0$. \pause
       \item \textbf{Paso 4.} $v_2$ tiene colores vecinos $S = \{0\}$ $\Rightarrow$  $v_2$ color $1$. \pause
       \item  \textbf{Paso 5.} $v_5$ tiene colores vecinos $S = \{0\}$ $\Rightarrow$  $v_5$ color $1$. \pause
       \item \textbf{Paso 6.} $v_1$ tiene colores vecinos $S = \{0, 1\}$ $\Rightarrow$  $v_1$ color $2$. \pause
   \end{itemize}

\end{frame}


\begin{frame}
    \frametitle{}

    Podemos representar en el  grafo la coloración: 

    \vskip .4 cm

    \begin{center}
        \begin{tikzpicture}[scale=0.55]
            %\SetVertexSimple[Shape=circle,FillColor=white]
            %
            \tikzset{VertexStyle/.append style={fill= red!50}}
            \Vertex[x=-1.50, y=2.60]{$v_1$}              
            \tikzset{VertexStyle/.append style={fill= blue!50}}
            \Vertex[x=1.50, y=2.60]{$v_2$}
            \Vertex[x=-1.50, y=-2.60]{$v_5$}
            \tikzset{VertexStyle/.append style={fill= green!50}}
            \Vertex[x=3.00, y=0.00]{$v_3$}
            \Vertex[x=-3.00, y=0.00]{$v_6$}
            \Vertex[x=1.50, y=-2.60]{$v_4$}
            \Edges($v_2$,$v_1$,$v_6$, $v_5$,$v_4$,$v_1$)
            \Edges($v_2$,$v_6$)
            \Edges($v_3$,$v_5$)
        \end{tikzpicture}
        \end{center}
\qed 
\pause
El coloreo queda igual que en la clase pasada.  \pause
\vskip .4 cm
El orden fue elegido de la siguiente forma: dado el coloreo con colores $\chi(G)$, se elije,  en el orden, primero los vértices de un solo color que haya mayor cantidad, luego de otro color que haya mayor cantidad,  etc.    
\end{frame}

    \begin{frame}
        \frametitle{}
        Más allá que el algoritmo greedy no soluciona el problema, el algoritmo es útil tanto en la teoría como en la práctica.  \pause
        
        \vskip .4 cm
        
        Probaremos ahora algunos resultados por medio de la estrategia greedy. \pause

        \vskip .4 cm
        
\begin{teorema}\label{t5.7.1} Si $G$ es un grafo con valencia máxima
    $k$, entonces
    \begin{enumerate}[label=\textit{\alph*)}]
    \item\label{it.com_a}  $\chi(G)\le k+1$,
    \item\label{it.com_b} Si $G$ es conexo y no regular , $\chi(G) \le k$.
    \end{enumerate}
    \end{teorema}
    
        
    
    \end{frame}


    \begin{frame}
        \frametitle{}
    
        \begin{proof} \pause
        
        \ref{it.com_a} Sea $v_1,v_2,\ldots,v_n$ un ordenamiento cualquiera de los vértices de $G$.  \pause
        \vskip .2 cm
        Para cada vértice $v$, si $S$ son los colores de los vecinos a $v$ $\Rightarrow$ $|S| \le k$.  \pause
        \vskip .2 cm
        \ref{it.com_b}
            \begin{enumerate}
                \item  Sea $v_n$ un vértice con $\delta(v_n) < k$.    \pause
                \vskip .2 cm
                \item Sean $v_{n-1},v_{n-2},\ldots,v_{n-r}$ los adyacentes a $v_n$. Hay a lo más $k-1$ de ellos.  \pause
                \vskip .2 cm
                \item Luego se van eligiendo los adyacentes a $v_i$  que no están listados antes ($n > i \ge 1$).   \pause
                \vskip .2 cm
                \item Si $i < n$ el vértice $v_i$ tiene un adyacente a nivel superior $\Rightarrow$ $v_i$ tiene  a lo más $k-1$ adyacentes a nivel inferior.   \pause
                \vskip .2 cm
                \item Si $i < n$, usando greedy y por (4),  se puede colorear $v_i$ con un color en $\{1,\ldots,k\}$.  \pause
                \vskip .2 cm
                \item Por (2) se puede colorear $v_n$   on un color en $\{1,\ldots,k\}$.\qed
        \end{enumerate}
        \end{proof}
    
    \end{frame}



    \begin{frame}
        \frametitle{Grafos bipartitos}
    
        \begin{definicion} Sea $G$  grafo. Diremos que $G$  es \textit{bipartito} si $\chi(G) = 2$. Es decir,  si se puede colorear con dos colores.
        \end{definicion}  \pause

        
        \begin{ejemplo}
        \begin{center}
        \begin{tabular}{llll}
            & 
            \begin{tikzpicture}[scale=1.2]
            \tikzset{VertexStyle/.append style={fill= red!50}}
            \Vertex[x=0.00, y=0.00]{0}
            \Vertex[x=2.00, y=-2.00]{2}
            \Vertex[x=2.00 + 1, y=0.00 + 1]{5}
            \Vertex[x=0.00 + 1, y=-2.00 + 1]{7}
            \tikzset{VertexStyle/.append style={fill= blue!50}}
            \Vertex[x=0.00 , y=-2.00]{3}
            \Vertex[x=0.00 + 1, y=0.00 + 1]{4}
            \Vertex[x=2.00, y=0.00]{1}
            \Vertex[x=2.00 + 1, y=-2.00 + 1]{6}
        
            \Edges(0,1,2,3,0,4,5,6,7,4)
            \Edges(1,5)
            \Edges(2,6)
            \Edges(3,7)
            \end{tikzpicture}
            &
            \qquad  \qquad
            & 
            \begin{tikzpicture}[scale=1.2]
            \tikzset{VertexStyle/.append style={fill= red!50}}
            \Vertex[x=0.00, y=0.00]{0}
            \Vertex[x=0.0, y=-1.00]{2}
            \Vertex[x=0.00 , y=-2.00]{5}
            \Vertex[x=0.00, y=-3.00]{7}
            \tikzset{VertexStyle/.append style={fill= blue!50}}
            \Vertex[x=2.00, y=0.00]{1}
            \Vertex[x=2.00 , y=-1.00]{3}
            \Vertex[x=2.00, y=-2.00]{4}
            \Vertex[x=2.00, y=-3.00]{6}
            \Edges(0,1,2,3,0,4,5,6,7,4)
            \Edges(1,5)
            \Edges(2,6)
            \Edges(3,7)
            \end{tikzpicture}
        \end{tabular}
    \end{center}
\end{ejemplo}
    \end{frame}



    \begin{frame}
        \frametitle{}

        Una consecuencia importante del algoritmo greedy es el siguiente teorema.   \pause

        \vskip .4cm 
        \begin{teorema}\label{t5.7.2} Un grafo es bipartito si  y sólo si no contiene ciclos de longitud impar.
        \end{teorema}  \pause
        \begin{proof}  \pause
            
        ($\Rightarrow$) El contrarrecíproco: \pause si $G$ tiene un ciclo de longitud impar $\Rightarrow$ no es bipartito. \pause
        
        \vskip .2cm 

        Esto es cierto, pues un ciclo de longitud impar requiere $3$ colores.

        \vskip .4cm 

        ($\Leftarrow$) Supongamos que $G$ es un grafo sin ciclos de longitud impar y conexo. \pause
        \vskip .2cm
        Construiremos un orden de $G$ para el cual el algoritmo greedy producirá una coloración de vértices con dos colores. 

        
    \end{proof}
    
    \end{frame}



    \begin{frame}
        \frametitle{}
    
        
        \begin{itemize}
            \item Elijamos cualquier vértice y llamémoslo $v_1$; diremos que $v_1$ esta en el \textit{nivel $0$}. \pause
            \vskip .2cm
            \item A continuación, listemos la lista de vecinos de $v_1$, llamémoslos $v_2,v_3, \dots,v_r$; diremos que estos vértices están en el \textit{nivel 1}. \pause
            \vskip .2cm
            \item Continuando de esta manera, definimos el \textit{nivel $i$} como todos aquellos vértices adyacentes a los del {nivel $i-1$}, exceptuando aquellos previamente listados en el {nivel $i-2$}. \pause
        \end{itemize}
        
        \vskip .3cm
        
        Cuando ningún nuevo vértice puede ser agregado de esta forma, obtenemos $G$ (pues es conexo).\pause
        \vskip .3cm
        El hecho crucial producido por este orden es que un vértice del nivel $i$ solo puede ser adyacente a vértices de los niveles $i-1$ y $i+1$, y no a vértices del mismo nivel. Probemos esto.
    \end{frame}


        \begin{frame}
            \frametitle{}
    

        Supongamos que $x$ e $y$ son vértices en el mismo nivel.\pause
        \vskip .2cm
        Entonces ellos son unidos por caminos de igual longitud $m$ a algún vértice $z$ de un nivel anterior.\pause 
        \vskip .2cm
        Los caminos pueden ser elegidos de tal manera que $z$ sea el único vértice común. \pause
        \vskip .2cm
        Si $x$ e $y$ fueran adyacentes, habría un ciclo de longitud $2m+1$, lo cual contradice
        la hipótesis. \pause
        \vskip .2cm
            \begin{center}
            \begin{tikzpicture}[scale=1]
            \SetVertexSimple[Shape=circle,MinSize=5 pt,FillColor=white]
            \Vertex[x=0.00, y=0.00]{0}
            \Vertex[x=-2.50, y=-1.3]{1}
            \Vertex[x=-3, y=-2.0]{2}
            \Vertex[x=-2.6 , y=-2.8]{3}
            \Vertex[x=1.00, y=-1.4]{4}
            \Vertex[x=1.00 , y=-2.1]{5}
            \Vertex[x=2, y=-2.8]{6}
            \draw (-2.6,-3.2) node {$x$};
            \draw (2,-3.2) node {$y$};
            \draw (0,0.3) node {$z$};
            \Edges(1,2,3)
            \Edges(4,5,6)
            \begin{scope}   [dashed]  % now dashed is for the lines inside the scope
            \Edge (0)(1)
            \Edge (0)(4)
            \Edge (6)(3)
            \end{scope}
            \end{tikzpicture}
            \end{center}


    
        
    
    \end{frame}



    \begin{frame}
        \frametitle{}
            
        Se deduce entonces que el algoritmo greedy asigna:\pause
        \vskip .4cm
        \begin{itemize}
            \item el color $1$ a los vértices en el nivel $0,2,4,\ldots$,\pause
            \vskip .2cm
            \item el color $2$ a los vértices en los niveles $1,3,5,\ldots$.\pause
        \end{itemize}
        \vskip .4cm
        Por consiguiente $\chi(G_0)=2$. \qed
        
        \vskip 3cm
        
    
    \end{frame}



\end{document}

