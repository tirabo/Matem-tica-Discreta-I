\documentclass{beamer} 
%\documentclass[handout]{beamer} % sin pausas
\usetheme{CambridgeUS}
\setbeamertemplate{background}[grid][step=8 ] % cuadriculado


\usepackage{etex}
\usepackage{t1enc}
\usepackage[spanish,es-nodecimaldot]{babel}
\usepackage{latexsym}
\usepackage[utf8]{inputenc}
\usepackage{verbatim}
\usepackage{multicol}
\usepackage{amsgen,amsmath,amstext,amsbsy,amsopn,amsfonts,amssymb}
\usepackage{amsthm}
\usepackage{calc}         % From LaTeX distribution
\usepackage{graphicx}     % From LaTeX distribution
\usepackage{ifthen}
%\usepackage{makeidx}
\input{random.tex}        % From CTAN/macros/generic
\usepackage{subfigure} 
\usepackage{tikz}
\usepackage[customcolors]{hf-tikz}
\usetikzlibrary{arrows}
\usetikzlibrary{matrix}
\tikzset{
    every picture/.append style={
        execute at begin picture={\deactivatequoting},
        execute at end picture={\activatequoting}
    }
}
\usetikzlibrary{decorations.pathreplacing,angles,quotes}
\usetikzlibrary{shapes.geometric}
\usepackage{mathtools}
\usepackage{stackrel}
%\usepackage{enumerate}
\usepackage{enumitem}
\usepackage{tkz-graph}
\usepackage{polynom}
\polyset{%
    style=B,
    delims={(}{)},
    div=:
}
\renewcommand\labelitemi{$\circ$}
\setlist[enumerate]{label={(\arabic*)}}
\setbeamertemplate{itemize item}{$\circ$}
\setbeamertemplate{enumerate items}[default]
\definecolor{links}{HTML}{2A1B81}
\hypersetup{colorlinks,linkcolor=,urlcolor=links}


\newcommand{\Id}{\operatorname{Id}}
\newcommand{\img}{\operatorname{Im}}
\newcommand{\nuc}{\operatorname{Nu}}
\newcommand{\im}{\operatorname{Im}}
\renewcommand\nu{\operatorname{Nu}}
\newcommand{\la}{\langle}
\newcommand{\ra}{\rangle}
\renewcommand{\t}{{\operatorname{t}}}
\renewcommand{\sin}{{\,\operatorname{sen}}}
\newcommand{\Q}{\mathbb Q}
\newcommand{\R}{\mathbb R}
\newcommand{\C}{\mathbb C}
\newcommand{\K}{\mathbb K}
\newcommand{\F}{\mathbb F}
\newcommand{\Z}{\mathbb Z}
\newcommand{\N}{\mathbb N}
\newcommand\sgn{\operatorname{sgn}}
\renewcommand{\t}{{\operatorname{t}}}
\renewcommand{\figurename }{Figura}

%
% Ver http://joshua.smcvt.edu/latex2e/_005cnewenvironment-_0026-_005crenewenvironment.html
%

\renewenvironment{block}[1]% environment name
{% begin code
	\par\vskip .2cm%
	{\color{blue}#1}%
	\vskip .2cm
}%
{%
	\vskip .2cm}% end code


\renewenvironment{alertblock}[1]% environment name
{% begin code
	\par\vskip .2cm%
	{\color{red!80!black}#1}%
	\vskip .2cm
}%
{%
	\vskip .2cm}% end code


\renewenvironment{exampleblock}[1]% environment name
{% begin code
	\par\vskip .2cm%
	{\color{blue}#1}%
	\vskip .2cm
}%
{%
	\vskip .2cm}% end code




\newenvironment{exercise}[1]% environment name
{% begin code
	\par\vspace{\baselineskip}\noindent
	\textbf{Ejercicio (#1)}\begin{itshape}%
		\par\vspace{\baselineskip}\noindent\ignorespaces
	}%
	{% end code
	\end{itshape}\ignorespacesafterend
}


\newenvironment{definicion}[1][]% environment name
{% begin code
	\par\vskip .2cm%
	{\color{blue}Definición #1}%
	\vskip .2cm
}%
{%
	\vskip .2cm}% end code

    \newenvironment{notacion}[1][]% environment name
    {% begin code
        \par\vskip .2cm%
        {\color{blue}Notación #1}%
        \vskip .2cm
    }%
    {%
        \vskip .2cm}% end code

\newenvironment{observacion}[1][]% environment name
{% begin code
	\par\vskip .2cm%
	{\color{blue}Observación #1}%
	\vskip .2cm
}%
{%
	\vskip .2cm}% end code

\newenvironment{ejemplo}[1][]% environment name
{% begin code
	\par\vskip .2cm%
	{\color{blue}Ejemplo #1}%
	\vskip .2cm
}%
{%
	\vskip .2cm}% end code


\newenvironment{preguntas}[1][]% environment name
{% begin code
    \par\vskip .2cm%
    {\color{blue}Preguntas #1}%
    \vskip .2cm
}%
{%
    \vskip .2cm}% end code

\newenvironment{ejercicio}[1][]% environment name
{% begin code
	\par\vskip .2cm%
	{\color{blue}Ejercicio #1}%
	\vskip .2cm
}%
{%
	\vskip .2cm}% end code


\renewenvironment{proof}% environment name
{% begin code
	\par\vskip .2cm%
	{\color{blue}Demostración}%
	\vskip .2cm
}%
{%
	\vskip .2cm}% end code



\newenvironment{demostracion}% environment name
{% begin code
	\par\vskip .2cm%
	{\color{blue}Demostración}%
	\vskip .2cm
}%
{%
	\vskip .2cm}% end code

\newenvironment{idea}% environment name
{% begin code
	\par\vskip .2cm%
	{\color{blue}Idea de la demostración}%
	\vskip .2cm
}%
{%
	\vskip .2cm}% end code

\newenvironment{solucion}% environment name
{% begin code
	\par\vskip .2cm%
	{\color{blue}Solución}%
	\vskip .2cm
}%
{%
	\vskip .2cm}% end code



\newenvironment{lema}[1][]% environment name
{% begin code
	\par\vskip .2cm%
	{\color{blue}Lema #1}\begin{itshape}%
		\par\vskip .2cm
	}%
	{% end code
	\end{itshape}\vskip .2cm\ignorespacesafterend
}

\newenvironment{proposicion}[1][]% environment name
{% begin code
	\par\vskip .2cm%
	{\color{blue}Proposición #1}\begin{itshape}%
		\par\vskip .2cm
	}%
	{% end code
	\end{itshape}\vskip .2cm\ignorespacesafterend
}

\newenvironment{teorema}[1][]% environment name
{% begin code
	\par\vskip .2cm%
	{\color{blue}Teorema #1}\begin{itshape}%
		\par\vskip .2cm
	}%
	{% end code
	\end{itshape}\vskip .2cm\ignorespacesafterend
}


\newenvironment{corolario}[1][]% environment name
{% begin code
	\par\vskip .2cm%
	{\color{blue}Corolario #1}\begin{itshape}%
		\par\vskip .2cm
	}%
	{% end code
	\end{itshape}\vskip .2cm\ignorespacesafterend
}

\newenvironment{propiedad}% environment name
{% begin code
	\par\vskip .2cm%
	{\color{blue}Propiedad}\begin{itshape}%
		\par\vskip .2cm
	}%
	{% end code
	\end{itshape}\vskip .2cm\ignorespacesafterend
}

\newenvironment{conclusion}% environment name
{% begin code
	\par\vskip .2cm%
	{\color{blue}Conclusión}\begin{itshape}%
		\par\vskip .2cm
	}%
	{% end code
	\end{itshape}\vskip .2cm\ignorespacesafterend
}


\newenvironment{definicion*}% environment name
{% begin code
	\par\vskip .2cm%
	{\color{blue}Definición}%
	\vskip .2cm
}%
{%
	\vskip .2cm}% end code

\newenvironment{observacion*}% environment name
{% begin code
	\par\vskip .2cm%
	{\color{blue}Observación}%
	\vskip .2cm
}%
{%
	\vskip .2cm}% end code


\newenvironment{obs*}% environment name
	{% begin code
		\par\vskip .2cm%
		{\color{blue}Observación}%
		\vskip .2cm
	}%
	{%
		\vskip .2cm}% end code

\newenvironment{ejemplo*}% environment name
{% begin code
	\par\vskip .2cm%
	{\color{blue}Ejemplo}%
	\vskip .2cm
}%
{%
	\vskip .2cm}% end code

\newenvironment{ejercicio*}% environment name
{% begin code
	\par\vskip .2cm%
	{\color{blue}Ejercicio}%
	\vskip .2cm
}%
{%
	\vskip .2cm}% end code

\newenvironment{propiedad*}% environment name
{% begin code
	\par\vskip .2cm%
	{\color{blue}Propiedad}\begin{itshape}%
		\par\vskip .2cm
	}%
	{% end code
	\end{itshape}\vskip .2cm\ignorespacesafterend
}

\newenvironment{conclusion*}% environment name
{% begin code
	\par\vskip .2cm%
	{\color{blue}Conclusión}\begin{itshape}%
		\par\vskip .2cm
	}%
	{% end code
	\end{itshape}\vskip .2cm\ignorespacesafterend
}






\newcommand{\nc}{\newcommand}

%%%%%%%%%%%%%%%%%%%%%%%%%LETRAS

\nc{\FF}{{\mathbb F}} \nc{\NN}{{\mathbb N}} \nc{\QQ}{{\mathbb Q}}
\nc{\PP}{{\mathbb P}} \nc{\DD}{{\mathbb D}} \nc{\Sn}{{\mathbb S}}
\nc{\uno}{\mathbb{1}} \nc{\BB}{{\mathbb B}} \nc{\An}{{\mathbb A}}

\nc{\ba}{\mathbf{a}} \nc{\bb}{\mathbf{b}} \nc{\bt}{\mathbf{t}}
\nc{\bB}{\mathbf{B}}

\nc{\cP}{\mathcal{P}} \nc{\cU}{\mathcal{U}} \nc{\cX}{\mathcal{X}}
\nc{\cE}{\mathcal{E}} \nc{\cS}{\mathcal{S}} \nc{\cA}{\mathcal{A}}
\nc{\cC}{\mathcal{C}} \nc{\cO}{\mathcal{O}} \nc{\cQ}{\mathcal{Q}}
\nc{\cB}{\mathcal{B}} \nc{\cJ}{\mathcal{J}} \nc{\cI}{\mathcal{I}}
\nc{\cM}{\mathcal{M}} \nc{\cK}{\mathcal{K}}

\nc{\fD}{\mathfrak{D}} \nc{\fI}{\mathfrak{I}} \nc{\fJ}{\mathfrak{J}}
\nc{\fS}{\mathfrak{S}} \nc{\gA}{\mathfrak{A}}
%%%%%%%%%%%%%%%%%%%%%%%%%LETRAS


\title[Clase 8 - Conteo 4]{Matemática Discreta I \\ Clase 8 - Conteo ($4^\circ$ parte)}
%\author[C. Olmos / A. Tiraboschi]{Carlos Olmos / Alejandro Tiraboschi}
\institute[]{\normalsize FAMAF / UNC
    \\[\baselineskip] ${}^{}$
    \\[\baselineskip]
}
\date[13/04/2023]{13 de abril  de 2023}



\begin{document}
    
    \frame{\titlepage} 
    
    
    \begin{frame}\frametitle{El teorema del binomio}
        En álgebra elemental aprendemos las fórmulas
        $$
        (a+b)^2 = a^2 +2ab +b^2, \qquad (a+b)^3 = a^3 + 3 a^2b +3ab^2 +
        b^3.
        $$
        \vskip .4cm\pause
        A veces nos piden desarrollar la formula para $(a+b)^4$ y
        potencias mayores de $a+b$. 
        \vskip .4cm\pause
        El resultado general que da una formula para $(a+b)^n$ es conocido como el
        {\it {teorema del binomio}} o {\it binomio de Newton}.  
        \vskip .4cm
    \end{frame}
    
    
    \begin{frame}\frametitle{El teorema del binomio}
        
        
        {\color{blue} Teorema}
        \vskip .2cm
        {\it
            Sea $n$ un entero positivo. El coeficiente del termino
            $a^{n-r}b^r$ en el desarrollo de $(a+b)^n$ es el número binomial
            $\binom{n}{r}$.\pause Explícitamente, tenemos
            $$
            (a+b)^n= \binom{n}{0} a^n + \binom{n}{1} a^{n-1}b+ \binom{n}{2}
            a^{n-2}b^2 + \cdots + \binom{n}{n} b^n.$$
        }
        \vskip .2cm        \pause
        Escrito de otra forma:
        \vskip .2cm    
        Si $n >0$, 
        \begin{equation*}
            (a+b)^n= \sum_{i=0}^n \binom{n}{i}     a^{n-i}b^i.
        \end{equation*}
        
    \end{frame}
    
    
    \begin{frame}
        
        
        {\color{blue}Observación}
        \vskip .2cm
        Los coeficientes binomiales que intervienen en la fórmula de $(a+b)^n$ forman una fila del triángulo de Pascal:\pause
        \begin{align*}
            &\binom{n}{0}&&\binom{n}{1}& \binom{n}{2} && &\cdots && &\binom{n}{n-2}&&&\binom{n}{n-1}&&\binom{n}{n}& && 
        \end{align*}
        
        \vskip .4cm\pause
        {\color{blue}Ejemplo}
        \begin{align*}
            (a+ b)^4 &= \binom{4}{0} a^4 + \binom{4}{1} a^{3}b+ \binom{4}{2} a^{2}b^2 + \binom{4}{3} ab^3 + \binom{4}{4} b^4 \\
            &{}\\
            &=  a^4 \;+\; 4 a^{3}b\;+\; 6 a^{2}b^2\; +\; 4 ab^3 \;+\; b^4.
        \end{align*}
    \end{frame}
    
    
    \begin{frame}
        El teorema del binomio puede usarse para deducir identidades en
        que estén involucrados los números binomiales.\pause
        
        \vskip .4cm
        {\color{blue}Ejemplo}
        \vskip .2cm
        Probemos que
        \begin{equation*}
            \binom{n}{0}+\binom{n}{1}+\binom{n}{2}+\cdots+\binom{n}{n}= 2^n.
        \end{equation*}
        
        \vskip .2cm\pause 
        {\color{blue}Demostración.}\pause
        
        \vskip .2cm
        Observemos que $2^n = (1 + 1)^n$. Por el  teorema del binomio sabemos que 
        \begin{align*}
            (1+1)^n &= \binom{n}{0} 1^n + \binom{n}{1} 1^{n-1}1+ \binom{n}{2}
            1^{n-2}1^2 + \cdots + \binom{n}{n} 1^n \\
            &= \binom{n}{0}  + \binom{n}{1} + \binom{n}{2}+ \cdots + \binom{n}{n}. \qed
        \end{align*}
        
    \end{frame}
    
    
    \begin{frame}
        {\color{blue}Observación}
        \vskip .2cm
        La fórmula 
        \begin{equation*}
            \binom{n}{0}+\binom{n}{1}+\binom{n}{2}+\cdots+\binom{n}{n}= 2^n. \tag{$*$}
        \end{equation*}
        tiene una interpretación combinatoria: nos permite calcular nuevamente la cantidad de subconjuntos de un conjuntos de $n$ elementos. 
        \pause
        \vskip .2cm 
        
        {\color{blue}1.} Es claro  que la cantidad de subconjuntos de $k$ elementos de un conjunto de $n$ elementos es $\binom{n}{k}$. 
        
        \vskip .2cm 
        \pause
        {\color{blue}2.} Los subconjuntos de un conjunto son los subconjuntos de 0  elementos, unión los subconjuntos de $1$ elemento, unión los subconjuntos de $2$ elementos, unión los subconjuntos de $3$ elementos, etc.
        
        \vskip .2cm 
        \pause
        {\color{blue}3 .} Por  el principio de adición y la fórmula $(*)$ obtenemos  que la cantidad de subconjuntos de un conjuntos de $n$ elementos es $2^n$.
        
        
    \end{frame}
    
    
    \begin{frame}\frametitle{Ejercicios de conteo}
        {\color{blue}Ejercicio}
        \vskip .2cm
        ¿Cuántos números naturales existen menores que $10^5$, cuyos dígitos sean todos distintos?
        \vskip .2cm \pause
        {\color{blue}Solución.}\pause
        \vskip .2cm
        Los números naturales menores de $10^5$ son todos aquellos que tienen como máximo 5 dígitos.
        \pause
        \begin{enumerate}
            \item[1.] 1 dígito $\to$ $9$ posibilidades,
            \item[2.] 2 dígitos $\to$ $9 \times 9$ posibilidades (81),
            \item[3.] 3 dígitos $\to$ $9 \times 9 \times 8$ posibilidades (648),
            \item[4.] 4 dígitos $\to$ $9 \times 9 \times  8\times 7$ posibilidades (4536),
            \item[5.] 5 dígitos $\to$ $9 \times 9 \times  8\times  7\times 6$ posibilidades (27216),
        \end{enumerate} 
        \vskip .2cm
        \textbf{Total:} $9 +81+ 648+4536+27216 = 32490$.\qed
    \end{frame}
    
        

    \begin{frame}
        
        {\color{blue}Ejercicio}
        \vskip .2cm
        ¿De cuántas formas distintas se pueden escoger $5$  cartas de una baraja de $52$ cartas?
            \begin{enumerate}
                \item Si no hay restricciones.
                \item Si debe haber tres picas y dos corazones.
                \item Si debe haber al menos una carta de cada palo.
                %\item (30 pts) Si de haber al menos $3$ picas.
            \end{enumerate}
            \vskip .2cm \pause
            {\color{blue}Solución.}\pause
            \vskip .3cm
            (1)  Como no nos imponen ninguna condición especial entonces solo debemos determinar cuántos subconjuntos hay con $5$ elementos  de un conjunto de $52$ objetos, es decir, debemos hacer una \textbf{\textit{selección sin orden de 5 cartas entre 52 cartas}}, esto es:
    \end{frame}
        

    \begin{frame}

        \begin{align*}
                \binom{52}{5} &= \frac{52!}{(52-5)!5!} = \frac{52!}{47!5!} = \frac{52\cdot51\cdot50\cdot49\cdot48\cdot{47!}}{{47!}5!}\\[0.3cm]
                &= \frac{52\cdot51\cdot{50}\cdot49\cdot{48}}{{5}\cdot{4\cdot3\cdot2}}
                = 52\cdot51\cdot49\cdot20.
        \end{align*}
        \vskip .2cm \pause
        (2) En este caso, debemos elegir $3$ cartas entre $13$ (para las picas), y la cantidad de elecciones posibles es \ $\binom{13}{3}$.  
        \vskip .2cm \pause
        Por otro lado, para el palo de corazones, hay \ $\binom{13}{2}$ \ formas de elegir $2$ cartas entre $13$.  
        \vskip .2cm \pause
        Luego, por el principio de multiplicación, el resultado es:
                    \begin{equation*}
                    \binom{13}{3}\cdot\binom{13}{2} = \frac{13!}{10!3!}\cdot\frac{13!}{11!2!} = 
                    \frac{13\cdot12\cdot11}{6}\cdot\frac{13\cdot{12}}{2} = 13^2\cdot12\cdot11.
                    \end{equation*}

    \end{frame}
        

    \begin{frame}
        (3) Como debe haber al menos una carta de cada palo, y hay $4$ palos, entonces en cada elección de $5$ cartas \textbf{ineludiblemente tiene que haber dos del mismo palo}.  Ahora bien, si fijamos el palo que se repite, hay
        $$\binom{13}{2}\cdot\binom{13}{1}\cdot\binom{13}{1}\cdot\binom{13}{1}$$
        formas de elegir las 5 cartas.  Como hay 4 palos, tenemos un total de:
        $$4\cdot\binom{13}{2}\cdot\binom{13}{1}\cdot\binom{13}{1}\cdot\binom{13}{1}=4\cdot13\cdot6\cdot13\cdot13\cdot13 = 24\cdot13^4.$$ \qed
        
    \end{frame}
    
    \begin{frame}
        
        {\color{blue}Ejercicio}
        \vskip .2cm
        ¿De cuántas formas diferentes pueden repartirse 5 bolas blancas, 3 rojas y
        2 negras en 10 urnas distintas (etiquetadas) de forma que cada urna
        contenga una bola? 
        \vskip .2cm \pause
        {\color{blue}Solución.}
        \vskip .2cm
        Observar que como hay 10  bolas y 10 urnas, cada urna debe contener una bola.
        \vskip .2cm\pause
        Primero nos preguntamos ¿de cuántas forma puedo poner las 5 bolas blancas en  las 10 urnas? 
        \vskip .3cm\pause
        Este problema es equivalente a elegir 5 urnas entre 10 (las urnas donde pondremos las bolas blancas) y sabemos que las posibilidades son $\binom{10}{5}$.
        \vskip .3cm\pause
        Ahora quedan 5 lugares y  queremos ver de cuántas formas ponemos 3 bolas rojas en 5 urnas y la respuesta es $\binom{5}{3}$ 
        \vskip 1cm
        
    \end{frame}
    
    \begin{frame}
        
        Finalmente,  hay dos lugares para las dos bolas negras y hay una sola forma de ponerlas.\pause
        \vskip .2cm
        El resultado es entonces
        
        \begin{equation*}
            \binom{10}{5}\binom{5}{3} = \frac{10!}{5!5!}\frac{5!}{2!3!} = \frac{10!}{5!2!3!} .
        \end{equation*}\qed
        
        {\color{blue}Observación} Este problema es equivalente a ``cuantas permutaciones hay de 5 bolas  blancas, 3 rojas y 2 negras''.
        \vskip .2cm\pause
        Podemos pensar primero que toda las bolas tienen distinto color,  con lo cual tenemos $10!$ permutaciones.
        \vskip .2cm\pause
        Luego,  hay que dividir por las permutaciones de 2 bolas blancas, 3 rojas  y 2 negras, y el resultado es

        \begin{equation*}
            \frac{10!}{5!2!3!} .
        \end{equation*}
        
    \end{frame}
    
    
    
    
    \begin{frame}
        
        
        {\color{blue}Ejercicio}
        \vskip .2cm
        Un ascensor de un centro comercial parte del sótano con 5 pasajeros y se detiene en 7 pisos. 
        ¿De cuántas maneras distintas pueden descender los pasajeros? ¿Y con la condición de que dos 
        pasajeros no bajen en el mismo piso?
        \vskip .2cm \pause
        {\color{blue}Solución.}\pause
        \vskip .3cm
        Hagamos un poco de  abstracción del problema y pensemos las  posibilidades que tiene 5 personas para ubicarse en 7 lugares: la primera persona tiene 7 posibilidades, la segunda también y así sucesivamente, luego el total de posibilidades es 
        \begin{equation*}
            7 \times     7 \times     7 \times     7 \times     7  = 7^5.
        \end{equation*}
        
        
        
        \vskip 5cm
        
        
    \end{frame}
    
    
    \begin{frame}
        
        
        Si  no estás convencido del razonamiento, lo podemos ver de la siguiente manera: supongamos  que tenemos 5 posiciones (que representan los 5 pasajeros) y  en cada posición podemos poner un número del 1 al 7 (el piso en que baja), entonces la pregunta se reduce a ¿cuántos números de 5 dígitos se pueden formar con los dígitos del 1 al 7?
        \vskip .2cm \pause
        Claramente, la respuesta es también $7^5$.
        
        \vskip 1cm\pause
        
        Para la segunda pregunta, simplemente hay que modificar la pregunta anterior ¿cuántos números de 5 dígitos con todos los dígitos diferentes se pueden formar con los dígitos del 1 al 7?
        \vskip .2cm 
        La respuesta es:
        \begin{equation*}
            7 \times 6 \times 5 \times 4 \times 3 = \frac{7!}{(7-5)!}=\frac{7!}{2!}.
        \end{equation*}
        \qed
        
    \end{frame}
    

    \begin{frame}

        {\color{blue}Ejercicio}
        \vskip .2cm
        Para participar en un torneo de tenis de dobles mixtos, es necesario presentar un equipo de 3 parejas, debiéndose elegir los jugadores entre los integrantes de un grupo constituido por 6 hombres y 3 mujeres. 
        \vskip .2cm
        ¿De cuántas maneras puede seleccionarse el equipo?

        \vskip .2cm \pause
        {\color{blue}Solución.}\pause
        \vskip .3cm
        1) Nos piden que armemos $3$ parejas, sin que importe el orden de las parejas.

        \vskip .2cm \pause

        2) Observar \textbf{todas} las mujeres deben formar parte del equipo, ya que requerimos de 3 mujeres para formar las 3 parejas. De donde, el problema lo podemos replantear como: 
        \vskip .2cm 
        \textbf{¿De cuántas maneras le podemos asignar un compañero (hombre) de juego a cada mujer?}

        
        

    \end{frame}
        

    \begin{frame}
        3) Para resolver esto, representamos al problema en un esquema de la siguiente manera:  Por comodidad, y sin perdida de generalidad, digamos que las mujeres las numeramos 1, 2 y 3. Entonces asignarle un compañero a cada una equivale a completar las siguientes casillas
        \begin{center}
        \underline{1} \, $\llcorner\lrcorner$ \qquad \underline{2} \, $\llcorner\lrcorner$ \qquad \underline{2} \, $\llcorner\lrcorner$
        \end{center}
        \vskip .2cm \pause
        \begin{itemize}
            \item 1º casilla: $6$ posibilidades (los 6 hombres).
            \item 2º casilla: $5$ posibilidades (5 hombres).
            \item 3º casilla: $4$ posibilidades (4 hombres).
        \end{itemize}
        \vskip .2cm \pause
        Así, por el principio de multiplicación, las maneras de seleccionar al equipo son: $$6\cdot5\cdot4 = 120.$$
        \qed
        
    \end{frame}


    \begin{frame}

        {\color{blue}Ejercicio}
        \vskip .2cm
        Tenemos tres cajas numeradas y 10  bolitas indistinguibles ¿De cuántas formas puedo distribuir las bolitas en las cajas?

        \vskip .2cm \pause
        {\color{blue}Solución.}\pause
        \vskip .3cm
        
        1) Pensemos en el problema equivalente: tengo 10 bolitas alineadas,  ¿de cuántas maneras puedo poner dos paredes que separen las bolitas en tres grupos?
        
        \vskip .2cm \pause
        2) Ahora bien, si pensamos que las bolitas son 0's y las paredes son 1's, entonces el problema se reduce a: ¿cuántas permutaciones hay de la palabra?
        $$
        1\,1\,0\,0\,0\,0\,0\,0\,0\,0\,0\,0
        $$

    \end{frame}

    \begin{frame}

        3) Lo resolvemos como siempre: primero consideramos que todos los caracteres son distintos y obtenemos $12!$ permutaciones, luego dividimos por $10!$ y $2!$ para eliminar las permutaciones de 0's seguidos y de 1's seguidos, respectivamente.
        Luego  el resultado es 
        $$
        \frac{12!}{10!2!} = \frac{12 \cdot 11 \cdot 10!}{10! 2} = \frac{12 \cdot 11 }{2} = 66.
        $$
        \qed

            \vskip .2cm
        {\color{blue}Ejercicio}
        \vskip .2cm
        Tenemos $m$ cajas numeradas y $n$  bolitas indistinguibles. Probar que hay 
        $$\binom{m+n-1}{m-1}$$
        formas de distribuir las bolitas en las cajas.

        \vskip .3cm\pause

        \textit{(Selecciones no ordenadas con repetición)}
    

    \end{frame}
    
\end{document}

