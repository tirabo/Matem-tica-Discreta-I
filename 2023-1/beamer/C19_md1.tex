%\documentclass{beamer} 
\documentclass[handout]{beamer} % sin pausas
\usetheme{CambridgeUS}
%\setbeamertemplate{background}[grid][step=8 ] % cuadriculado


\usepackage{etex}
\usepackage{t1enc}
\usepackage[spanish,es-nodecimaldot]{babel}
\usepackage{latexsym}
\usepackage[utf8]{inputenc}
\usepackage{verbatim}
\usepackage{multicol}
\usepackage{amsgen,amsmath,amstext,amsbsy,amsopn,amsfonts,amssymb}
\usepackage{amsthm}
\usepackage{calc}         % From LaTeX distribution
\usepackage{graphicx}     % From LaTeX distribution
\usepackage{ifthen}
%\usepackage{makeidx}
\input{random.tex}        % From CTAN/macros/generic
\usepackage{subfigure} 
\usepackage{tikz}
\usepackage[customcolors]{hf-tikz}
\usetikzlibrary{arrows}
\usetikzlibrary{matrix}
\tikzset{
    every picture/.append style={
        execute at begin picture={\deactivatequoting},
        execute at end picture={\activatequoting}
    }
}
\usetikzlibrary{decorations.pathreplacing,angles,quotes}
\usetikzlibrary{shapes.geometric}
\usepackage{mathtools}
\usepackage{stackrel}
%\usepackage{enumerate}
\usepackage{enumitem}
\usepackage{tkz-graph}
\usepackage{polynom}
\polyset{%
    style=B,
    delims={(}{)},
    div=:
}
\renewcommand\labelitemi{$\circ$}
\setlist[enumerate]{label={(\arabic*)}}
\setbeamertemplate{itemize item}{$\circ$}
\setbeamertemplate{enumerate items}[default]
\definecolor{links}{HTML}{2A1B81}
\hypersetup{colorlinks,linkcolor=,urlcolor=links}


\newcommand{\Id}{\operatorname{Id}}
\newcommand{\img}{\operatorname{Im}}
\newcommand{\nuc}{\operatorname{Nu}}
\newcommand{\im}{\operatorname{Im}}
\renewcommand\nu{\operatorname{Nu}}
\newcommand{\la}{\langle}
\newcommand{\ra}{\rangle}
\renewcommand{\t}{{\operatorname{t}}}
\renewcommand{\sin}{{\,\operatorname{sen}}}
\newcommand{\Q}{\mathbb Q}
\newcommand{\R}{\mathbb R}
\newcommand{\C}{\mathbb C}
\newcommand{\K}{\mathbb K}
\newcommand{\F}{\mathbb F}
\newcommand{\Z}{\mathbb Z}
\newcommand{\N}{\mathbb N}
\newcommand\sgn{\operatorname{sgn}}
\renewcommand{\t}{{\operatorname{t}}}
\renewcommand{\figurename }{Figura}

%
% Ver http://joshua.smcvt.edu/latex2e/_005cnewenvironment-_0026-_005crenewenvironment.html
%

\renewenvironment{block}[1]% environment name
{% begin code
	\par\vskip .2cm%
	{\color{blue}#1}%
	\vskip .2cm
}%
{%
	\vskip .2cm}% end code


\renewenvironment{alertblock}[1]% environment name
{% begin code
	\par\vskip .2cm%
	{\color{red!80!black}#1}%
	\vskip .2cm
}%
{%
	\vskip .2cm}% end code


\renewenvironment{exampleblock}[1]% environment name
{% begin code
	\par\vskip .2cm%
	{\color{blue}#1}%
	\vskip .2cm
}%
{%
	\vskip .2cm}% end code




\newenvironment{exercise}[1]% environment name
{% begin code
	\par\vspace{\baselineskip}\noindent
	\textbf{Ejercicio (#1)}\begin{itshape}%
		\par\vspace{\baselineskip}\noindent\ignorespaces
	}%
	{% end code
	\end{itshape}\ignorespacesafterend
}


\newenvironment{definicion}[1][]% environment name
{% begin code
	\par\vskip .2cm%
	{\color{blue}Definición #1}%
	\vskip .2cm
}%
{%
	\vskip .2cm}% end code

    \newenvironment{notacion}[1][]% environment name
    {% begin code
        \par\vskip .2cm%
        {\color{blue}Notación #1}%
        \vskip .2cm
    }%
    {%
        \vskip .2cm}% end code

\newenvironment{observacion}[1][]% environment name
{% begin code
	\par\vskip .2cm%
	{\color{blue}Observación #1}%
	\vskip .2cm
}%
{%
	\vskip .2cm}% end code

\newenvironment{ejemplo}[1][]% environment name
{% begin code
	\par\vskip .2cm%
	{\color{blue}Ejemplo #1}%
	\vskip .2cm
}%
{%
	\vskip .2cm}% end code


\newenvironment{preguntas}[1][]% environment name
{% begin code
    \par\vskip .2cm%
    {\color{blue}Preguntas #1}%
    \vskip .2cm
}%
{%
    \vskip .2cm}% end code

\newenvironment{ejercicio}[1][]% environment name
{% begin code
	\par\vskip .2cm%
	{\color{blue}Ejercicio #1}%
	\vskip .2cm
}%
{%
	\vskip .2cm}% end code


\renewenvironment{proof}% environment name
{% begin code
	\par\vskip .2cm%
	{\color{blue}Demostración}%
	\vskip .2cm
}%
{%
	\vskip .2cm}% end code



\newenvironment{demostracion}% environment name
{% begin code
	\par\vskip .2cm%
	{\color{blue}Demostración}%
	\vskip .2cm
}%
{%
	\vskip .2cm}% end code

\newenvironment{idea}% environment name
{% begin code
	\par\vskip .2cm%
	{\color{blue}Idea de la demostración}%
	\vskip .2cm
}%
{%
	\vskip .2cm}% end code

\newenvironment{solucion}% environment name
{% begin code
	\par\vskip .2cm%
	{\color{blue}Solución}%
	\vskip .2cm
}%
{%
	\vskip .2cm}% end code



\newenvironment{lema}[1][]% environment name
{% begin code
	\par\vskip .2cm%
	{\color{blue}Lema #1}\begin{itshape}%
		\par\vskip .2cm
	}%
	{% end code
	\end{itshape}\vskip .2cm\ignorespacesafterend
}

\newenvironment{proposicion}[1][]% environment name
{% begin code
	\par\vskip .2cm%
	{\color{blue}Proposición #1}\begin{itshape}%
		\par\vskip .2cm
	}%
	{% end code
	\end{itshape}\vskip .2cm\ignorespacesafterend
}

\newenvironment{teorema}[1][]% environment name
{% begin code
	\par\vskip .2cm%
	{\color{blue}Teorema #1}\begin{itshape}%
		\par\vskip .2cm
	}%
	{% end code
	\end{itshape}\vskip .2cm\ignorespacesafterend
}


\newenvironment{corolario}[1][]% environment name
{% begin code
	\par\vskip .2cm%
	{\color{blue}Corolario #1}\begin{itshape}%
		\par\vskip .2cm
	}%
	{% end code
	\end{itshape}\vskip .2cm\ignorespacesafterend
}

\newenvironment{propiedad}% environment name
{% begin code
	\par\vskip .2cm%
	{\color{blue}Propiedad}\begin{itshape}%
		\par\vskip .2cm
	}%
	{% end code
	\end{itshape}\vskip .2cm\ignorespacesafterend
}

\newenvironment{conclusion}% environment name
{% begin code
	\par\vskip .2cm%
	{\color{blue}Conclusión}\begin{itshape}%
		\par\vskip .2cm
	}%
	{% end code
	\end{itshape}\vskip .2cm\ignorespacesafterend
}


\newenvironment{definicion*}% environment name
{% begin code
	\par\vskip .2cm%
	{\color{blue}Definición}%
	\vskip .2cm
}%
{%
	\vskip .2cm}% end code

\newenvironment{observacion*}% environment name
{% begin code
	\par\vskip .2cm%
	{\color{blue}Observación}%
	\vskip .2cm
}%
{%
	\vskip .2cm}% end code


\newenvironment{obs*}% environment name
	{% begin code
		\par\vskip .2cm%
		{\color{blue}Observación}%
		\vskip .2cm
	}%
	{%
		\vskip .2cm}% end code

\newenvironment{ejemplo*}% environment name
{% begin code
	\par\vskip .2cm%
	{\color{blue}Ejemplo}%
	\vskip .2cm
}%
{%
	\vskip .2cm}% end code

\newenvironment{ejercicio*}% environment name
{% begin code
	\par\vskip .2cm%
	{\color{blue}Ejercicio}%
	\vskip .2cm
}%
{%
	\vskip .2cm}% end code

\newenvironment{propiedad*}% environment name
{% begin code
	\par\vskip .2cm%
	{\color{blue}Propiedad}\begin{itshape}%
		\par\vskip .2cm
	}%
	{% end code
	\end{itshape}\vskip .2cm\ignorespacesafterend
}

\newenvironment{conclusion*}% environment name
{% begin code
	\par\vskip .2cm%
	{\color{blue}Conclusión}\begin{itshape}%
		\par\vskip .2cm
	}%
	{% end code
	\end{itshape}\vskip .2cm\ignorespacesafterend
}






\newcommand{\nc}{\newcommand}

%%%%%%%%%%%%%%%%%%%%%%%%%LETRAS

\nc{\FF}{{\mathbb F}} \nc{\NN}{{\mathbb N}} \nc{\QQ}{{\mathbb Q}}
\nc{\PP}{{\mathbb P}} \nc{\DD}{{\mathbb D}} \nc{\Sn}{{\mathbb S}}
\nc{\uno}{\mathbb{1}} \nc{\BB}{{\mathbb B}} \nc{\An}{{\mathbb A}}

\nc{\ba}{\mathbf{a}} \nc{\bb}{\mathbf{b}} \nc{\bt}{\mathbf{t}}
\nc{\bB}{\mathbf{B}}
 
\nc{\cP}{\mathcal{P}} \nc{\cU}{\mathcal{U}} \nc{\cX}{\mathcal{X}}
\nc{\cE}{\mathcal{E}} \nc{\cS}{\mathcal{S}} \nc{\cA}{\mathcal{A}}
\nc{\cC}{\mathcal{C}} \nc{\cO}{\mathcal{O}} \nc{\cQ}{\mathcal{Q}}
\nc{\cB}{\mathcal{B}} \nc{\cJ}{\mathcal{J}} \nc{\cI}{\mathcal{I}}
\nc{\cM}{\mathcal{M}} \nc{\cK}{\mathcal{K}}

\nc{\fD}{\mathfrak{D}} \nc{\fI}{\mathfrak{I}} \nc{\fJ}{\mathfrak{J}}
\nc{\fS}{\mathfrak{S}} \nc{\gA}{\mathfrak{A}}
%%%%%%%%%%%%%%%%%%%%%%%%%LETRAS



\title[Clase 20 - Isomorfismo / Caminatas, caminos y ciclos]{Matemática Discreta I \\ Clase 20 -  Isomorfismo / Caminatas, caminos y ciclos}
%\author[C. Olmos / A. Tiraboschi]{Carlos Olmos / Alejandro Tiraboschi}
\institute[]{\normalsize FAMAF / UNC
    \\[\baselineskip] ${}^{}$
    \\[\baselineskip]
}
\date[06/06/2023]{6 de junio de 2023}




\begin{document}
    
    \frame{\titlepage} 
    
    
    
    
    \begin{frame}
        Una aplicación importante de la noción de valencia es en el problema de determinar si dos grafos son o no isomorfos. \pause
        
        \vskip .6cm
        
        Si     $\alpha:V_1 \to  V_2$ es un isomorfismo entre $G_1$ y $G_2$, y $\alpha(v)=w$, entonces cada arista que contiene a $v$ se transforma en una arista que contiene a $w$. \pause
        
        \vskip .6cm
        
        En consecuencia $\delta(v)=\delta(w)$. Por otro lado, si $G_1$ tiene un vértice $x$, con valencia $\delta(x)=\delta_0$, y $G_2$ no tiene vértices    con valencia $\delta_0$, entonces $G_1$ y $G_2$ no pueden ser    isomorfos. 
        
        
        
        
        
        
        
        
    \end{frame}
    
    
    \begin{frame}
        Esto nos da otra manera de ver que no son isomorfos los siguientes grafos:
        
        \vskip .5cm
        \begin{figure}[ht]
            \begin{center}
                \begin{tabular}{llll}
                    &
                    \begin{tikzpicture}[scale=1]
                        %\SetVertexSimple[Shape=circle,FillColor=white]
                        \Vertex[x=0.00, y=2.00]{$a$}
                        \Vertex[x=1.90, y=0.62]{$b$}
                        \Vertex[x=1.18, y=-1.62]{$c$}
                        \Vertex[x=-1.18, y=-1.62]{$d$}
                        \Vertex[x=-1.90, y=0.62]{$e$}
                        \Edges($c$, $b$,$a$,$e$,$d$,$b$,$a$,$d$)
                        \Edges($e$,$b$)
                    \end{tikzpicture}
                    &
                    \qquad
                    & 
                    \begin{tikzpicture}[scale=1]
                        %\SetVertexSimple[Shape=circle,FillColor=white]
                        \Vertex[x=0.00, y=2.00]{1}
                        \Vertex[x=1.90, y=0.62]{2}
                        \Vertex[x=1.18, y=-1.62]{3}
                        \Vertex[x=-1.18, y=-1.62]{4}
                        \Vertex[x=-1.90, y=0.62]{5}
                        \Edges(1,2,3,4,5,1)
                        \Edges(4,2,5)
                    \end{tikzpicture}
                \end{tabular}
            \end{center}
        \end{figure}
        \vskip .4cm
        Puesto que el primer grafo tiene un vértice de
        valencia 1 y el segundo no.
    \end{frame}
    
    
    \begin{frame}
        Una extensión de esta idea se da en la siguiente proposición.\pause
        
        \vskip .4cm
        \begin{proposicion}\label{criterioiso}Sean  $G_1$ y $G_2$ grafos isomorfos. Para cada $k\ge 0$ sea $n_i(k)$ el
            número de vértices de $G_i$ que tienen valencia $k$ ($i=1,2$).
            Entonces $n_1(k)=n_2(k)$.
        \end{proposicion}\pause
        \vskip .4cm
        \begin{proof}\pause  Hemos visto más arriba que si $\alpha:V_1 \to  V_2$ es un isomorfismo entre $G_1$ y $G_2$ y $v\in V_1$, entonces $\delta(v)=\delta(\alpha(v))$. Luego la cantidad de vértices con valencia $k$ en $G_1$ es igual  a la cantidad de vértices con valencia $k$ en $G_2$.  
            
            \qed
        \end{proof}
    \end{frame}
    
    \begin{frame}
        
        
        No hay ningún criterio general eficiente para determinar si dos grafos son isomorfos o no. En  el siguiente ejemplo, no se aplica el criterio anterior. \pause
        
        \vskip .4cm
        \begin{ejemplo}
            Probar que los siguientes grafos no son isomorfos.
            
            \begin{figure}[ht]
                \begin{center}
                    \begin{tabular}{llll}
                        &
                        \begin{tikzpicture}[scale=1]
                            \SetVertexSimple[Shape=circle,FillColor=white,MinSize=8 pt]
                            \Vertex[x=0.00, y=2.00]{a}
                            \Vertex[x=2., y=-1.50]{b}
                            \Vertex[x=-2., y=-1.50]{c}
                            \Edges(a,b,c,a)
                            \Vertex[x=0.00, y=0.85]{1}
                            \Vertex[x=1., y=-0.9]{2}
                            \Vertex[x=-1., y=-0.9]{3}
                            \Edges(1,2,3,1)
                            \Edges(a,1,3,c,b,2)
                            \draw (0,-2.2) node {$G_1$};
                        \end{tikzpicture}
                        &
                        \qquad
                        & 
                        \begin{tikzpicture}[scale=0.65]
                            \SetVertexSimple[Shape=circle,FillColor=white,MinSize=8 pt]
                            %                
                            \Vertex[x=3.00, y=0.00]{1}
                            \Vertex[x=1.50, y=2.60]{2}
                            \Vertex[x=-1.50, y=2.60]{3}
                            \Vertex[x=-3.00, y=0.00]{4}
                            \Vertex[x=-1.50, y=-2.60]{5}
                            \Vertex[x=1.50, y=-2.60]{6}
                            \Edges(1,2,3,4,5,6,1)
                            \Edges(1,4) \Edges(3,6) \Edges(2,5)
                            \draw (0,-3.8) node {$G_2$};
                        \end{tikzpicture}
                    \end{tabular}
                \end{center}
            \end{figure}
        \end{ejemplo}
        
        
    \end{frame}
    
    
    
    \begin{frame}
        \begin{solucion}\pause
            Ambos tienen $6$ vértices, $9$ aristas y todos los vértices son de valencia $3$. 
            \pause
            \vskip.2cm
            Por lo tanto no podemos utilizar los criterios anteriores para decir que no son isomorfos.
            \pause 
            \vskip .2cm
            
            Sin embargo, observar que  $G_1$ tiene un subgrafo $K_3$:
            \vskip .2cm
            \begin{center}
                \begin{tikzpicture}[scale=0.6]
                    \SetVertexSimple[Shape=circle,FillColor=white,MinSize=8 pt]
                    \Vertex[x=0.00, y=2.00]{a}
                    \Vertex[x=2., y=-1.50]{b}
                    \Vertex[x=-2., y=-1.50]{c}
                    \Edges(a,b,c,a)
                \end{tikzpicture}
            \end{center}\pause
            \vskip .4cm
            Mientras que $G_2$ no lo tiene. Por lo tanto, $G_1$ y $G_2$ no son isomorfos.
            
            \qed
        \end{solucion}
    \end{frame}
    
    \begin{frame}\frametitle{Caminatas, caminos y ciclos}
        \begin{definicion}  Una {\em caminata} en un grafo $G$ es  \index{caminata}
            una secuencia de vértices
            $$
            v_1,v_2,\ldots,v_k,
            $$
            tal que $v_i$ y $v_{i+1}$ son adyacentes ($1 \le i \le k-1$). \pause
            
            \vskip .4cm
            Si todos los vértices son distintos, una caminata es llamada un {\em  camino}.\pause
            
            \vskip .4cm
            
            Un \textit{recorrido} es una caminata donde todas las aristas $\{v_i,v_{i+1}\}$ con $1 \le i \le k-1$ son distintas.  

            \vskip .4cm

            Un {\em ciclo} a una caminata $v_1,v_2,\ldots,v_{k}, v_1$ con $v_1,v_2,\ldots,v_{k}$ camino y $k \ge 3$. A menudo diremos que es un {\em
                $k$-ciclo}, o un ciclo de {\em longitud} $k$ en $G$.
        \end{definicion}\pause

    
        
    \end{frame}


    \begin{frame}
        
        \begin{observacion}
            \begin{itemize}
                \item Por definición de caminata $k\ge2$, es decir una caminata tiene al menos una arista ($\{v_1, v_2\}$).
                \item Por definición de ciclo  $v_1,v_2,\ldots,v_{k}, v_1$, $k \ge 3$; un ciclo tiene al menos 3 aristas (claramente, no hay 2-ciclos.). 
            \end{itemize}

        \end{observacion}
    
        \vskip 3cm
    
    \end{frame}
    
    \begin{frame}
        \begin{ejemplo}
            Dibujemos caminatas, caminos, recorridos y ciclos en el siguiente grafo:
            \vskip .4cm
            
            \begin{center}
                \begin{tikzpicture}[scale=1]
                    \def\rvar{1.2}
                    \Vertex[x=0.00, y=-2.00]{u}
                    \Vertex[x=\rvar*1.90, y=-0.62]{t}
                    \Vertex[x=\rvar*1.18, y=1.62]{q}
                    \Vertex[x=-1.18*\rvar, y=1.62]{p}
                    \Vertex[x=-1.90*\rvar, y=-0.62]{r}
                    \Vertex[x=0, y=0]{s}
                    \Edge(u)(t)
                    \Edges(t,q,p)
                    \Edges(r,u)
                    \Edges(s,t)
                    \Edges(r,s,q,r)
                    \Edges(p,t,s)
                    \Edges(s,p)
                    
                    %\Edge[label=1](u)(t)
                \end{tikzpicture}
            \end{center}
            
        \end{ejemplo}
    \end{frame}

    
    \begin{frame}
        
        Caminata: p,q,t,s,q,r,u,r
        
        \begin{figure}[ht]
            \begin{center}
                \begin{tikzpicture}[scale=1]
                    %\SetVertexSimple[Shape=circle,FillColor=white]
                    \def\rvar{1.2}
                    \Vertex[x=0.00, y=-2.00]{u}
                    \Vertex[x=\rvar*1.90, y=-0.62]{t}
                    \Vertex[x=\rvar*1.18, y=1.62]{q}
                    \Vertex[x=-1.18*\rvar, y=1.62]{p}
                    \Vertex[x=-1.90*\rvar, y=-0.62]{r}
                    \Vertex[x=0, y=0]{s}
                    \Edges(u,t,q,p)
                    \Edges(r,u)
                    \Edges(s,t)
                    \Edges(r,s,q,r)
                    \Edges(p,t,s)
                    \tikzset{EdgeStyle/.style = {->,color=blue}}
                    %\tikzset{EdgeStyle/.append style = {bend left,->,color=blue}}
                    \Edge[label=1](p)(q)
                    \Edge[label=2](q)(t)
                    \Edge[label=3](t)(s)
                    \Edge[label=4](s)(q)
                    \Edge[label=5](q)(r)
                    \Edge[label=6](r)(u)
                    \Edge[label=7,style={bend left}](u)(r)
                \end{tikzpicture}
            \end{center}
        \end{figure}
    \end{frame}

    \begin{frame}
        
        Recorrido: p,q,t,s,q,r,u
        
        \begin{figure}[ht]
            \begin{center}
                \begin{tikzpicture}[scale=1]
                    %\SetVertexSimple[Shape=circle,FillColor=white]
                    \def\rvar{1.2}
                    \Vertex[x=0.00, y=-2.00]{u}
                    \Vertex[x=\rvar*1.90, y=-0.62]{t}
                    \Vertex[x=\rvar*1.18, y=1.62]{q}
                    \Vertex[x=-1.18*\rvar, y=1.62]{p}
                    \Vertex[x=-1.90*\rvar, y=-0.62]{r}
                    \Vertex[x=0, y=0]{s}
                    \Edges(u,t,q,p)
                    \Edges(r,u)
                    \Edges(s,t)
                    \Edges(r,s,q,r)
                    \Edges(p,t,s)
                    \tikzset{EdgeStyle/.style = {->,color=blue}}
                    %\tikzset{EdgeStyle/.append style = {bend left,->,color=blue}}
                    \Edge[label=1](p)(q)
                    \Edge[label=2](q)(t)
                    \Edge[label=3](t)(s)
                    \Edge[label=4](s)(q)
                    \Edge[label=5](q)(r)
                    \Edge[label=6](r)(u)
                \end{tikzpicture}
            \end{center}
        \end{figure}
    \end{frame}
    
    
    \begin{frame}
        
        Camino: p,q,s,r,u,t
        
        \begin{figure}[ht]
            \begin{center}
                \begin{tikzpicture}[scale=1]
                    %\SetVertexSimple[Shape=circle,FillColor=white]
                    \def\rvar{1.2}
                    \Vertex[x=0.00, y=-2.00]{u}
                    \Vertex[x=\rvar*1.90, y=-0.62]{t}
                    \Vertex[x=\rvar*1.18, y=1.62]{q}
                    \Vertex[x=-1.18*\rvar, y=1.62]{p}
                    \Vertex[x=-1.90*\rvar, y=-0.62]{r}
                    \Vertex[x=0, y=0]{s}
                    \Edges(u,t,q,p)
                    \Edges(r,u)
                    \Edges(s,t)
                    \Edges(r,s,q,r)
                    \Edges(p,t,s)
                    \tikzset{EdgeStyle/.style = {->,color=blue}}
                    \Edge[label=1](p)(q)
                    \Edge[label=2](q)(s)
                    \Edge[label=3](s)(r)
                    \Edge[label=4](r)(u)
                    \Edge[label=5](u)(t)
                \end{tikzpicture}
            \end{center}
        \end{figure}
    \end{frame}
    
    \begin{frame}
        
        Ciclo: p,q,s,r,u,t,p
        
        \begin{figure}[ht]
            \begin{center}
                \begin{tikzpicture}[scale=1]
                    %\SetVertexSimple[Shape=circle,FillColor=white]
                    \def\rvar{1.2}
                    \Vertex[x=0.00, y=-2.00]{u}
                    \Vertex[x=\rvar*1.90, y=-0.62]{t}
                    \Vertex[x=\rvar*1.18, y=1.62]{q}
                    \Vertex[x=-1.18*\rvar, y=1.62]{p}
                    \Vertex[x=-1.90*\rvar, y=-0.62]{r}
                    \Vertex[x=0, y=0]{s}
                    \Edges(u,t,q,p)
                    \Edges(r,u)
                    \Edges(s,t)
                    \Edges(r,s,q,r)
                    \Edges(p,t,s)
                    \tikzset{EdgeStyle/.style = {->,color=blue}}
                    \Edge[label=1](p)(q)
                    \Edge[label=2](q)(s)
                    \Edge[label=3](s)(r)
                    \Edge[label=4](r)(u)
                    \Edge[label=5](u)(t)
                    \Edge(t)(p)
    
                \end{tikzpicture}
            \end{center}
        \end{figure}
    \end{frame}
    
    
    \begin{frame}
        
        \begin{lema} Sea $G$ un grafo.  Entonces, $x$ e $y$ pueden ser unidos por una caminata  si  y sólo si  $x$ e $y$ pueden ser unidos por un camino.
        \end{lema}
        \begin{idea} ($\Leftarrow$) es trivial (un camino es una caminata).
            
            \vskip .2cm
            ($\Rightarrow$) Eliminar "bucles".
            
            \begin{figure}[ht]
                \begin{center}
                    \begin{tikzpicture}[scale=1]
                        \SetVertexSimple[Shape=circle,FillColor=white,MinSize=5 pt]
                        %
                        %\ponertz{-7}{-30}{$x=$}
                        \draw (-0.6,0) node {$x = $};                
                        \draw (0,0.4) node {$v_0$};
                        \Vertex[x=0.00, y=0]{v0}
                        \Vertex[x=1, y=0]{v1}
                        \draw (1,0.4) node {$v_1$};
                        \Vertex[x=3, y=0]{vi}
                        \draw (3,0.4) node {$v_i$};
                        \Vertex[x=2.5, y=0.5]{vi1}
                        \Vertex[x=2.5, y=1]{vi2}
                        \Vertex[x=3.5, y=0.5]{vi3}
                        \Vertex[x=3.5, y=1]{vi4}
                        \Edges(v0,v1)
                        \Edges(vi,vi1)
                        \Edges(vi1,vi2)
                        \draw (2.7,1.3) node {$\scriptstyle\bullet$};
                        \draw (2.9,1.5) node {$\scriptstyle\bullet$};
                        \draw (3.1,1.5) node {$\scriptstyle\bullet$};
                        \draw (3.3,1.3) node {$\scriptstyle\bullet$};
                        \draw (2,0) node {$\scriptstyle\bullet$};
                        \draw (2.3,0) node {$\scriptstyle\bullet$};
                        \Edges(vi3,vi4)
                        \Edges(vi3,vi)
                        \Vertex[x=5, y=0]{vr1}
                        \Vertex[x=6, y=0]{vr}
                        \Edges(vr1,vr)
                        \draw (6,0.4) node {$v_r$};
                        \draw (5,0.4) node {$v_{r-1}$};
                        \draw (6.6,0) node {$=y$};
                        
                        \SetVertexNormal[LineColor=white]
                        \Vertex[x=1.8, y=0]{s1}
                        \Vertex[x=2.2, y=0]{s2}
                        \Edges(v1,s1)
                        \Edges(vi,s2)
                        \draw (1.7,0) node {$\scriptstyle\bullet$};
                        \draw (2,0) node {$\scriptstyle\bullet$};
                        \draw (2.3,0) node {$\scriptstyle\bullet$};
                        
                        \Vertex[x=4.2, y=0]{t1}
                        \Vertex[x=3.8, y=0]{t2}
                        \Edges(vr1,t1)
                        \Edges(vi,t2)
                        \draw (3.7,0) node {$\scriptstyle\bullet$};
                        \draw (4,0) node {$\scriptstyle\bullet$};
                        \draw (4.3,0) node {$\scriptstyle\bullet$};
                    \end{tikzpicture}
                \end{center}
                Se transforma en 
                \begin{center}
                    \begin{tikzpicture}[scale=1]
                        \SetVertexSimple[Shape=circle,FillColor=white,MinSize=5 pt]
                        %
                        %\ponertz{-7}{-30}{$x=$}
                        \draw (-0.6,0) node {$x = $};                
                        \draw (0,0.4) node {$v_0$};
                        \Vertex[x=0.00, y=0]{v0}
                        \Vertex[x=1, y=0]{v1}
                        \draw (1,0.4) node {$v_1$};
                        \Vertex[x=3, y=0]{vi}
                        \draw (3,0.4) node {$v_i$};
                        \Edges(v0,v1)
                        \Vertex[x=5, y=0]{vr1}
                        \Vertex[x=6, y=0]{vr}
                        \Edges(vr1,vr)
                        \draw (6,0.4) node {$v_r$};
                        \draw (5,0.4) node {$v_{r-1}$};
                        \draw (6.6,0) node {$=y$};
                        
                        \SetVertexNormal[LineColor=white]
                        \Vertex[x=1.8, y=0]{s1}
                        \Vertex[x=2.2, y=0]{s2}
                        \Edges(v1,s1)
                        \Edges(vi,s2)
                        \draw (1.7,0) node {$\scriptstyle\bullet$};
                        \draw (2,0) node {$\scriptstyle\bullet$};
                        \draw (2.3,0) node {$\scriptstyle\bullet$};
                        
                        \Vertex[x=4.2, y=0]{t1}
                        \Vertex[x=3.8, y=0]{t2}
                        \Edges(vr1,t1)
                        \Edges(vi,t2)
                        \draw (3.7,0) node {$\scriptstyle\bullet$};
                        \draw (4,0) node {$\scriptstyle\bullet$};
                        \draw (4.3,0) node {$\scriptstyle\bullet$};
                    \end{tikzpicture}
                \end{center}    \qed
            \end{figure}
            
            
            
            \vskip 4cm
        \end{idea}
        
        
        
    \end{frame}
    
    \begin{frame} 
        Escribamos $x \sim y$ siempre y cuando los vértices $x$ e $y$ de
        $G$ puedan ser unidos por un camino en $G$ o $x=y$: hablando en forma
        rigurosa, esto significa que si $x \ne y$ hay un camino $v_1,v_2,\ldots,v_k$ en
        $G$ con $x=v_1$ e $y=v_k$. 
        \vskip .3cm
        
        \begin{definicion}
            Sea $G$ grafo,  diremos que es conexo si  para $x\sim y$ para cualesquiera  $x,y$ vértices en $G$. 
        \end{definicion}
        
        \vskip .6cm
        El  lema de la página anterior implica que  $\sim$  es una relación de equivalencia. 

    
        
 
    \end{frame}
    
    \begin{frame}
        \begin{proposicion} Sea $G$ grafo  y $x$, $y$, $z$ vértices de $G$. Entonces, 
            \begin{enumerate}
                \item $x \sim x$ (reflexividad de $\sim$).
                \item $x \sim y$, entonces $y \sim x$ (simetría de $\sim$).
                \item $x \sim y$,  $y \sim z$, entonces  $x \sim z$ (transitividad  de $\sim$).
            \end{enumerate}
        \end{proposicion}
        
        \begin{proof}
            (1) Por definición $x \sim x$.
            \vskip .4cm

            (2) Si $x=x_1,x_2, \ldots,x_k= y$  es un camino de $x$ a $y$,  entonces $y=x_k,\ldots, x_2, x_1 =x$  es un camino de $y$ a $x$.
            \vskip .4cm

            (3) 
            
            $x \sim y$ $\Rightarrow$  un camino de $x$ a $y$. 

            $y \sim z$ $\Rightarrow$  un camino de $y$ a $z$.
            
            Pegando los caminos en $y$,  obtenemos  una caminata de $x$ a $z$ (que se reduce a un camino por el lema). \qed

        \end{proof}
        
    
    \end{frame}

\end{document}

