%\documentclass{beamer} 
\documentclass[handout]{beamer} % sin pausas
\usetheme{CambridgeUS}
%\setbeamertemplate{background}[grid][step=8 ] % cuadriculado


\usepackage{etex}
\usepackage{t1enc}
\usepackage[spanish,es-nodecimaldot]{babel}
\usepackage{latexsym}
\usepackage[utf8]{inputenc}
\usepackage{verbatim}
\usepackage{multicol}
\usepackage{amsgen,amsmath,amstext,amsbsy,amsopn,amsfonts,amssymb}
\usepackage{amsthm}
\usepackage{calc}         % From LaTeX distribution
\usepackage{graphicx}     % From LaTeX distribution
\usepackage{ifthen}
%\usepackage{makeidx}
\input{random.tex}        % From CTAN/macros/generic
\usepackage{subfigure} 
\usepackage{tikz}
\usepackage[customcolors]{hf-tikz}
\usetikzlibrary{arrows}
\usetikzlibrary{matrix}
\tikzset{
	every picture/.append style={
		execute at begin picture={\deactivatequoting},
		execute at end picture={\activatequoting}
	}
}
\usetikzlibrary{decorations.pathreplacing,angles,quotes}
\usetikzlibrary{shapes.geometric}
\usepackage{mathtools}
\usepackage{stackrel}
%\usepackage{enumerate}
\usepackage{enumitem}
\usepackage{tkz-graph}
\usepackage{polynom}
\polyset{%
	style=B,
	delims={(}{)},
	div=:
}
\renewcommand\labelitemi{$\circ$}
\setlist[enumerate]{label={(\arabic*)}}

\setbeamertemplate{itemize item}{$\circ$}
\setbeamertemplate{enumerate items}[default]
\definecolor{links}{HTML}{2A1B81}
\hypersetup{colorlinks,linkcolor=,urlcolor=links}


\newcommand{\Id}{\operatorname{Id}}
\newcommand{\img}{\operatorname{Im}}
\newcommand{\nuc}{\operatorname{Nu}}
\newcommand{\im}{\operatorname{Im}}
\renewcommand\nu{\operatorname{Nu}}
\newcommand{\la}{\langle}
\newcommand{\ra}{\rangle}
\renewcommand{\t}{{\operatorname{t}}}
\renewcommand{\sin}{{\,\operatorname{sen}}}
\newcommand{\Q}{\mathbb Q}
\newcommand{\R}{\mathbb R}
\newcommand{\C}{\mathbb C}
\newcommand{\K}{\mathbb K}
\newcommand{\F}{\mathbb F}
\newcommand{\Z}{\mathbb Z}
\newcommand{\N}{\mathbb N}
\newcommand\sgn{\operatorname{sgn}}
\renewcommand{\t}{{\operatorname{t}}}
\renewcommand{\figurename }{Figura}

%
% Ver http://joshua.smcvt.edu/latex2e/_005cnewenvironment-_0026-_005crenewenvironment.html
%

\renewenvironment{block}[1]% environment name
{% begin code
	\par\vskip .2cm%
	{\color{blue}#1}%
	\vskip .2cm
}%
{%
	\vskip .2cm}% end code


\renewenvironment{alertblock}[1]% environment name
{% begin code
	\par\vskip .2cm%
	{\color{red!80!black}#1}%
	\vskip .2cm
}%
{%
	\vskip .2cm}% end code


\renewenvironment{exampleblock}[1]% environment name
{% begin code
	\par\vskip .2cm%
	{\color{blue}#1}%
	\vskip .2cm
}%
{%
	\vskip .2cm}% end code




\newenvironment{exercise}[1]% environment name
{% begin code
	\par\vspace{\baselineskip}\noindent
	\textbf{Ejercicio (#1)}\begin{itshape}%
		\par\vspace{\baselineskip}\noindent\ignorespaces
	}%
	{% end code
	\end{itshape}\ignorespacesafterend
}


\newenvironment{definicion}[1][]% environment name
{% begin code
	\par\vskip .2cm%
	{\color{blue}Definición #1}%
	\vskip .2cm
}%
{%
	\vskip .2cm}% end code

    \newenvironment{notacion}[1][]% environment name
    {% begin code
        \par\vskip .2cm%
        {\color{blue}Notación #1}%
        \vskip .2cm
    }%
    {%
        \vskip .2cm}% end code

\newenvironment{observacion}[1][]% environment name
{% begin code
	\par\vskip .2cm%
	{\color{blue}Observación #1}%
	\vskip .2cm
}%
{%
	\vskip .2cm}% end code

\newenvironment{ejemplo}[1][]% environment name
{% begin code
	\par\vskip .2cm%
	{\color{blue}Ejemplo #1}%
	\vskip .2cm
}%
{%
	\vskip .2cm}% end code


\newenvironment{preguntas}[1][]% environment name
{% begin code
    \par\vskip .2cm%
    {\color{blue}Preguntas #1}%
    \vskip .2cm
}%
{%
    \vskip .2cm}% end code

\newenvironment{ejercicio}[1][]% environment name
{% begin code
	\par\vskip .2cm%
	{\color{blue}Ejercicio #1}%
	\vskip .2cm
}%
{%
	\vskip .2cm}% end code


\renewenvironment{proof}% environment name
{% begin code
	\par\vskip .2cm%
	{\color{blue}Demostración}%
	\vskip .2cm
}%
{%
	\vskip .2cm}% end code



\newenvironment{demostracion}% environment name
{% begin code
	\par\vskip .2cm%
	{\color{blue}Demostración}%
	\vskip .2cm
}%
{%
	\vskip .2cm}% end code

\newenvironment{idea}% environment name
{% begin code
	\par\vskip .2cm%
	{\color{blue}Idea de la demostración}%
	\vskip .2cm
}%
{%
	\vskip .2cm}% end code

\newenvironment{solucion}% environment name
{% begin code
	\par\vskip .2cm%
	{\color{blue}Solución}%
	\vskip .2cm
}%
{%
	\vskip .2cm}% end code



\newenvironment{lema}[1][]% environment name
{% begin code
	\par\vskip .2cm%
	{\color{blue}Lema #1}\begin{itshape}%
		\par\vskip .2cm
	}%
	{% end code
	\end{itshape}\vskip .2cm\ignorespacesafterend
}

\newenvironment{proposicion}[1][]% environment name
{% begin code
	\par\vskip .2cm%
	{\color{blue}Proposición #1}\begin{itshape}%
		\par\vskip .2cm
	}%
	{% end code
	\end{itshape}\vskip .2cm\ignorespacesafterend
}

\newenvironment{teorema}[1][]% environment name
{% begin code
	\par\vskip .2cm%
	{\color{blue}Teorema #1}\begin{itshape}%
		\par\vskip .2cm
	}%
	{% end code
	\end{itshape}\vskip .2cm\ignorespacesafterend
}


\newenvironment{corolario}[1][]% environment name
{% begin code
	\par\vskip .2cm%
	{\color{blue}Corolario #1}\begin{itshape}%
		\par\vskip .2cm
	}%
	{% end code
	\end{itshape}\vskip .2cm\ignorespacesafterend
}

\newenvironment{propiedad}% environment name
{% begin code
	\par\vskip .2cm%
	{\color{blue}Propiedad}\begin{itshape}%
		\par\vskip .2cm
	}%
	{% end code
	\end{itshape}\vskip .2cm\ignorespacesafterend
}

\newenvironment{conclusion}% environment name
{% begin code
	\par\vskip .2cm%
	{\color{blue}Conclusión}\begin{itshape}%
		\par\vskip .2cm
	}%
	{% end code
	\end{itshape}\vskip .2cm\ignorespacesafterend
}


\newenvironment{definicion*}% environment name
{% begin code
	\par\vskip .2cm%
	{\color{blue}Definición}%
	\vskip .2cm
}%
{%
	\vskip .2cm}% end code

\newenvironment{observacion*}% environment name
{% begin code
	\par\vskip .2cm%
	{\color{blue}Observación}%
	\vskip .2cm
}%
{%
	\vskip .2cm}% end code


\newenvironment{obs*}% environment name
	{% begin code
		\par\vskip .2cm%
		{\color{blue}Observación}%
		\vskip .2cm
	}%
	{%
		\vskip .2cm}% end code

\newenvironment{ejemplo*}% environment name
{% begin code
	\par\vskip .2cm%
	{\color{blue}Ejemplo}%
	\vskip .2cm
}%
{%
	\vskip .2cm}% end code

\newenvironment{ejercicio*}% environment name
{% begin code
	\par\vskip .2cm%
	{\color{blue}Ejercicio}%
	\vskip .2cm
}%
{%
	\vskip .2cm}% end code

\newenvironment{propiedad*}% environment name
{% begin code
	\par\vskip .2cm%
	{\color{blue}Propiedad}\begin{itshape}%
		\par\vskip .2cm
	}%
	{% end code
	\end{itshape}\vskip .2cm\ignorespacesafterend
}

\newenvironment{conclusion*}% environment name
{% begin code
	\par\vskip .2cm%
	{\color{blue}Conclusión}\begin{itshape}%
		\par\vskip .2cm
	}%
	{% end code
	\end{itshape}\vskip .2cm\ignorespacesafterend
}






\newcommand{\nc}{\newcommand}

%%%%%%%%%%%%%%%%%%%%%%%%%LETRAS

\nc{\FF}{{\mathbb F}} \nc{\NN}{{\mathbb N}} \nc{\QQ}{{\mathbb Q}}
\nc{\PP}{{\mathbb P}} \nc{\DD}{{\mathbb D}} \nc{\Sn}{{\mathbb S}}
\nc{\uno}{\mathbb{1}} \nc{\BB}{{\mathbb B}} \nc{\An}{{\mathbb A}}

\nc{\ba}{\mathbf{a}} \nc{\bb}{\mathbf{b}} \nc{\bt}{\mathbf{t}}
\nc{\bB}{\mathbf{B}}

\nc{\cP}{\mathcal{P}} \nc{\cU}{\mathcal{U}} \nc{\cX}{\mathcal{X}}
\nc{\cE}{\mathcal{E}} \nc{\cS}{\mathcal{S}} \nc{\cA}{\mathcal{A}}
\nc{\cC}{\mathcal{C}} \nc{\cO}{\mathcal{O}} \nc{\cQ}{\mathcal{Q}}
\nc{\cB}{\mathcal{B}} \nc{\cJ}{\mathcal{J}} \nc{\cI}{\mathcal{I}}
\nc{\cM}{\mathcal{M}} \nc{\cK}{\mathcal{K}}

\nc{\fD}{\mathfrak{D}} \nc{\fI}{\mathfrak{I}} \nc{\fJ}{\mathfrak{J}}
\nc{\fS}{\mathfrak{S}} \nc{\gA}{\mathfrak{A}}
%%%%%%%%%%%%%%%%%%%%%%%%%LETRAS


\title[Clase 8 - Conteo 3]{Matemática Discreta I \\ Clase 8 - Conteo ($3^\circ$ parte)}
%\author[C. Olmos / A. Tiraboschi]{Carlos Olmos / Alejandro Tiraboschi}
\institute[]{\normalsize FAMAF / UNC
	\\[\baselineskip] ${}^{}$
	\\[\baselineskip]
}
\date[11/04/2023]{11 de abril  de 2023}



\begin{document}
	
	\frame{\titlepage} 
	
	
	
	\begin{frame}\frametitle{Selecciones sin orden}
		
		$X$ finito de $n$ elementos. 
		\begin{center}
			{\it ¿Cuántos subconjuntos de $m$ elementos hay en $X$?}
		\end{center} \pause
		
		\vskip .4cm
		{\color{blue}Ejemplo}
		\vskip .2cm
		%\begin{ejemplo}
		Por ejemplo, sea $\mathbb I_5 = \{ 1, 2, 3, 4, 5 \}$ y nos interesan los
		subconjuntos de tres ele\-men\-tos. ¿Cuántos habrá? 
		
		\vskip .2cm \pause
		
		{\color{blue}Solución.} \pause
		\vskip .2cm
		Contemos directamente: 
		
		$\{1,2,3\}$, $\{1,2,4\}$, $\{1,2,5\}$, $\{1,3,4\}$, $\{1,3,5\}$, $\{1,4,5\}$,
		
		$\{2,3,4\}$, $\{2,3,5\}$, $\{2,4,5\}$,
		
		$\{3,4,5\}$.
		\vskip .2cm \pause
		La respuesta, entonces, es 10. 
		
	\end{frame}
	
	
	
	
	
	
	
	\begin{frame}	
		
		¿Cómo podemos calcular este número sin contar caso por caso?
		\vskip .2cm \pause
		
		Una forma de individualizar un subconjunto de tres elementos en $\mathbb I_5$, consiste
		en, primero, seleccionar  ordenadamente $3$ elementos.
		
		\vskip .2cm \pause
		
		Habría, a priori, $5 \cdot 4 \cdot 3$ subconjuntos pues ese es el número
		de selecciones ordenadas y sin repetición de $3$ elementos de $\mathbb I_5$. 
		
		\vskip .2cm
		Es claro que algunas de las selecciones ordenadas pueden determinar el mismo subconjunto. En efecto, por ejemplo, cualesquiera de las selecciones
		\begin{align*}
			1 2 3, \qquad  1 3 2, \qquad  2 1 3, \qquad 2 3 1, \qquad  3 1 2, \qquad  3 2 1
		\end{align*}
		determina el subconjunto $\{ 1, 2, 3\}$. 
		
		\vskip .2cm \pause
		
		Es decir las permutaciones de $\{ 1, 2, 3\}$ determinan el mismo subconjunto.  
		
		\vskip .2cm 
		
		Veamos todos los casos: 
		
		
	\end{frame}
	
	
	
	
	
	\begin{frame}	
		
		\pause
		\begin{align*}
			&1 2 3 \qquad  1 3 2  \qquad  2 1 3  \qquad 2 3 1  \qquad  3 1 2  \qquad  3 2 1 \qquad	&\to \qquad &\{1,2,3\}\\
			&1 2 4  \qquad  1 4 2  \qquad  2 1 4  \qquad 2 4 1  \qquad  4 1 2  \qquad  4 2 1  \qquad	&\to \qquad &\{1,2,4\}\\
			&1 2 5  \qquad  1 5 2  \qquad  2 1 5  \qquad 2 5 1  \qquad  3 1 2  \qquad  5 2 1  \qquad	&\to \qquad &\{1,2,5\}\\
			&1 3 4  \qquad  1 43  \qquad  3 1 4  \qquad 34 1  \qquad  4 1 3  \qquad  431  \qquad	&\to \qquad &\{1,3,4\}\\
			&1 3 5  \qquad  1 53  \qquad  3 1 5  \qquad 35 1  \qquad  5 1 3  \qquad  531  \qquad	&\to \qquad &\{1,3,5\}\\
			&1 45  \qquad  1 54  \qquad  4 1 5  \qquad 45 1  \qquad  5 1 4  \qquad  541  \qquad	&\to \qquad &\{1,4,5\}\\
			&234  \qquad  243  \qquad  324  \qquad 342  \qquad  423  \qquad  432  \qquad	&\to \qquad &\{2,3,4\}\\
			&235  \qquad  253  \qquad  325  \qquad 352  \qquad  523  \qquad  532  \qquad	&\to \qquad &\{2,3,5\}\\
			&245  \qquad  254  \qquad  425  \qquad 452  \qquad  524  \qquad  542  \qquad	&\to \qquad &\{2,4,5\}\\
			&345  \qquad  354  \qquad  435 \qquad 453  \qquad  534  \qquad  543  \qquad	&\to \qquad &\{3,4,5\}
		\end{align*}	
		
	\end{frame}
	
	
	
	\begin{frame}
		
		Es decir  $3! = 6$ selecciones ordenadas nos determinan un subconjunto de tres elementos. Por lo tanto, el número total de
		subconjuntos de $3$ elementos es
		
		$$
		\frac{5 \cdot 4 \cdot 3}{3!} =  \frac{5!}{3! (5 - 3)!}
		$$
		\vskip .4cm \pause
		
		Podemos  hacer el mismo razonamiento, si tuviéramos que elegir subconjuntos de 2 elementos:
		
		\vskip .4cm \pause
		
		\begin{itemize}
			\item Tenemos $5 \cdot 4 = \displaystyle \frac{5!}{3!}$  selecciones ordenadas de 2 elementos.
			\item La cantidad de permutaciones de dos elementos es $2!$, luego,
			\item hay  $\displaystyle \frac{5!}{3!2!} = 10$ subconjuntos de 2 elementos. 
		\end{itemize}
		
		
		
	\end{frame}
	
	\begin{frame}
		En el caso general de subconjuntos de $m$ elementos de un
		conjunto de $n$ elementos ($m \le n$) podemos razonar en forma análoga. 
		
		\vskip .4cm \pause
		
		\begin{itemize}
			\item Tenemos $\displaystyle \frac{n!}{(n-m)!}$  selecciones ordenadas de $m$ elementos entre $n$. \pause
			\item La cantidad de permutaciones de $m$ elementos es $m!$, luego, \pause
			\item hay  $\displaystyle \frac{n!}{(n-m)!m!}$ subconjuntos de $m$ elementos. 
		\end{itemize}
		
		
		
		\vskip .4cm \pause
		
		
		
		{\color{blue}Teorema}
		\vskip .2cm
		{\it Sea $X$ un conjunto de $n$ elementos. Entonces   el número total de subconjuntos de $m$ elementos
			de $X$ es
			$$
			\frac{n!}{(n - m)!\; m!}
			$$}
		
		\vskip .4cm
	\end{frame}
	
	
	\begin{frame}	\frametitle{Número combinatorio}
		
		Sean $n, m \in \mathbb N_0$, $m \le n$. Definimos
		$$
		\binom{n}{m} = \frac{n!}{(n - m)! \; m!}
		$$
		el  {\em número combinatorio} asociado al par $n$, $m$  o {\em número combinatorio $n$, $m$}.
		
		\vskip .4cm \pause
		Por razones que se verán más adelante se denomina también el {\em coeficiente binomial} asociado al par $n$, $m$ con $m \le n$.
		
		\vskip .4cm  \pause
		Definimos también
		$$
		\binom{n}{m} = 0,\qquad \text{ si } m > n.
		$$
		
	\end{frame}
	
	
	
	\begin{frame}
		Hay unos pocos números combinatorios que son fácilmente calculables: 
		$$
		\binom{n}{0} = \binom{0}{0} = 1 \qquad \text{ y }\qquad  \binom{n}{1} = \binom{n}{n-1} = n. 
		$$
		Estos resultados se obtienen por aplicación directa de la definición (recordar que  $0! =1$). 
		
		
		\vskip .4cm  \pause
		
		Con el número combinatorio podemos reescribir el resultado visto más arriba:
		
		\vskip .4cm 
		{\color{blue}Teorema}
		\vskip .2cm
		{\it Sean $n, m \in \mathbb N_0$, $m \le n$, y supongamos que el conjunto $X$ tiene $n$ elementos.
			Entonces la cantidad de subconjuntos de $X$ con $m$ elementos es
			$
			\displaystyle\binom{n}{m}
			$.}
		
	\end{frame}
	
	
	\begin{frame}
		
		
		{\color{blue}Ejemplo}
		\vskip .3cm
		¿Cuántos comités pueden formarse de un conjunto de $6$ mujeres y $4$ hombres, si el comité debe estar compuesto por $4$ mujeres y $2$ hombres?
		\vskip .3cm \pause
		{\color{blue}Solución} \pause
		\vskip .2cm Debemos elegir $4$ mujeres entre $6$, y la cantidad de elecciones posibles es   $\displaystyle\binom{6}{4}$. Por otro lado, hay $ \displaystyle\binom{4}{2}$ formas de elegir $2$ hombres entre $4$. Luego, por el principio de multiplicación,  el resultado es
		\begin{align*}
            \binom{6}{4}\cdot \binom{4}{2} &= \frac{6!}{(6-4)!4!}\cdot\frac{4!}{(4-2)!2!}= \frac{6!}{2!4!}\cdot\frac{4!}{2!2!} = \frac{6\cdot 5}{2}\cdot\frac{4\cdot 3}{2} \\[0.3cm] & = 15 \cdot 6= 90.
        \end{align*}
        \qed
		
	
		\vskip .3cm
	\end{frame}
	
	
	\begin{frame}\frametitle{Simetría del número combinatorio}
		
		
		{\color{blue} Proposición}
		\vskip .2cm
		{\it	Sean $m,n \in \mathbb N_0$, tal que $m \le n$. Entonces
			$$
			\binom{n}{m} = \binom{n}{n-m}.
			$$} \pause  
		{\color{blue} Demostración.} \pause  
		\begin{align*}
			\binom{n}{n-m} &= \frac{n!}{(n-(n-m))!\,(n-m)!} \\
			&=  \frac{n!}{m!\,(n-m)!} \\
			&=    \frac{n!}{(n-m)!\,m!}=  \binom{n}{m}.
		\end{align*}
		\qed
		
		
		
	\end{frame}
	
	
	\begin{frame}	
		
		El hecho  de que 
		$$
		\binom{n}{m} = \binom{n}{n-m}.
		$$
		se puede interpretar en términos de conteo:  $\displaystyle\binom{n}{m}$ es el número de subconjuntos de $m$ ele\-men\-tos de un conjunto de $n$ elementos. 
		\vskip .2cm\pause  
		
		Puesto que con cada subconjunto de $m$ ele\-men\-tos hay unívocamente asociado un subconjunto de $n - m$ elementos, su complemento en $X$, es claro que $$\displaystyle\binom{n}{m} = \binom{n}{n-m}.$$
		
		
	\end{frame}
	
	
	\begin{frame}\frametitle{Triángulo de Pascal}
		
		
		{\color{blue} Teorema}
		\vskip .2cm
		{\it
			Sean $m,n \in \mathbb N$, tal que $m \le n$. Entonces
			$$
			\binom{n}{m-1} + \binom{n}{m}   = \binom{n+1}{m} 
			$$
		}\pause  
		
		{\color{blue} Demostración.}\pause  
		\vskip .2cm
		
		El enunciado nos dice que debemos demostrar que 
		\begin{equation*}
			\frac{n!}{(n-m+1)!\,(m-1)!} +  \frac{n!}{(n-m)!\,m!} = \frac{(n+1)!}{(n-m+1)!\,m!}
		\end{equation*}
		
		y lo dejamos como ejercicio.\qed
		
	\end{frame}
	
	\begin{frame}
		
		Veamos unos pocos ejemplos de la fórmula $\displaystyle\binom{n}{m-1} + \binom{n}{m}   = \binom{n+1}{m}$ o  equivalentemente (dando vuelta la igualdad).
		$$
		\binom{n+1}{m} = \binom{n}{m-1} + \binom{n}{m} .
		$$
		\vskip .3cm \pause  

        \begin{block}{Ejemplos}
        \begin{align*}
            \binom{4}{2} &= \binom{3}{1} + \binom{3}{2} = 3+3 =6. \\[.5cm]
            \binom{5}{2} &= \binom{4}{1} + \binom{4}{2} = 4 + 6 = 10, \\[.3cm]
            \binom{5}{3} &= \binom{4}{2} + \binom{4}{3} = 6 + 4 = 10.
        \end{align*}
    \end{block}
		
	\end{frame}
	
	\begin{frame}
		El teorema precedente permite calcular los coeficientes binomiales:
		\pause  
		\begin{align*}
			&& && && && && &\binom{0}{0}& && && && && &&  \\
			&& && && && &\binom{1}{0}& && &\binom{1}{1}& && && && &&  \\
			&& && && &\binom{2}{0}& && &\binom{2}{1}& && &\binom{2}{2}& && && &&  \\
			&& && &\binom{3}{0}& && &\binom{3}{1}& && &\binom{3}{2}& && &\binom{3}{3}& && &&  \\
			&& &\binom{4}{0}& && &\binom{4}{1}& && &\binom{4}{2}& && &\binom{4}{3}& && &\binom{4}{4}& && 
		\end{align*}\pause  
		Cada término interior es suma de los dos términos inmediatos superiores. 
		
		Los elementos en los lados valen 1.
		
	\end{frame}
	
	
	\begin{frame}
		Escribamos los valores de cada fila:\pause  
		\begin{align*}
			&& && && && && &1& && && && && &&  \\
			&& && && && &1& && &1& && && && &&  \\
			&& && && &1& && &2& && &1& && && &&  \\
			&& && &1& && &3& && &3& && &1& && &&  \\
			&& &1& && &4& && &6& && &4& && &1& &&  
		\end{align*}
		
		\vskip .2cm
		El  triángulo es denominado \emph{triángulo de Pascal}. 
		\vskip .2cm\pause  
		Entre las propiedades que cuenta el triángulo de Pascal está la de ser simétrico respecto al eje vertical central, como consecuencia de la simetría de los números combinatorios. 
	\end{frame}
	
	
	
\end{document}

