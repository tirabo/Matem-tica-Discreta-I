%\documentclass{beamer} 
\documentclass[handout]{beamer} % sin pausas
\usetheme{CambridgeUS}

\usepackage{etex}
\usepackage{t1enc}
\usepackage[spanish,es-nodecimaldot]{babel}
\usepackage{latexsym}
\usepackage[utf8]{inputenc}
\usepackage{verbatim}
\usepackage{multicol}
\usepackage{amsgen,amsmath,amstext,amsbsy,amsopn,amsfonts,amssymb}
\usepackage{amsthm}
\usepackage{calc}         % From LaTeX distribution
\usepackage{graphicx}     % From LaTeX distribution
\usepackage{ifthen}
%\usepackage{makeidx}
\input{random.tex}        % From CTAN/macros/generic
\usepackage{subfigure} 
\usepackage{tikz}
\usepackage[customcolors]{hf-tikz}
\usetikzlibrary{arrows}
\usetikzlibrary{matrix}
\tikzset{
	every picture/.append style={
		execute at begin picture={\deactivatequoting},
		execute at end picture={\activatequoting}
	}
}
\usetikzlibrary{decorations.pathreplacing,angles,quotes}
\usetikzlibrary{shapes.geometric}
\usepackage{mathtools}
\usepackage{stackrel}
%\usepackage{enumerate}
\usepackage{enumitem}
\usepackage{tkz-graph}
\usepackage{polynom}
\polyset{%
	style=B,
	delims={(}{)},
	div=:
}
\renewcommand\labelitemi{$\circ$}
\setlist[enumerate]{label={(\arabic*)}}
%\setbeamertemplate{background}[grid][step=8 ] % cuadriculado
\setbeamertemplate{itemize item}{$\circ$}
\setbeamertemplate{enumerate items}[default]
\definecolor{links}{HTML}{2A1B81}
\hypersetup{colorlinks,linkcolor=,urlcolor=links}


\newcommand{\Id}{\operatorname{Id}}
\newcommand{\img}{\operatorname{Im}}
\newcommand{\nuc}{\operatorname{Nu}}
\newcommand{\im}{\operatorname{Im}}
\renewcommand\nu{\operatorname{Nu}}
\newcommand{\la}{\langle}
\newcommand{\ra}{\rangle}
\renewcommand{\t}{{\operatorname{t}}}
\renewcommand{\sin}{{\,\operatorname{sen}}}
\newcommand{\Q}{\mathbb Q}
\newcommand{\R}{\mathbb R}
\newcommand{\C}{\mathbb C}
\newcommand{\K}{\mathbb K}
\newcommand{\F}{\mathbb F}
\newcommand{\Z}{\mathbb Z}
\newcommand{\N}{\mathbb N}
\newcommand\sgn{\operatorname{sgn}}
\renewcommand{\t}{{\operatorname{t}}}
\renewcommand{\figurename }{Figura}

%
% Ver http://joshua.smcvt.edu/latex2e/_005cnewenvironment-_0026-_005crenewenvironment.html
%

\renewenvironment{block}[1]% environment name
{% begin code
	\par\vskip .2cm%
	{\color{blue}#1}%
	\vskip .2cm
}%
{%
	\vskip .2cm}% end code


\renewenvironment{alertblock}[1]% environment name
{% begin code
	\par\vskip .2cm%
	{\color{red!80!black}#1}%
	\vskip .2cm
}%
{%
	\vskip .2cm}% end code


\renewenvironment{exampleblock}[1]% environment name
{% begin code
	\par\vskip .2cm%
	{\color{blue}#1}%
	\vskip .2cm
}%
{%
	\vskip .2cm}% end code




\newenvironment{exercise}[1]% environment name
{% begin code
	\par\vspace{\baselineskip}\noindent
	\textbf{Ejercicio (#1)}\begin{itshape}%
		\par\vspace{\baselineskip}\noindent\ignorespaces
	}%
	{% end code
	\end{itshape}\ignorespacesafterend
}


\newenvironment{definicion}[1][]% environment name
{% begin code
	\par\vskip .2cm%
	{\color{blue}Definición #1}%
	\vskip .2cm
}%
{%
	\vskip .2cm}% end code

    \newenvironment{notacion}[1][]% environment name
    {% begin code
        \par\vskip .2cm%
        {\color{blue}Notación #1}%
        \vskip .2cm
    }%
    {%
        \vskip .2cm}% end code

\newenvironment{observacion}[1][]% environment name
{% begin code
	\par\vskip .2cm%
	{\color{blue}Observación #1}%
	\vskip .2cm
}%
{%
	\vskip .2cm}% end code

\newenvironment{ejemplo}[1][]% environment name
{% begin code
	\par\vskip .2cm%
	{\color{blue}Ejemplo #1}%
	\vskip .2cm
}%
{%
	\vskip .2cm}% end code


\newenvironment{preguntas}[1][]% environment name
{% begin code
    \par\vskip .2cm%
    {\color{blue}Preguntas #1}%
    \vskip .2cm
}%
{%
    \vskip .2cm}% end code

\newenvironment{ejercicio}[1][]% environment name
{% begin code
	\par\vskip .2cm%
	{\color{blue}Ejercicio #1}%
	\vskip .2cm
}%
{%
	\vskip .2cm}% end code


\renewenvironment{proof}% environment name
{% begin code
	\par\vskip .2cm%
	{\color{blue}Demostración}%
	\vskip .2cm
}%
{%
	\vskip .2cm}% end code



\newenvironment{demostracion}% environment name
{% begin code
	\par\vskip .2cm%
	{\color{blue}Demostración}%
	\vskip .2cm
}%
{%
	\vskip .2cm}% end code

\newenvironment{idea}% environment name
{% begin code
	\par\vskip .2cm%
	{\color{blue}Idea de la demostración}%
	\vskip .2cm
}%
{%
	\vskip .2cm}% end code

\newenvironment{solucion}% environment name
{% begin code
	\par\vskip .2cm%
	{\color{blue}Solución}%
	\vskip .2cm
}%
{%
	\vskip .2cm}% end code



\newenvironment{lema}[1][]% environment name
{% begin code
	\par\vskip .2cm%
	{\color{blue}Lema #1}\begin{itshape}%
		\par\vskip .2cm
	}%
	{% end code
	\end{itshape}\vskip .2cm\ignorespacesafterend
}

\newenvironment{proposicion}[1][]% environment name
{% begin code
	\par\vskip .2cm%
	{\color{blue}Proposición #1}\begin{itshape}%
		\par\vskip .2cm
	}%
	{% end code
	\end{itshape}\vskip .2cm\ignorespacesafterend
}

\newenvironment{teorema}[1][]% environment name
{% begin code
	\par\vskip .2cm%
	{\color{blue}Teorema #1}\begin{itshape}%
		\par\vskip .2cm
	}%
	{% end code
	\end{itshape}\vskip .2cm\ignorespacesafterend
}


\newenvironment{corolario}[1][]% environment name
{% begin code
	\par\vskip .2cm%
	{\color{blue}Corolario #1}\begin{itshape}%
		\par\vskip .2cm
	}%
	{% end code
	\end{itshape}\vskip .2cm\ignorespacesafterend
}

\newenvironment{propiedad}% environment name
{% begin code
	\par\vskip .2cm%
	{\color{blue}Propiedad}\begin{itshape}%
		\par\vskip .2cm
	}%
	{% end code
	\end{itshape}\vskip .2cm\ignorespacesafterend
}

\newenvironment{conclusion}% environment name
{% begin code
	\par\vskip .2cm%
	{\color{blue}Conclusión}\begin{itshape}%
		\par\vskip .2cm
	}%
	{% end code
	\end{itshape}\vskip .2cm\ignorespacesafterend
}


\newenvironment{definicion*}% environment name
{% begin code
	\par\vskip .2cm%
	{\color{blue}Definición}%
	\vskip .2cm
}%
{%
	\vskip .2cm}% end code

\newenvironment{observacion*}% environment name
{% begin code
	\par\vskip .2cm%
	{\color{blue}Observación}%
	\vskip .2cm
}%
{%
	\vskip .2cm}% end code


\newenvironment{obs*}% environment name
	{% begin code
		\par\vskip .2cm%
		{\color{blue}Observación}%
		\vskip .2cm
	}%
	{%
		\vskip .2cm}% end code

\newenvironment{ejemplo*}% environment name
{% begin code
	\par\vskip .2cm%
	{\color{blue}Ejemplo}%
	\vskip .2cm
}%
{%
	\vskip .2cm}% end code

\newenvironment{ejercicio*}% environment name
{% begin code
	\par\vskip .2cm%
	{\color{blue}Ejercicio}%
	\vskip .2cm
}%
{%
	\vskip .2cm}% end code

\newenvironment{propiedad*}% environment name
{% begin code
	\par\vskip .2cm%
	{\color{blue}Propiedad}\begin{itshape}%
		\par\vskip .2cm
	}%
	{% end code
	\end{itshape}\vskip .2cm\ignorespacesafterend
}

\newenvironment{conclusion*}% environment name
{% begin code
	\par\vskip .2cm%
	{\color{blue}Conclusión}\begin{itshape}%
		\par\vskip .2cm
	}%
	{% end code
	\end{itshape}\vskip .2cm\ignorespacesafterend
}






\newcommand{\nc}{\newcommand}

%%%%%%%%%%%%%%%%%%%%%%%%%LETRAS

\nc{\FF}{{\mathbb F}} \nc{\NN}{{\mathbb N}} \nc{\QQ}{{\mathbb Q}}
\nc{\PP}{{\mathbb P}} \nc{\DD}{{\mathbb D}} \nc{\Sn}{{\mathbb S}}
\nc{\uno}{\mathbb{1}} \nc{\BB}{{\mathbb B}} \nc{\An}{{\mathbb A}}

\nc{\ba}{\mathbf{a}} \nc{\bb}{\mathbf{b}} \nc{\bt}{\mathbf{t}}
\nc{\bB}{\mathbf{B}}

\nc{\cP}{\mathcal{P}} \nc{\cU}{\mathcal{U}} \nc{\cX}{\mathcal{X}}
\nc{\cE}{\mathcal{E}} \nc{\cS}{\mathcal{S}} \nc{\cA}{\mathcal{A}}
\nc{\cC}{\mathcal{C}} \nc{\cO}{\mathcal{O}} \nc{\cQ}{\mathcal{Q}}
\nc{\cB}{\mathcal{B}} \nc{\cJ}{\mathcal{J}} \nc{\cI}{\mathcal{I}}
\nc{\cM}{\mathcal{M}} \nc{\cK}{\mathcal{K}}

\nc{\fD}{\mathfrak{D}} \nc{\fI}{\mathfrak{I}} \nc{\fJ}{\mathfrak{J}}
\nc{\fS}{\mathfrak{S}} \nc{\gA}{\mathfrak{A}}
%%%%%%%%%%%%%%%%%%%%%%%%%LETRAS


\title[Clase 1 - Enteros]{Matemática Discreta I \\ Clase 1 - Los números enteros}
%\author[A. Tiraboschi]{Alejandro Tiraboschi}
\institute[]{\normalsize FAMAF / UNC
	\\[\baselineskip] ${}^{}$
	\\[\baselineskip]
}
\date[15/03/2022]{15 de marzo  de 2022}



\begin{document}
%\title{El centro geográfico de Argentina}   
%\author{} 
%\date{Villa Huidobro \\ 4/12/2018} 



\frame{\titlepage} 

%\frame{\frametitle{Índice}\tableofcontents} 


\begin{frame}\frametitle{Axiomas de los números enteros} 

    Todos conocemos los \textit{enteros}. \pause

    \vskip .3cm
    
    Primero, los  ``números naturales'' $$1,2,3,4,5,\ldots$$ \pause

    \vskip .3cm
    
    Más adelante introducimos el $0$ (cero).\pause
    
    \vskip .3cm
    
    Luego,  los enteros negativos $$ -1,-2,-3,-4,-5,\ldots $$ \pause
    
    \begin{itemize}
        \item En este curso no nos preocupamos demasiado por el significado lógico y filosófico de estos objetos, pero necesitamos saber las propiedades que se supone que tienen. 
    
    \end{itemize}
    
    
\end{frame}



\begin{frame}\frametitle{ }  
	
    \begin{itemize}
        \item La idea es: si todos parten de las mismas suposiciones entonces todos llegarán a los mismos resultados. \pause
        \vskip .4cm 
        \item Estos supuestos son los llamados \textit{axiomas.}\pause
        \vskip .4cm 
        \item  Aceptamos sin reparo que existe un conjunto de objetos llamados \textit{enteros}.\pause
        \vskip .4cm 
        \item  El conjunto de enteros se denotará por el símbolo especial ${\mathbb Z}$.\pause
        \vskip .4cm 
        \item Las propiedades de ${\mathbb Z}$ serán dadas por una lista de axiomas.\pause
        \vskip .4cm 
        \item  Un  resultado acerca de $\Z$ es válido  si se deduce lógicamente de los axiomas. 
    \end{itemize}
    


    

\end{frame}



\begin{frame}\frametitle{ }  
	

     Empezaremos listando aquellos axiomas que tratan la suma y la multiplicación.\pause
     \vskip .6cm
    \begin{notacion}
        Sean $a, b \in \Z$ (números enteros)\pause
        \begin{itemize}
            \item $a + b$  denota la suma.\pause
            \item  $a \cdot b$ ( o $ab$ o también $a \times b$) el producto de enteros.\pause
        \end{itemize}

    \end{notacion}

    \vskip .6cm
    
     El hecho de que $a \cdot b$ y $a+b$ son enteros, es nuestro primer axioma.     
	
\end{frame}



\begin{frame}\frametitle{ }  

    En la siguiente lista de axiomas $a$, $b$, $c$ denotan enteros arbitrarios, y $0$ y $1$ denotan enteros especiales que cumplen las propiedades especificadas más abajo.\pause
\vskip .4cm
    \begin{enumerate}[label=\textbf{I\arabic*)}, ref=\textbf{I\arabic*}]\pause
    \item \label{axioma-i1} $a+b$ y $a\cdot b$ pertenecen a ${\mathbb Z}$.\pause
    \item \label{axioma-i2} {\em Conmutatividad.}\, $a+b = b+a$;\qquad $ab=ba$. \pause
    \item \label{axioma-i3} {\em Asociatividad.}\, $(a+b)+c = a+(b+c)$;\qquad $(a\cdot b)\cdot c = a\cdot (b\cdot c)$. \pause
    \item \label{axioma-i4} {\em Existencia de elemento neutro.}\, Existen números $0$, $1 \in \mathbb Z$ con $0\not=1$ tal que $a+0=a$;\qquad $a\cdot 1=a$. \pause
    \end{enumerate}

    Los axiomas anteriores involucran a la suma y al producto por separado. 


    
\end{frame}


\begin{frame}
    El axioma siguiente  relaciona el suma y el producto. \pause
    \vskip .4cm
	\begin{enumerate}
        \item[\textbf{I5)}] \label{axioma-i5} {\em Distributividad.}\, $a\cdot (b+c)=a\cdot b+a\cdot c$. 
        \end{enumerate}
        \vskip .8cm\pause
        También tenemos propiedades de cancelación. 
        \vskip .4cm
    	\begin{enumerate}
            \item[\textbf{I6)}] \label{axioma-i6} {\em Existencia del inverso aditivo, también llamado opuesto.}\, Por cada $a$ en ${\mathbb Z}$ existe un único entero $-a$ en ${\mathbb Z}$ tal que $a+(-a)=0$. \pause
            \item[\textbf{I7)}] \label{axioma-i7} {\em Cancelación.}\, Si $a$ es distinto de $0$ y $a\cdot b=a\cdot c$, entonces $b=c$. 
        \end{enumerate}
	 
	
\end{frame}

\begin{frame}

    \begin{itemize}
        \item Debido a la ley de asociatividad para la suma axioma (\textbf{I3}) $(a+b)+c$ es igual a $a+(b+c)$ y por lo tanto podemos eliminar los paréntesis sin ambigüedad. Es decir, denotamos
        $$
        a+b+c := (a+b)+c = a+(b+c).
        $$\pause
        \vskip .2cm
        \item De forma análoga, usaremos la notación
        $$
        abc = (ab)c = a(bc).
        $$\pause
        \vskip .2cm
        \item Debido a la ley de conmutatividad axioma (\textbf{I2}), es claro que  del axioma (\textbf{I4}) se deduce que  $0+a=a+0=a$ y $1\cdot a = a\cdot 1=a$.
        \vskip .2cm\pause
        \item Análogamente,  por  (\textbf{I2}) e  (\textbf{I6}) obtenemos que  $-a+a =    a+(-a)=0$.
    \end{itemize}

\end{frame}

\begin{frame}
    
    Una propiedad que debemos mencionar es la siguiente: 
    
    \begin{center}
        \textit{Si $a,b, c \in \mathbb Z$  y $a=b$, entonces $a+c = b+c$ y $ac = bc$.}    
    \end{center}
    \vskip .2cm    \vskip .2cm
    \pause
    
    Esto se debe a que:
    \vskip .2cm
    \begin{itemize}
        \item la suma y el producto son operaciones que devuelven enteros.\pause
         \vskip .2cm
        \item Si $a=b$, entonces el  par $a,c$ es igual al par $b,c$ y por lo tanto devuelven la misma suma y el mismo producto.\pause
    \end{itemize}
    \vskip .6cm\pause
    Esta propiedad \textit{no es un axioma,} sino  una mera aplicación de la lógica formal. 
	
\end{frame}



\begin{frame}\begin{ejemplo}\label{Ej.opuesto_opuesto} Demostremos que, para todo $n$ entero, el opuesto de $-n$ es $n$, es decir que 
    $$-(-n) = n.$$ 
    \end{ejemplo}\pause
    \begin{proof}\pause El axioma (\textbf{I6}) nos dice que $-(-n)$ es el único número que sumado a $-n$, da cero.  Por lo tanto, para demostrar que $-(-n) = n$ basta ver que $(-n)+n=0$. \pause Esto se cumple puesto que 
    \begin{alignat*}2
    (-n)+n&=n+(-n) &\qquad &\text{axioma (\textbf{I2})} \\
    &=0&\qquad &\text{axioma (\textbf{I6})}
    \end{alignat*}\pause
    Por lo tanto  $(-n)+n=0$. \qed
    \end{proof}
	
\end{frame}


\begin{frame} 
    
    A continuación definimos la resta o sustracción. \pause

    \begin{definicion} Si $a,b\in\mathbb{Z}$ definimos $a-b$ como la suma de $a$ más el opuesto de $b$, es decir que  $a-b=a+(-b)$ por definición.  
    \end{definicion}
    
    
    
	
\end{frame}


\begin{frame}
    
    Ahora demostremos una propiedad básica de la resta.\pause
    
    \begin{ejemplo*} Sean  $m$ y $n$ enteros, entonces
    $$m-(-n) = m+n.$$ 
    \end{ejemplo*}\pause
    \begin{proof}\pause Por la definición de sustracción, $m-(-n)$ es la suma $m+(-(-n))$, es decir 
    $$m-(-n)=m+(-(-n)).$$ 
    Por el ejemplo de la p. \ref{Ej.opuesto_opuesto} sabemos que $-(-n)=n$ y por lo tanto $m-(-n)=m+(-(-n))=m+n$.\qed
    \end{proof}


\end{frame}



\begin{frame}
    
    \begin{ejemplo*} Supongamos que existen dos enteros $0$ y $0'$ ambos cumpliendo el  axioma (\textbf{I4}), esto es
        $$
        a+0= a, \qquad a+0'=a
        $$
        para todo $a$ de $\mathbb Z$.  Entonces $0= 0'$. 
        \end{ejemplo*}\pause
        \begin{proof}\pause
            \vskip -.6cm
        \begin{alignat*}2
        0 &= 0 + 0'&\qquad &\text{axioma (\textbf{I4}) aplicado a $0$ y con $0'$ como neutro} \\
        &=0'+0&\qquad &\text{axioma (\textbf{I2})}\\
        &= 0' &\qquad &\text{axioma (\textbf{I4}) aplicado a $0'$ y con $0$ como neutro}.
        \end{alignat*} \qed
        \end{proof}
        \pause
        \vskip .1cm

    \begin{observacion}
        Vale el resultado análogo  para el producto: el  elemento neutro del producto es único. 
    \end{observacion}
	
\end{frame}


\begin{frame}
    
    \begin{ejemplo*}\label{a0=0} Sea $a \in \Z$,  entonces 
        $$
        a \cdot 0 = 0
        $$
    \end{ejemplo*}\pause
    \begin{proof}\pause
        \vskip -.8cm
    \begin{alignat*}2
    a \cdot 0  &= a \cdot (0 + 0)  &\qquad &\text{axioma (\textbf{I4})} \\
    a \cdot 0  &= a \cdot 0 + a \cdot 0  &\qquad &\text{axioma (\textbf{I5})} \\
    a \cdot 0 -  a \cdot 0&= a \cdot 0 + a \cdot 0 - a \cdot 0 &\qquad &\text{lógica} \\
     0  &= a \cdot 0 +  0  &\qquad &\text{2 veces axioma (\textbf{I6})} \\
      0  &= a \cdot 0   &\qquad &\text{axioma (\textbf{I4}).}
    \end{alignat*}
    \qed
    \end{proof}
\end{frame}

 


\begin{frame}
    \begin{ejemplo*} (Regla de los signos) Veamos que  si $a,b \in \mathbb Z$ entonces
        $$
        (1)\quad (-a)(-b) = ab ,\qquad (2)\quad a(-b) = (-a)b = -(ab).
        $$
        \end{ejemplo*}\pause
        \begin{proof}\pause
        (2)  Probaremos $a(-b)= -(ab)$.  
        \vskip .2cm
        Para ello, veremos que $a(-b)$ es el inverso aditivo de $ab$. 
        \vskip .2cm
        Por unicidad del inverso aditivo (axioma \textbf{I6}), de deduce que $a(-b) = -(ab)$.
        
        \end{proof}
    \end{frame}

    \begin{frame}
        
        \begin{alignat*}2
        ab + a(-b) &=a(b-b) &\qquad &\text{axioma (\textbf{I5})} \\
        &=a\cdot 0 &\qquad &\text{axioma (\textbf{I6})}\\
        &= 0 &\qquad &\text{ejercicio p. \ref{a0=0}}.
        \end{alignat*}
        
        Es complétamente análogo probar $(-a)b = -(ab)$.

        \vskip .4cm

        (1) Ejercicio. 

        
        \qed 
    \vskip 3cm
        
    
    \end{frame}

\end{document}

