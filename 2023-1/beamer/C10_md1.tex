%\documentclass{beamer} 
\documentclass[handout]{beamer} % sin pausas
\usetheme{CambridgeUS}
%\setbeamertemplate{background}[grid][step=8 ] % cuadriculado

\usepackage{etex}
\usepackage{t1enc}
\usepackage[spanish,es-nodecimaldot]{babel}
\usepackage{latexsym}
\usepackage[utf8]{inputenc}
\usepackage{verbatim}
\usepackage{multicol}
\usepackage{amsgen,amsmath,amstext,amsbsy,amsopn,amsfonts,amssymb}
\usepackage{amsthm}
\usepackage{calc}        % From LaTeX distribution
\usepackage{graphicx}     % From LaTeX distribution
\usepackage{ifthen}
%\usepackage{makeidx}
\input{random.tex}        % From CTAN/macros/generic
\usepackage{subfigure} 
\usepackage{tikz}
\usepackage[customcolors]{hf-tikz}
\usetikzlibrary{arrows}
\usetikzlibrary{matrix}
\tikzset{
    every picture/.append style={
        execute at begin picture={\deactivatequoting},
        execute at end picture={\activatequoting}
    }
}
\usetikzlibrary{decorations.pathreplacing,angles,quotes}
\usetikzlibrary{shapes.geometric}
\usepackage{mathtools}
\usepackage{stackrel}
%\usepackage{enumerate}
\usepackage{enumitem}
\usepackage{tkz-graph}
\usepackage{polynom}
\polyset{%
    style=B,
    delims={(}{)},
    div=:
}
\renewcommand\labelitemi{$\circ$}
\setlist[enumerate]{label={(\arabic*)}}

\setbeamertemplate{itemize item}{$\circ$}
\setbeamertemplate{enumerate items}[default]
\definecolor{links}{HTML}{2A1B81}
\hypersetup{colorlinks,linkcolor=,urlcolor=links}


\newcommand{\Id}{\operatorname{Id}}
\newcommand{\img}{\operatorname{Im}}
\newcommand{\nuc}{\operatorname{Nu}}
\newcommand{\im}{\operatorname{Im}}
\renewcommand\nu{\operatorname{Nu}}
\newcommand{\la}{\langle}
\newcommand{\ra}{\rangle}
\renewcommand{\t}{{\operatorname{t}}}
\renewcommand{\sin}{{\,\operatorname{sen}}}
\newcommand{\Q}{\mathbb Q}
\newcommand{\R}{\mathbb R}
\newcommand{\C}{\mathbb C}
\newcommand{\K}{\mathbb K}
\newcommand{\F}{\mathbb F}
\newcommand{\Z}{\mathbb Z}
\newcommand{\N}{\mathbb N}
\newcommand\sgn{\operatorname{sgn}}
\renewcommand{\t}{{\operatorname{t}}}
\renewcommand{\figurename }{Figura}

%
% Ver http://joshua.smcvt.edu/latex2e/_005cnewenvironment-_0026-_005crenewenvironment.html
%

\renewenvironment{block}[1]% environment name
{% begin code
	\par\vskip .2cm%
	{\color{blue}#1}%
	\vskip .2cm
}%
{%
	\vskip .2cm}% end code


\renewenvironment{alertblock}[1]% environment name
{% begin code
	\par\vskip .2cm%
	{\color{red!80!black}#1}%
	\vskip .2cm
}%
{%
	\vskip .2cm}% end code


\renewenvironment{exampleblock}[1]% environment name
{% begin code
	\par\vskip .2cm%
	{\color{blue}#1}%
	\vskip .2cm
}%
{%
	\vskip .2cm}% end code




\newenvironment{exercise}[1]% environment name
{% begin code
	\par\vspace{\baselineskip}\noindent
	\textbf{Ejercicio (#1)}\begin{itshape}%
		\par\vspace{\baselineskip}\noindent\ignorespaces
	}%
	{% end code
	\end{itshape}\ignorespacesafterend
}


\newenvironment{definicion}[1][]% environment name
{% begin code
	\par\vskip .2cm%
	{\color{blue}Definición #1}%
	\vskip .2cm
}%
{%
	\vskip .2cm}% end code

    \newenvironment{notacion}[1][]% environment name
    {% begin code
        \par\vskip .2cm%
        {\color{blue}Notación #1}%
        \vskip .2cm
    }%
    {%
        \vskip .2cm}% end code

\newenvironment{observacion}[1][]% environment name
{% begin code
	\par\vskip .2cm%
	{\color{blue}Observación #1}%
	\vskip .2cm
}%
{%
	\vskip .2cm}% end code

\newenvironment{ejemplo}[1][]% environment name
{% begin code
	\par\vskip .2cm%
	{\color{blue}Ejemplo #1}%
	\vskip .2cm
}%
{%
	\vskip .2cm}% end code


\newenvironment{preguntas}[1][]% environment name
{% begin code
    \par\vskip .2cm%
    {\color{blue}Preguntas #1}%
    \vskip .2cm
}%
{%
    \vskip .2cm}% end code

\newenvironment{ejercicio}[1][]% environment name
{% begin code
	\par\vskip .2cm%
	{\color{blue}Ejercicio #1}%
	\vskip .2cm
}%
{%
	\vskip .2cm}% end code


\renewenvironment{proof}% environment name
{% begin code
	\par\vskip .2cm%
	{\color{blue}Demostración}%
	\vskip .2cm
}%
{%
	\vskip .2cm}% end code



\newenvironment{demostracion}% environment name
{% begin code
	\par\vskip .2cm%
	{\color{blue}Demostración}%
	\vskip .2cm
}%
{%
	\vskip .2cm}% end code

\newenvironment{idea}% environment name
{% begin code
	\par\vskip .2cm%
	{\color{blue}Idea de la demostración}%
	\vskip .2cm
}%
{%
	\vskip .2cm}% end code

\newenvironment{solucion}% environment name
{% begin code
	\par\vskip .2cm%
	{\color{blue}Solución}%
	\vskip .2cm
}%
{%
	\vskip .2cm}% end code



\newenvironment{lema}[1][]% environment name
{% begin code
	\par\vskip .2cm%
	{\color{blue}Lema #1}\begin{itshape}%
		\par\vskip .2cm
	}%
	{% end code
	\end{itshape}\vskip .2cm\ignorespacesafterend
}

\newenvironment{proposicion}[1][]% environment name
{% begin code
	\par\vskip .2cm%
	{\color{blue}Proposición #1}\begin{itshape}%
		\par\vskip .2cm
	}%
	{% end code
	\end{itshape}\vskip .2cm\ignorespacesafterend
}

\newenvironment{teorema}[1][]% environment name
{% begin code
	\par\vskip .2cm%
	{\color{blue}Teorema #1}\begin{itshape}%
		\par\vskip .2cm
	}%
	{% end code
	\end{itshape}\vskip .2cm\ignorespacesafterend
}


\newenvironment{corolario}[1][]% environment name
{% begin code
	\par\vskip .2cm%
	{\color{blue}Corolario #1}\begin{itshape}%
		\par\vskip .2cm
	}%
	{% end code
	\end{itshape}\vskip .2cm\ignorespacesafterend
}

\newenvironment{propiedad}% environment name
{% begin code
	\par\vskip .2cm%
	{\color{blue}Propiedad}\begin{itshape}%
		\par\vskip .2cm
	}%
	{% end code
	\end{itshape}\vskip .2cm\ignorespacesafterend
}

\newenvironment{conclusion}% environment name
{% begin code
	\par\vskip .2cm%
	{\color{blue}Conclusión}\begin{itshape}%
		\par\vskip .2cm
	}%
	{% end code
	\end{itshape}\vskip .2cm\ignorespacesafterend
}


\newenvironment{definicion*}% environment name
{% begin code
	\par\vskip .2cm%
	{\color{blue}Definición}%
	\vskip .2cm
}%
{%
	\vskip .2cm}% end code

\newenvironment{observacion*}% environment name
{% begin code
	\par\vskip .2cm%
	{\color{blue}Observación}%
	\vskip .2cm
}%
{%
	\vskip .2cm}% end code


\newenvironment{obs*}% environment name
	{% begin code
		\par\vskip .2cm%
		{\color{blue}Observación}%
		\vskip .2cm
	}%
	{%
		\vskip .2cm}% end code

\newenvironment{ejemplo*}% environment name
{% begin code
	\par\vskip .2cm%
	{\color{blue}Ejemplo}%
	\vskip .2cm
}%
{%
	\vskip .2cm}% end code

\newenvironment{ejercicio*}% environment name
{% begin code
	\par\vskip .2cm%
	{\color{blue}Ejercicio}%
	\vskip .2cm
}%
{%
	\vskip .2cm}% end code

\newenvironment{propiedad*}% environment name
{% begin code
	\par\vskip .2cm%
	{\color{blue}Propiedad}\begin{itshape}%
		\par\vskip .2cm
	}%
	{% end code
	\end{itshape}\vskip .2cm\ignorespacesafterend
}

\newenvironment{conclusion*}% environment name
{% begin code
	\par\vskip .2cm%
	{\color{blue}Conclusión}\begin{itshape}%
		\par\vskip .2cm
	}%
	{% end code
	\end{itshape}\vskip .2cm\ignorespacesafterend
}






\newcommand{\nc}{\newcommand}

%%%%%%%%%%%%%%%%%%%%%%%%%LETRAS

\nc{\FF}{{\mathbb F}} \nc{\NN}{{\mathbb N}} \nc{\QQ}{{\mathbb Q}}
\nc{\PP}{{\mathbb P}} \nc{\DD}{{\mathbb D}} \nc{\Sn}{{\mathbb S}}
\nc{\uno}{\mathbb{1}} \nc{\BB}{{\mathbb B}} \nc{\An}{{\mathbb A}}

\nc{\ba}{\mathbf{a}} \nc{\bb}{\mathbf{b}} \nc{\bt}{\mathbf{t}}
\nc{\bB}{\mathbf{B}}

\nc{\cP}{\mathcal{P}} \nc{\cU}{\mathcal{U}} \nc{\cX}{\mathcal{X}}
\nc{\cE}{\mathcal{E}} \nc{\cS}{\mathcal{S}} \nc{\cA}{\mathcal{A}}
\nc{\cC}{\mathcal{C}} \nc{\cO}{\mathcal{O}} \nc{\cQ}{\mathcal{Q}}
\nc{\cB}{\mathcal{B}} \nc{\cJ}{\mathcal{J}} \nc{\cI}{\mathcal{I}}
\nc{\cM}{\mathcal{M}} \nc{\cK}{\mathcal{K}}

\nc{\fD}{\mathfrak{D}} \nc{\fI}{\mathfrak{I}} \nc{\fJ}{\mathfrak{J}}
\nc{\fS}{\mathfrak{S}} \nc{\gA}{\mathfrak{A}}
%%%%%%%%%%%%%%%%%%%%%%%%%LETRAS


\title[Clase 11 - Divisibilidad]{Matemática Discreta I \\ Clase 11 - Divisibilidad}
%\author[C. Olmos / A. Tiraboschi]{Carlos Olmos / Alejandro Tiraboschi}
\institute[]{\normalsize FAMAF / UNC
    \\[\baselineskip] ${}^{}$
    \\[\baselineskip]
}
\date[25/04/2023]{25 de abril  de 2023}




\begin{document}

\frame{\titlepage} 


\begin{frame}
    
{\color{blue}Definición}
\vskip .2cm Dados dos enteros $x$ e $y$ decimos que $y$ {\em divide a } $x$, y escribimos $y|x$, si
    $$
    x=yq\quad\text{ para algún }\quad q\in \mathbb Z.
    $$
    También decimos que $y$ es un {\em factor} de $x$, que $y$ es un {\em divisor} de $x$, que $x$ es {\em divisible} por $y$, y que $x$ es {\em múltiplo}\index{múltiplo} de $y$.

\vskip .8cm\pause

{\color{blue}Observación.}
\vskip .2cm 

1. Si $y|x$, es decir si $y$ es divisor de $x$, existe $q$ tal que $x = yq$. Luego $q$ es
también un divisor de $x$.\pause

\vskip .2cm 
2. Si $y|x$ e $y \not=0$, denotamos $\frac{x}{y}$ al cociente de $x$ dividido  $y$, es decir 
$$
x = \frac{x}{y}\cdot y
$$

\end{frame}


\begin{frame}
    
    {\color{blue}Notación. }
    \vskip .3cm
    Si $y$ no  divide a $x$ escribimos $y  \not| x$.
\vskip .8cm\pause
{\color{blue}Ejemplos. }
\vskip .3cm

(1) $3 | 12$, pues $12 = 3 \cdot 4$. Es decir, 3 es divisor de 12 y luego 4 es otro divisor de 12.
\vskip .3cm\pause
(2) $12 \not| 3$, pues no existe ningún entero $q$ tal que $3 = q \cdot 12$. 
\vskip .3cm\pause
(3) $6 | 18$, pues $18 = 6 \cdot 3$. Luego también vale que $18 = (-6) \cdot (-3)$ y que $ -18 = (-6) \cdot 3 $ y $ -18 = 6 \cdot (-3)$.  De esto se sigue que
\begin{align*}
    6 &| 18,& -6 &| 18,& -6 &| -18,& 6 &| -18,\\
    3 &| 18,& -3 &| 18,& 3 &| -18,& -3 &| -18.
\end{align*}



\end{frame}


\begin{frame}\frametitle{Propiedades básicas 1}
    
Veamos ahora alguna propiedades básicas de la relación ``divide a''. \pause

\vskip .3cm

Sean $a,b,c$ enteros, entonces\pause

\vskip .3cm

{\color{blue}1. }        $1|a$\quad $a|\pm a$;
        
        \vskip .3cm \pause
{\color{blue}Demostración. }\pause
\vskip .3cm        
        \;\textbullet\; $a = 1 \cdot a$.
\vskip .3cm    
        \;\textbullet\; $ a =     a \cdot 1$.     
\vskip .3cm    
        \;\textbullet\; $ - a =     a \cdot (-1)$.    
\vskip .3cm            



\end{frame}




\begin{frame}\frametitle{Propiedades básicas 2}
    
        
    {\color{blue}2. }             $a|0$ y $0$ sólo divide a $0$;
    \vskip .3cm  \pause
    {\color{blue}Demostración. }\pause
    \vskip .3cm
    \;\textbullet\; $0 = a\cdot 0$.
    \vskip .3cm    
    \;\textbullet\; Si $0|a$ entonces existe $q$ tal que $a = 0 \cdot q =0$. 
    \vskip .8cm    

    {\color{blue}3. }            si $a|b$, entonces $a|bc$ para cualquier $c$;\pause
    \vskip .3cm
    {\color{blue}Demostración. } \pause
    \vskip .3cm
    \;\textbullet\; $a | b \;\Rightarrow\; b = a \cdot q \;\Rightarrow\;  bc = a \cdot qc \;\Rightarrow\; a|bc$. 
    

    
    
\end{frame}




\begin{frame}\frametitle{Propiedades básicas 2}
        {\color{blue}4. }           si $a|b$ y $a|c$, entonces $a|(b+c)$;\pause
        \vskip .3cm
        {\color{blue}Demostración. } \pause
    \vskip .3cm
    \;\textbullet\; $a|b$ y $a|c$ $\;\Rightarrow\;$ $b = a \cdot q$ y $c = a \cdot q'$ $\;\Rightarrow\;$ 
    \vskip .2cm
    \quad\; $b+c =  a \cdot q +   a \cdot q' =  a \cdot(q + q')$  $\;\Rightarrow\;$  $a|(b+c)$.
    
    
    \vskip .8cm    
    {\color{blue}5. }               si $a|b$ y $a|c$, entonces $a|(rb+sc)$ para cualesquiera $r,s \in \mathbb Z$.
        \vskip .3cm\pause
        {\color{blue}Demostración. } \pause
        \vskip .3cm
    \;\textbullet\; $a|b$ y $a|c$ $\;\Rightarrow\;$ $b = a \cdot q$ y $c = a \cdot q'$ $\;\Rightarrow\;$ 
    \vskip .2cm
    \quad\; $rb+sc =  a \cdot rq +   a \cdot sq' =  a \cdot(rq + sq')$  $\;\Rightarrow\;$  $a|(rb+sc)$.
\end{frame}


\begin{frame}\frametitle{Propiedades básicas 3}
    {\color{blue}6. }           si $a|b + c$ y $a|c$, entonces $a|b$;\pause
    \vskip .3cm 
    {\color{blue}Demostración. } \pause
    \vskip .3cm
    
    \;\textbullet\; $a|b+c$ y $a|c$ $\;\Rightarrow\;$ $b + c = a \cdot q$ y $c = a \cdot q'$ $\;\Rightarrow\;$ 
    \vskip .2cm
    \quad\; $ b = (b+c) -c  =  a \cdot q -   a \cdot q' =  a \cdot(q -q')$  $\;\Rightarrow\;$  $a|b$.
    
    \vskip 0.4cm \pause

    {\color{blue}7. }           si $a|b$, entonces $\pm a| \pm b$;\pause
    \vskip .3cm 
    {\color{blue}Demostración. } \pause
    \vskip .3cm
    
    \;\textbullet\; $a|b$\;$\Rightarrow$\; $b = a \cdot q$\;$\Rightarrow$\;
    \vskip .2cm
    \qquad$\begin{matrix}
        -b = a \cdot (-q ) \;\Rightarrow\; a|-b, &\quad &b = -a \cdot (-q ) \;\Rightarrow\; -a|b\\-b = -a \cdot q  \;\Rightarrow\; -a|-b. & &
    \end{matrix}$

    \qed

\end{frame}

\begin{frame}
    
    
{\color{blue}Proposición }
\vskip .3cm
{\it Sean $a,b \in \mathbb{N}$. Entonces} 
$$ab = 1\quad \Rightarrow \quad a=1 \wedge b=1.
$$
 \pause
{\color{blue}Demostración.}  \pause 
\vskip .2cm
Como  $a,b \in \mathbb{N}$, entonces  $a\ge 1$ y $b\ge 1$. 
\vskip .2cm
Si $a=1$, como $ab =1$, obtenemos  $1 = ab = 1\cdot b = b$. 
\vskip .2cm
Si $a>1$, como  $b>0$ por compatibilidad de $<$ con el producto tenemos que $ab>b$, es decir $1 > b$, lo cual no es cierto ($b \in \mathbb{N}$). \qed
\vskip .5cm
{\color{blue}Observación}
\vskip .2cm
A partir de la proposición no es difícil probar que si $a,b \in \mathbb{Z}$ y $ab = 1$, tenemos que $a=1$ y $b= 1$ o $a=-1$ y $b= -1$. 
\end{frame}


\begin{frame}
    
{\color{blue}Proposición }
\vskip .3cm
{\it Sean $a,b,c \in \mathbb{N}$, entonces 
\begin{enumerate}
    \item[(D1)] $a | a$ (reflexividad);
    \item[(D2)]  si $a | b$ y $b | a$, entonces $a = b$ (antisimetría);
    \item[(D3)]  si $a | b$ y $b | c$, entonces $a | c$ (transitividad).
\end{enumerate} }
\vskip .3cm
{\color{blue}Demostración. }
\vskip .3cm\pause
(D1) Esto ya fue probado antes.
\vskip .3cm\pause
(D2) $a|b \Rightarrow$ existe $q \in \mathbb N$ tal que $b = aq$.  \pause

\quad\quad\,\,$b|a\Rightarrow$ existe $q' \in \mathbb N$ tal que $a = bq'$. 

\pause
Luego $$b = aq = (bq')q = b(q'q).$$
\pause
Por el axioma de cancelación (cancelando $b$) obtenemos que $1 = q'q \Rightarrow q = q' =1$. \pause Luego $a = b$.
\end{frame}


\begin{frame}
    
    
    (D3) $a|b \Rightarrow$ existe $q \in \mathbb N$ tal que $b = aq$.  \pause
    \vskip .3cm
    \quad\quad\,$b|c\Rightarrow$ existe $q' \in \mathbb N$ tal que $c = bq'$. 
    \vskip .3cm
    \pause
    Luego $$c =bq' = aqq' =a(qq').$$
    \pause
    Luego $a|c$. \qed\pause
    \vskip .6cm
    {\color{blue}Observación. }
    \vskip .3cm
    Las propiedades (D1), (D2) y (D3) nos dicen que ``divide a'' es una \textit{relación de orden}.
    \vskip .3cm
    Habíamos visto que ``$\le$'' también era una relación de orden.
    
    \vskip 1cm
\end{frame}


\begin{frame}\frametitle{Ejercicios}
    
    {\color{blue}Ejercicio}
        \vskip .3cm 
        ¿Es cierto que si $a | bc$, entonces $a | b$ ó $a | c$?
        \vskip .3cm
        \pause
    {\color{blue}Solución. } No necesariamente (es decir la respuesta es NO).
    \vskip .3cm
        \;\textbullet\; Es cierto, por ejemplo que  $3 | 6 \cdot 2$ y que $3 | 6$.
        \vskip .3cm
        \;\textbullet\; Pero $6|4 \cdot 3$ y $6\not|4$, $6\not|3 $.
        \vskip .8cm
        
        {\color{blue}Ejercicio}
        \vskip .3cm 
        Determinar todos los divisores de 12.
        \vskip .3cm
        \pause
        {\color{blue}Solución. }
        \vskip .3cm
        \;\textbullet\; $\pm1$, $\pm2$, $\pm3$, $\pm4$, $\pm6$, $\pm12$, (12 divisores).
    
\end{frame}


\begin{frame}\frametitle{Ejercicios}
        {\color{blue}Ejercicio.}
        \vskip .3cm 
           Mostrar que $4^n -1$ es divisible por 3 para todo $n \in \mathbb{N}$.
    \pause\vskip .3cm 
    
    {\color{blue}Solución. } 
    \vskip .3cm 
    Este ejercicio se puede hacer de dos formas.
    \vskip .3cm 
    \textbf{1° demostración.} Por inducción sobre $n$.
    \vskip .3cm 
    Caso base $n=1$. $4^n -1 = 4^1- 1 = 4-1 = 3$ que obviamente es divisible por 3.
    \vskip .3cm 
    Paso  inductivo.  Debemos probar que  $3 | 4^k -1$ para $k \ge 1$ (HI) entonces, se deduce que $3 | 4^{k+1} -1$.
    \vskip .3cm
    Ahora bien,
    \begin{equation*}
        4^{k+1} -1 = 4 \cdot 4^k - 1 = 3 \cdot 4^k + (4^k -1).
    \end{equation*} 
\end{frame}


\begin{frame}
    \textbf{Opción a)} Ahora bien,
    \begin{equation*}
        4^{k+1} -1 = 4 \cdot 4^k - 1 = 3 \cdot 4^k + (4^k -1).
    \end{equation*} 
        Como $3| 3 \cdot 4^k$ y por (HI) $3|4^k -1$    , tenemos
        \begin{equation*}
            3| 3 \cdot 4^k+ 4^k -1 =4^{k+1} -1.
        \end{equation*}
    \vskip .3cm 
    \textbf{Opción b)} Ahora bien, como por (HI) $3 | 4^k -1$, existe $q$ tal que $4^k -1 = 3q$. Entonces,
    \begin{equation*}
        4^{k+1} -1 = 4 \cdot (4^k - 1) + 3 = 4 \cdot 3q + 3 = 3(4  q +1).
    \end{equation*} 
    Luego 
    \begin{equation*}
        3 | 3(4q +1) =4^{k+1} -1 .
    \end{equation*}
    
    
\end{frame}


\begin{frame}
    \textbf{2° demostración.} 
    Observemos que $4 = 3 +1$, luego $4^n -1 = (3+1)^n -1$.
    \begin{align*}
        4^n -1 = (3+1)^n -1 &= \sum_{i=0}^{n} \binom{n}{i} 3^i 1^{n-i}- 1&&(\text{binomio de Newton}) \\
        &= \sum_{i=0}^{n} \binom{n}{i} 3^i- 1 &&\\
    &= 1 + \sum_{i=1}^{n} \binom{n}{i} 3^i - 1 &&\\
    &= 1 + 3 (\sum_{i=1}^{n} \binom{n}{i} 3^{i-1}) - 1 &&(\text{$i > 0$ en la sumatoria})\\
    &=  3 (\sum_{i=1}^{n} \binom{n}{i} 3^{i-1}) &&
    \end{align*}
    Luego $4^n -1 = 3 \cdot q$. 
    \vskip .3cm 
    Por lo tanto, $3|4^n -1$.

    \qed
\end{frame}


\begin{frame}\frametitle{Ejercicios}
    
    {\color{blue}Ejercicio}
    \vskip .3cm 
¿Cuál es el menor natural que es divisible por 6 y por 15?

    \vskip .3cm
    \pause
    {\color{blue}Solución. } Hagamos una lista de múltiplos de 6 y 15.
    \vskip .3cm
    \;\textbullet\; Múltiplos de 6: 6, 12, 18, 24, 30, 36, 42, 48, 54, 60, ...
    \vskip .3cm
    \;\textbullet\; Múltiplos de 15: 15, 30, 45, 60 , 75, ...
    \vskip .3cm
    \;\textbullet\; Luego, el menor natural que es divisible por 6 y por 15 es 30.
    \vskip .8cm
\end{frame}



\end{document}


