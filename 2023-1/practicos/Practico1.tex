% PDFLaTeX
\documentclass[a4paper,12pt,twoside,spanish,reqno]{amsbook}
%%%---------------------------------------------------

%\renewcommand{\familydefault}{\sfdefault} % la font por default es sans serif
%\usepackage[T1]{fontenc}

% Para hacer el  indice en linea de comando hacer 
% makeindex main
%% En http://www.tug.org/pracjourn/2006-1/hartke/hartke.pdf hay ejemplos de packages de fonts libres, como los siguientes:
%\usepackage{cmbright}
%\usepackage{pxfonts}
%\usepackage[varg]{txfonts}
%\usepackage{ccfonts}
%\usepackage[math]{iwona}
\usepackage[math]{kurier}

\usepackage{etex}
\usepackage{t1enc}
\usepackage{latexsym}
\usepackage[utf8]{inputenc}
\usepackage{verbatim}
\usepackage{multicol}
\usepackage{amsgen,amsmath,amstext,amsbsy,amsopn,amsfonts,amssymb}
\usepackage{amsthm}
\usepackage{calc}         % From LaTeX distribution
\usepackage{graphicx}     % From LaTeX distribution
\usepackage{ifthen}
\input{random.tex}        % From CTAN/macros/generic
\usepackage{subfigure} 
\usepackage{tikz}
\usetikzlibrary{arrows}
\usetikzlibrary{matrix}
\usepackage{mathtools}
\usepackage{stackrel}
\usepackage{enumitem}
\usepackage{tkz-graph}
%\usepackage{makeidx}
\usepackage{hyperref}
\hypersetup{
    colorlinks=true,
    linkcolor=blue,
    filecolor=magenta,      
    urlcolor=cyan,
}
\usepackage{hypcap}
\numberwithin{equation}{section}
% http://www.texnia.com/archive/enumitem.pdf (para las labels de los enumerate)
\renewcommand\labelitemi{$\circ$}
\setlist[enumerate, 1]{label={(\arabic*)}}
\setlist[enumerate, 2]{label=\emph{\alph*)}}


%%% FORMATOS %%%%%%%%%%%%%%%%%%%%%%%%%%%%%%%%%%%%%%%%%%%%%%%%%%%%%%%%%%%%%%%%%%%%%
\tolerance=10000
\renewcommand{\baselinestretch}{1.3}
\usepackage[a4paper, top=3cm, left=3cm, right=2cm, bottom=2.5cm]{geometry}
\usepackage{setspace}
%\setlength{\parindent}{0,7cm}% tamaño de sangria.
\setlength{\parskip}{0,4cm} % separación entre parrafos.
\renewcommand{\baselinestretch}{0.90}% separacion del interlineado
\setlist[1]{topsep=8pt,itemsep=.4cm,partopsep=4pt, parsep=4pt} %espacios nivel 1 listas
\setlist[2]{itemsep=.15cm}  %espacios nivel 2 listas
%%%%%%%%%%%%%%%%%%%%%%%%%%%%%%%%%%%%%%%%%%%%%%%%%%%%%%%%%%%%%%%%%%%%%%%%%%%%%%%%%%%
%\end{comment}
%%% FIN FORMATOS  %%%%%%%%%%%%%%%%%%%%%%%%%%%%%%%%%%%%%%%%%%%%%%%%%%%%%%%%%%%%%%%%%

\newcommand{\rta}{\noindent\textit{Rta: }} 

\newcommand \Z{{\mathbb Z}}
\newcommand \N{{\mathbb N}}
\newcommand \mcd{\operatorname{mcd}}
\newcommand \mcm{\operatorname{mcm}}

\begin{document}
    \baselineskip=0.55truecm %original
    

{\bf \begin{center}\large  Práctico 1 \\ Matemática Discreta I -- Año 2023/1 \\ FAMAF \end{center}}

%\begin{enumerate}[topsep=8pt,itemsep=.4cm,partopsep=4pt, parsep=4pt]

\begin{enumerate}
\item\label{prob1} Demostrar las siguientes afirmaciones, donde $a$, $b$ son siempre números \linebreak enteros. Justificar cada uno de los pasos en cada demostración indicando el axioma o resultado que utiliza.
\begin{enumerate}
	\item $a+a=a$\, implica que \,  $a=0$.
	\item $0\cdot a = 0$.
	\item $-a = (-1)\cdot a$.
    \item  $-(-a) = a$.
    \item  $a=b\,$ si y sólo si $\,-a=-b$.
\end{enumerate}




\item Idem \ref{prob1}, donde $a$, $b$, $c$ son siempre números enteros.

\begin{enumerate}
	\item $c<0$\, implica que \, $0 < -c$.
	\item $a+c <b+c$\, implica que \, $a<b$.
    \item $0<a\,$ y $\,0<b\,$ implican $\,0<a\cdot b$
    \item $a<b\,$ y $\,c<0$\, implican $\,b\cdot c<a\cdot c$
\end{enumerate}



\item  Probar las siguientes afirmaciones, justificando los pasos que realiza.
\begin{enumerate}
  \item Si $0 < a$  y $\,0<b\,$ entonces $\,a<b\,$ si y sólo si $a^2<b^2$.
  \item Si $a\neq 0$  entonces $a^2>0$.
  \item Si $a\neq b$  entonces $a^2+b^2>0$.
\end{enumerate}


\item Calcular, evaluando, las siguientes expresiones:
\vskip .2cm
\begin{enumerate}
\begin{minipage}{0.40 \textwidth}
\item \quad $\displaystyle{\sum_{r=0}^4 r}$
\end{minipage}
\begin{minipage}{0.40 \textwidth}
\item \quad $\displaystyle{\prod_{i=1}^5 i}$
 \end{minipage}

\vskip .2cm

\begin{minipage}{0.40 \textwidth}
\item  \quad $\displaystyle{\sum_{k=-3}^{-1} \frac{1}{k(k+4)}}$
\end{minipage}
\begin{minipage}{0.40 \textwidth}
\item \quad $\displaystyle{\prod_{n=2}^7 \frac{n}{n-1}}$
\end{minipage}
\end{enumerate}


\begin{comment}
\item Dado un natural $m$, probar que $\forall n \in {\mathbb N} $; $x$, $y \in {\mathbb R}$, se cumple:
\begin{enumerate}
	\begin{minipage}{0.25 \textwidth}\item $x^n \cdot x^m = x^{n+m}$
	\end{minipage}
	\begin{minipage}{0.25 \textwidth}
		\item $(x\cdot y)^n=x^n\cdot y^n$
	\end{minipage}
	\begin{minipage}{0.25\textwidth}
		\item $(x^n)^m = x^{n\cdot m}$
	\end{minipage}
\end{enumerate}
\end{comment}


\item Usando las propiedades de las potencias, calcular:
 \begin{enumerate}
\begin{minipage}{0.40 \textwidth}
\item \quad $2^{10} - 2^{9}$
\end{minipage}
\begin{minipage}{0.40 \textwidth}
\item \quad $3^2 2^5 - 3^5 2^2$
\end{minipage}
\vskip .2cm
\begin{minipage}{0.40 \textwidth}
\item \quad $\left(2^2\right)^n - \left(2^n\right)^2$
\end{minipage}
\begin{minipage}{0.40 \textwidth}
\item \quad $\left(2^{2^n} + 1\right)\left(2^{2^n} - 1\right)$
\end{minipage}
\end{enumerate}



\item Analizar la validez de las siguientes afirmaciones:
 \begin{enumerate}
\item  $\left(2^{2^n}\right)^{2^k} = 2^{2^{n+k}}$,  \ $n$, $k \in {\mathbb N}$.
\item $(2^n)^2 = 4^n$, $n \in {\mathbb N}$.
\item $2^{7+11} = 2^7 + 2^{11}$.
\end{enumerate}

\item Probar que \,  $\sum_{i=0}^n 2^i = 2^{n+1} -1$,\ para todo \ $n \in \N_0$. 

\item Demostrar por inducción  las siguientes igualdades:
  \begin{enumerate}
  \item  $\displaystyle{ \sum_{j=1}^n j = \frac{n(n+1)}{2}}$, \ $n\in \mathbb N$.
  \item  $\displaystyle{ \sum_{i=1}^n i^2 = \frac{n(n+1)(2n+1)}{6}}$, \  $n\in \mathbb N$.
  \item  $\displaystyle{ \sum_{k=0}^n (2k+1) = (n+1)^2}$, \ $n\in \mathbb N_0$.
  \item  $\displaystyle{ \sum_{i=1}^n i^3 = \left( \frac{n(n+1)}{2 }\right)^2}$, \ $n\in \mathbb N$.
  \item  $\displaystyle{ \sum_{k=0}^n a^k = \frac{a^{n+1}-1}{a-1}}$, \  donde $a\in {\mathbb R}$, $a \neq 0$, $a \neq 1$, $n\in \mathbb N_0$.
  \item  $\displaystyle{ \sum_{i=1}^n (i^2+1) i! = n (n+1)!}$, \ $n\in \N$.
  \item  $\displaystyle{ \prod_{i=1}^n (4i - 2) = \dfrac{(2n)!}{n!}}$, \ $n\in \N$.
  \end{enumerate}

\item Hallar $n_0 \in {\mathbb N}$ tal que $\forall n \ge n_0$ se cumpla que $n^2 \ge 11 n + 3$, y usar el principio de inducción para probar dicha desigualdad.

\item Sea $\{u_n\}_{n \in \mathbb N_0}$ la sucesión definida por recurrencia como sigue: \ $u_0=2$, $u_1=4$ y $u_n=4 u_{n-1} - 3 u_{n-2}$ con $n\in \mathbb N$, $n\geq 2$. Probar que $u_n=3^n+1$, para todo $n\in\mathbb{N}_0$.

\item Sea $\{ u_n \}_{n \in \mathbb N}$ la sucesión definida por recurrencia como sigue: $u_1 = 9$, $u_2 = 33$, $u_n = 7u_{n-1} - 10u_{n-2}$, $\forall n \geq 3$. Probar que $u_n = 2^{n+1} + 5^n$, para todo $n \in \mathbb N$.


\item  Sea $\{a_n\}_{n\in\mathbb N_0}$ la sucesi\'on definida recursivamente por
$$\begin{cases}
a_0=1, \\a_1=1, \\a_{n} = 3a_{n-1}+(n-1)(n-3)a_{n-2}, \text{ para $n\geq 2$}.
\end{cases}$$
Probar que $a_n=n!$ para todo $n\in \mathbb N_0$.

\item Sea $\{a_n\}_{n\in\mathbb N_0}$ la sucesi\'on definida recursivamente por
$$\begin{cases}
   a_0=0, \\a_1=7, \\a_{n} = 5a_{n-1}+6a_{n-2}, \text{ para $n\geq 2$}.
  \end{cases}$$
Probar que $a_n=6^n + (-1)^{n+1}$ para todo $n\in \mathbb N_0$.

\newpage

\item Sea $u_n$ definida recursivamente por: $u_1=2$, $u_n=2+\sum_{i=1}^{n-1}2^{n-2i}u_i \;\;\forall\; n >1$.
\begin{enumerate}
	\item Calcule $u_2$ y $u_3$.
	\item Proponga una fórmula para el término general $u_n$ y pruébela por inducción.
\end{enumerate}


\item Las siguientes proposiciones no son válidas para todo $n \in {\mathbb N}$. Indicar en qué paso del principio de inducción falla la demostración:
\begin{enumerate}
\begin{minipage}{0.20 \textwidth}
\item  $n=n^2$,
\end{minipage}
\begin{minipage}{0.20 \textwidth}
\item  $n=n+1$,
\end{minipage}
\begin{minipage}{0.20 \textwidth}
\item  $3^n = 3^{n+2}$,
\end{minipage}
\begin{minipage}{0.20 \textwidth}
\item  $3^{3n} = 3^{n+2}$.
\end{minipage}
\end{enumerate}
\end{enumerate}

\subsection*{$\S$ Ejercicios de repaso} Los ejercicios marcados con ${}^{(*)}$ son de mayor dificultad.

\begin{enumerate}[resume]
\item Demostrar las siguientes igualdades:
  \begin{enumerate}
  \item  $\displaystyle{ \prod_{i=1}^n \frac{i+1}{i} = n+1}$, \ $n\in \mathbb N$.

  \item $\displaystyle{ \sum_{i=1}^n \frac{1}{4i^2-1} = \frac{n}{2n+1}}$, \  $n\in \mathbb N$.

  \item $\displaystyle{ \sum_{i=1}^n i^2\, /\, \sum_{j=1}^n j = \frac{2n+1}{3}}$, \ $n\in \mathbb N$.

  \item $\displaystyle{ \prod_{i=2}^n \left(1-\frac{1}{i^2}\right) = \frac{n+1}{2n}}$, \ $n\in \mathbb N$ y $ n\ge 2$.
\vskip .2cm
  \item Si $a\in \mathbb R$ y $a\geq -1$, entonces $(1+a)^n\geq 1+n\cdot a$, \ $\forall \, n \in \mathbb N$.
\vskip .2cm
  \item Si $a_1,\dots,a_n \in \mathbb R$, entonces $\displaystyle{\sum_{k=1}^n a_{k}^{2}\leq \left(\sum_{k=1}^n |a_{k}|\right)^{2}}$, $n\in \mathbb N$.
\vskip .2cm

  \item Si $a_1,\dots,a_n \in \mathbb R$ y $0<a_i<1\, \forall \, i$, entonces $\forall \, n\in \mathbb N$,
  $$(1-a_1)\cdots(1-a_n)\ge 1-a_1-\cdots -a_n.$$

  \end{enumerate}

  \item   Sea $\{a_n\}_{n\in\mathbb N}$ la sucesión definida recursivamente por
  $$\begin{cases}
  a_1=1, \\a_2=2, \\a_{n} = (n-2)a_{n-1}+2(n-1)a_{n-2}, \text{ para $n\geq 3$}.
  \end{cases}$$
  Probar que $a_n=n!$ para todo $n\in \mathbb N$.

  \item   Sea $\{a_n\}_{n\in\mathbb N_0}$ la sucesión definida recursivamente por
  $$\begin{cases}
     a_0=0, \\a_1=5, \\a_{n} = a_{n-1}+6a_{n-2}, \text{ para $n\geq 2$}.
    \end{cases}$$
  Probar que $a_n=3^n + (-1)^{n+1}2^n$ para todo $n\in \mathbb N_0$.

\newpage

\item${}^{(*)}$ Encuentre el error en los siguientes argumentos de inducción.
\begin{enumerate}
\item  Demostraremos que $5n+3$ es múltiplo de 5 para todo $n\in \mathbb N$.

Supongamos que $5k+3$ es múltiplo de 5, siendo $k\in \mathbb N$. Entonces existe
$p\in \mathbb N$ tal que  $5k+3=5p$. Probemos que $5(k+1)+3$ es múltiplo de 5:
Como
$$
5(k+1)+3=(5k+5)+3=(5k+3)+5=5p+5=5(p+1),
$$
entonces obtenemos que $5(k+1)+3$ es múltiplo de 5. Por lo tanto, por el principio
de inducción, demostramos que $5n+3$ es múltiplo de 5 para todo $n\in \mathbb
N$.

\item Sea $a\in\mathbb R$, con $a\neq 0$. Vamos a demostrar que para todo entero no negativo $n$, $a^n=1$.

Como $a^0=1$ por definición, la proposición es verdadera para $n=0$. Supongamos
que para  un entero $k$, $a^m=1$ para $0\leq m \leq k$. Entonces
$a^{k+1}= \frac{a^k a^k}{a^{k-1}}=\frac{1\cdot1}1=1$.
Por lo tanto, el principio de inducción fuerte implica que $a^n=1$ para todo $n\in \mathbb N$.
\end{enumerate}

\item${}^{(*)}$ La \emph{sucesión de Fibonacci} se define recursivamente de la siguiente manera:
$$
u_1=1,\quad u_2=1,\quad u_{n+1}=u_{n}+u_{n-1}, \, n\geq 2.
$$
Los primeros términos de esta sucesión son: $1,1,2,3,5,8,13,\ldots$

Demostrar por inducción que el término general de esta sucesión se puede calcular mediante la fórmula
\[ u_n= \frac{1}{\sqrt{5}}\left[\left(\frac{1+\sqrt{5}}{2}\right)^n-\left(\frac{1-\sqrt{5}}{2}\right)^n\right].\]
\vskip .2cm
\textit{Ayuda:} usar que $\frac{1+\sqrt{5}}{2}$ y $\frac{1-\sqrt{5}}{2}$ son las raíces de la ecuación cuadrática $x^2-x-1=0$ y por lo tanto  $\left(\frac{1\pm\sqrt{5}}{2}\right)^{n+1} = \left(\frac{1\pm\sqrt{5}}{2}\right)^{n}+\left(\frac{1\pm\sqrt{5}}{2}\right)^{n-1}$.

    \item Probar las siguientes afirmaciones usando inducción en $n$:
    \begin{enumerate}
    	\item $2n+1 < 2^n$ para todo $n\in{\mathbb N}$, $n>2$.
        \item $n^2\leq 2^n$ para todo $n\in{\mathbb N}$, $n>3$.
        \item $\forall n \in {\mathbb N}$,\ $3^n \ge 1 + 2^n$.
    \end{enumerate}

\end{enumerate}


\end{document}
