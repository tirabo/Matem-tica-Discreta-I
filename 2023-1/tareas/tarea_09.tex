% PDFLaTeX
\documentclass[a4paper,12pt,twoside,spanish]{amsbook}

\usepackage{etex}
\tolerance=10000
\renewcommand{\baselinestretch}{1.3}

\renewcommand{\familydefault}{\sfdefault} % la font por default es sans serif

% Para hacer el  indice en linea de comando hacer 
% makeindex main
%% En http://www.tug.org/pracjourn/2006-1/hartke/hartke.pdf hay ejemplos de packages de fonts libres, como los siguientes:
%\usepackage{cmbright}
%\usepackage{pxfonts}
%\usepackage[varg]{txfonts}
%\usepackage{ccfonts}
%\usepackage[math]{iwona}
%\usepackage[math]{kurier}


\usepackage{t1enc}
%\usepackage[spanish]{babel}
\usepackage{latexsym}
\usepackage[utf8]{inputenc}
\usepackage{verbatim}
\usepackage{multicol}
\usepackage{amsgen,amsmath,amstext,amsbsy,amsopn,amsfonts,amssymb}
\usepackage{amsthm}
\usepackage{calc}         % From LaTeX distribution
\usepackage{graphicx}     % From LaTeX distribution
\usepackage{ifthen}
\input{random.tex}        % From CTAN/macros/generic
\usepackage{subfigure} 
\usepackage{tikz}
\usetikzlibrary{arrows}
\usetikzlibrary{matrix}
%\usetikzlibrary{graphs}
%\usepackage{tikz-3dplot} %for tikz-3dplot functionality
%\usepackage{pgfplots}
\usepackage{mathtools}
\usepackage{stackrel}
\usepackage{enumerate}
\usepackage{tkz-graph}
%\usepackage{makeidx}
\makeindex

%%%----------------------------------------------------------------------------
\usepackage[a4paper, top=3cm, left=3cm, right=2cm, bottom=2.5cm]{geometry}
%% CONTROLADORES DE.
% Tamaño de la hoja de impresión.
% Tamaños de los laterales del documento. 
%%%%%%%%%%%%%%%%%%%%%%%%%%%%%%%%%%%%%%%%%%%%%%%%%%%%%%%%%%%%%%%%%%%%%%%%%%%%%%%%%
%%% \theoremstyle{plain} %% This is the default
%\oddsidemargin 0.0in \evensidemargin -1.0cm \topmargin 0in
%\headheight .3in \headsep .2in \footskip .2in
%\setlength{\textwidth}{16cm} %ancho para apunte
%\setlength{\textheight}{21cm} %largo para apunte
%%%%\leftmargin 2.5cm
%%%%\rightmargin 2.5cm
%\topmargin 0.5 cm
%%%%%%%%%%%%%%%%%%%%%%%%%%%%%%%%%%%%%%%%%%%%%%%%%%%%%%%%%%%%%%%%%%%%%%%%%%%%%%%%%%%

\usepackage{hyperref}
\hypersetup{
	colorlinks=true,
	linkcolor=blue,
	filecolor=magenta,      
	urlcolor=cyan,
}
\usepackage{hypcap}


\renewcommand{\thesection}{\thechapter.\arabic{section}}
\renewcommand{\thesubsection}{\thesection.\arabic{subsection}}

\newtheorem{teorema}{Teorema}[section]
\newtheorem{proposicion}[teorema]{Proposici\'on}
\newtheorem{corolario}[teorema]{Corolario}
\newtheorem{lema}[teorema]{Lema}
\newtheorem{propiedad}[teorema]{Propiedad}

\theoremstyle{definition}

\newtheorem{definicion}{Definici\'on}[section]
\newtheorem{ejemplo}{Ejemplo}[section]
\newtheorem{problema}{Problema}[section]
\newtheorem{ejercicio}{Ejercicio}[section]
\newtheorem{ejerciciof}{}[section]

\theoremstyle{remark}
\newtheorem{observacion}{Observaci\'on}[section]
\newtheorem{nota}{Nota}[section]

\renewcommand{\partname }{Parte }
\renewcommand{\indexname}{Indice }
\renewcommand{\figurename }{Figura }
\renewcommand{\tablename }{Tabla }
\renewcommand{\proofname}{Demostraci\'on}
\renewcommand{\appendixname }{}
\renewcommand{\contentsname }{Contenidos }
\renewcommand{\chaptername }{}
\renewcommand{\bibname }{Bibliograf\'\i a }


\newcommand{\tarea}[1]{
	\begin{center}
		{\Large Matemática Discreta I - 2021/1} \vskip.4cm
		{\Large Tarea #1}\vskip .4cm
\end{center}}

\renewenvironment{ejercicio}[1][]% environment name
{% begin code
	\par\vskip .2cm%
	{\noindent\color{blue}Ejercicio #1}%
	\vskip .2cm
}%
{%
	\vskip .2cm}% end code


\newenvironment{solucion}% environment name
{% begin code
	\par\vskip .2cm%
	{\noindent\color{blue}Solución}%
	\vskip .2cm
}%
{%
	\vskip .2cm}% end code

 


\begin{document}


\tarea{9}


\begin{ejercicio}[1]
	
		(60 pts)
		\begin{enumerate}
			\item[(i)] Encontrar todas las soluciones de la ecuación en congruencia
			$$30\,x\equiv 2 \quad (67)$$
			usando el método visto en clase.
			\item[(ii)] Dar todas las soluciones $x$ de la ecuación anterior tales que $0 < x < 300$.
		\end{enumerate}
	\end{ejercicio}
    \begin{ejercicio}[2]
	 (40 pts)  Encontrar los últimos dos dígitos de $2^{338}$. (\textit{Ayuda:} Se cumple que $2^{22} \equiv 4 \; (100)$).
	
\end{ejercicio}
	
\begin{comment}^{5}
\begin{solucion}

1(i). (40 pts) 
Primero encontramos el mcd entre $59$ y $10$:
\begin{alignat}3
59 &= 10 \cdot 5 + 9& \qquad&\Rightarrow& \qquad 9 &= 59 -  10 \cdot 5 \label{eq:cle1}\\ 
10 &=  9\cdot 1 +1 & \qquad&\Rightarrow& \qquad 1 &= 10 - 9 \label{eq:cle2}\qquad\\
9 &=  1\cdot 9  +0 & \qquad&& \qquad  &
\end{alignat}
Luego, $1=(59,10)$ y 
\begin{align*}
1 & \overset{(\ref{eq:cle2})}{=} 10 -9  \\
& \overset{(\ref{eq:cle1})}{=}  10 -(59 -  10 \cdot 5) = 10 -59 + 5\cdot  10 \\
&=  6 \cdot  10 + (-1) \cdot 59.
\end{align*}

Por lo tanto
\begin{equation*}
	10 \cdot 6 \equiv 1 (59). 
\end{equation*}
Multiplicando  por $8$,  tenemos
\begin{equation*}
10 \cdot 48 \equiv 8 (59). 
\end{equation*}
Por lo tanto $x_0=48$ es solución y todas las soluciones son de la forma $48+ k\cdot 59$, con $k \in \mathbb Z$.
\vskip .4cm



1(ii). (20 pts) Por (i), las soluciones son de la forma  $48+ k\cdot 59$, con $k \in \mathbb Z$, por lo tanto son soluciones
\begin{equation*}
	48 +(-2)\cdot 59 = -70,\quad 48 +(-1)\cdot 59 = -11,\quad  48 +0 \cdot 59= 48, \quad   48 +1 \cdot 59 =107, \quad   48 +2 \cdot 59 =166.,
\end{equation*}
Luego, las soluciones $x$ tal que $0 < x < 150$ son $x= 48$ y $x=107$. 
\vskip .4cm
\newpage
	2. Por una de la versiones del teorema de Fermat, si  $p$ ves primo  y $p\not= a$, tenemos que
	\begin{equation*}
		a^{p-1} \equiv 1 \,(p).
	\end{equation*}
	Como $13$ es primo y $1=(7,13)$,  tenemos que:
	\begin{equation*}
	7^{12} \equiv 1 \,(13).
	\end{equation*}
	Como $98 = 12 \cdot 8 + 2$, 
	\begin{equation*}
		7^{98} \equiv 7^{12 \cdot 8 + 2}  \equiv 7^{12 \cdot 8}\, 7^{2} \equiv (7^{12})^8\, 7^{2}\equiv 1^8\, 7^{2}\equiv 49 \equiv 3\cdot 13 +10  \equiv 10  \, (13).
	\end{equation*}
	
	
\end{solucion}

\end{comment}
\end{document}


	2. 