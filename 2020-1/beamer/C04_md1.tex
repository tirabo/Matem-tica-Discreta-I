%\documentclass{beamer} 
\documentclass[handout]{beamer} % sin pausas
\usetheme{CambridgeUS}

\usepackage{etex}
\usepackage{t1enc}
\usepackage[spanish,es-nodecimaldot]{babel}
\usepackage{latexsym}
\usepackage[utf8]{inputenc}
\usepackage{verbatim}
\usepackage{multicol}
\usepackage{amsgen,amsmath,amstext,amsbsy,amsopn,amsfonts,amssymb}
\usepackage{amsthm}
\usepackage{calc}         % From LaTeX distribution
\usepackage{graphicx}     % From LaTeX distribution
\usepackage{ifthen}
%\usepackage{makeidx}
\input{random.tex}        % From CTAN/macros/generic
\usepackage{subfigure} 
\usepackage{tikz}
\usepackage[customcolors]{hf-tikz}
\usetikzlibrary{arrows}
\usetikzlibrary{matrix}
\tikzset{
	every picture/.append style={
		execute at begin picture={\deactivatequoting},
		execute at end picture={\activatequoting}
	}
}
\usetikzlibrary{decorations.pathreplacing,angles,quotes}
\usetikzlibrary{shapes.geometric}
\usepackage{mathtools}
\usepackage{stackrel}
%\usepackage{enumerate}
\usepackage{enumitem}
\usepackage{tkz-graph}
\usepackage{polynom}
\polyset{%
	style=B,
	delims={(}{)},
	div=:
}
\renewcommand\labelitemi{$\circ$}
\setlist[enumerate]{label={(\arabic*)}}
%\setbeamertemplate{background}[grid][step=8 ] % cuadriculado
\setbeamertemplate{itemize item}{$\circ$}
\setbeamertemplate{enumerate items}[default]
\definecolor{links}{HTML}{2A1B81}
\hypersetup{colorlinks,linkcolor=,urlcolor=links}


\newcommand{\Id}{\operatorname{Id}}
\newcommand{\img}{\operatorname{Im}}
\newcommand{\nuc}{\operatorname{Nu}}
\newcommand{\im}{\operatorname{Im}}
\renewcommand\nu{\operatorname{Nu}}
\newcommand{\la}{\langle}
\newcommand{\ra}{\rangle}
\renewcommand{\t}{{\operatorname{t}}}
\renewcommand{\sin}{{\,\operatorname{sen}}}
\newcommand{\Q}{\mathbb Q}
\newcommand{\R}{\mathbb R}
\newcommand{\C}{\mathbb C}
\newcommand{\K}{\mathbb K}
\newcommand{\F}{\mathbb F}
\newcommand{\Z}{\mathbb Z}
\newcommand{\N}{\mathbb N}
\newcommand\sgn{\operatorname{sgn}}
\renewcommand{\t}{{\operatorname{t}}}
\renewcommand{\figurename }{Figura}

%
% Ver http://joshua.smcvt.edu/latex2e/_005cnewenvironment-_0026-_005crenewenvironment.html
%

\renewenvironment{block}[1]% environment name
{% begin code
	\par\vskip .2cm%
	{\color{blue}#1}%
	\vskip .2cm
}%
{%
	\vskip .2cm}% end code


\renewenvironment{alertblock}[1]% environment name
{% begin code
	\par\vskip .2cm%
	{\color{red!80!black}#1}%
	\vskip .2cm
}%
{%
	\vskip .2cm}% end code


\renewenvironment{exampleblock}[1]% environment name
{% begin code
	\par\vskip .2cm%
	{\color{blue}#1}%
	\vskip .2cm
}%
{%
	\vskip .2cm}% end code




\newenvironment{exercise}[1]% environment name
{% begin code
	\par\vspace{\baselineskip}\noindent
	\textbf{Ejercicio (#1)}\begin{itshape}%
		\par\vspace{\baselineskip}\noindent\ignorespaces
	}%
	{% end code
	\end{itshape}\ignorespacesafterend
}


\newenvironment{definicion}[1][]% environment name
{% begin code
	\par\vskip .2cm%
	{\color{blue}Definición #1}%
	\vskip .2cm
}%
{%
	\vskip .2cm}% end code

    \newenvironment{notacion}[1][]% environment name
    {% begin code
        \par\vskip .2cm%
        {\color{blue}Notación #1}%
        \vskip .2cm
    }%
    {%
        \vskip .2cm}% end code

\newenvironment{observacion}[1][]% environment name
{% begin code
	\par\vskip .2cm%
	{\color{blue}Observación #1}%
	\vskip .2cm
}%
{%
	\vskip .2cm}% end code

\newenvironment{ejemplo}[1][]% environment name
{% begin code
	\par\vskip .2cm%
	{\color{blue}Ejemplo #1}%
	\vskip .2cm
}%
{%
	\vskip .2cm}% end code


\newenvironment{preguntas}[1][]% environment name
{% begin code
    \par\vskip .2cm%
    {\color{blue}Preguntas #1}%
    \vskip .2cm
}%
{%
    \vskip .2cm}% end code

\newenvironment{ejercicio}[1][]% environment name
{% begin code
	\par\vskip .2cm%
	{\color{blue}Ejercicio #1}%
	\vskip .2cm
}%
{%
	\vskip .2cm}% end code


\renewenvironment{proof}% environment name
{% begin code
	\par\vskip .2cm%
	{\color{blue}Demostración}%
	\vskip .2cm
}%
{%
	\vskip .2cm}% end code



\newenvironment{demostracion}% environment name
{% begin code
	\par\vskip .2cm%
	{\color{blue}Demostración}%
	\vskip .2cm
}%
{%
	\vskip .2cm}% end code

\newenvironment{idea}% environment name
{% begin code
	\par\vskip .2cm%
	{\color{blue}Idea de la demostración}%
	\vskip .2cm
}%
{%
	\vskip .2cm}% end code

\newenvironment{solucion}% environment name
{% begin code
	\par\vskip .2cm%
	{\color{blue}Solución}%
	\vskip .2cm
}%
{%
	\vskip .2cm}% end code



\newenvironment{lema}[1][]% environment name
{% begin code
	\par\vskip .2cm%
	{\color{blue}Lema #1}\begin{itshape}%
		\par\vskip .2cm
	}%
	{% end code
	\end{itshape}\vskip .2cm\ignorespacesafterend
}

\newenvironment{proposicion}[1][]% environment name
{% begin code
	\par\vskip .2cm%
	{\color{blue}Proposición #1}\begin{itshape}%
		\par\vskip .2cm
	}%
	{% end code
	\end{itshape}\vskip .2cm\ignorespacesafterend
}

\newenvironment{teorema}[1][]% environment name
{% begin code
	\par\vskip .2cm%
	{\color{blue}Teorema #1}\begin{itshape}%
		\par\vskip .2cm
	}%
	{% end code
	\end{itshape}\vskip .2cm\ignorespacesafterend
}


\newenvironment{corolario}[1][]% environment name
{% begin code
	\par\vskip .2cm%
	{\color{blue}Corolario #1}\begin{itshape}%
		\par\vskip .2cm
	}%
	{% end code
	\end{itshape}\vskip .2cm\ignorespacesafterend
}

\newenvironment{propiedad}% environment name
{% begin code
	\par\vskip .2cm%
	{\color{blue}Propiedad}\begin{itshape}%
		\par\vskip .2cm
	}%
	{% end code
	\end{itshape}\vskip .2cm\ignorespacesafterend
}

\newenvironment{conclusion}% environment name
{% begin code
	\par\vskip .2cm%
	{\color{blue}Conclusión}\begin{itshape}%
		\par\vskip .2cm
	}%
	{% end code
	\end{itshape}\vskip .2cm\ignorespacesafterend
}


\newenvironment{definicion*}% environment name
{% begin code
	\par\vskip .2cm%
	{\color{blue}Definición}%
	\vskip .2cm
}%
{%
	\vskip .2cm}% end code

\newenvironment{observacion*}% environment name
{% begin code
	\par\vskip .2cm%
	{\color{blue}Observación}%
	\vskip .2cm
}%
{%
	\vskip .2cm}% end code


\newenvironment{obs*}% environment name
	{% begin code
		\par\vskip .2cm%
		{\color{blue}Observación}%
		\vskip .2cm
	}%
	{%
		\vskip .2cm}% end code

\newenvironment{ejemplo*}% environment name
{% begin code
	\par\vskip .2cm%
	{\color{blue}Ejemplo}%
	\vskip .2cm
}%
{%
	\vskip .2cm}% end code

\newenvironment{ejercicio*}% environment name
{% begin code
	\par\vskip .2cm%
	{\color{blue}Ejercicio}%
	\vskip .2cm
}%
{%
	\vskip .2cm}% end code

\newenvironment{propiedad*}% environment name
{% begin code
	\par\vskip .2cm%
	{\color{blue}Propiedad}\begin{itshape}%
		\par\vskip .2cm
	}%
	{% end code
	\end{itshape}\vskip .2cm\ignorespacesafterend
}

\newenvironment{conclusion*}% environment name
{% begin code
	\par\vskip .2cm%
	{\color{blue}Conclusión}\begin{itshape}%
		\par\vskip .2cm
	}%
	{% end code
	\end{itshape}\vskip .2cm\ignorespacesafterend
}






\newcommand{\nc}{\newcommand}

%%%%%%%%%%%%%%%%%%%%%%%%%LETRAS

\nc{\FF}{{\mathbb F}} \nc{\NN}{{\mathbb N}} \nc{\QQ}{{\mathbb Q}}
\nc{\PP}{{\mathbb P}} \nc{\DD}{{\mathbb D}} \nc{\Sn}{{\mathbb S}}
\nc{\uno}{\mathbb{1}} \nc{\BB}{{\mathbb B}} \nc{\An}{{\mathbb A}}

\nc{\ba}{\mathbf{a}} \nc{\bb}{\mathbf{b}} \nc{\bt}{\mathbf{t}}
\nc{\bB}{\mathbf{B}}

\nc{\cP}{\mathcal{P}} \nc{\cU}{\mathcal{U}} \nc{\cX}{\mathcal{X}}
\nc{\cE}{\mathcal{E}} \nc{\cS}{\mathcal{S}} \nc{\cA}{\mathcal{A}}
\nc{\cC}{\mathcal{C}} \nc{\cO}{\mathcal{O}} \nc{\cQ}{\mathcal{Q}}
\nc{\cB}{\mathcal{B}} \nc{\cJ}{\mathcal{J}} \nc{\cI}{\mathcal{I}}
\nc{\cM}{\mathcal{M}} \nc{\cK}{\mathcal{K}}

\nc{\fD}{\mathfrak{D}} \nc{\fI}{\mathfrak{I}} \nc{\fJ}{\mathfrak{J}}
\nc{\fS}{\mathfrak{S}} \nc{\gA}{\mathfrak{A}}
%%%%%%%%%%%%%%%%%%%%%%%%%LETRAS

\title[Clase 4 - Inducción]{Matemática Discreta I \\ Clase 4 - Inducción}
%\author[C. Olmos / A. Tiraboschi]{Carlos Olmos / Alejandro Tiraboschi}
\institute[]{\normalsize FAMAF / UNC
	\\[\baselineskip] ${}^{}$
	\\[\baselineskip]
}
\date[02/04/2020]{31 de marzo   de 2020}




\begin{document}
%\title{El centro geográfico de Argentina}   
%\author{} 
%\date{Villa Huidobro \\ 4/12/2018} 



\frame{\titlepage} 

%\frame{\frametitle{Índice}\tableofcontents} 




\begin{frame}\frametitle{El principio de inducción} 

	Queremos  analizar la suma de los primeros $n$ números impares, es decir
	$$
	1+3+5+\cdots+(2n-1).
	$$
	
	Por la definición recursiva de la sumatoria,  tenemos que 
	\begin{equation*}
	a_1 = 1, \;\text{  y } \; a_n = a_{n-1} + 2n-1,
	\end{equation*}
	
	
	Analicemos los primero valores
	
	\begin{itemize}
		\item $a_1 = 1$,
		\item $a_2 = 1 + 3 = 4$, 
		\item $a_3 = 1 + 3 + 5 = 9$, 
		\item $a_4 = 1 + 3 + 5 + 7= 16 $, 
	\end{itemize}
	
\end{frame}


\begin{frame}	
	Entonces, podemos conjeturar que
	$$
	1+3+5+\cdots+(2n-1) = n^2.
	$$
	
	Para convencernos de que la fórmula es ciertamente correcta procedemos de la siguiente manera: cuando $n=1$ puesto que
	$1=1^2$.
	Supongamos que es correcta para un valor específico
	de $n$, digamos para $n=k$, de modo que
	$$
	1+3+5+\cdots+(2k-1) = k^2.
	$$



\end{frame}

\begin{frame}
		Podemos usar esto para simplificar la expresión definida
	recursivamente a la izquierda cuando $n$ es igual a $k+1$,
	$$
	\begin{aligned}
	1+3+5+\cdots+(2k+1) &= 1+3+5+\cdots+(2k-1) +(2k+1) \\
	&=k^2 +(2k+1) \\
	&=(k+1)^2.
	\end{aligned}
	$$
	Por lo tanto si el resultado es correcto cuando $n=k$, entonces lo es cuando $n=k+1$. Se comienza observando que si es correcto cuando $n=1$, debe ser por lo tanto correcto cuando $n=2$. Con el mismo argumento como es correcto cuando $n=2$ debe serlo cuando $n=3$. Continuando de esta forma veremos que es correcto para todos los enteros positivos $n$.
\end{frame}

\begin{frame}
		
	La esencia de este argumento es comúnmente llamada {\it principio de inducción}.
	
	\medskip 
	
	 Con $S$ denotemos al subconjunto de $\mathbb N$ para el cual el resultado es correcto: por supuesto, nuestra intención es probar que $S$ es todo $\mathbb N$. 
	 
	 \medskip 
	 
	%\begin{teorema}\label{t1.4} 
	{\color{blue} Teorema }
	\vskip .2cm 
	{\it
		Supongamos que $S$ es un subconjunto de $\mathbb N$ que satisface las condiciones \index{principio de inducción}
		\begin{enumerate}
			\item[a)] $1 \in S$,
			\item[b)] para cada $k \in \mathbb N$, si $ k \in S$ entonces $k+1\in S$.
		\end{enumerate}
		Entonces se sigue que $S=\mathbb N$.
}
	%\end{teorema}
\end{frame}

\begin{frame}


	%\begin{proof}[Demostración]
	
	{\color{blue} Demostración.}
	\vskip .2cm 
		Si la conclusión es falsa, $S \not= \mathbb N$ y
		el conjunto complementario $S^{\text{c}}$ definido por
		$$
		S^{\text{c}}= \{ r \in \mathbb N | r\not\in S\}
		$$
		es no vacío. \pause 
		
		\medskip 
		
		Por el axioma del buen orden, $S^{\text{c}}$ tiene un menor
		elemento (mínimo) $m$. 
		
		\pause\medskip 
		
		Como $1$ pertenece a $S$, $m\not=1$. Se sigue que
		$m-1$ pertenece a $\mathbb N$ y como $m$ es el mínimo de
		$S^{\text{c}}$, $m-1$ debe pertenecer a $S$.
		
		\pause\medskip 
		
		 Poniendo $k=m-1$ en
		la condición (b), concluimos que $m$ esta en $S$, lo cual
		contradice el hecho de que $m$ pertenece a $S^{\text{c}}$.
		
		\pause\medskip 
		
		 De este
		modo, la suposición $S \not= \mathbb N$ nos lleva a un absurdo, y
		por lo tanto tenemos $S= \mathbb N$.
	%\end{proof}
\end{frame}

\begin{frame}	

	
	El principio de inducción es útil para probar la veracidad de propiedades relativas a los
	números naturales. Por ejemplo, consideremos la siguiente propiedade $P(n)$:
	\medskip 
	\begin{itemize}
		\item $P(n)$ es la propiedad: $2n -1 < n^2 + 1$,
	\end{itemize}
\medskip 
	Intuitivamente notamos que $P(n)$ es verdadera para cualquier $n$ natural y lo podemos probar usando el siguiente teorema: 
	\medskip 



\end{frame}

\begin{frame}	
	
	
	
	%\begin{teorema}[Principio de inducción] 
	{\color{blue} Teorema (Principio de inducción)}
	\vskip .2cm 
	{\it Sea $P(n)$ una propiedad para $n \in \mathbb N$ tal que:
		\begin{enumerate}
			\item[a)] $P(1)$ es verdadera.
			\item[b)] Para todo $k \in \mathbb N$, $P(k)$ verdadera implica $P(k + 1)$ verdadera.
		\end{enumerate}
		Entonces $P(n)$ es verdadera para todo $n \in \mathbb N$. }
	%\end{teorema}
	\vskip .2cm 
%\begin{proof}[Demostración] 
	{\color{blue} Demostración.}
	\vskip .2cm 
	{\it 
	\pause
	 Basta tomar
	$$S = \{n \in \mathbb N| P(n) \text{ es verdadera} \}.$$ \pause
	Entonces $S$ es un subconjunto de $\mathbb N$ y las condiciones (a) y (b) nos dicen que $1 \in S$ y  si $ k \in S$ entonces $k+1\in S$. 
	
	\pause Por el teorema anterior se sigue que $S= \mathbb N$, es decir que $P(n)$ es verdadera para todo $n$.
	natural.
	
	\vskip 1cm}
%\end{proof}
	 
	
\end{frame}

\begin{frame}	

	
		En la práctica, generalmente presentamos una ``demostración por
	inducción'' en términos más descriptivos. 
	
	\medskip
	
	En la notación del teorema anterior, 
	\medskip
	\begin{itemize}
		\item (a) es llamado  el {\em caso base},
		\item  (b) es llamado el  {\em paso inductivo} y
		\item  $P(k)$ es llamada la {\em hipótesis inductiva}.
	\end{itemize}
	\medskip
	 El paso inductivo  consiste en probar que $P(k) \Rightarrow P(k + 1)$ o, equivalentemente, podemos suponer $P(k)$ verdadera y a partir de ella probar $P(k + 1)$. 
	
\end{frame}

\begin{frame}
		
	%\begin{ejemplo} 
	{\color{blue}Ejemplo}
	\vskip .2cm 
	El entero $x_n$ esta definido recursivamente por
		$$
		x_1=2, \qquad x_n=x_{n-1} +2n, \qquad n\ge 2.
		$$
		Demostremos que  \qquad 
		$
		x_n = n(n+1)  \text{ para todo } n\in \mathbb N.
		$
	%\end{ejemplo}
	\vskip .2cm
%\begin{proof}[Demostración]
	{\color{blue} Demostración.}
	\vskip 3.5cm
	
	${}^{}$
%\end{proof}

\end{frame}

\begin{frame}

	%\begin{proof}[Demostración]
		\vskip .2cm
	%\begin{proof}[Demostración]
	{\color{blue} Demostración.}\pause
			\vskip .2cm
		\noindent({\it Caso  base}) El resultado es verdadero
		cuando $n=1$ pues $ 2 = 1 \cdot 2$.
		
		\medskip\pause
		
		\noindent ({\it Paso  inductivo})
		Supongamos que el resultado verdadero cuando $n=k$, o
		sea, que 
		\begin{center}
			$x_k = k(k+1)$ hipótesis inductiva (HI).
		\end{center}
		
		\pause
	 Entonces

		\medskip
		
		
		\begin{tabular}{lllll}
		$x_{k+1}$ &$=$& $x_k + 2(k+1)$ &\qquad &$\text{(por la definición recursiva)}$ \\
		&$=$& $k(k+1)+2(k+1)$ &\qquad &$\text{(por hipótesis inductiva)}$ \\
		&$=$& $(k+1)(k+2).$&\qquad &$\text{($(k+1)$ factor común)}$
		\end{tabular}
		\medskip
		
		
		Luego el resultado es verdadero cuando $n=k+1$ y por el principio de inducción, es verdadero para todos los enteros positivos $n$.
	%\end{proof}
	
\end{frame}

\begin{frame}
		
	Existen varias formas modificadas del principio de inducción. A veces es conveniente tomar como base inductiva el valor $n=0$, por otro lado puede ser apropiado tomar un valor como $2$ o $3$ porque los primeros casos pueden ser excepcionales. 
	
	\medskip
	
	Cada problema debe ser tratado según sus características. 
	
	\medskip
	
	Otra modificación útil es tomar como hipótesis inductiva la suposición de que el resultado es verdadero para todos los valores $n\le k$, más que para $n=k$ solamente.
	
	\medskip
	
	Esta formulación es llamada a veces el {\it principio de inducción completa}. Todas esas modificaciones pueden justificarse con cambios triviales en la demostración del principio de inducción.
	
\end{frame}

\begin{frame}	
	El siguiente teorema incorpora todas las modificaciones del principio de inducción mencionadas más arriba.


	\medskip
	

	
	%\begin{teorema}[Inducción completa]\label{ind-completa} 
	{\color{blue} Teorema (Inducción completa)}
	\vskip .2cm	
		Sea $n_0$ número entero y sea $P(n)$ una propiedad para $n \ge n_0$ tal que:
		\begin{enumerate}
			\item[a)] $P(n_0)$ es verdadera.
			\item[b)] Si $P(h)$ verdadera para toda $h$ tal que $n_0 \le h \le k$ implica $P(k + 1)$ verdadera.
		\end{enumerate}
		Entonces $P(n)$ es verdadera para todo $n \ge n_0$.
	%\end{teorema}

	
\end{frame}



\begin{frame}		
	
	%\begin{ejemplo}
		{\color{blue} Ejemplo}	
	$$u_1 = 3,\qquad u_2 = 5,\qquad u_n = 3u_{n-1}- 2u_{n-2},\qquad  n \ge 3.$$
		Probemos que $u_n = 2^n + 1$, para todo $n \in  \mathbb N$.
	%\end{ejemplo}
		%\begin{proof}[Solución] 
		\vskip .2cm	
		{\color{blue} Solución.}
		\vskip .2cm	
		\pause
			
			\noindent({\it Caso  base}) $n= 1$ :  $3 = 2^1+1$ \checkmark, \qquad   $n=2$ :  $ 5 =2^2+1$ \checkmark.\pause
			
			\medskip
			
			\noindent ({\it Paso  inductivo}) Hipótesis inductiva:
			$$
			u_h = 2^h+1 \text{ para } 1 \le h \le k \text{  y } k \ge 2 \text{ (HI), }
			$$ 
			\pause
			entonces,
			
			\begin{tabular}{llll}
			$u_{k+1}$ &$= 3u_k -2u_{k-1}$ &\qquad &{(por definición recursiva)}  \\
				&$= 3(2^k+1)-2(2^{k-1}+1)$  &\qquad &{(por hipótesis inductiva)} \\
				&= $3\cdot 2^k+3-2\cdot 2^{k-1}-2$ &\qquad & \\
				&= $3\cdot 2^k+1- 2^{k}$  &\qquad & \\
				&= $2\cdot 2^k+1$  &\qquad & \\
				&= $2^{k+1}+1.$  &\qquad & 
			\end{tabular}

		%\end{proof}
	
	
\end{frame}

\end{document}

