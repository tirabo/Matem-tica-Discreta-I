%\documentclass{beamer} 
\documentclass[handout]{beamer} % sin pausas
\usetheme{CambridgeUS}

\usepackage{etex}
\usepackage{t1enc}
\usepackage[spanish,es-nodecimaldot]{babel}
\usepackage{latexsym}
\usepackage[utf8]{inputenc}
\usepackage{verbatim}
\usepackage{multicol}
\usepackage{amsgen,amsmath,amstext,amsbsy,amsopn,amsfonts,amssymb}
\usepackage{amsthm}
\usepackage{calc}         % From LaTeX distribution
\usepackage{graphicx}     % From LaTeX distribution
\usepackage{ifthen}
%\usepackage{makeidx}
\input{random.tex}        % From CTAN/macros/generic
\usepackage{subfigure} 
\usepackage{tikz}
\usepackage[customcolors]{hf-tikz}
\usetikzlibrary{arrows}
\usetikzlibrary{matrix}
\tikzset{
	every picture/.append style={
		execute at begin picture={\deactivatequoting},
		execute at end picture={\activatequoting}
	}
}
\usetikzlibrary{decorations.pathreplacing,angles,quotes}
\usetikzlibrary{shapes.geometric}
\usepackage{mathtools}
\usepackage{stackrel}
%\usepackage{enumerate}
\usepackage{enumitem}
\usepackage{tkz-graph}
\usepackage{polynom}
\polyset{%
	style=B,
	delims={(}{)},
	div=:
}
\renewcommand\labelitemi{$\circ$}
\setlist[enumerate]{label={(\arabic*)}}
%\setbeamertemplate{background}[grid][step=8 ] % cuadriculado
\setbeamertemplate{itemize item}{$\circ$}
\setbeamertemplate{enumerate items}[default]
\definecolor{links}{HTML}{2A1B81}
\hypersetup{colorlinks,linkcolor=,urlcolor=links}


\newcommand{\Id}{\operatorname{Id}}
\newcommand{\img}{\operatorname{Im}}
\newcommand{\nuc}{\operatorname{Nu}}
\newcommand{\im}{\operatorname{Im}}
\renewcommand\nu{\operatorname{Nu}}
\newcommand{\la}{\langle}
\newcommand{\ra}{\rangle}
\renewcommand{\t}{{\operatorname{t}}}
\renewcommand{\sin}{{\,\operatorname{sen}}}
\newcommand{\Q}{\mathbb Q}
\newcommand{\R}{\mathbb R}
\newcommand{\C}{\mathbb C}
\newcommand{\K}{\mathbb K}
\newcommand{\F}{\mathbb F}
\newcommand{\Z}{\mathbb Z}
\newcommand{\N}{\mathbb N}
\newcommand\sgn{\operatorname{sgn}}
\renewcommand{\t}{{\operatorname{t}}}
\renewcommand{\figurename }{Figura}

%
% Ver http://joshua.smcvt.edu/latex2e/_005cnewenvironment-_0026-_005crenewenvironment.html
%

\renewenvironment{block}[1]% environment name
{% begin code
	\par\vskip .2cm%
	{\color{blue}#1}%
	\vskip .2cm
}%
{%
	\vskip .2cm}% end code


\renewenvironment{alertblock}[1]% environment name
{% begin code
	\par\vskip .2cm%
	{\color{red!80!black}#1}%
	\vskip .2cm
}%
{%
	\vskip .2cm}% end code


\renewenvironment{exampleblock}[1]% environment name
{% begin code
	\par\vskip .2cm%
	{\color{blue}#1}%
	\vskip .2cm
}%
{%
	\vskip .2cm}% end code




\newenvironment{exercise}[1]% environment name
{% begin code
	\par\vspace{\baselineskip}\noindent
	\textbf{Ejercicio (#1)}\begin{itshape}%
		\par\vspace{\baselineskip}\noindent\ignorespaces
	}%
	{% end code
	\end{itshape}\ignorespacesafterend
}


\newenvironment{definicion}[1][]% environment name
{% begin code
	\par\vskip .2cm%
	{\color{blue}Definición #1}%
	\vskip .2cm
}%
{%
	\vskip .2cm}% end code

    \newenvironment{notacion}[1][]% environment name
    {% begin code
        \par\vskip .2cm%
        {\color{blue}Notación #1}%
        \vskip .2cm
    }%
    {%
        \vskip .2cm}% end code

\newenvironment{observacion}[1][]% environment name
{% begin code
	\par\vskip .2cm%
	{\color{blue}Observación #1}%
	\vskip .2cm
}%
{%
	\vskip .2cm}% end code

\newenvironment{ejemplo}[1][]% environment name
{% begin code
	\par\vskip .2cm%
	{\color{blue}Ejemplo #1}%
	\vskip .2cm
}%
{%
	\vskip .2cm}% end code


\newenvironment{preguntas}[1][]% environment name
{% begin code
    \par\vskip .2cm%
    {\color{blue}Preguntas #1}%
    \vskip .2cm
}%
{%
    \vskip .2cm}% end code

\newenvironment{ejercicio}[1][]% environment name
{% begin code
	\par\vskip .2cm%
	{\color{blue}Ejercicio #1}%
	\vskip .2cm
}%
{%
	\vskip .2cm}% end code


\renewenvironment{proof}% environment name
{% begin code
	\par\vskip .2cm%
	{\color{blue}Demostración}%
	\vskip .2cm
}%
{%
	\vskip .2cm}% end code



\newenvironment{demostracion}% environment name
{% begin code
	\par\vskip .2cm%
	{\color{blue}Demostración}%
	\vskip .2cm
}%
{%
	\vskip .2cm}% end code

\newenvironment{idea}% environment name
{% begin code
	\par\vskip .2cm%
	{\color{blue}Idea de la demostración}%
	\vskip .2cm
}%
{%
	\vskip .2cm}% end code

\newenvironment{solucion}% environment name
{% begin code
	\par\vskip .2cm%
	{\color{blue}Solución}%
	\vskip .2cm
}%
{%
	\vskip .2cm}% end code



\newenvironment{lema}[1][]% environment name
{% begin code
	\par\vskip .2cm%
	{\color{blue}Lema #1}\begin{itshape}%
		\par\vskip .2cm
	}%
	{% end code
	\end{itshape}\vskip .2cm\ignorespacesafterend
}

\newenvironment{proposicion}[1][]% environment name
{% begin code
	\par\vskip .2cm%
	{\color{blue}Proposición #1}\begin{itshape}%
		\par\vskip .2cm
	}%
	{% end code
	\end{itshape}\vskip .2cm\ignorespacesafterend
}

\newenvironment{teorema}[1][]% environment name
{% begin code
	\par\vskip .2cm%
	{\color{blue}Teorema #1}\begin{itshape}%
		\par\vskip .2cm
	}%
	{% end code
	\end{itshape}\vskip .2cm\ignorespacesafterend
}


\newenvironment{corolario}[1][]% environment name
{% begin code
	\par\vskip .2cm%
	{\color{blue}Corolario #1}\begin{itshape}%
		\par\vskip .2cm
	}%
	{% end code
	\end{itshape}\vskip .2cm\ignorespacesafterend
}

\newenvironment{propiedad}% environment name
{% begin code
	\par\vskip .2cm%
	{\color{blue}Propiedad}\begin{itshape}%
		\par\vskip .2cm
	}%
	{% end code
	\end{itshape}\vskip .2cm\ignorespacesafterend
}

\newenvironment{conclusion}% environment name
{% begin code
	\par\vskip .2cm%
	{\color{blue}Conclusión}\begin{itshape}%
		\par\vskip .2cm
	}%
	{% end code
	\end{itshape}\vskip .2cm\ignorespacesafterend
}


\newenvironment{definicion*}% environment name
{% begin code
	\par\vskip .2cm%
	{\color{blue}Definición}%
	\vskip .2cm
}%
{%
	\vskip .2cm}% end code

\newenvironment{observacion*}% environment name
{% begin code
	\par\vskip .2cm%
	{\color{blue}Observación}%
	\vskip .2cm
}%
{%
	\vskip .2cm}% end code


\newenvironment{obs*}% environment name
	{% begin code
		\par\vskip .2cm%
		{\color{blue}Observación}%
		\vskip .2cm
	}%
	{%
		\vskip .2cm}% end code

\newenvironment{ejemplo*}% environment name
{% begin code
	\par\vskip .2cm%
	{\color{blue}Ejemplo}%
	\vskip .2cm
}%
{%
	\vskip .2cm}% end code

\newenvironment{ejercicio*}% environment name
{% begin code
	\par\vskip .2cm%
	{\color{blue}Ejercicio}%
	\vskip .2cm
}%
{%
	\vskip .2cm}% end code

\newenvironment{propiedad*}% environment name
{% begin code
	\par\vskip .2cm%
	{\color{blue}Propiedad}\begin{itshape}%
		\par\vskip .2cm
	}%
	{% end code
	\end{itshape}\vskip .2cm\ignorespacesafterend
}

\newenvironment{conclusion*}% environment name
{% begin code
	\par\vskip .2cm%
	{\color{blue}Conclusión}\begin{itshape}%
		\par\vskip .2cm
	}%
	{% end code
	\end{itshape}\vskip .2cm\ignorespacesafterend
}






\newcommand{\nc}{\newcommand}

%%%%%%%%%%%%%%%%%%%%%%%%%LETRAS

\nc{\FF}{{\mathbb F}} \nc{\NN}{{\mathbb N}} \nc{\QQ}{{\mathbb Q}}
\nc{\PP}{{\mathbb P}} \nc{\DD}{{\mathbb D}} \nc{\Sn}{{\mathbb S}}
\nc{\uno}{\mathbb{1}} \nc{\BB}{{\mathbb B}} \nc{\An}{{\mathbb A}}

\nc{\ba}{\mathbf{a}} \nc{\bb}{\mathbf{b}} \nc{\bt}{\mathbf{t}}
\nc{\bB}{\mathbf{B}}

\nc{\cP}{\mathcal{P}} \nc{\cU}{\mathcal{U}} \nc{\cX}{\mathcal{X}}
\nc{\cE}{\mathcal{E}} \nc{\cS}{\mathcal{S}} \nc{\cA}{\mathcal{A}}
\nc{\cC}{\mathcal{C}} \nc{\cO}{\mathcal{O}} \nc{\cQ}{\mathcal{Q}}
\nc{\cB}{\mathcal{B}} \nc{\cJ}{\mathcal{J}} \nc{\cI}{\mathcal{I}}
\nc{\cM}{\mathcal{M}} \nc{\cK}{\mathcal{K}}

\nc{\fD}{\mathfrak{D}} \nc{\fI}{\mathfrak{I}} \nc{\fJ}{\mathfrak{J}}
\nc{\fS}{\mathfrak{S}} \nc{\gA}{\mathfrak{A}}
%%%%%%%%%%%%%%%%%%%%%%%%%LETRAS



\title[Clase 10 - Cociente y resto]{Matemática Discreta I \\ Clase 10 - Cociente y resto}
%\author[C. Olmos / A. Tiraboschi]{Carlos Olmos / Alejandro Tiraboschi}
\institute[]{\normalsize FAMAF / UNC
	\\[\baselineskip] ${}^{}$
	\\[\baselineskip]
}
\date[21/04/2020]{21 de abril  de 2020}




\begin{document}
	
	\frame{\titlepage} 
	
	
	\begin{frame}\frametitle{Cociente y resto}
		Cuando somos chicos aprendemos que $6$ ``cabe'' cuatro veces en $27$ y
		el resto es $3$, o sea
		$$
		27=6 \cdot 4 + 3.
		$$
		Un punto importante es que el resto debe ser menor que 6. Aunque,
		también es verdadero que, por ejemplo
		$$
		27=6 \cdot 3 + 9,
		$$
		debemos tomar el menor valor para el resto, de forma que ``lo que
		queda'' sea un número no negativo lo más chico posible. 
		\vskip .2cm\pause
		El hecho de que el conjunto de
		posibles ``restos'' tenga un mínimo es una consecuencia del \textit{axioma
			del buen orden.}
		
		
	\end{frame}
	
	
	\begin{frame}
		
		
		\vskip .4cm
		{\color{blue}Teorema (Algoritmo de división)}
		\vskip .2cm
		{\it Sean $a$ y $b$ números enteros
			cualesquiera con $b \in \mathbb N$, entonces existen enteros únicos $q$ y
			$r$ tales que
			$$
			a=b \cdot q + r\qquad\text{ y }\qquad 0\le r<b.
			$$
		}\pause
		\vskip 1cm
		A la $q$ del teorema anterior lo llamaremos el cociente de $b$  por $a$.
		\vskip .2cm
		A $r$ lo llamamos el resto de dividir $b$ por $a$
		\vskip .2cm
		
	\end{frame}
	
	
	\begin{frame}
		
		{\color{blue}Idea de la prueba.}
		\vskip .2cm
		Debemos aplicar el axioma del buen orden al
		conjunto de los ``restos''
		$$ R=\{x \in \mathbb N_0 | a = by + x \ \text{ para algún }\ y \in \mathbb Z\}.
		$$
		\pause
		
		Primero se demuestra que $R$ no es vacío. 
		\pause
		
		Entonces, por el axioma del buen orden se encuentra un  mínimo $r \in R$. Es decir existe $q \in \mathbb{Z}$ tal que 
		$$
		a=b \cdot q + r\qquad\text{ y $r$ es mínimo de R} .
		$$\pause  
		Luego se prueba  que $0\le r<b$.
	\end{frame}
	
	
	\begin{frame}
		
		
		{\color{blue}Ejemplos}
		\vskip .2cm
		\begin{enumerate}
			\item[1. ] Si $a=10$ y $b=3$, entonces $10 = 3 \cdot 3 +1$. Es decir $q= 3$, $r=1$. \pause\vskip .3cm
			\item[2. ] Si $a=2$ y $b=5$, entonces $2 = 5 \cdot 0 +2$. Es decir $q= 0$, $r=2$.  \pause\vskip .3cm
			\item[3. ] Si $a=-10$ y $b=3$, entonces $-10 = 3 \cdot (-4) +2$. Es decir $q= -4$, $r=2$.    \pause\vskip .3cm 
			\item[4. ] Si $a=-2$ y $b=3$, entonces $-2 = 3 \cdot (-1) +1$. Es decir $q= -1$, $r=1$. 
		\end{enumerate}
		
	\end{frame}
	
	
	\begin{frame}\frametitle{ Desarrollos en base $b$, ($b \ge 2$)}
		
		
		Los números tal como los escribimos se encuentran en \textit{base 10.} 
		\vskip .2cm
		¿Qué significa  esto?
		\vskip .4cm\pause
		{\color{blue}Ejemplo}
		\vskip .2cm
		El número 407 se puede escribir como 
		$$
		407 = 4 \cdot 10^2 +0 \cdot 10^{1}+ 7 \cdot 10^0,
		$$
		\vskip .2cm\pause
		El número 23827 se puede escribir como 
		$$
		23827 = 2 \cdot 10^4 +3 \cdot 10^3 +8 \cdot 10^2 +2 \cdot 10^{1}+ 7 \cdot 10^0,
		$$
		
		
		
	\end{frame}
	
	
	\begin{frame}
		
		En  general:
		\vskip .4cm
		{\color{blue}Teorema }
		\vskip .2cm
		{\it Todo número natural $x$  se puede escribir de una única forma como  
			$$
			x = r_n10^n +r_{n-1} 10^{n-1}+\cdots + r_1 10 + r_0,
			$$
			donde $0 < r_n < 10$ y  $0 \le r_i < 10$ para $i= 0, 1, \ldots, {n-1}$.}
		
		\vskip .8cm
		
		Podemos decir entonces que
		\vskip .4cm
		
		\begin{center}
			{\it $x$ se representa como la cadena $ r_n r_{n-1}  \ldots  r_1   r_0$ en base 10.}
		\end{center}
		\vskip .2cm
		o también que 
		
		\begin{equation*}
			x = ( r_n r_{n-1}  \ldots  r_1   r_0)_{10}.
		\end{equation*}
		\vskip 4cm
	\end{frame}
	
	
	\begin{frame}
		Este teorema es	una consecuencia importante del algoritmo de división  y es el que
		justifica nuestro método usual de representación de enteros. 
		
		\vskip .2cm
		
		También podemos desarrollar en potencias de $b$ con $b$ cualquier entero $\ge 2$ (y vale un análogo al teorema anterior ).
		
		\vskip .8cm\pause
		{\color{blue}Ejemplo}
		\vskip .2cm
		Deseamos escribir el número $407$ con una expresión de la forma 
		$$
		407 = r_n5^n +r_{n-1} 5^{n-1}+\cdots + r_1 5 + r_0,
		$$
		con $0 \le r_i < 5$. Veamos que esto es posible y se puede hacer de forma algorítmica. 
		
	\end{frame}
	
	
	\begin{frame}
		La forma de hacerlo  es, primero, dividir el número original y los sucesivos cocientes por $5$:   \pause
		\begin{alignat}2
			407 &=5\cdot 81 &+& 2 \label{ib3}\\
			81 & = 5\cdot 16 &+& 1  \label{ib4}\\
			16 & = 5\cdot 3 &+& 1  \label{ibi5}\\
			3 & = 5\cdot 0 &+& 3.
		\end{alignat} \pause
		Observar entonces que
		\begin{alignat*}2
			407 &=5\cdot 81 + 2  &\qquad& \text{por (\ref{ib3})}\\
			&= 5\cdot (5\cdot 16 + 1) + 2  &\qquad& \text{por (\ref{ib4})}\\
			&= 5^2 \cdot 16+ 5\cdot 1 + 2 &\qquad& \text{}\\
			&= 5^2 \cdot (5\cdot 3 + 1)+ 5\cdot 1 + 2   &\qquad& \text{por (\ref{ibi5})}\\
			&= 5^3 \cdot 3+ 5^2 \cdot 1 + 5\cdot 1 + 2.  &\qquad& \text{}
		\end{alignat*} \pause
		En este caso diremos que el desarrollo en base $5$ de $407$ es $3112$ o, resumidamente, $407 = (3112)_5$.  
	\end{frame}
	
	
	\begin{frame}
		Observar que 
		\begin{alignat*}2
			407 &=5\cdot 81 &+\quad& \textbf{2} \\
			81 & = 5\cdot 16 &+\quad& \textbf{1} \\
			16 & = 5\cdot 3 &+\quad& \textbf{1}  \\
			3 & = 5\cdot 0 &+\quad& \textbf{3}.
		\end{alignat*} \pause
		
		y  que 
		
		$$407 = (\textbf{3112})_5.$$ 
		\pause
		
		Es decir los restos leídos de abajo hacia arriba forman la escritura en base 5 de 407. Esto es general. 
	\end{frame}
	
	\begin{frame}
		Sea
		$b \ge 2$ un número entero, llamado {\em base}\index{base (de un sistema de numeración)} para los cálculos.
		Para cualquier entero positivo $x$ tenemos, por la aplicación
		repetida delalgoritmo de división,
		\begin{alignat*}2
			x&=bq_0 &+& r_0 \\
			q_0 & = bq_1 &+&r_1 \\
			\cdots & \\
			q_{n-2} & = bq_{n-1} &+&r_{n-1} \\
			q_{n-1} & = bq_n &+&r_n.
		\end{alignat*}
		Aquí cada resto es uno de los enteros $0, 1,\ldots,b-1$, y paramos
		cuando $q_n=0$. Reemplazando sucesivamente los cocientes $q_i$, como lo hicimos en el ejemplo, obtenemos
		$$
		x=r_nb^n +r_{n-1} b^{n-1}+\cdots + r_1 b + r_0.
		$$
	\end{frame}
	
	
	\begin{frame}
		Hemos representado $x$ (con respecto a la base $b$) por la
		secuencia de los restos, y escribimos 	\vskip .2cm
		$$x=(r_nr_{n-1}\dots r_1
		r_0)_b.$$
		\vskip .2cm
		Convencionalmente $b=10$ es la base para los cálculos
		hechos ``a mano'' y omitimos ponerle el subíndice, entonces
		tenemos la notación usual
		\vskip .2cm\vskip .2cm
		Por ejemplo,
		
		$$
		1984=(1\cdot 10^3 ) + (9\cdot 10^2 )+(8\cdot 10 ) + 4.
		$$
	\end{frame}
	
	
	\begin{frame}
		
		
		La base $b=2$ es particularmente adaptable para
		los cálculos en computadoras. 
		
		\vskip .2cm\pause
		{\color{blue}Ejemplo}
		\vskip .2cm ?`Cuál es la representación en base $2$ de
		$(109)_{10}$?
		\vskip .2cm\pause
		{\color{blue}Solución.} 	Dividiendo repetidamente por $2$ obtenemos
		$$\begin{aligned}
			109&=2\cdot 54+1\\ 54&=2\cdot 27+0\\ 27&=2\cdot 13+1\\ 13&=2\cdot 6+1\\
			6&=2\cdot 3+0 \\ 3&=2\cdot 1+1 \\1&=2\cdot 0+1
		\end{aligned}
		$$\pause
		Por lo tanto
		$$ (109)_{10} = (1101101)_2.
		$$
		
	\end{frame}
	
	
	\begin{frame}
		La base $16$ también es  muy usada en computación.
		\vskip .2cm\pause
		Los dígitos disponibles (del $0$  al $9$) no nos alcanzan  para representar un  número en base $16$, pues se requieren $16$ símbolos. La convención usada es 
		$${\tt A}=10,\quad {\tt B}=11,\quad {\tt C} =12,\quad {\tt D} = 13,\quad {\tt E} = 14,\quad {\tt F} = 15.$$
		\vskip .2cm\pause
		{\color{blue}Ejemplo}
		\vskip .2cm
		
		Representemos  $12488$ en  base $16$.
		\begin{alignat*}2
			12488 &= 16 \cdot 780 &+&  8\\
			780 & = 16 \cdot 48 &+& 12\\
			48 & = 16\cdot 3 &+& 0\\
			3 & = 16 \cdot 0  &+& 3.
		\end{alignat*}
		Luego $12488 = (30{\tt C}8)_{16}$.
		
		
	\end{frame}
	
	
	
	\begin{frame}\frametitle{Conversiones de base}
		¿Cómo convertimos un número en  base $b$  a base 10?
		\vskip .2cm
		Lo \textit{calculamos}.
		\vskip .5cm\pause
		{\color{blue}Ejemplo}
		\vskip .2cm
		Convertir $(1304)_6$  a base 10.
		\vskip .2cm\pause
		{\color{blue}Solución.} 
		\begin{align*}
			(1304)_6 &= 1 \cdot 6^3 + 3 \cdot 6^2 + 0 \cdot 6^1 + 4 \cdot 6^0 \\ 
			&= 1 \cdot 216 + 3 \cdot 36 + 0 \cdot 6 + 4  \\
			&= 216 + 108 +  4  \\
			&= 328. \\
		\end{align*}
		Luego $(1304)_6 = 328$.
		
	\end{frame}
	
	
	\begin{frame}\frametitle{Conversiones de base}
		
		¿Cómo convertimos un número en  base $b$  a base $b'$?
		\vskip .2cm
		
		\begin{enumerate}
			\item[1. ] Convertimos el número de base $b$  a  base 10 .
			\item[2. ] Convertimos el número obtenido a  base $b'$.
		\end{enumerate}
		
		\vskip .4cm\pause
		{\color{blue}Ejemplo}
		\vskip .2cm
		Convertir $(1304)_6$  a base 5.
		\vskip .2cm\pause
		{\color{blue}Solución.} 
		\vskip .2cm 
		Calculamos antes que $(1304)_6 = 328$. Ahora debemos ver 328  en base 5.
		\begin{align*}
			328 &= 5 \cdot 65   + 3\\ 
			65 &= 5 \cdot 13   + 0\\ 
			13 &= 5 \cdot 2   + 3\\ 
			2 &= 5 \cdot 0   + 2
		\end{align*}
		
		Luego  $(1304)_6 = (2303)_5$.
	\end{frame}
	
	
	
	
\end{document}