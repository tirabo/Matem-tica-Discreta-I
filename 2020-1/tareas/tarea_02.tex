% PDFLaTeX
\documentclass[a4paper,12pt,twoside,spanish]{amsbook}
%\documentclass[a4paper,11pt,twoside]{book}
%\documentclass[a4paper,11pt,twoside,spanish]{amsbook}

%%%---------------------------------------------------


\usepackage{etex}
\tolerance=10000
\renewcommand{\baselinestretch}{1.3}

\renewcommand{\familydefault}{\sfdefault} % la font por default es sans serif

% Para hacer el  indice en linea de comando hacer 
% makeindex main
%% En http://www.tug.org/pracjourn/2006-1/hartke/hartke.pdf hay ejemplos de packages de fonts libres, como los siguientes:
%\usepackage{cmbright}
%\usepackage{pxfonts}
%\usepackage[varg]{txfonts}
%\usepackage{ccfonts}
%\usepackage[math]{iwona}
%\usepackage[math]{kurier}


\usepackage{t1enc}
%\usepackage[spanish]{babel}
\usepackage{latexsym}
\usepackage[utf8]{inputenc}
\usepackage{verbatim}
\usepackage{multicol}
\usepackage{amsgen,amsmath,amstext,amsbsy,amsopn,amsfonts,amssymb}
\usepackage{amsthm}
\usepackage{calc}         % From LaTeX distribution
\usepackage{graphicx}     % From LaTeX distribution
\usepackage{ifthen}
\input{random.tex}        % From CTAN/macros/generic
\usepackage{subfigure} 
\usepackage{tikz}
\usetikzlibrary{arrows}
\usetikzlibrary{matrix}
%\usetikzlibrary{graphs}
%\usepackage{tikz-3dplot} %for tikz-3dplot functionality
%\usepackage{pgfplots}
\usepackage{mathtools}
\usepackage{stackrel}
\usepackage{enumerate}
\usepackage{tkz-graph}
%\usepackage{makeidx}
\makeindex

%%%----------------------------------------------------------------------------
\usepackage[a4paper, top=3cm, left=3cm, right=2cm, bottom=2.5cm]{geometry}
%% CONTROLADORES DE.
% Tamaño de la hoja de impresión.
% Tamaños de los laterales del documento. 
%%%%%%%%%%%%%%%%%%%%%%%%%%%%%%%%%%%%%%%%%%%%%%%%%%%%%%%%%%%%%%%%%%%%%%%%%%%%%%%%%
%%% \theoremstyle{plain} %% This is the default
%\oddsidemargin 0.0in \evensidemargin -1.0cm \topmargin 0in
%\headheight .3in \headsep .2in \footskip .2in
%\setlength{\textwidth}{16cm} %ancho para apunte
%\setlength{\textheight}{21cm} %largo para apunte
%%%%\leftmargin 2.5cm
%%%%\rightmargin 2.5cm
%\topmargin 0.5 cm
%%%%%%%%%%%%%%%%%%%%%%%%%%%%%%%%%%%%%%%%%%%%%%%%%%%%%%%%%%%%%%%%%%%%%%%%%%%%%%%%%%%

\usepackage{hyperref}
\hypersetup{
	colorlinks=true,
	linkcolor=blue,
	filecolor=magenta,      
	urlcolor=cyan,
}
\usepackage{hypcap}


\renewcommand{\thesection}{\thechapter.\arabic{section}}
\renewcommand{\thesubsection}{\thesection.\arabic{subsection}}

\newtheorem{teorema}{Teorema}[section]
\newtheorem{proposicion}[teorema]{Proposici\'on}
\newtheorem{corolario}[teorema]{Corolario}
\newtheorem{lema}[teorema]{Lema}
\newtheorem{propiedad}[teorema]{Propiedad}

\theoremstyle{definition}

\newtheorem{definicion}{Definici\'on}[section]
\newtheorem{ejemplo}{Ejemplo}[section]
\newtheorem{problema}{Problema}[section]
\newtheorem{ejercicio}{Ejercicio}[section]
\newtheorem{ejerciciof}{}[section]

\theoremstyle{remark}
\newtheorem{observacion}{Observaci\'on}[section]
\newtheorem{nota}{Nota}[section]

\renewcommand{\partname }{Parte }
\renewcommand{\indexname}{Indice }
\renewcommand{\figurename }{Figura }
\renewcommand{\tablename }{Tabla }
\renewcommand{\proofname}{Demostraci\'on}
\renewcommand{\appendixname }{}
\renewcommand{\contentsname }{Contenidos }
\renewcommand{\chaptername }{}
\renewcommand{\bibname }{Bibliograf\'\i a }


\newcommand{\tarea}[1]{
	\begin{center}
		{\Large Matemática Discreta I - 2020/1} \vskip.4cm
		{\Large Tarea #1}\vskip .4cm
\end{center}}

\renewenvironment{ejercicio}% environment name
{% begin code
	\par\vskip .5cm%
	{\noindent\color{blue}Ejercicio}%
	\vskip .2cm
}%
{%
	\vskip .2cm}% end code


\newenvironment{solucion}% environment name
{% begin code
	\par\vskip .2cm%
	{\noindent\color{blue}Solución}%
	\vskip .2cm
}%
{%
	\vskip .2cm}% end code






\begin{document}

%\frame{\titlepage} 


\tarea{2}

\begin{ejercicio}
	Sea $u_n$ definida recursivamente por:
	\begin{equation*}
	u_0 = 1, \; u_1 = 0, \;\qquad u_n = 5u_{n-1} - 6 u_{n-2}, \; \forall n \ge 2. 
	\end{equation*}
	\begin{enumerate}
		\item[1.]  Calcule $u_2$ y $u_3$ usando  recursión. (30 pts)
		\item[2.] Pruebe por inducción que $u_n = 3\cdot 2^n - 2 \cdot 3^n$. (70 pts)
	\end{enumerate}
	\vskip .2cm
\end{ejercicio}
\begin{comment}
	\begin{solucion}
	{\noindent 1.} Por definición $u_0 = 1, \; u_1 = 0$, luego:
	\begin{equation*}
	u_2 = 5u_{2-1} - 6 u_{2-2} = 5u_{1} - 6 u_{0} = 5 \cdot 0 - 6 \cdot 1 = -6 
	\end{equation*}
	Ahora  $u_1 = 0, \; u_2 = -6$, luego:
	\begin{equation*}
	u_3 = 5u_{3-1} - 6 u_{3-2} = 5u_{2} - 6 u_{1} = 5 \cdot (-6) - 6 \cdot 0 = -30. 
	\end{equation*}
	
	\vskip .2cm
	
	
	{\noindent 2.} \textbf{Caso base.} 
	Por un lado $u_0 =1$, por definición, por otro lado  si calculamos con la fórmula: $u_0 = 3 \cdot 2^0 - 2 \cdot 3^0 = 3- 2 = 1$ y listo.
	\vskip .2cm
	Por un lado $u_1 =0$, por definición, por otro lado  si calculamos con la fórmula: $u_1 = 3 \cdot 2^1 - 2 \cdot 3^1 = 3 \cdot 2- 2 \cdot 3 =  6-6 = 0$ y listo.
	
	\vskip .8cm
	
	\textbf{Paso inductivo.} Debemos probar que si  vale
	\begin{equation*}
	u_h = 3\cdot 2^h - 2 \cdot 3^h \text{ \;para\; }  0 \le h \le k \text{\; con\; }  k \ge 0, \tag{HI}
	\end{equation*}
	eso implica que
	\begin{equation*}
	u_{k+1} = 3\cdot 2^{k+1} - 2 \cdot 3^{k+1}. \tag{*}
	\end{equation*}
	
	Comenzamos por el lado izquierdo de (*):
	
	
	\begin{align*}
	u_{k+1} &=   5u_{k+1-1} - 6 u_{k+1-2}&&\text{ \;(por definición de $u_n$)\; } \\
	&=   5u_{k} - 6 u_{k-1}&&\\
	&=   5( 3\cdot 2^{k} - 2 \cdot 3^{k}) - 6 ( 3\cdot 2^{k-1} - 2 \cdot 3^{k-1})&&\text{ \;(por HI)\; } \\
	&=   5\cdot  3\cdot 2^{k} - 5 \cdot 2 \cdot 3^{k} - 6 \cdot  3\cdot 2^{k-1} + 6 \cdot  2 \cdot 3^{k-1}&& \\
	&=   5\cdot  3\cdot 2\cdot 2^{k-1} - 5 \cdot 2 \cdot 3 \cdot  3^{k-1} - 6 \cdot  3\cdot 2^{k-1} + 6 \cdot  2 \cdot 3^{k-1}&& \\
	&=   30\cdot 2^{k-1} -30 \cdot  3^{k-1} - 18\cdot 2^{k-1} + 12 \cdot 3^{k-1}&&\\
	&=   (30-18)\cdot 2^{k-1} +(-30+12) \cdot  3^{k-1}&& \text{ \;(factor común $ 2^{k-1}$ y $ 3^{k-1}$ )\; } \\
	&=   12\cdot 2^{k-1} -18 \cdot  3^{k-1}&&\\
	&=   3 \cdot 2^2 \cdot 2^{k-1} - 2 \cdot 3^2 \cdot  3^{k-1}&&\\
	&=   3 \cdot 2^{2+k-1} - 2  \cdot  3^{2+k-1}&&\\
	&=   3 \cdot 2^{k+1} - 2  \cdot  3^{k+1}&&.
	\end{align*}
	
	Esto  prueba (*).
	\end{solucion}
\end{comment}



\end{document}

