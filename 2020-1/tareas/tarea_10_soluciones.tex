% PDFLaTeX
\documentclass[a4paper,12pt,twoside,spanish]{amsbook}
%\documentclass[a4paper,11pt,twoside]{book}
%\documentclass[a4paper,11pt,twoside,spanish]{amsbook}

%%%---------------------------------------------------


\usepackage{etex}
\tolerance=10000
\renewcommand{\baselinestretch}{1.3}

\renewcommand{\familydefault}{\sfdefault} % la font por default es sans serif

% Para hacer el  indice en linea de comando hacer
% makeindex main
%% En http://www.tug.org/pracjourn/2006-1/hartke/hartke.pdf hay ejemplos de packages de fonts libres, como los siguientes:
%\usepackage{cmbright}
%\usepackage{pxfonts}
%\usepackage[varg]{txfonts}
%\usepackage{ccfonts}
%\usepackage[math]{iwona}
%\usepackage[math]{kurier}


\usepackage{t1enc}
%\usepackage[spanish]{babel}
\usepackage{latexsym}
\usepackage[utf8]{inputenc}
\usepackage{verbatim}
\usepackage{multicol}
\usepackage{amsgen,amsmath,amstext,amsbsy,amsopn,amsfonts,amssymb}
\usepackage{amsthm}
\usepackage{calc}         % From LaTeX distribution
\usepackage{graphicx}     % From LaTeX distribution
\usepackage{ifthen}
\input{random.tex}        % From CTAN/macros/generic
\usepackage{subfigure}
\usepackage{tikz}
\usetikzlibrary{arrows}
\usetikzlibrary{matrix}
%\usetikzlibrary{graphs}
%\usepackage{tikz-3dplot} %for tikz-3dplot functionality
%\usepackage{pgfplots}
\usepackage{mathtools}
\usepackage{stackrel}
\usepackage{enumerate}
\usepackage{tkz-graph}
%\usepackage{makeidx}
\makeindex

%%%----------------------------------------------------------------------------
\usepackage[a4paper, top=3cm, left=3cm, right=2cm, bottom=2.5cm]{geometry}
%% CONTROLADORES DE.
% Tamaño de la hoja de impresión.
% Tamaños de los laterales del documento.
%%%%%%%%%%%%%%%%%%%%%%%%%%%%%%%%%%%%%%%%%%%%%%%%%%%%%%%%%%%%%%%%%%%%%%%%%%%%%%%%%
%%% \theoremstyle{plain} %% This is the default
%\oddsidemargin 0.0in \evensidemargin -1.0cm \topmargin 0in
%\headheight .3in \headsep .2in \footskip .2in
%\setlength{\textwidth}{16cm} %ancho para apunte
%\setlength{\textheight}{21cm} %largo para apunte
%%%%\leftmargin 2.5cm
%%%%\rightmargin 2.5cm
%\topmargin 0.5 cm
%%%%%%%%%%%%%%%%%%%%%%%%%%%%%%%%%%%%%%%%%%%%%%%%%%%%%%%%%%%%%%%%%%%%%%%%%%%%%%%%%%%

\usepackage{hyperref}
\hypersetup{
	colorlinks=true,
	linkcolor=blue,
	filecolor=magenta,
	urlcolor=cyan,
}
\usepackage{hypcap}

%----------------------------------------------------------------------------------------------------------------------
\usepackage{pgf,tikz,pgfplots}
\pgfplotsset{compat=1.15}
\usepackage{mathrsfs}
\usetikzlibrary{arrows}

%-----------------------------------------------------------------------------------------------------------------------


\renewcommand{\thesection}{\thechapter.\arabic{section}}
\renewcommand{\thesubsection}{\thesection.\arabic{subsection}}

\newtheorem{teorema}{Teorema}[section]
\newtheorem{proposicion}[teorema]{Proposici\'on}
\newtheorem{corolario}[teorema]{Corolario}
\newtheorem{lema}[teorema]{Lema}
\newtheorem{propiedad}[teorema]{Propiedad}

\theoremstyle{definition}

\newtheorem{definicion}{Definici\'on}[section]
\newtheorem{ejemplo}{Ejemplo}[section]
\newtheorem{problema}{Problema}[section]
\newtheorem{ejercicio}{Ejercicio}[section]
\newtheorem{ejerciciof}{}[section]

\theoremstyle{remark}
\newtheorem{observacion}{Observaci\'on}[section]
\newtheorem{nota}{Nota}[section]

\renewcommand{\partname }{Parte }
\renewcommand{\indexname}{Indice }
\renewcommand{\figurename }{Figura }
\renewcommand{\tablename }{Tabla }
\renewcommand{\proofname}{Demostraci\'on}
\renewcommand{\appendixname }{}
\renewcommand{\contentsname }{Contenidos }
\renewcommand{\chaptername }{}
\renewcommand{\bibname }{Bibliograf\'\i a }






\begin{document}
\definecolor{uuuuuu}{rgb}{0.26666666666666666,0.26666666666666666,0.26666666666666666}

%\frame{\titlepage}


\begin{center}
	{\Large Matemática Discreta I - 2020/1}
	
	{\Large Tarea 10}
\end{center}
\vskip .4cm
	\noindent{\color{blue}Ejercicio.} \ Dados los siguientes grafos:

\vspace{-2cm}\begin{figure}[h]
	\centering
\begin{tikzpicture}[line cap=round,line join=round,>=triangle 45,x=1cm,y=1cm]
\clip(-4.630210027906974,-3.7534003348837173) rectangle (13.62467369302325,6.642878734883723);
\draw [line width=2pt] (2.5,3)-- (4.5,3);
\draw [line width=2pt] (8,3)-- (10,3);
\draw [line width=2pt] (7,1.2679491924311226)-- (8,3);
\draw [line width=2pt] (7,1.2679491924311226)-- (11,1.267949192431121);
\draw [line width=2pt] (11,1.267949192431121)-- (10,3);
\draw [line width=2pt] (7,1.2679491924311226)-- (8,-0.46410161513775505);
\draw [line width=2pt] (8,-0.46410161513775505)-- (11,1.267949192431121);
\draw [line width=2pt] (7,1.2679491924311226)-- (10,-0.4641016151377557);
\draw [line width=2pt] (10,-0.4641016151377557)-- (11,1.267949192431121);
\draw [line width=2pt] (8,-0.46410161513775505)-- (8,3);
\draw [line width=2pt] (-4,1.2679491924311226)-- (-1,3);
\draw [line width=2pt] (-1,3)-- (0,1.267949192431121);
\draw [line width=2pt] (0,1.267949192431121)-- (-4,1.2679491924311226);
\draw [line width=2pt] (-4,1.2679491924311226)-- (-3,-0.46410161513775505);
\draw [line width=2pt] (-3,-0.46410161513775505)-- (0,1.267949192431121);
\draw [line width=2pt] (-4,1.2679491924311226)-- (-1,-0.4641016151377557);
\draw [line width=2pt] (-1,-0.4641016151377557)-- (0,1.267949192431121);
\draw [line width=2pt] (1.5,1.2679491924311226)-- (4.5,3);
\draw [line width=2pt] (4.5,3)-- (5.5,1.267949192431121);
\draw [line width=2pt] (5.5,1.267949192431121)-- (1.5,1.2679491924311226);
\draw [line width=2pt] (1.5,1.2679491924311226)-- (2.5,3);
\draw [line width=2pt] (1.5,1.2679491924311226)-- (2.5,-0.46410161513775505);
\draw [line width=2pt] (2.5,-0.46410161513775505)-- (5.5,1.267949192431121);
\draw [line width=2pt] (-4,1.2679491924311226)-- (-1.2,1.9607695154586726);
\draw [line width=2pt] (-1.2,1.9607695154586726)-- (-1,3);
\draw [line width=2pt] (5.5,1.267949192431121) -- (2.7,0.5751288694035712);
\draw [line width=2pt] (2.7,0.5751288694035712)-- (1.5,1.2679491924311226);
\draw (-3.9957914232558114,3.8186926883720957) node[anchor=north west] {(1)};
\draw (1.5093248558139538,3.8186926883720957) node[anchor=north west] {(2)};
\draw (6.993976018604649,3.8186926883720957) node[anchor=north west] {(3)};
\begin{scriptsize}
\draw [fill=black] (-1,3) circle (2.5pt);
\draw[color=black] (-0.8441635162790686,3.3) node {B};
\draw [fill=black] (2.5,3) circle (2.5pt);
\draw[color=black] (2.65537136744186,3.3) node {H};
\draw [fill=black] (4.5,3) circle (2.5pt);
\draw[color=black] (4.660952762790696,3.3) node {I};
\draw [fill=black] (8,3) circle (2.5pt);
\draw[color=black] (8.160487646511625,3.3) node {P};
\draw [fill=black] (10,3) circle (2.5pt);
\draw[color=black] (10.166069041860462,3.3) node {R};
\draw [fill=uuuuuu] (-4,1.2679491924311226) circle (2.5pt);
\draw[color=uuuuuu] (-4.036721655813952,1.6) node {A};
\draw [fill=black] (-3,-0.46410161513775505) circle (2.5pt);
\draw[color=black] (-2.829279795348835,-0.018516613953485167) node {F};
\draw [fill=black] (-1,-0.4641016151377557) circle (2.5pt);
\draw[color=black] (-0.6804425860465104,-0.30502824186046185) node {D};
\draw [fill=black] (0,1.267949192431121) circle (2.5pt);
\draw[color=black] (0.15862718139534965,1.6) node {C};
\draw [fill=black] (1.5,1.2679491924311226) circle (2.5pt);
\draw[color=black] (1.4274643906976747,1.6) node {G};
\draw [fill=black] (2.5,-0.46410161513775505) circle (2.5pt);
\draw[color=black] (2.8600225302325577,-0.5096794046511596) node {L};
\draw [fill=black] (5.5,1.267949192431121) circle (2.5pt);
\draw[color=black] (5.663743460465114,1.6) node {J};
\draw [fill=black] (7,1.2679491924311226) circle (2.5pt);
\draw[color=black] (6.809789972093021,1.6) node {M};
\draw [fill=black] (8,-0.46410161513775505) circle (2.5pt);
\draw[color=black] (8.5,-0.4073538232558107) node {N};
\draw [fill=black] (10,-0.4641016151377557) circle (2.5pt);
\draw[color=black] (10.32978997209302,-0.30502824186046185) node {Q};
\draw [fill=black] (11,1.267949192431121) circle (2.5pt);
\draw[color=black] (11.168859739534879,1.6) node {O};
\draw [fill=black] (2.7,0.5751288694035712) circle (2.5pt);
\draw[color=black] (2.5,0.9024136186046543) node {K};
\draw [fill=black] (-1.2,1.9607695154586726) circle (2.5pt);
\draw[color=black] (-0.9055588651162778,1.884739200000003) node {E};
\end{scriptsize}
\end{tikzpicture}
\end{figure}

\vspace{-2.5cm}\begin{enumerate}
		%\item[(a)(20 pts)] En el grafo $(1)$, ¿existen \textbf{subgrafos completos} de tres vértices? \ ¿y de 4 vértices? En caso
%                           de existir, exhibir al menos un subgrafo; en caso contrario, justificar la no existencia.
		\item[(a) (25 pts)] En el grafo $(2)$, determine si existe una \textbf{caminata euleriana}, y en caso de ser así, encuentre una.
        \item[(b) (25 pts)] Dé un \textbf{ciclo hamiltoniano} en el grafo $(3)$.
        \item[(c) (50 pts)] Determinar cuales de los siguientes pares de grafos son isomorfos. En el caso de ser isomorfos, especifique un isomorfismo;
                           en caso contrario, justificar por que no son isomorfos.

                           (i) (1) y (2).\quad

                           (ii) (2) y (3).
	\end{enumerate}
\vskip .2cm
%\begin{comment}

\noindent{\color{blue}Solución.}


\noindent\text{{(a)}} 
Como el vértice I y el vértice G tienen valencia impar (3 y 5 respectivamente) y todos los demás vértices tienen valencia par, existe una caminata euleriana que va de  I a G. Una posible es:
\begin{center}
	I, J, G, K, J, L, G, H, I, G.
\end{center}	
\vskip .4cm	

\noindent\text{{(b)}} 
\begin{center}
	O, Q, M, N, P, R, O
\end{center}
\vskip .4cm	

\noindent\text{{(c)(i)}} 
Observar que si uno "da vuelta" el vértice E de (1) queda como el H de (2). Si "mete adentro" el vértice D de (1) queda como el K de (2). Luego,  el isomorfismo es:

\begin{align*}
	\text{E} \; &\mapsto \;\text{H} \\
	\text{A} \; &\mapsto \;\text{G} \\
	\text{B} \; &\mapsto \;\text{I} \\
	\text{C} \; &\mapsto \;\text{J} \\
	\text{D} \; &\mapsto \;\text{K} \\
	\text{F} \; &\mapsto \;\text{L} 
\end{align*}
\vskip .4cm	

\noindent\text{{(c)(ii)}} 
El grafo (2) tiene un vértice con valencia $5$ (el vértice G), mientras que el grafo (3) tiene todos sus vértices con valencia $4$ o  menor. Cuando dos grafos son isomorfos la lista de valencias de ambos debe ser la misma. Esto no se cumple entre (2) y (3) y por lo tanto no son isomorfos.
\vskip .4cm	

%\end{comment}
\end{document}

