% PDFLaTeX
\documentclass[a4paper,12pt,twoside,spanish]{amsbook}
%\documentclass[a4paper,11pt,twoside]{book}
%\documentclass[a4paper,11pt,twoside,spanish]{amsbook}

%%%---------------------------------------------------


\usepackage{etex}
\tolerance=10000
\renewcommand{\baselinestretch}{1.3}

\renewcommand{\familydefault}{\sfdefault} % la font por default es sans serif

% Para hacer el  indice en linea de comando hacer 
% makeindex main
%% En http://www.tug.org/pracjourn/2006-1/hartke/hartke.pdf hay ejemplos de packages de fonts libres, como los siguientes:
%\usepackage{cmbright}
%\usepackage{pxfonts}
%\usepackage[varg]{txfonts}
%\usepackage{ccfonts}
%\usepackage[math]{iwona}
%\usepackage[math]{kurier}


\usepackage{t1enc}
%\usepackage[spanish]{babel}
\usepackage{latexsym}
\usepackage[utf8]{inputenc}
\usepackage{verbatim}
\usepackage{multicol}
\usepackage{amsgen,amsmath,amstext,amsbsy,amsopn,amsfonts,amssymb}
\usepackage{amsthm}
\usepackage{calc}         % From LaTeX distribution
\usepackage{graphicx}     % From LaTeX distribution
\usepackage{ifthen}
\input{random.tex}        % From CTAN/macros/generic
\usepackage{subfigure} 
\usepackage{tikz}
\usetikzlibrary{arrows}
\usetikzlibrary{matrix}
%\usetikzlibrary{graphs}
%\usepackage{tikz-3dplot} %for tikz-3dplot functionality
%\usepackage{pgfplots}
\usepackage{mathtools}
\usepackage{stackrel}
\usepackage{enumerate}
\usepackage{tkz-graph}
%\usepackage{makeidx}
\makeindex

%%%----------------------------------------------------------------------------
\usepackage[a4paper, top=3cm, left=3cm, right=2cm, bottom=2.5cm]{geometry}
%% CONTROLADORES DE.
% Tamaño de la hoja de impresión.
% Tamaños de los laterales del documento. 
%%%%%%%%%%%%%%%%%%%%%%%%%%%%%%%%%%%%%%%%%%%%%%%%%%%%%%%%%%%%%%%%%%%%%%%%%%%%%%%%%
%%% \theoremstyle{plain} %% This is the default
%\oddsidemargin 0.0in \evensidemargin -1.0cm \topmargin 0in
%\headheight .3in \headsep .2in \footskip .2in
%\setlength{\textwidth}{16cm} %ancho para apunte
%\setlength{\textheight}{21cm} %largo para apunte
%%%%\leftmargin 2.5cm
%%%%\rightmargin 2.5cm
%\topmargin 0.5 cm
%%%%%%%%%%%%%%%%%%%%%%%%%%%%%%%%%%%%%%%%%%%%%%%%%%%%%%%%%%%%%%%%%%%%%%%%%%%%%%%%%%%

\usepackage{hyperref}
\hypersetup{
	colorlinks=true,
	linkcolor=blue,
	filecolor=magenta,      
	urlcolor=cyan,
}
\usepackage{hypcap}


\renewcommand{\thesection}{\thechapter.\arabic{section}}
\renewcommand{\thesubsection}{\thesection.\arabic{subsection}}

\newtheorem{teorema}{Teorema}[section]
\newtheorem{proposicion}[teorema]{Proposici\'on}
\newtheorem{corolario}[teorema]{Corolario}
\newtheorem{lema}[teorema]{Lema}
\newtheorem{propiedad}[teorema]{Propiedad}

\theoremstyle{definition}

\newtheorem{definicion}{Definici\'on}[section]
\newtheorem{ejemplo}{Ejemplo}[section]
\newtheorem{problema}{Problema}[section]
\newtheorem{ejercicio}{Ejercicio}[section]
\newtheorem{ejerciciof}{}[section]

\theoremstyle{remark}
\newtheorem{observacion}{Observaci\'on}[section]
\newtheorem{nota}{Nota}[section]

\renewcommand{\partname }{Parte }
\renewcommand{\indexname}{Indice }
\renewcommand{\figurename }{Figura }
\renewcommand{\tablename }{Tabla }
\renewcommand{\proofname}{Demostraci\'on}
\renewcommand{\appendixname }{}
\renewcommand{\contentsname }{Contenidos }
\renewcommand{\chaptername }{}
\renewcommand{\bibname }{Bibliograf\'\i a }



\newcommand{\tarea}[1]{
	\begin{center}
		{\Large Matemática Discreta I - 2020/1} \vskip.4cm
		{\Large Tarea #1}\vskip .4cm
\end{center}}

\renewenvironment{ejercicio}% environment name
{% begin code
	\par\vskip .5cm%
	{\noindent\color{blue}Ejercicio}%
	\vskip .2cm
}%
{%
	\vskip .2cm}% end code


\newenvironment{solucion}% environment name
{% begin code
	\par\vskip .2cm%
	{\noindent\color{blue}Solución}%
	\vskip .2cm
}%
{%
	\vskip .2cm}% end code


\begin{document}

%\frame{\titlepage} 


\tarea{7}

\begin{ejercicio}
	\begin{enumerate}
		%\item[1.] (60 pts) Probar que para $n \ge 0$ se satisface $3 | 2^{3n+2} + 5^{n+3}$.  
		\item[1.] (50 pts) Probar que $\sqrt[3]{9}$ no es un número racional.
		\item[2.]  (50 pts) Calcular el máximo común divisor y el mínimo común múltiplo de $1176$ y $450$  usando la descomposición en números primos.
	\end{enumerate}
\end{ejercicio}
	
\begin{comment}
\begin{solucion}

1. Deebemos probar que $\sqrt[3]{9}$ no es cociente de dos números naturales. Lo haremos por el absurdo: supongamos que $\sqrt[3]{9} = \frac{m}{n}$  con $m, n$ números naturales. 
Luego,
\begin{equation*}
	\sqrt[3]{9} = \frac{m}{n} \quad \Rightarrow  \quad  (\sqrt[3]{9})^3 = \left(\frac{m}{n}\right)^3  \quad \Rightarrow  \quad  9 = \frac{m^3}{n^3}  \quad \Rightarrow  \quad 9 {n^3}= {m^3}.
\end{equation*}
Veamos entonces que no es posible que 	$ 9 {n^3}= {m^3}$. Podemos escribir,
\begin{align*}
	 m &= 3^{k}p_1^{e_1}p_2^{e_2}\ldots p_r^{e_r},\\
	 n &= 3^{h}p_1^{f_1}p_2^{f_2}\ldots p_r^{f_r},
\end{align*}
con $p_1,\ldots,p_r$ primos distintos entre sí y distintos de 3 y cada exponente ($k,h,e_i,f_i$) no negativo.

Luego
\begin{align*}
m^3 &= 3^{3k}p_1^{3e_1}p_2^{3e_2}\ldots p_r^{3e_r},\\
n^3 &= 3^{3h}p_1^{3f_1}p_2^{3f_2}\ldots p_r^{3f_r},
\end{align*}
y por lo tanto
\begin{align*}
9 {n^3}&= {m^3}&\quad &\Leftrightarrow \\
	3^2  3^{3h}p_1^{3f_1}p_2^{3f_2}\ldots p_r^{3f_r}&= 3^{3k}p_1^{3e_1}p_2^{3e_2}\ldots p_r^{3e_r}&\quad &\Leftrightarrow  \\
		  3^{3h+2}p_1^{3f_1}p_2^{3f_2}\ldots p_r^{3f_r}&= 3^{3k}p_1^{3e_1}p_2^{3e_2}\ldots p_r^{3e_r}. &&
\end{align*}
Por el teorema fundamental de la aritmética, el exponente de $3$  a la izquierda de la igualdad debe ser igual al exponente de $3$  a la derecha de la igualdad,  es decir $3h+2 = 3k$, lo cual implica que $2= 3(k-h)$ $\Rightarrow$ $3|2$, absurdo.

El absurdo vino de suponer que  $\sqrt[3]{9}$ es racional. Por lo tanto   $\sqrt[3]{9}$ no es racional.

\vskip .4cm

	2. Primero  hagamos la descomposición prima de cada número:
	\begin{align*}
		1176  &= 2 \cdot 588 = 2 \cdot 2 \cdot 294=  2 \cdot  2 \cdot 2 \cdot 147 =  2 \cdot  2 \cdot 2 \cdot 3  \cdot 49 = \; 2^3 \cdot 3  \cdot 7^2.\\
		450 &= 10 \cdot 45 =  2 \cdot 5 \cdot 3 \cdot 15  = 2 \cdot 5 \cdot 3 \cdot 3 \cdot 5  =  2 \cdot 5^2 \cdot 3^2 .
	\end{align*}
	
	Podemos los números anteriores incluyendo los primos que participan en ambos:
	\begin{align*}
	1176  &=\; 2^3 \cdot 3^1 \cdot 5^0  \cdot 7^2.\\
	450 &= \; 2^1 \cdot 3^2 \cdot 5^2  \cdot 7^0.
	\end{align*}
	
	Luego 
	\begin{align*}
	\operatorname{mcd}(1176,450)  &=\; 2^1 \cdot 3^1 \cdot 5^0  \cdot 7^0 = 6 .\\
	\operatorname{mcm}(1176,450)  &= \; 2^3 \cdot 3^2 \cdot 5^2  \cdot 7^2 = 88200.
	\end{align*}
	
\end{solucion}

%\end{comment}
\end{document}

	1. Lo haremos por inducción.
\vskip .3cm 
Caso base $n=0$. 

En  este caso  $2^{3n+2} + 5^{n+3} =  2^{2} + 5^{3} = 4 +125  = 129 = 3 \cdot 43$,  es decir $3| 2^{2} + 5^{3}$.

\vskip .3cm 

Paso  inductivo. Supongamos que $3 | 2^{3k+2} + 5^{k+3}$ para un $k \ge 0$ (HI),  debemos probar que 
\begin{equation}\label{eq-q}
3 | 2^{3(k+1)+2} + 5^{(k+1)+3}. \tag{*}
\end{equation}
Ahora bien, 
\begin{align*}
2^{3(k+1)+2} + 5^{(k+1)+3} &= 2^{3k+3+2} + 5^{k+1+3}\\
&= 2^{3k+2}\cdot 2^{3} + 5^{k+3}\cdot 5^{1} \\
&= 8 \cdot 2^{3k+2} + 5\cdot 5^{k+3}\\
&= 5( 2^{3k+2} + 5^{k+3})  +3 \cdot 2^{3k+2}.
\end{align*}
Es decir
\begin{equation}
2^{3(k+1)+2} + 5^{(k+1)+3} = 5( 2^{3k+2} + 5^{k+3})  +3 \cdot 2^{3k+2}. \tag{**}
\end{equation}
Por hipótesis inductiva (HI), $3 | 2^{3k+2} + 5^{k+3}$ y por lo tanto $ 3|5( 2^{3k+2} + 5^{k+3})$. Por otro lado es claro  que $3 | 3 \cdot 2^{3k+2}$. Entonces,  3 divide a la suma de  $5( 2^{3k+2} + 5^{k+3})$ y $3 \cdot 2^{3k+2}$:
\begin{equation*}
3|5( 2^{3k+2} + 5^{k+3}) + 3 \cdot 2^{3k+2} \overset{(**)}{=}2^{3(k+1)+2} + 5^{(k+1)+3},
\end{equation*}
y esto  prueba (*), y por lo tanto prueba el ejercicio.
\vskip .4cm 


	2. 