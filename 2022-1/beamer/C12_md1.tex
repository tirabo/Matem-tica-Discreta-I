\documentclass[handout]{beamer} % sin pausas
%\documentclass{beamer} 
%\setbeamertemplate{background}[grid][step=8 ] % cuadriculado

\usetheme{CambridgeUS}

\usepackage{etex}
\usepackage{t1enc}
\usepackage[spanish,es-nodecimaldot]{babel}
\usepackage{latexsym}
\usepackage[utf8]{inputenc}
\usepackage{verbatim}
\usepackage{multicol}
\usepackage{amsgen,amsmath,amstext,amsbsy,amsopn,amsfonts,amssymb}
\usepackage{amsthm}
\usepackage{calc}         % From LaTeX distribution
\usepackage{graphicx}     % From LaTeX distribution
\usepackage{ifthen}
%\usepackage{makeidx}
\input{random.tex}        % From CTAN/macros/generic
\usepackage{subfigure} 
\usepackage{tikz}
\usepackage[customcolors]{hf-tikz}
\usetikzlibrary{arrows}
\usetikzlibrary{matrix}
\tikzset{
    every picture/.append style={
        execute at begin picture={\deactivatequoting},
        execute at end picture={\activatequoting}
    }
}
\usetikzlibrary{decorations.pathreplacing,angles,quotes}
\usetikzlibrary{shapes.geometric}
\usepackage{mathtools}
\usepackage{stackrel}
%\usepackage{enumerate}
\usepackage{enumitem}
\usepackage{tkz-graph}
\usepackage{polynom}
\polyset{%
    style=B,
    delims={(}{)},
    div=:
}
\renewcommand\labelitemi{$\circ$}
\setlist[enumerate]{label={(\arabic*)}}

\setbeamertemplate{itemize item}{$\circ$}
\setbeamertemplate{enumerate items}[default]
\definecolor{links}{HTML}{2A1B81}
\hypersetup{colorlinks,linkcolor=,urlcolor=links}


\newcommand{\Id}{\operatorname{Id}}
\newcommand{\img}{\operatorname{Im}}
\newcommand{\nuc}{\operatorname{Nu}}
\newcommand{\im}{\operatorname{Im}}
\renewcommand\nu{\operatorname{Nu}}
\newcommand{\la}{\langle}
\newcommand{\ra}{\rangle}
\renewcommand{\t}{{\operatorname{t}}}
\renewcommand{\sin}{{\,\operatorname{sen}}}
\newcommand{\Q}{\mathbb Q}
\newcommand{\R}{\mathbb R}
\newcommand{\C}{\mathbb C}
\newcommand{\K}{\mathbb K}
\newcommand{\F}{\mathbb F}
\newcommand{\Z}{\mathbb Z}
\newcommand{\N}{\mathbb N}
\newcommand\sgn{\operatorname{sgn}}
\renewcommand{\t}{{\operatorname{t}}}
\renewcommand{\figurename }{Figura}

%
% Ver http://joshua.smcvt.edu/latex2e/_005cnewenvironment-_0026-_005crenewenvironment.html
%

\renewenvironment{block}[1]% environment name
{% begin code
	\par\vskip .2cm%
	{\color{blue}#1}%
	\vskip .2cm
}%
{%
	\vskip .2cm}% end code


\renewenvironment{alertblock}[1]% environment name
{% begin code
	\par\vskip .2cm%
	{\color{red!80!black}#1}%
	\vskip .2cm
}%
{%
	\vskip .2cm}% end code


\renewenvironment{exampleblock}[1]% environment name
{% begin code
	\par\vskip .2cm%
	{\color{blue}#1}%
	\vskip .2cm
}%
{%
	\vskip .2cm}% end code




\newenvironment{exercise}[1]% environment name
{% begin code
	\par\vspace{\baselineskip}\noindent
	\textbf{Ejercicio (#1)}\begin{itshape}%
		\par\vspace{\baselineskip}\noindent\ignorespaces
	}%
	{% end code
	\end{itshape}\ignorespacesafterend
}


\newenvironment{definicion}[1][]% environment name
{% begin code
	\par\vskip .2cm%
	{\color{blue}Definición #1}%
	\vskip .2cm
}%
{%
	\vskip .2cm}% end code

    \newenvironment{notacion}[1][]% environment name
    {% begin code
        \par\vskip .2cm%
        {\color{blue}Notación #1}%
        \vskip .2cm
    }%
    {%
        \vskip .2cm}% end code

\newenvironment{observacion}[1][]% environment name
{% begin code
	\par\vskip .2cm%
	{\color{blue}Observación #1}%
	\vskip .2cm
}%
{%
	\vskip .2cm}% end code

\newenvironment{ejemplo}[1][]% environment name
{% begin code
	\par\vskip .2cm%
	{\color{blue}Ejemplo #1}%
	\vskip .2cm
}%
{%
	\vskip .2cm}% end code


\newenvironment{preguntas}[1][]% environment name
{% begin code
    \par\vskip .2cm%
    {\color{blue}Preguntas #1}%
    \vskip .2cm
}%
{%
    \vskip .2cm}% end code

\newenvironment{ejercicio}[1][]% environment name
{% begin code
	\par\vskip .2cm%
	{\color{blue}Ejercicio #1}%
	\vskip .2cm
}%
{%
	\vskip .2cm}% end code


\renewenvironment{proof}% environment name
{% begin code
	\par\vskip .2cm%
	{\color{blue}Demostración}%
	\vskip .2cm
}%
{%
	\vskip .2cm}% end code



\newenvironment{demostracion}% environment name
{% begin code
	\par\vskip .2cm%
	{\color{blue}Demostración}%
	\vskip .2cm
}%
{%
	\vskip .2cm}% end code

\newenvironment{idea}% environment name
{% begin code
	\par\vskip .2cm%
	{\color{blue}Idea de la demostración}%
	\vskip .2cm
}%
{%
	\vskip .2cm}% end code

\newenvironment{solucion}% environment name
{% begin code
	\par\vskip .2cm%
	{\color{blue}Solución}%
	\vskip .2cm
}%
{%
	\vskip .2cm}% end code



\newenvironment{lema}[1][]% environment name
{% begin code
	\par\vskip .2cm%
	{\color{blue}Lema #1}\begin{itshape}%
		\par\vskip .2cm
	}%
	{% end code
	\end{itshape}\vskip .2cm\ignorespacesafterend
}

\newenvironment{proposicion}[1][]% environment name
{% begin code
	\par\vskip .2cm%
	{\color{blue}Proposición #1}\begin{itshape}%
		\par\vskip .2cm
	}%
	{% end code
	\end{itshape}\vskip .2cm\ignorespacesafterend
}

\newenvironment{teorema}[1][]% environment name
{% begin code
	\par\vskip .2cm%
	{\color{blue}Teorema #1}\begin{itshape}%
		\par\vskip .2cm
	}%
	{% end code
	\end{itshape}\vskip .2cm\ignorespacesafterend
}


\newenvironment{corolario}[1][]% environment name
{% begin code
	\par\vskip .2cm%
	{\color{blue}Corolario #1}\begin{itshape}%
		\par\vskip .2cm
	}%
	{% end code
	\end{itshape}\vskip .2cm\ignorespacesafterend
}

\newenvironment{propiedad}% environment name
{% begin code
	\par\vskip .2cm%
	{\color{blue}Propiedad}\begin{itshape}%
		\par\vskip .2cm
	}%
	{% end code
	\end{itshape}\vskip .2cm\ignorespacesafterend
}

\newenvironment{conclusion}% environment name
{% begin code
	\par\vskip .2cm%
	{\color{blue}Conclusión}\begin{itshape}%
		\par\vskip .2cm
	}%
	{% end code
	\end{itshape}\vskip .2cm\ignorespacesafterend
}


\newenvironment{definicion*}% environment name
{% begin code
	\par\vskip .2cm%
	{\color{blue}Definición}%
	\vskip .2cm
}%
{%
	\vskip .2cm}% end code

\newenvironment{observacion*}% environment name
{% begin code
	\par\vskip .2cm%
	{\color{blue}Observación}%
	\vskip .2cm
}%
{%
	\vskip .2cm}% end code


\newenvironment{obs*}% environment name
	{% begin code
		\par\vskip .2cm%
		{\color{blue}Observación}%
		\vskip .2cm
	}%
	{%
		\vskip .2cm}% end code

\newenvironment{ejemplo*}% environment name
{% begin code
	\par\vskip .2cm%
	{\color{blue}Ejemplo}%
	\vskip .2cm
}%
{%
	\vskip .2cm}% end code

\newenvironment{ejercicio*}% environment name
{% begin code
	\par\vskip .2cm%
	{\color{blue}Ejercicio}%
	\vskip .2cm
}%
{%
	\vskip .2cm}% end code

\newenvironment{propiedad*}% environment name
{% begin code
	\par\vskip .2cm%
	{\color{blue}Propiedad}\begin{itshape}%
		\par\vskip .2cm
	}%
	{% end code
	\end{itshape}\vskip .2cm\ignorespacesafterend
}

\newenvironment{conclusion*}% environment name
{% begin code
	\par\vskip .2cm%
	{\color{blue}Conclusión}\begin{itshape}%
		\par\vskip .2cm
	}%
	{% end code
	\end{itshape}\vskip .2cm\ignorespacesafterend
}




 % definiciones propias

\newcommand{\nc}{\newcommand}

%%%%%%%%%%%%%%%%%%%%%%%%%LETRAS

\nc{\FF}{{\mathbb F}} \nc{\NN}{{\mathbb N}} \nc{\QQ}{{\mathbb Q}}
\nc{\PP}{{\mathbb P}} \nc{\DD}{{\mathbb D}} \nc{\Sn}{{\mathbb S}}
\nc{\uno}{\mathbb{1}} \nc{\BB}{{\mathbb B}} \nc{\An}{{\mathbb A}}

\nc{\ba}{\mathbf{a}} \nc{\bb}{\mathbf{b}} \nc{\bt}{\mathbf{t}}
\nc{\bB}{\mathbf{B}}

\nc{\cP}{\mathcal{P}} \nc{\cU}{\mathcal{U}} \nc{\cX}{\mathcal{X}}
\nc{\cE}{\mathcal{E}} \nc{\cS}{\mathcal{S}} \nc{\cA}{\mathcal{A}}
\nc{\cC}{\mathcal{C}} \nc{\cO}{\mathcal{O}} \nc{\cQ}{\mathcal{Q}}
\nc{\cB}{\mathcal{B}} \nc{\cJ}{\mathcal{J}} \nc{\cI}{\mathcal{I}}
\nc{\cM}{\mathcal{M}} \nc{\cK}{\mathcal{K}}

\nc{\fD}{\mathfrak{D}} \nc{\fI}{\mathfrak{I}} \nc{\fJ}{\mathfrak{J}}
\nc{\fS}{\mathfrak{S}} \nc{\gA}{\mathfrak{A}}
%%%%%%%%%%%%%%%%%%%%%%%%%LETRAS




\title[Clase 12 - MCD (1)]{Matemática Discreta I \\ Clase 12 - Máximo común divisor (1)}
%\author[C. Olmos / A. Tiraboschi]{Carlos Olmos / Alejandro Tiraboschi}
\institute[]{\normalsize FAMAF / UNC
    \\[\baselineskip] ${}^{}$
    \\[\baselineskip]
}
\date[03/05/2022]{3 de mayo  de 2022}

\newcommand{\mcd}{\operatorname{mcd}}


\begin{document}
    
    \frame{\titlepage} 
    
    
    \begin{frame}\frametitle{Definición de MCD}
        
        
        \begin{definicion}
        Si $a$ y $b$ son enteros algunos de ellos no nulo, decimos que un entero positivo $d$ es un {\em máximo común divisor}\index{máximo común divisor}, o {\em mcd}, de $a$ y $b$ si
        \begin{enumerate}
            \item[{\color{blue} a)}] $ d|a$  y $d|b$;
            \item[{\color{blue} b)}]  si $ c|a $ y $c|b$ entonces $ c|d$.
        \end{enumerate}
        \end{definicion}\pause
        \vskip .4cm 
        \begin{itemize}
            \item[$\bullet$]  La condición (a) nos dice que $d$ es un común divisor de $a$ y $b$.
            \vskip .4cm \pause
            
            \item[$\bullet$] La condición (b) nos dice que cualquier divisor común de
            $a$ y $b$ es también divisor de $d$. 
        \end{itemize}
        
        \vskip .4cm 
        
    \end{frame}
    
    
    \begin{frame}
        
        \begin{ejemplo}
            ¿Cuál es el mcd entre $60$ y $84$?
        \end{ejemplo}
         \pause
        \begin{solucion}
            \pause
            \begin{itemize}
                \item[$\bullet$] $6$ es un divisor común de $60$ y $84$, pero no es el mayor divisor común, porque
                $12|60$ y $12|84$ pero $12{\not|}6$.
                \vskip .2cm 
                \item[$\bullet$] Los divisores positivos comunes de 60  y 84 son 1, 2, 3, 6 y 12, luego aunque 6  es un divisor común, no satisface (2) de la definición. 
                \vskip .2cm 
                \item[$\bullet$] En este caso, 12  claramente es  el  \textbf{máximo común divisor}.
            \end{itemize}
        \end{solucion}

        
        
    \end{frame}
    
    
    \begin{frame}
        
        \begin{preguntas}
            \pause

            \begin{itemize}
                \item[$\bullet$] Dados $a,b \in \mathbb Z$ arbitrarios, alguno de ellos no nulo ¿existe el máximo común divisor? \pause \newline \textbf{Rta:} Sí.
                \vskip .4cm \pause
                \item[$\bullet$]
                Si existe, ¿hay una forma eficiente de calcularlo? \pause \newline \textbf{Rta:} Sí.
                \vskip .4cm \pause
                \item[$\bullet$]  ¿Cuántos máximos común divisores puede tener un  par de enteros? \pause\newline \textbf{Rta:} 1. 
            \end{itemize}
            \vskip .8cm 
        \end{preguntas}

        
    \end{frame}
    
    
    \begin{frame}
        La primera y tercera pregunta son respondidas por el siguiente:
        \vskip .4cm
        \begin{teorema}\label{th-mcd}
            {\it Dados $a,b \in \mathbb Z$, alguno de ellos no nulo, existe un único $d \in \mathbb Z$ que es el máximo común divisor. }
            \vskip .2cm \pause
        \end{teorema}
        \begin{idea}
            \pause 
            \begin{equation*}
                S = \{m a+n b : m,n\in \mathbb Z, m a+n b>0\} \subset \mathbb N.
            \end{equation*}
            
            El mínimo de $S$ es el mcd.\qed
        \end{idea}
        \pause 
        \vskip .4cm 
        \begin{notacion}
            Sean $a,b \in \mathbb Z$, alguno de ellos no nulo, denotamos $\mcd(a,b)$ o $(a,b)$ al máximo común divisor entre $a$ y $b$.
        \end{notacion}

        
    \end{frame}
    
    
    \begin{frame}
        
        
        \begin{ejemplo}
            Hallar   $\mcd(174,72)$.
        \end{ejemplo}
         \pause    
        \begin{solucion}
            \pause
        Divisores de 174: 1, 2, 3, 6, 29, 58, 87, 174\pause
        \vskip .2cm         
        Divisores de 72: 1, 2, 3, 4, 6, 8, 9, 12, 18, 24, 36, 72 \pause
        \vskip .2cm         
        Luego, $6$ es divisor común de 174 y 72, y todos los demás divisores comunes ($1$, $2$ y $3$) dividen a $6$. 
        \vskip .2cm     
        
        Por lo tanto $\mcd(174,72) =6$.\qed
        \end{solucion} 
        
    \end{frame}
    
    
    \begin{frame}
        
        \begin{proposicion} 
        Sean $a,b \in \mathbb Z$, alguno de ellos no nulo. Entonces existen $s,t \in \mathbb Z$ tal que
        \begin{equation*}
            (a,b) = sa + tb. 
        \end{equation*}
        \end{proposicion}
        \begin{proof}
            Es consecuencia inmediata de la demostración del  teorema de la p. \ref{th-mcd}. \qed
        \end{proof}
        \pause
        \vskip .2cm 
        \begin{corolario}
            Sean $a$ y $b$ enteros, $b$ no nulo, entonces
            $$
            (a,b) = 1 \Leftrightarrow \text{existen $s,t \in \mathbb Z$ tales que $1 = sa+tb$.}
            $$
        \end{corolario}
        \pause
    
        
    \end{frame}
    
    \begin{frame}
        \frametitle{}
        
    \begin{definicion}
        Si  $(a,b)=1$ entonces decimos que $a$ y $ b$ son {\em coprimos}\index{coprimos}.
    \end{definicion}

    \begin{observacion}
        Por el corolario de la página anterior 
        $$
        a,b \text{ coprimos }\quad \Leftrightarrow\quad \text{existen $s,t \in \mathbb Z$ tales que $1 = sa+tb$.}
        $$
    \end{observacion}

    \begin{observacion}
        \textit{NO} es cierto  que si existen $s,t \in \mathbb Z$ tales que $d = sa+tb$ $\Rightarrow$ $d = (a,b)$.
\vskip .4cm
        Por ejemplo, $4 = 2 \cdot 6 + (-2) \cdot 4$ y $(6,4) =2$. 
    \end{observacion}
        
        
    
    \end{frame}
    
    \begin{frame}
        
        \begin{proposicion} Sean $a,b$ enteros con $a \not = 0$, entonces
            \begin{enumerate}
                \item[1.] $\mcd(b,a) = \mcd(a,b) = \mcd(\pm a, \pm b)$,\pause
                \item[2.] si $a>0$,  $\mcd(a,0) = a$ y $\mcd(a,a) = a$,\pause
                \item[3.] $\mcd(1,b) = 1$.
            \end{enumerate}
        \end{proposicion}
        \pause
        \begin{proof}
            Estas propiedades son de demostración casi trivial, por ejemplo para demostrar que  $\mcd(1,b) = 1$ comprobamos que 1 cumple con la definición:
            \begin{enumerate}
                \item[({a})] $ 1|1$ y $1|b$;
                \item[({b})] si $ c|1 $ y $c|b$ entonces $ c|1$,
            \end{enumerate}
            propiedades que son obviamente verdaderas.
            
            1. y 2.  se dejan a cargo del lector. \qed
        \end{proof}
        \vskip .8cm
        
        
    \end{frame}
    
    
    \begin{frame}
        La siguiente propiedad no es tan obvia y resulta muy importante. 
        \vskip .2cm
        \begin{propiedad}
            Si $a \not=0, b \in \mathbb Z$, entonces $\mcd(a,b) = \mcd(a,b-a)$. 
        \end{propiedad}\pause
        \begin{proof}
            Sea $d =  \mcd(a,b)$, luego 
            \vskip .2cm
            (a) $ d|a$ y $d|b$ \; \quad \;\,\, y \; (b) si $ c|a $ y $c|b$ entonces $ c|d$.\pause
            \vskip .2cm
            Debemos probar que 
            \vskip .2cm
            (a') $ d|a$ y $d|b -a$ \; y \; (b') si $ c|a $ y $c|b -a$ entonces $ c|d$.\pause
            \vskip .6cm
            Por (a),  $ d|a$ y $d|b$ $\Rightarrow$  $d|b -a$  $\Rightarrow$ (a').\pause
            \vskip .2cm
            Si $ c|a $ y $c|b -a$ $\Rightarrow$  $c|a +(b -a) = b$ $\stackrel{(b)}{\Rightarrow}$ $ c|d$ $\Rightarrow$  (b').

            \qed

        \end{proof}
        
    \end{frame}
    
    
    \begin{frame}
        \begin{ejemplo} Encontrar el mcd entre 72 y 174.
        \end{ejemplo} \pause\vskip -.6cm
        $$
        \begin{matrix*}[l]
            \text{{\color{blue}Solución:}\qquad} & (72, 174)     &=  (72,174-72) =  (72,102)\qquad\qquad\qquad{}^{}\\
            &&=  (72,102 - 72) = (72,30)\\
            &&= (30,72)  \\
            &&= (30,72 -30) = (42, 30) \\
            &&=  (30,42)\\
            &&= (30,42 -30) =(30,12) \\
            &&= (12,30) \\
            &&= (12, 30-12) = (12,18)\\
            &&=  (12, 18-12)= (12,6)\\
            &&= (6,12) \\
            &&= (6,12-6) =(6,6)\\
            &&= (6, 6-6) = (6,0) = 6. \qquad\qquad\qquad\qquad\qquad\qquad\qed 
        \end{matrix*}
        $$
        
    \end{frame}
    
    
    \begin{frame}
        
        $\bullet$ En  general no es sencillo encontrar todos los divisores de un número entero grande. 
        \vskip .2cm
        $\bullet$ No es factible calcular el  mcd de números grandes revisando todos los divisores comunes.  
        \vskip .2cm
        $\bullet$ El algoritmo anterior  nos da un método práctico y relativamente eficiente para calcular el mcd. 
        \vskip .6cm
        \pause
        La próxima proposición nos provee una herramienta aún mejor para calcular el mcd.
        \pause
        \begin{proposicion}\label{prop-alg-eucl} Sean  $a,b$ enteros no negativos con $b \not=0$, entonces 
            \begin{equation}\label{bec}
                a=bq+r\quad \Rightarrow \quad\mcd(a,b)=\mcd(b,r).
            \end{equation}
        \end{proposicion}
    \end{frame}
    
    
    \begin{frame}
        \begin{ejemplo} Encuentre el mcd de 174 y 72.
        \end{ejemplo}\pause
        \begin{solucion}\pause Con el uso repetido de la proposición anterior, obtenemos
            \begin{alignat*}3
                174&=72\cdot 2+30,&\quad&\text{ entonces  }& \quad(174,72)&= (72,30)\\
                72&=30\cdot 2+12,&\quad&\text{ entonces  }& (72,30)&= (30,12)\\
                30&=12\cdot 2+6,&\quad&\text{ entonces  }& (30,12)&= (12,6)\\
                12&=6\cdot 2+0,&\quad&\text{ entonces  }&(12,6)&= (6,0)=6.\\
            \end{alignat*}
            
            Por lo tanto $(174,72) = 6$. \qed
            
        \end{solucion}
    \end{frame}
    
    
    
\end{document}

