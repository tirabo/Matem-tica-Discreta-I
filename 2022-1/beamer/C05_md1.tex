%\documentclass{beamer} 
\documentclass[handout]{beamer} % sin pausas
\usetheme{CambridgeUS}

%\setbeamertemplate{background}[grid][step=8 ] % cuadriculado

\usepackage{etex}
\usepackage{t1enc}
\usepackage[spanish,es-nodecimaldot]{babel}
\usepackage{latexsym}
\usepackage[utf8]{inputenc}
\usepackage{verbatim}
\usepackage{multicol}
\usepackage{amsgen,amsmath,amstext,amsbsy,amsopn,amsfonts,amssymb}
\usepackage{amsthm}
\usepackage{calc}         % From LaTeX distribution
\usepackage{graphicx}     % From LaTeX distribution
\usepackage{ifthen}
%\usepackage{makeidx}
\input{random.tex}        % From CTAN/macros/generic
\usepackage{subfigure} 
\usepackage{tikz}
\usepackage[customcolors]{hf-tikz}
\usepackage[most]{tcolorbox}
\usetikzlibrary{arrows}
\usetikzlibrary{matrix}
\tikzset{
	every picture/.append style={
		execute at begin picture={\deactivatequoting},
		execute at end picture={\activatequoting}
	}
}
\usetikzlibrary{decorations.pathreplacing,angles,quotes}
\usetikzlibrary{shapes.geometric}
\usepackage{mathtools}
\usepackage{stackrel}
%\usepackage{enumerate}
\usepackage{enumitem}
\usepackage{tkz-graph}
\usepackage{polynom}
\polyset{%
	style=B,
	delims={(}{)},
	div=:
}
\renewcommand\labelitemi{$\circ$}
\setlist[enumerate]{label={(\arabic*)}}
%\setbeamertemplate{background}[grid][step=8 ] % cuadriculado
\setbeamertemplate{itemize item}{$\circ$}
\setbeamertemplate{enumerate items}[default]
\definecolor{links}{HTML}{2A1B81}
\hypersetup{colorlinks,linkcolor=,urlcolor=links}


\newcommand{\Id}{\operatorname{Id}}
\newcommand{\img}{\operatorname{Im}}
\newcommand{\nuc}{\operatorname{Nu}}
\newcommand{\im}{\operatorname{Im}}
\renewcommand\nu{\operatorname{Nu}}
\newcommand{\la}{\langle}
\newcommand{\ra}{\rangle}
\renewcommand{\t}{{\operatorname{t}}}
\renewcommand{\sin}{{\,\operatorname{sen}}}
\newcommand{\Q}{\mathbb Q}
\newcommand{\R}{\mathbb R}
\newcommand{\C}{\mathbb C}
\newcommand{\K}{\mathbb K}
\newcommand{\F}{\mathbb F}
\newcommand{\Z}{\mathbb Z}
\newcommand{\N}{\mathbb N}
\newcommand\sgn{\operatorname{sgn}}
\renewcommand{\t}{{\operatorname{t}}}
\renewcommand{\figurename }{Figura}

%
% Ver http://joshua.smcvt.edu/latex2e/_005cnewenvironment-_0026-_005crenewenvironment.html
%

\renewenvironment{block}[1]% environment name
{% begin code
	\par\vskip .2cm%
	{\color{blue}#1}%
	\vskip .2cm
}%
{%
	\vskip .2cm}% end code


\renewenvironment{alertblock}[1]% environment name
{% begin code
	\par\vskip .2cm%
	{\color{red!80!black}#1}%
	\vskip .2cm
}%
{%
	\vskip .2cm}% end code


\renewenvironment{exampleblock}[1]% environment name
{% begin code
	\par\vskip .2cm%
	{\color{blue}#1}%
	\vskip .2cm
}%
{%
	\vskip .2cm}% end code




\newenvironment{exercise}[1]% environment name
{% begin code
	\par\vspace{\baselineskip}\noindent
	\textbf{Ejercicio (#1)}\begin{itshape}%
		\par\vspace{\baselineskip}\noindent\ignorespaces
	}%
	{% end code
	\end{itshape}\ignorespacesafterend
}


\newenvironment{definicion}[1][]% environment name
{% begin code
	\par\vskip .2cm%
	{\color{blue}Definición #1}%
	\vskip .2cm
}%
{%
	\vskip .2cm}% end code

    \newenvironment{notacion}[1][]% environment name
    {% begin code
        \par\vskip .2cm%
        {\color{blue}Notación #1}%
        \vskip .2cm
    }%
    {%
        \vskip .2cm}% end code

\newenvironment{observacion}[1][]% environment name
{% begin code
	\par\vskip .2cm%
	{\color{blue}Observación #1}%
	\vskip .2cm
}%
{%
	\vskip .2cm}% end code

\newenvironment{ejemplo}[1][]% environment name
{% begin code
	\par\vskip .2cm%
	{\color{blue}Ejemplo #1}%
	\vskip .2cm
}%
{%
	\vskip .2cm}% end code


\newenvironment{preguntas}[1][]% environment name
{% begin code
    \par\vskip .2cm%
    {\color{blue}Preguntas #1}%
    \vskip .2cm
}%
{%
    \vskip .2cm}% end code

\newenvironment{ejercicio}[1][]% environment name
{% begin code
	\par\vskip .2cm%
	{\color{blue}Ejercicio #1}%
	\vskip .2cm
}%
{%
	\vskip .2cm}% end code


\renewenvironment{proof}% environment name
{% begin code
	\par\vskip .2cm%
	{\color{blue}Demostración}%
	\vskip .2cm
}%
{%
	\vskip .2cm}% end code



\newenvironment{demostracion}% environment name
{% begin code
	\par\vskip .2cm%
	{\color{blue}Demostración}%
	\vskip .2cm
}%
{%
	\vskip .2cm}% end code

\newenvironment{idea}% environment name
{% begin code
	\par\vskip .2cm%
	{\color{blue}Idea de la demostración}%
	\vskip .2cm
}%
{%
	\vskip .2cm}% end code

\newenvironment{solucion}% environment name
{% begin code
	\par\vskip .2cm%
	{\color{blue}Solución}%
	\vskip .2cm
}%
{%
	\vskip .2cm}% end code



\newenvironment{lema}[1][]% environment name
{% begin code
	\par\vskip .2cm%
	{\color{blue}Lema #1}\begin{itshape}%
		\par\vskip .2cm
	}%
	{% end code
	\end{itshape}\vskip .2cm\ignorespacesafterend
}

\newenvironment{proposicion}[1][]% environment name
{% begin code
	\par\vskip .2cm%
	{\color{blue}Proposición #1}\begin{itshape}%
		\par\vskip .2cm
	}%
	{% end code
	\end{itshape}\vskip .2cm\ignorespacesafterend
}

\newenvironment{teorema}[1][]% environment name
{% begin code
	\par\vskip .2cm%
	{\color{blue}Teorema #1}\begin{itshape}%
		\par\vskip .2cm
	}%
	{% end code
	\end{itshape}\vskip .2cm\ignorespacesafterend
}


\newenvironment{corolario}[1][]% environment name
{% begin code
	\par\vskip .2cm%
	{\color{blue}Corolario #1}\begin{itshape}%
		\par\vskip .2cm
	}%
	{% end code
	\end{itshape}\vskip .2cm\ignorespacesafterend
}

\newenvironment{propiedad}% environment name
{% begin code
	\par\vskip .2cm%
	{\color{blue}Propiedad}\begin{itshape}%
		\par\vskip .2cm
	}%
	{% end code
	\end{itshape}\vskip .2cm\ignorespacesafterend
}

\newenvironment{conclusion}% environment name
{% begin code
	\par\vskip .2cm%
	{\color{blue}Conclusión}\begin{itshape}%
		\par\vskip .2cm
	}%
	{% end code
	\end{itshape}\vskip .2cm\ignorespacesafterend
}


\newenvironment{definicion*}% environment name
{% begin code
	\par\vskip .2cm%
	{\color{blue}Definición}%
	\vskip .2cm
}%
{%
	\vskip .2cm}% end code

\newenvironment{observacion*}% environment name
{% begin code
	\par\vskip .2cm%
	{\color{blue}Observación}%
	\vskip .2cm
}%
{%
	\vskip .2cm}% end code


\newenvironment{obs*}% environment name
	{% begin code
		\par\vskip .2cm%
		{\color{blue}Observación}%
		\vskip .2cm
	}%
	{%
		\vskip .2cm}% end code

\newenvironment{ejemplo*}% environment name
{% begin code
	\par\vskip .2cm%
	{\color{blue}Ejemplo}%
	\vskip .2cm
}%
{%
	\vskip .2cm}% end code

\newenvironment{ejercicio*}% environment name
{% begin code
	\par\vskip .2cm%
	{\color{blue}Ejercicio}%
	\vskip .2cm
}%
{%
	\vskip .2cm}% end code

\newenvironment{propiedad*}% environment name
{% begin code
	\par\vskip .2cm%
	{\color{blue}Propiedad}\begin{itshape}%
		\par\vskip .2cm
	}%
	{% end code
	\end{itshape}\vskip .2cm\ignorespacesafterend
}

\newenvironment{conclusion*}% environment name
{% begin code
	\par\vskip .2cm%
	{\color{blue}Conclusión}\begin{itshape}%
		\par\vskip .2cm
	}%
	{% end code
	\end{itshape}\vskip .2cm\ignorespacesafterend
}






\newcommand{\nc}{\newcommand}

%%%%%%%%%%%%%%%%%%%%%%%%%LETRAS

\nc{\FF}{{\mathbb F}} \nc{\NN}{{\mathbb N}} \nc{\QQ}{{\mathbb Q}}
\nc{\PP}{{\mathbb P}} \nc{\DD}{{\mathbb D}} \nc{\Sn}{{\mathbb S}}
\nc{\uno}{\mathbb{1}} \nc{\BB}{{\mathbb B}} \nc{\An}{{\mathbb A}}

\nc{\ba}{\mathbf{a}} \nc{\bb}{\mathbf{b}} \nc{\bt}{\mathbf{t}}
\nc{\bB}{\mathbf{B}}

\nc{\cP}{\mathcal{P}} \nc{\cU}{\mathcal{U}} \nc{\cX}{\mathcal{X}}
\nc{\cE}{\mathcal{E}} \nc{\cS}{\mathcal{S}} \nc{\cA}{\mathcal{A}}
\nc{\cC}{\mathcal{C}} \nc{\cO}{\mathcal{O}} \nc{\cQ}{\mathcal{Q}}
\nc{\cB}{\mathcal{B}} \nc{\cJ}{\mathcal{J}} \nc{\cI}{\mathcal{I}}
\nc{\cM}{\mathcal{M}} \nc{\cK}{\mathcal{K}}

\nc{\fD}{\mathfrak{D}} \nc{\fI}{\mathfrak{I}} \nc{\fJ}{\mathfrak{J}}
\nc{\fS}{\mathfrak{S}} \nc{\gA}{\mathfrak{A}}
%%%%%%%%%%%%%%%%%%%%%%%%%LETRAS

\title[Clase 5 - Inducción]{Matemática Discreta I \\ Clase 5 - Ejercicios de inducción}
%\author[C. Olmos / A. Tiraboschi]{Carlos Olmos / Alejandro Tiraboschi}
\institute[]{\normalsize FAMAF / UNC
	\\[\baselineskip] ${}^{}$
	\\[\baselineskip]
}
\date[31/03/2022]{31 de marzo   de 2022}




\begin{document}

\frame{\titlepage} 

\begin{frame}\frametitle{} 

    \begin{ejercicio}[(Serie aritmética)]\label{serie-aritmetica}
        Probar que 
        $$
        1 + 2+ 3 + \cdots + n = \frac{n(n+1)}{2},
        $$
        para $n \in \N$.
    \end{ejercicio}

    \vskip .2cm\pause
    \begin{proof}
        \vskip .4cm \pause
        \noindent({\it Caso  base}) El resultado es verdadero
            cuando $n=1$ pues $ 1 = \frac{1 \cdot (1+1)}{2}$.
        
        \vskip .4cm 

    \end{proof}
	
\end{frame}


\begin{frame}
    \frametitle{}

		\noindent ({\it Paso  inductivo}) 
        \pause \vskip .4cm
		
        Supongamos que el resultado verdadero cuando $n=k$, o
		sea, que 
        $$
        \sum_{i=1}^{k} i = \frac{k(k+1)}{2}\quad \text{hipótesis inductiva (HI).}
        $$
        \vskip .4cm\pause
        En el paso inductivo queremos probar que 
        $$
        \sum_{i=1}^{k} i = \frac{k(k+1)}{2} \quad \Rightarrow \quad    \sum_{i=1}^{k+1} i =\frac{(k+1)(k+2)}{2}.
        $$
\vskip 1cm

\end{frame}

\begin{frame}
    \frametitle{}

    Entonces

    \medskip
    
    
    \begin{tabular}{lllll}
        $\displaystyle\sum_{i=1}^{k+1} i$ &$=$& $\displaystyle\sum_{i=0}^{k} i + (k+1)$ &\qquad &$\text{(por la definición recursiva de $\sum$)}$ \\[0.6cm]
    &$=$& $\displaystyle\frac{k(k+1)}{2}+(k+1)$ &\qquad &$\text{(por hipótesis inductiva)}$ \\[0.6cm]
    &$=$& $\displaystyle(k+1)(\frac{k}{2}+1))$ &\qquad &$\text{($(k+1)$ factor común)}$ \\[0.6cm]
    &$=$& $\displaystyle(k+1)\frac{(k+2)}{2}.$&\qquad &\\[0.6cm]
    &$=$& $\displaystyle\frac{(k+1)(k+2)}{2}.$&\qquad &
    \end{tabular}
    \medskip
    \pause
    
    Luego el resultado es verdadero cuando $n=k+1$ y por el principio de inducción, es verdadero para todos los enteros positivos $n$.\qed

\end{frame}

\begin{frame}\frametitle{} 

    \begin{ejercicio}[(Serie geométrica)]\label{serie-geometrica}
        Probar que 
        $$
        1 + q+ q^2 + \cdots + q^n = \sum_{i=0}^{n} q^i = \frac{q^{n+1}-1}{q -1},
        $$
        para $n \in \N_0$, $q >0$ y $q \ne 1$.
    \end{ejercicio}

    \vskip .2cm\pause
    \begin{proof} \pause
        \vskip .2cm 
        \noindent({\it Caso  base}) El resultado es verdadero
            cuando $n=0$ pues 
            $$q^0 = 1 =  \frac{q^{0+1}-1}{q -1}.$$
        
        \vskip .4cm 

    \end{proof}
	
\end{frame}

\begin{frame}
    \frametitle{}

		\noindent ({\it Paso  inductivo})
		Supongamos que el resultado verdadero cuando $n=k$, o
		sea, que 
        $$
        \sum_{i=0}^{k} q^i = \frac{q^{k+1}-1}{q -1},\quad \text{hipótesis inductiva (HI).}
        $$
        \vskip .4cm\pause
        En el paso inductivo queremos probar que 
        $$
        \sum_{i=0}^{k} q^i = \frac{q^{k+1}-1}{q -1} \quad \Rightarrow \quad   \sum_{i=0}^{k+1} q^i = \frac{q^{k+2}-1}{q -1}.
        $$
\vskip 1cm

\end{frame}

\begin{frame}
    \frametitle{}

    Entonces

    \medskip
    
    
    \begin{tabular}{lllll}
        $\displaystyle\sum_{i=0}^{k+1} q^i $ &$=$& $\displaystyle\sum_{i=0}^{k} q^i  + q^{k+1}$ &\qquad &$\text{(por la definición recursiva)}$ \\[0.6cm]
    &$=$& $\displaystyle\ \frac{q^{k+1}-1}{q -1}+ q^{k+1}$ &\qquad &$\text{(por hipótesis inductiva)}$ \\[0.6cm]
    &$=$& $\displaystyle\ \frac{q^{k+1}-1+ (q-1) q^{k+1}}{q -1}$ &\qquad & \\[0.6cm]
    &$=$& $\displaystyle\ \frac{q^{k+1}-1+ q\cdot q^{k+1}- q^{k+1}}{q -1}$ &\qquad &\\[0.6cm]
    &$=$& $\displaystyle\ \frac{q^{k+2}-1}{q -1}$ &\qquad &
    \end{tabular}
    \medskip
    \pause
    
    Luego el resultado es verdadero cuando $n=k+1$ y por el principio de inducción, es verdadero para todos los enteros positivos $n$.\qed

\end{frame}

\begin{frame}\frametitle{} 

    \begin{ejercicio}
        Probar que 
        $$
        (1 + 2+ 3 + \cdots + n)^2 = 1^3 +2^3 +3^3 + \cdots +n^3,
        $$
        para $n \in \N$.
    \end{ejercicio}\pause
\begin{proof}\pause
    Para probar esto primero debemos usar el resultado de la página \ref{serie-aritmetica} que nos dice  que 
    $$
    (1 + 2+ 3 + \cdots + n)^2 = \left(\sum_{i=1}^n i\right)^2 = \left(\frac{n(n+1)}{2}\right)^2 = \frac{n^2(n+1)^2}{4}.
    $$\pause
    Luego debemos probar que 
    $$
    \sum_{i=1}^{n} i^3 =  \frac{n^2(n+1)^2}{4}.
    $$
\end{proof}
\end{frame}


\begin{frame}
    \frametitle{}

        \vskip .4cm 
        \noindent({\it Caso  base})\pause El resultado es verdadero
            cuando $n=1$ pues 
            $$ \sum_{i=1}^{1} i^3 = 1\quad \text{y}\quad\frac{1^2(1+1)^2}{4} =\frac{4}{4} =1.$$
        
        \vskip .4cm \pause
        \noindent ({\it Paso  inductivo})\pause
		Debemos probar que si $k \ge 1$, 
        $$
        \sum_{i=0}^{k} i^3 = \frac{k^2(k+1)^2}{4} \;\; \text{ (HI) }\quad \Rightarrow \quad    \sum_{i=0}^{k+1} i^3 =\frac{(k+1)^2(k+2)^2}{4}.
        $$

    

\end{frame}


\begin{frame}
    \frametitle{}
    
    \begin{tabular}{lllll}
        $\displaystyle\sum_{i=0}^{k+1} i^3$ &$=$& $\displaystyle\sum_{i=0}^{k} i^3 + (k+1)^3$ &\qquad &$\text{(por la definición recursiva de $\sum$ )}$ \\[0.6cm]
    &$=$& $\displaystyle\frac{k^2(k+1)^2}{4}+(k+1)^3$ &\quad &$\text{(por hipótesis inductiva)}$ \\[0.6cm]
    &$=$& $\displaystyle(k+1)^2(\frac{k^2}{4}+(k+1)))$ &\quad &$\text{($(k+1)^2$ factor común)}$ \\[0.6cm]
    &$=$& $\displaystyle(k+1)^2\frac{(k^2+4k+4)}{4}.$&\quad &\\[0.6cm]
    &$=$& $\displaystyle\frac{(k+1)^2(k+2)^2}{4}.$&\quad &
    \end{tabular}
    \medskip
    \pause
    
    Luego el resultado es verdadero cuando $n=k+1$ y por el principio de inducción, es verdadero para todos los enteros positivos $n$.\qed

\end{frame}

\begin{frame}
    \frametitle{Inducción completa}

    \begin{ejercicio}
        Sea $u_n$ definida recursivamente por:
        \begin{equation*}
        u_0 = 1, \; u_1 = 0, \;\qquad u_n = 5u_{n-1} - 6 u_{n-2}, \; \forall n \ge 2. 
        \end{equation*}
        \begin{enumerate}
            \item[1.]  Calcule $u_2$ y $u_3$ usando  recursión.
            \item[2.] Pruebe por inducción que $u_n = 3\cdot 2^n - 2 \cdot 3^n$ ($n \in \N_0$). 
        \end{enumerate}
        \vskip .2cm
    \end{ejercicio}\pause

    \begin{solucion}\pause
        {\noindent 1.} Por definición $u_0 = 1, \; u_1 = 0$, luego:
        \begin{equation*}
        u_2 = 5u_{2-1} - 6 u_{2-2} = 5u_{1} - 6 u_{0} = 5 \cdot 0 - 6 \cdot 1 = -6 
        \end{equation*}\pause
        Ahora  $u_1 = 0, \; u_2 = -6$, luego:\pause
        \begin{equation*}
        u_3 = 5u_{3-1} - 6 u_{3-2} = 5u_{2} - 6 u_{1} = 5 \cdot (-6) - 6 \cdot 0 = -30. 
        \end{equation*}
    \end{solucion}

\end{frame}


\begin{frame}
    \frametitle{}


        {\noindent 2.} \noindent ({\it Caso base})\pause
        \vskip .2cm
        Por un lado $u_0 =1$, por definición, por otro lado  si calculamos con la fórmula: $u_0 = 3 \cdot 2^0 - 2 \cdot 3^0 = 3- 2 = 1$ y listo.
        \vskip .2cm\pause
        Por un lado $u_1 =0$, por definición, por otro lado  si calculamos con la fórmula: $u_1 = 3 \cdot 2^1 - 2 \cdot 3^1 = 3 \cdot 2- 2 \cdot 3 =  6-6 = 0$ y listo.\pause
        
        \vskip .6cm
        
        \noindent ({\it Paso  inductivo}) \pause
        \vskip .2cm
        Debemos probar que si  vale
        \begin{equation*}
        u_h = 3\cdot 2^h - 2 \cdot 3^h \text{ \;para\; }  0 \le h \le k \text{\; con\; }  k \ge 1, \tag{HI}
        \end{equation*}\pause
        eso implica que
        \begin{equation*}
        u_{k+1} = 3\cdot 2^{k+1} - 2 \cdot 3^{k+1}. \tag{*}
        \end{equation*}
        
\end{frame}


\begin{frame}
    \frametitle{}
    Comenzamos por el lado izquierdo de (*):
        \begin{align*}
        u_{k+1} &=   5u_{k+1-1} - 6 u_{k+1-2}&&\text{ \;(def. $u_n$)\; } \\
        &=   5u_{k} - 6 u_{k-1}&&\\
        &=   5( 3\cdot 2^{k} - 2 \cdot 3^{k}) - 6 ( 3\cdot 2^{k-1} - 2 \cdot 3^{k-1})&&\text{ \;(por HI)\; }
        \end{align*}\vskip .2cm
        En  lo que se refiere al procedimiento de inducción hemos terminado,  ahora solo queda por probar que
        \begin{equation}
            5( 3\cdot 2^{k} - 2 \cdot 3^{k}) - 6 ( 3\cdot 2^{k-1} - 2 \cdot 3^{k-1}) =
        3\cdot 2^{k+1} - 2 \cdot 3^{k+1} \tag{**}
        \end{equation}
        
        \vskip .2cm
        Desarrollemos el término de la izquierda
        \pause
        \begin{align*}
        \qquad &=  5( 3\cdot 2^{k} - 2 \cdot 3^{k}) - 6 ( 3\cdot 2^{k-1} - 2 \cdot 3^{k-1})&& \\
        &=   5\cdot  3\cdot 2^{k} - 5 \cdot 2 \cdot 3^{k} - 6 \cdot  3\cdot 2^{k-1} + 6 \cdot  2 \cdot 3^{k-1}&& \\
        &=   5\cdot  3\cdot 2^{k} - 5 \cdot 2 \cdot 3^{k} - 3 \cdot  3\cdot 2^{k} + 2 \cdot  2 \cdot 3^{k}&& \\
        &=   6\cdot 2^{k} - 6 \cdot 3^{k} && \\
        \end{align*}
        \pause
        
\end{frame}

\begin{frame}
    \frametitle{}
Luego (**) se transforma  en 
\begin{gather*}
    6\cdot 2^{k} - 6 \cdot 3^{k} =
    3\cdot 2^{k+1} - 2 \cdot 3^{k+1} \tag{***} \\
\Updownarrow\\
3\cdot 2 \cdot 2^{k} - 2 \cdot 3 \cdot 3^{k} =
    3\cdot 2^{k+1} - 2 \cdot 3^{k+1} \\
    \Updownarrow\\
    3\cdot 2^{k+1} - 2 \cdot 3^{k+1}  = 3\cdot 2^{k+1} - 2 \cdot 3^{k+1}. 
\end{gather*}

Como esto último es verdadero,  es verdadero (***) y por lo tanto son verdaderos (**) y (*). 

\qed

\end{frame}




\begin{frame}
    \frametitle{}

Nos adelantamos un poco en los temas de la materia para ejemplificar el principio de inducción completa. \pause

\begin{definicion} Sea $n \in \N$ diremos que es \textit{primo} si $n \ne 1$ y el  único natural menor que lo divide es  $1$.
\end{definicion}
    \vskip .4cm\pause
\begin{ejercicio}
    Probar que si $n \in \N$, $n >1$,  entonces $n$  es producto de primos. 
\end{ejercicio}\pause

\begin{proof}\pause
    Lo haremos por inducción en $n \ge 2$. 
\end{proof}



\end{frame}


\begin{frame}
    \frametitle{}
    \noindent ({\it Caso base})\pause
    \vskip .4cm
    El caso base es $n=2$,  y es claro que $2$ es primo (y por lo tanto producto de primos  en un sentido generalizado).  
    \vskip .6cm\pause
    \noindent ({\it Paso inductivo})\pause
    \vskip .4cm
    Probaremos que dado $k>1$, 

    \begin{gather*}
        \text{todo $h$ tal que $2 \le h \le k$, es producto de primos (HI)} \\[.2cm]
        \Downarrow \\[.2cm]
        \text{$k+1$  es producto de primos. }
    \end{gather*}

\end{frame}


\begin{frame}
    \frametitle{}
Si $k+1$ es primo, listo,  es producto de primos.\pause
\vskip .4cm
Si $k+1$ no es primo,  significa que $k+1= d \cdot e$ donde $d, e < k+1$. \pause
\vskip .4cm

Por (HI), $d$ y $e$ son productos de primos,  es decir
\begin{align*}
    d &= p_1\cdot p_2 \cdots p_r \\ 
    e &= q_1\cdot q_2 \cdots q_s \\ 
\end{align*}\vskip -.6cm
donde $p_i, q_i$ son primos.\pause Luego
\begin{equation*}
    k +1 = d \cdot e = p_1\cdot p_2 \cdots p_r \cdot q_1\cdot q_2 \cdots q_s. 
\end{equation*}
\vskip .4cm
\pause
Por lo tanto $k+1$  es producto de primos.

\qed
    

\end{frame}



\end{document}

