\documentclass[handout]{beamer} % sin pausas
%\documentclass{beamer} % con pausas
%\setbeamertemplate{background}[grid][step=8 ] % cuadriculado

\usetheme{CambridgeUS}


\usepackage{etex}
\usepackage{t1enc}
\usepackage[spanish,es-nodecimaldot]{babel}
\usepackage{latexsym}
\usepackage[utf8]{inputenc}
\usepackage{verbatim}
\usepackage{multicol}
\usepackage{amsgen,amsmath,amstext,amsbsy,amsopn,amsfonts,amssymb}
\usepackage{amsthm}
\usepackage{calc}         % From LaTeX distribution
\usepackage{graphicx}     % From LaTeX distribution
\usepackage{ifthen}
%\usepackage{makeidx}
\input{random.tex}        % From CTAN/macros/generic
\usepackage{subfigure} 
\usepackage{tikz}
\usepackage[customcolors]{hf-tikz}
\usetikzlibrary{arrows}
\usetikzlibrary{matrix}
\tikzset{
    every picture/.append style={
        execute at begin picture={\deactivatequoting},
        execute at end picture={\activatequoting}
    }
}
\usetikzlibrary{decorations.pathreplacing,angles,quotes}
\usetikzlibrary{shapes.geometric}
\usepackage{mathtools}
\usepackage{stackrel}
%\usepackage{enumerate}
\usepackage{enumitem}
\usepackage{tkz-graph}
\usepackage{polynom}
\polyset{%
    style=B,
    delims={(}{)},
    div=:
}
\renewcommand\labelitemi{$\circ$}
\setlist[enumerate]{label={(\arabic*)}}
\setbeamertemplate{itemize item}{$\circ$}
\setbeamertemplate{enumerate items}[default]
\definecolor{links}{HTML}{2A1B81}
\hypersetup{colorlinks,linkcolor=,urlcolor=links}


\newcommand{\Id}{\operatorname{Id}}
\newcommand{\img}{\operatorname{Im}}
\newcommand{\nuc}{\operatorname{Nu}}
\newcommand{\im}{\operatorname{Im}}
\renewcommand\nu{\operatorname{Nu}}
\newcommand{\la}{\langle}
\newcommand{\ra}{\rangle}
\renewcommand{\t}{{\operatorname{t}}}
\renewcommand{\sin}{{\,\operatorname{sen}}}
\newcommand{\Q}{\mathbb Q}
\newcommand{\R}{\mathbb R}
\newcommand{\C}{\mathbb C}
\newcommand{\K}{\mathbb K}
\newcommand{\F}{\mathbb F}
\newcommand{\Z}{\mathbb Z}
\newcommand{\N}{\mathbb N}
\newcommand\sgn{\operatorname{sgn}}
\renewcommand{\t}{{\operatorname{t}}}
\renewcommand{\figurename }{Figura}

%
% Ver http://joshua.smcvt.edu/latex2e/_005cnewenvironment-_0026-_005crenewenvironment.html
%

\renewenvironment{block}[1]% environment name
{% begin code
	\par\vskip .2cm%
	{\color{blue}#1}%
	\vskip .2cm
}%
{%
	\vskip .2cm}% end code


\renewenvironment{alertblock}[1]% environment name
{% begin code
	\par\vskip .2cm%
	{\color{red!80!black}#1}%
	\vskip .2cm
}%
{%
	\vskip .2cm}% end code


\renewenvironment{exampleblock}[1]% environment name
{% begin code
	\par\vskip .2cm%
	{\color{blue}#1}%
	\vskip .2cm
}%
{%
	\vskip .2cm}% end code




\newenvironment{exercise}[1]% environment name
{% begin code
	\par\vspace{\baselineskip}\noindent
	\textbf{Ejercicio (#1)}\begin{itshape}%
		\par\vspace{\baselineskip}\noindent\ignorespaces
	}%
	{% end code
	\end{itshape}\ignorespacesafterend
}


\newenvironment{definicion}[1][]% environment name
{% begin code
	\par\vskip .2cm%
	{\color{blue}Definición #1}%
	\vskip .2cm
}%
{%
	\vskip .2cm}% end code

    \newenvironment{notacion}[1][]% environment name
    {% begin code
        \par\vskip .2cm%
        {\color{blue}Notación #1}%
        \vskip .2cm
    }%
    {%
        \vskip .2cm}% end code

\newenvironment{observacion}[1][]% environment name
{% begin code
	\par\vskip .2cm%
	{\color{blue}Observación #1}%
	\vskip .2cm
}%
{%
	\vskip .2cm}% end code

\newenvironment{ejemplo}[1][]% environment name
{% begin code
	\par\vskip .2cm%
	{\color{blue}Ejemplo #1}%
	\vskip .2cm
}%
{%
	\vskip .2cm}% end code


\newenvironment{preguntas}[1][]% environment name
{% begin code
    \par\vskip .2cm%
    {\color{blue}Preguntas #1}%
    \vskip .2cm
}%
{%
    \vskip .2cm}% end code

\newenvironment{ejercicio}[1][]% environment name
{% begin code
	\par\vskip .2cm%
	{\color{blue}Ejercicio #1}%
	\vskip .2cm
}%
{%
	\vskip .2cm}% end code


\renewenvironment{proof}% environment name
{% begin code
	\par\vskip .2cm%
	{\color{blue}Demostración}%
	\vskip .2cm
}%
{%
	\vskip .2cm}% end code



\newenvironment{demostracion}% environment name
{% begin code
	\par\vskip .2cm%
	{\color{blue}Demostración}%
	\vskip .2cm
}%
{%
	\vskip .2cm}% end code

\newenvironment{idea}% environment name
{% begin code
	\par\vskip .2cm%
	{\color{blue}Idea de la demostración}%
	\vskip .2cm
}%
{%
	\vskip .2cm}% end code

\newenvironment{solucion}% environment name
{% begin code
	\par\vskip .2cm%
	{\color{blue}Solución}%
	\vskip .2cm
}%
{%
	\vskip .2cm}% end code



\newenvironment{lema}[1][]% environment name
{% begin code
	\par\vskip .2cm%
	{\color{blue}Lema #1}\begin{itshape}%
		\par\vskip .2cm
	}%
	{% end code
	\end{itshape}\vskip .2cm\ignorespacesafterend
}

\newenvironment{proposicion}[1][]% environment name
{% begin code
	\par\vskip .2cm%
	{\color{blue}Proposición #1}\begin{itshape}%
		\par\vskip .2cm
	}%
	{% end code
	\end{itshape}\vskip .2cm\ignorespacesafterend
}

\newenvironment{teorema}[1][]% environment name
{% begin code
	\par\vskip .2cm%
	{\color{blue}Teorema #1}\begin{itshape}%
		\par\vskip .2cm
	}%
	{% end code
	\end{itshape}\vskip .2cm\ignorespacesafterend
}


\newenvironment{corolario}[1][]% environment name
{% begin code
	\par\vskip .2cm%
	{\color{blue}Corolario #1}\begin{itshape}%
		\par\vskip .2cm
	}%
	{% end code
	\end{itshape}\vskip .2cm\ignorespacesafterend
}

\newenvironment{propiedad}% environment name
{% begin code
	\par\vskip .2cm%
	{\color{blue}Propiedad}\begin{itshape}%
		\par\vskip .2cm
	}%
	{% end code
	\end{itshape}\vskip .2cm\ignorespacesafterend
}

\newenvironment{conclusion}% environment name
{% begin code
	\par\vskip .2cm%
	{\color{blue}Conclusión}\begin{itshape}%
		\par\vskip .2cm
	}%
	{% end code
	\end{itshape}\vskip .2cm\ignorespacesafterend
}


\newenvironment{definicion*}% environment name
{% begin code
	\par\vskip .2cm%
	{\color{blue}Definición}%
	\vskip .2cm
}%
{%
	\vskip .2cm}% end code

\newenvironment{observacion*}% environment name
{% begin code
	\par\vskip .2cm%
	{\color{blue}Observación}%
	\vskip .2cm
}%
{%
	\vskip .2cm}% end code


\newenvironment{obs*}% environment name
	{% begin code
		\par\vskip .2cm%
		{\color{blue}Observación}%
		\vskip .2cm
	}%
	{%
		\vskip .2cm}% end code

\newenvironment{ejemplo*}% environment name
{% begin code
	\par\vskip .2cm%
	{\color{blue}Ejemplo}%
	\vskip .2cm
}%
{%
	\vskip .2cm}% end code

\newenvironment{ejercicio*}% environment name
{% begin code
	\par\vskip .2cm%
	{\color{blue}Ejercicio}%
	\vskip .2cm
}%
{%
	\vskip .2cm}% end code

\newenvironment{propiedad*}% environment name
{% begin code
	\par\vskip .2cm%
	{\color{blue}Propiedad}\begin{itshape}%
		\par\vskip .2cm
	}%
	{% end code
	\end{itshape}\vskip .2cm\ignorespacesafterend
}

\newenvironment{conclusion*}% environment name
{% begin code
	\par\vskip .2cm%
	{\color{blue}Conclusión}\begin{itshape}%
		\par\vskip .2cm
	}%
	{% end code
	\end{itshape}\vskip .2cm\ignorespacesafterend
}






\newcommand{\nc}{\newcommand}

%%%%%%%%%%%%%%%%%%%%%%%%%LETRAS

\nc{\FF}{{\mathbb F}} \nc{\NN}{{\mathbb N}} \nc{\QQ}{{\mathbb Q}}
\nc{\PP}{{\mathbb P}} \nc{\DD}{{\mathbb D}} \nc{\Sn}{{\mathbb S}}
\nc{\uno}{\mathbb{1}} \nc{\BB}{{\mathbb B}} \nc{\An}{{\mathbb A}}

\nc{\ba}{\mathbf{a}} \nc{\bb}{\mathbf{b}} \nc{\bt}{\mathbf{t}}
\nc{\bB}{\mathbf{B}}

\nc{\cP}{\mathcal{P}} \nc{\cU}{\mathcal{U}} \nc{\cX}{\mathcal{X}}
\nc{\cE}{\mathcal{E}} \nc{\cS}{\mathcal{S}} \nc{\cA}{\mathcal{A}}
\nc{\cC}{\mathcal{C}} \nc{\cO}{\mathcal{O}} \nc{\cQ}{\mathcal{Q}}
\nc{\cB}{\mathcal{B}} \nc{\cJ}{\mathcal{J}} \nc{\cI}{\mathcal{I}}
\nc{\cM}{\mathcal{M}} \nc{\cK}{\mathcal{K}}

\nc{\fD}{\mathfrak{D}} \nc{\fI}{\mathfrak{I}} \nc{\fJ}{\mathfrak{J}}
\nc{\fS}{\mathfrak{S}} \nc{\gA}{\mathfrak{A}}
%%%%%%%%%%%%%%%%%%%%%%%%%LETRAS


\title[Clase 18 - Grafos y sus representaciones]{Matemática Discreta I \\ Clase 18 - Grafos y sus representaciones}
%\author[C. Olmos / A. Tiraboschi]{Carlos Olmos / Alejandro Tiraboschi}
\institute[]{\normalsize FAMAF / UNC
    \\[\baselineskip] ${}^{}$
    \\[\baselineskip]
}
\date[31/05/2022]{31 de mayo de 2022}




\begin{document}
    
    \frame{\titlepage} 
    
    \begin{frame}\frametitle{Grafos y sus representaciones}
        Usaremos la siguiente definición en lo que sigue: dado un conjunto $X$ un {\em $2$-subconjunto} es un subconjunto de $X$ de dos elementos.  \pause
        \vskip .4cm
        \begin{definicion} Un {\em grafo} $G$ consiste de un  \index{grafo}
            conjunto finito $V$, cuyos miembros son llamados
            {\em vértices},  \index{vértices de un grafo}
            y un conjunto de $2$-subconjuntos de $V$, cuyos miembros son
            llamados {\em aristas}.  \index{aristas de un grafo}
            \pause
            \vskip .4cm
            Nosotros
            usualmente escribiremos $G=(V,E)$ y diremos que $V$ es el {\em
                conjunto de vértices} y $E$ es el {\em conjunto de aristas}.
        \end{definicion}
        
    \end{frame}
    
    
    \begin{frame}
        Un ejemplo típico de un grafo $G=(V,E)$ es dado por los conjuntos
        \begin{equation*}\label{grafosimple}
            V=\{a,b,c,d,z\}, \qquad\quad
            E=\{\{a,b\},\{a,d\},\{b,z\},\{c,d\},\{d,z\}\}.
        \end{equation*}\pause
        Este ejemplo y la definición misma no son demasiado
        esclarecedores, y solamente cuando con\-si\-de\-ra\-mos la {\it
            representación pictórica} de un grafo es cuando entendemos un poco más la definición.
        
        \pause
        \begin{figure}[ht]
            \begin{center}
                \begin{tikzpicture}[scale=1]
                    %\SetVertexSimple[Shape=circle,FillColor=white]
                    \Vertex[x=0.00, y=2.00]{$a$}
                    \Vertex[x=1.90, y=0.62]{$b$}
                    \Vertex[x=1.18, y=-1.62]{$c$}
                    \Vertex[x=-1.18, y=-1.62]{$d$}
                    \Vertex[x=-1.90, y=0.62]{$z$}
                    \Edges($c$, $d$,$a$,$b$,$z$,$d$)
                \end{tikzpicture}
            \end{center}    
        \end{figure}
    \end{frame}
    
    \begin{frame}
        
        \begin{itemize}
            \item[$\circ$]  La representación pictórica es intuitivamente atractiva
            pero no es útil cuando deseamos comunicarnos con una computadora.\pause
            \item[$\circ$] Debemos     re\-pre\-sen\-tar el grafo mediante conjuntos o una tabla.
        \end{itemize}
        \vskip .4cm\pause
        \begin{definicion}
            Diremos que dos vértices $x$ e $y$ de un grafo son {\em    adyacentes} cuando $\{x,y\}$ es una arista.
        \end{definicion}
        \pause
        \begin{definicion}
            Podemos representar un grafo $G=(V,E)$ por su {\em lista de    adyacencia}, donde cada vértice $v$     encabeza una lista de aquellos vértices que son adyacentes a $v$:
        \end{definicion}
        
        
        %\vskip.4cm
        
    \end{frame}
    
    
    \begin{frame}
        \begin{ejemplo}
            Vimos el $G=(V,E)$ es dado por
            \begin{equation*}\label{grafosimple}
                V=\{a,b,c,d,z\}, \qquad\quad
                E=\{\{a,b\},\{a,d\},\{b,z\},\{c,d\},\{d,z\}\}.
            \end{equation*}
            \pause    
            Su lista de adyacencia es 
            
            \begin{center}
                \begin{tabular}{ccccc}
                    $a$&$b$&$c$&$d$&$z$ \\ \hline
                    $b$&$a$&$d$&$a$&$b$ \\
                    $d$&$z$&&$c$&$d$\\
                    &&&$z$&
                \end{tabular}
            \end{center}
            \pause    
            En Python el grafo estaría representado por la lista de listas\\
            \begin{equation*}
                [[b,d],[a,z],[d],[a,c,z],[b,d]].
            \end{equation*}
        \end{ejemplo}
    \end{frame}    
    
    
    \begin{frame}
        
        
        
        
        \begin{definicion}
            Por cada entero positivo $n$ definimos el {\em grafo completo  \index{grafo completo}
                $K_n$} como el grafo con $n$ vértices y en el cual cada par de vértices es adyacente. 
        \end{definicion}
        
        \vskip.4cm\pause
        
        La lista de adyacencia de $K_n$ es unas lista donde en la columna del vértice $i$ están todos los vértices menos $i$ ($n-1$ vértices). 
        
        \vskip.4cm\pause
        
        
        ?`Cuántas aristas tiene $K_n$? \pause
        
        \vskip.2cm\pause
        
        De cada vértice ``salen'' $n-1$ aristas, las que van a otros vértices. 
        \vskip.2cm
        Si  sumamos $n$-veces las $n-1$ aristas es claro que estamos contando cada arista dos veces, luego el número total de aristas es $n(n-1)/2$.
        \pause
        \vskip.2cm
        Observar que esta es una demostración, usando  grafos, de que $\sum_{i=1}^{n-1} i = n(n-1)/2$.
        
        
    \end{frame}    
    
    
    
    \begin{frame}
        
        \begin{ejemplo} Mario y su mujer Abril dan una
            fiesta en la cual hay otras cuatro parejas de casados. Las
            parejas, cuando arriban, estrechan la mano a algunas personas,
            pero, naturalmente, no se estrechan la mano entre marido y mujer.
            Cuando la fiesta finaliza el profesor pregunta a los otros a
            cuantas personas han estrechado la mano, recibiendo $9$ respuestas
            diferentes. ?`Cuántas personas estrecharon la mano de Abril?
        \end{ejemplo}\pause
        \begin{solucion}\pause Construyamos un grafo cuyos vértices son las personas que asisten a la
            fiesta. Las aristas del grafo son las  $\{x,y\}$ siempre y cuando $x$ e $y$ se
            hayan estrechado las manos.
            \vskip.2cm\pause
            Puesto que hay nueve personas aparte
            de Mario, y que una persona puede estrechar a lo sumo
            a otras $8$ personas, se sigue que las $9$ respuestas diferentes que
            ha recibido el profesor deben ser $0, 1, 2, 3, 4, 5, 6, 7, 8.$
        \end{solucion}
        
        
    \end{frame}
    
    \begin{frame}
        \begin{figure}[h]
            \begin{center}
                \begin{tikzpicture}[scale=2.5]
                    \draw[-,line width=1pt] (0.81,0.59) -- (0.7*0.81,0.7*0.59);
                    \draw[-,line width=1pt] (0.31,0.95) -- (0.04, 0.81) -- (0.31,0.95) -- (0.45, 0.68) -- (0.31,0.95);
                    \draw[-,line width=1pt] (-0.31,0.95) -- (-0.45, 0.68) -- (-0.31,0.95) -- (-0.22, 0.66) -- (-0.31,0.95) -- (-0.04, 0.81) -- (-0.31,0.95);
                    \draw[-,line width=1pt] (-0.81,0.59) -- (-0.76, 0.29) -- (-0.81,0.59) -- (-0.62, 0.36) -- (-0.81,0.59) -- (-0.53, 0.48) -- (-0.81,0.59) -- (-0.51, 0.64) -- (-0.81,0.59);
                    \draw[-,line width=1pt] (-1.0,-0.0) -- (-0.79, -0.21) -- (-1.0,-0.0) -- (-0.72, -0.12) -- (-1.0,-0.0) -- (-0.7, -0.0) -- (-1.0,-0.0) -- (-0.72, 0.11) -- (-1.0,-0.0) -- (-0.79, 0.21) -- (-1.0,-0.0);
                    \draw[-,line width=1pt] (-0.81,-0.59) -- (-0.51, -0.64) -- (-0.81,-0.59) -- (-0.51, -0.54) -- (-0.81,-0.59) -- (-0.54, -0.45) -- (-0.81,-0.59) -- (-0.6, -0.38) -- (-0.81,-0.59) -- (-0.67, -0.32) -- (-0.81,-0.59) -- (-0.76, -0.29) -- (-0.81,-0.59);
                    \draw[-,line width=1pt] (-0.31,-0.95) -- (-0.04, -0.81) -- (-0.31,-0.95) -- (-0.09, -0.75) -- (-0.31,-0.95) -- (-0.15, -0.7) -- (-0.31,-0.95) -- (-0.22, -0.66) -- (-0.31,-0.95) -- (-0.29, -0.65) -- (-0.31,-0.95) -- (-0.37, -0.66) -- (-0.31,-0.95) -- (-0.45, -0.68) -- (-0.31,-0.95);
                    \draw[-,line width=1pt] (0.31,-0.95) -- (0.45, -0.68) -- (0.31,-0.95) -- (0.38, -0.66) -- (0.31,-0.95) -- (0.32, -0.65) -- (0.31,-0.95) -- (0.25, -0.66) -- (0.31,-0.95) -- (0.18, -0.68) -- (0.31,-0.95) -- (0.13, -0.71) -- (0.31,-0.95) -- (0.08, -0.76) -- (0.31,-0.95) -- (0.04, -0.81) -- (0.31,-0.95);
                    
                    \GraphInit[vstyle=Welsh]
                    \Vertices[]{circle}{0,1,2,3,4,5,6,7,8,$M$}
                    \draw (0.6,-0.45) node {?};
                \end{tikzpicture} 
            \end{center}
            \caption{La fiesta de Abril}\label{f5.2}
        \end{figure}
    \end{frame}
    
    \begin{frame}
        
        
        Ahora, el vértice $8$ alcanza a todos los otros vértices excepto
        uno, el cual debe por lo tanto representar a la esposa de $8$. Este
        vértice debe ser el $0$ el cual por cierto que no está unido al $8$
        (ni ob\-via\-men\-te a ningún otro). 
        \vskip.2cm\pause
        Luego $8$ y $0$ son una pareja de
        casados y $8$ está unido a $1, 2, 3, 4, 5, 6, 7$ y $M$. 
        \pause
        En particular el $1$ está unido al $8$ y ésta es la única arista que parte del $1$.
        \vskip.2cm
        \pause
        Por consiguiente $7$ no esta unido al $0$ y al $1$ y sí está unido a $2,3,4$, $5,6,8$ y $M$. La
        esposa de $7$ debe ser $1$, puesto que $0$ esta casado con $8$.
        \vskip.2cm
        \pause
        Continuando con este razonamiento vemos que $6$ y $2$, y $5$ y $3$ son
        parejas de casados. 
        \vskip.2cm\pause
        Se sigue entonces que $M$ y $4$ están casados,
        luego el vértice $4$ representa a Abril, quien estrechó la mano de
        cuatro personas. \qed
    \end{frame}
    
    \begin{frame}
        \begin{ejemplo}
            Los senderos de un jardín han sido diseñados dándoles forma de {\em grafo rueda}
            $W_n$, cuyos vértices son $V=\{0,1,2,\ldots,n\}$ y sus aristas son
            $$
            \begin{aligned}
                &\{0,1\},\{0,2\},\ldots,\{0,n\}, \\
                &\{1,2\}, \{2,3\},\ldots,\{n-1,n\},\{n,1\}.
            \end{aligned}
            $$
            \pause
            Describir una ruta por los senderos de tal forma que empiece y
            termine en le vértice $0$ y que pase por cada vértice una sola vez.
        \end{ejemplo}
    \end{frame}
    
    \begin{frame}
        
        \begin{solucion}
            Primero  dibujemos el grafo para darnos cuenta de por que se llama ``rueda''. Dibujemos $W_6$:\pause
            \vskip .2cm
            \begin{figure}
                \begin{tikzpicture}[scale=0.65]
                    %\SetVertexSimple[Shape=circle,FillColor=white,MinSize=8 pt]
                    %
                    \Vertex[x=0.00, y=0.00]{0}            
                    \Vertex[x=3.00, y=0.00]{1}
                    \Vertex[x=1.50, y=2.60]{2}
                    \Vertex[x=-1.50, y=2.60]{3}
                    \Vertex[x=-3.00, y=0.00]{4}
                    \Vertex[x=-1.50, y=-2.60]{5}
                    \Vertex[x=1.50, y=-2.60]{6}
                    \Edges(1,2,3,4,5,6,1)
                    \Edges(1,0,4) \Edges(2,0,5) \Edges(3,0,6) 
                    
                \end{tikzpicture}
            \end{figure}\pause
            El dibujo nos orienta de como puede ser una ruta: $0,1,2,3,4,5,6,0$. 
            \vskip .2cm\pause
            En  general una respuesta es: $0,1,2,3,\cdots,n-1,n,0$.    \qed\end{solucion} 
        
    \end{frame}
    
    
\end{document}

