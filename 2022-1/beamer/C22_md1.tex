%\documentclass{beamer} 
\documentclass[handout]{beamer} % sin pausas
\usetheme{CambridgeUS}

\usepackage{etex}
\usepackage{t1enc}
\usepackage[spanish,es-nodecimaldot]{babel}
\usepackage{latexsym}
\usepackage[utf8]{inputenc}
\usepackage{verbatim}
\usepackage{multicol}
\usepackage{amsgen,amsmath,amstext,amsbsy,amsopn,amsfonts,amssymb}
\usepackage{amsthm}
\usepackage{calc}         % From LaTeX distribution
\usepackage{graphicx}     % From LaTeX distribution
\usepackage{ifthen}
%\usepackage{makeidx}
\input{random.tex}        % From CTAN/macros/generic
\usepackage{subfigure} 
\usepackage{tikz}
\usepackage[customcolors]{hf-tikz}
\usetikzlibrary{arrows}
\usetikzlibrary{matrix}
\tikzset{
    every picture/.append style={
        execute at begin picture={\deactivatequoting},
        execute at end picture={\activatequoting}
    }
}
\usetikzlibrary{decorations.pathreplacing,angles,quotes}
\usetikzlibrary{shapes.geometric}
\usepackage{mathtools}
\usepackage{stackrel}
%\usepackage{enumerate}
\usepackage{enumitem}
\usepackage{tkz-graph}
\usepackage{polynom}
\polyset{%
    style=B,
    delims={(}{)},
    div=:
}
\renewcommand\labelitemi{$\circ$}
\setlist[enumerate]{label={(\arabic*)}}
%\setbeamertemplate{background}[grid][step=8 ] % cuadriculado
\setbeamertemplate{itemize item}{$\circ$}
\setbeamertemplate{enumerate items}[default]
\definecolor{links}{HTML}{2A1B81}
\hypersetup{colorlinks,linkcolor=,urlcolor=links}


\newcommand{\Id}{\operatorname{Id}}
\newcommand{\img}{\operatorname{Im}}
\newcommand{\nuc}{\operatorname{Nu}}
\newcommand{\im}{\operatorname{Im}}
\renewcommand\nu{\operatorname{Nu}}
\newcommand{\la}{\langle}
\newcommand{\ra}{\rangle}
\renewcommand{\t}{{\operatorname{t}}}
\renewcommand{\sin}{{\,\operatorname{sen}}}
\newcommand{\Q}{\mathbb Q}
\newcommand{\R}{\mathbb R}
\newcommand{\C}{\mathbb C}
\newcommand{\K}{\mathbb K}
\newcommand{\F}{\mathbb F}
\newcommand{\Z}{\mathbb Z}
\newcommand{\N}{\mathbb N}
\newcommand\sgn{\operatorname{sgn}}
\renewcommand{\t}{{\operatorname{t}}}
\renewcommand{\figurename }{Figura}

%
% Ver http://joshua.smcvt.edu/latex2e/_005cnewenvironment-_0026-_005crenewenvironment.html
%

\renewenvironment{block}[1]% environment name
{% begin code
	\par\vskip .2cm%
	{\color{blue}#1}%
	\vskip .2cm
}%
{%
	\vskip .2cm}% end code


\renewenvironment{alertblock}[1]% environment name
{% begin code
	\par\vskip .2cm%
	{\color{red!80!black}#1}%
	\vskip .2cm
}%
{%
	\vskip .2cm}% end code


\renewenvironment{exampleblock}[1]% environment name
{% begin code
	\par\vskip .2cm%
	{\color{blue}#1}%
	\vskip .2cm
}%
{%
	\vskip .2cm}% end code




\newenvironment{exercise}[1]% environment name
{% begin code
	\par\vspace{\baselineskip}\noindent
	\textbf{Ejercicio (#1)}\begin{itshape}%
		\par\vspace{\baselineskip}\noindent\ignorespaces
	}%
	{% end code
	\end{itshape}\ignorespacesafterend
}


\newenvironment{definicion}[1][]% environment name
{% begin code
	\par\vskip .2cm%
	{\color{blue}Definición #1}%
	\vskip .2cm
}%
{%
	\vskip .2cm}% end code

    \newenvironment{notacion}[1][]% environment name
    {% begin code
        \par\vskip .2cm%
        {\color{blue}Notación #1}%
        \vskip .2cm
    }%
    {%
        \vskip .2cm}% end code

\newenvironment{observacion}[1][]% environment name
{% begin code
	\par\vskip .2cm%
	{\color{blue}Observación #1}%
	\vskip .2cm
}%
{%
	\vskip .2cm}% end code

\newenvironment{ejemplo}[1][]% environment name
{% begin code
	\par\vskip .2cm%
	{\color{blue}Ejemplo #1}%
	\vskip .2cm
}%
{%
	\vskip .2cm}% end code


\newenvironment{preguntas}[1][]% environment name
{% begin code
    \par\vskip .2cm%
    {\color{blue}Preguntas #1}%
    \vskip .2cm
}%
{%
    \vskip .2cm}% end code

\newenvironment{ejercicio}[1][]% environment name
{% begin code
	\par\vskip .2cm%
	{\color{blue}Ejercicio #1}%
	\vskip .2cm
}%
{%
	\vskip .2cm}% end code


\renewenvironment{proof}% environment name
{% begin code
	\par\vskip .2cm%
	{\color{blue}Demostración}%
	\vskip .2cm
}%
{%
	\vskip .2cm}% end code



\newenvironment{demostracion}% environment name
{% begin code
	\par\vskip .2cm%
	{\color{blue}Demostración}%
	\vskip .2cm
}%
{%
	\vskip .2cm}% end code

\newenvironment{idea}% environment name
{% begin code
	\par\vskip .2cm%
	{\color{blue}Idea de la demostración}%
	\vskip .2cm
}%
{%
	\vskip .2cm}% end code

\newenvironment{solucion}% environment name
{% begin code
	\par\vskip .2cm%
	{\color{blue}Solución}%
	\vskip .2cm
}%
{%
	\vskip .2cm}% end code



\newenvironment{lema}[1][]% environment name
{% begin code
	\par\vskip .2cm%
	{\color{blue}Lema #1}\begin{itshape}%
		\par\vskip .2cm
	}%
	{% end code
	\end{itshape}\vskip .2cm\ignorespacesafterend
}

\newenvironment{proposicion}[1][]% environment name
{% begin code
	\par\vskip .2cm%
	{\color{blue}Proposición #1}\begin{itshape}%
		\par\vskip .2cm
	}%
	{% end code
	\end{itshape}\vskip .2cm\ignorespacesafterend
}

\newenvironment{teorema}[1][]% environment name
{% begin code
	\par\vskip .2cm%
	{\color{blue}Teorema #1}\begin{itshape}%
		\par\vskip .2cm
	}%
	{% end code
	\end{itshape}\vskip .2cm\ignorespacesafterend
}


\newenvironment{corolario}[1][]% environment name
{% begin code
	\par\vskip .2cm%
	{\color{blue}Corolario #1}\begin{itshape}%
		\par\vskip .2cm
	}%
	{% end code
	\end{itshape}\vskip .2cm\ignorespacesafterend
}

\newenvironment{propiedad}% environment name
{% begin code
	\par\vskip .2cm%
	{\color{blue}Propiedad}\begin{itshape}%
		\par\vskip .2cm
	}%
	{% end code
	\end{itshape}\vskip .2cm\ignorespacesafterend
}

\newenvironment{conclusion}% environment name
{% begin code
	\par\vskip .2cm%
	{\color{blue}Conclusión}\begin{itshape}%
		\par\vskip .2cm
	}%
	{% end code
	\end{itshape}\vskip .2cm\ignorespacesafterend
}


\newenvironment{definicion*}% environment name
{% begin code
	\par\vskip .2cm%
	{\color{blue}Definición}%
	\vskip .2cm
}%
{%
	\vskip .2cm}% end code

\newenvironment{observacion*}% environment name
{% begin code
	\par\vskip .2cm%
	{\color{blue}Observación}%
	\vskip .2cm
}%
{%
	\vskip .2cm}% end code


\newenvironment{obs*}% environment name
	{% begin code
		\par\vskip .2cm%
		{\color{blue}Observación}%
		\vskip .2cm
	}%
	{%
		\vskip .2cm}% end code

\newenvironment{ejemplo*}% environment name
{% begin code
	\par\vskip .2cm%
	{\color{blue}Ejemplo}%
	\vskip .2cm
}%
{%
	\vskip .2cm}% end code

\newenvironment{ejercicio*}% environment name
{% begin code
	\par\vskip .2cm%
	{\color{blue}Ejercicio}%
	\vskip .2cm
}%
{%
	\vskip .2cm}% end code

\newenvironment{propiedad*}% environment name
{% begin code
	\par\vskip .2cm%
	{\color{blue}Propiedad}\begin{itshape}%
		\par\vskip .2cm
	}%
	{% end code
	\end{itshape}\vskip .2cm\ignorespacesafterend
}

\newenvironment{conclusion*}% environment name
{% begin code
	\par\vskip .2cm%
	{\color{blue}Conclusión}\begin{itshape}%
		\par\vskip .2cm
	}%
	{% end code
	\end{itshape}\vskip .2cm\ignorespacesafterend
}






\newcommand{\nc}{\newcommand}

%%%%%%%%%%%%%%%%%%%%%%%%%LETRAS

\nc{\FF}{{\mathbb F}} \nc{\NN}{{\mathbb N}} \nc{\QQ}{{\mathbb Q}}
\nc{\PP}{{\mathbb P}} \nc{\DD}{{\mathbb D}} \nc{\Sn}{{\mathbb S}}
\nc{\uno}{\mathbb{1}} \nc{\BB}{{\mathbb B}} \nc{\An}{{\mathbb A}}

\nc{\ba}{\mathbf{a}} \nc{\bb}{\mathbf{b}} \nc{\bt}{\mathbf{t}}
\nc{\bB}{\mathbf{B}}

\nc{\cP}{\mathcal{P}} \nc{\cU}{\mathcal{U}} \nc{\cX}{\mathcal{X}}
\nc{\cE}{\mathcal{E}} \nc{\cS}{\mathcal{S}} \nc{\cA}{\mathcal{A}}
\nc{\cC}{\mathcal{C}} \nc{\cO}{\mathcal{O}} \nc{\cQ}{\mathcal{Q}}
\nc{\cB}{\mathcal{B}} \nc{\cJ}{\mathcal{J}} \nc{\cI}{\mathcal{I}}
\nc{\cM}{\mathcal{M}} \nc{\cK}{\mathcal{K}}

\nc{\fD}{\mathfrak{D}} \nc{\fI}{\mathfrak{I}} \nc{\fJ}{\mathfrak{J}}
\nc{\fS}{\mathfrak{S}} \nc{\gA}{\mathfrak{A}}
%%%%%%%%%%%%%%%%%%%%%%%%%LETRAS


\title[Clase 22 - Árboles / Coloreo de vértices]{Matemática Discreta I \\ Clase 22 - Árboles / Coloreo de vértices}
%\author[C. Olmos / A. Tiraboschi]{Carlos Olmos / Alejandro Tiraboschi}
\institute[]{\normalsize FAMAF / UNC
    \\[\baselineskip] ${}^{}$
    \\[\baselineskip]
}
\date[08/06/2021]{8 de junio de 2021}




\begin{document}
    
    \frame{\titlepage} 
    
    
\begin{frame}
    \frametitle{Árboles}
    \begin{definicion}
    Diremos que un grafo $T$ es un \textit{árbol} si cumple que es conexo y no hay ciclos en $T$.
    \end{definicion}

    \vskip .8cm
    
    \begin{figure}[ht]
        \begin{center}
        \begin{tabular}{llllllll}
            &
            \begin{tikzpicture}[scale=0.8]
            \SetVertexSimple[Shape=circle,FillColor=white,MinSize=8 pt]
            \Vertex[x=0.00, y=0]{a}
            \Vertex[x=0, y=-1]{b}
            \Vertex[x=0., y=-2]{c}
            \Vertex[x=0, y=-3]{d}
            \Vertex[x=0., y=-4]{e}
            \Edges(a,b,c,d,e)
            \end{tikzpicture}
            &
            \qquad
            & 
            \begin{tikzpicture}[scale=0.8]
            \SetVertexSimple[Shape=circle,FillColor=white,MinSize=8 pt]
            %                
            \Vertex[x=0.00, y=0]{a}
            \Vertex[x=-1.5, y=-0.5]{b}
            \Vertex[x=1.5, y=-0.5]{c}
            \Vertex[x=-1.5, y=-1.5]{d}
            \Vertex[x=1.5, y=-1.5]{e}
            \Vertex[x=0, y=-1.5]{f}
            \Vertex[x=-0.7, y=-1]{g}
            \Vertex[x=0.7, y=-1]{h}
            \Vertex[x=0, y=-4]{i}
            \Edges(d,b,a,c,e)
            \Edges(g,f,h)
            \Edges(a,f,i)
            \end{tikzpicture}
            &
            \qquad
            & 
            \begin{tikzpicture}[scale=0.8]
            \SetVertexSimple[Shape=circle,FillColor=white,MinSize=8 pt]
            %                
            \Vertex[x=0.00, y=0]{a}
            \Vertex[x=0, y=-1.0]{b}
            \Vertex[x=0, y=-2.5]{c}
            \Vertex[x=1.2, y=-2]{e}
            \Vertex[x=-1.2, y=-2]{f}
            \Vertex[x=-1.2, y=-3.5]{g}
            \Vertex[x=1.2, y=-3.5]{h}
            \Edges(a,b,c)
            \Edges(f,b,e)
            \Edges(g,c,h)
            \end{tikzpicture}
            &
            \qquad
            & 
            \begin{tikzpicture}[scale=0.50]
            \SetVertexSimple[Shape=circle,FillColor=white,MinSize=8 pt]
            %
            \Vertex[x=0.00, y=0.00]{0}
            \Vertex[x=3.00, y=0.00]{1}
            \Vertex[x=2.12, y=2.12]{2}
            \Vertex[x=0.00, y=3.00]{3}
            \Vertex[x=-2.12, y=2.12]{4}
            \Vertex[x=-3.00, y=0.00]{5}
            \Vertex[x=-2.12, y=-2.12]{6}
            \Vertex[x=0.00, y=-3.00]{7}
            \Vertex[x=2.12, y=-2.12]{8}
            \Edges(1,0,5) \Edges(3,0,7) \Edges(2,0,6)\Edges(4,0,8)
            \end{tikzpicture}
        \end{tabular}
    \end{center}
        \caption{Algunos árboles} \label{f5.8}
    \end{figure}
    

    

\end{frame}


\begin{frame}
    \frametitle{}

    A causa de su particular estructura y propiedades, los árboles aparecen en diversas aplicaciones de la matemática, especialmente en investigación operativa y ciencias de la computación. 
\vskip .9cm
    El siguiente lema nos resultará útil para probar una parte del teorema fundamental de esta sección.
\vskip .4cm

\begin{lema} Sea $G=(V,E)$ un grafo conexo, entonces $|E| \ge |V| -1$.  
\end{lema}


\end{frame}


\begin{frame}
    \frametitle{}

    \begin{proof} Como $G$ es conexo existe una caminata que recorre todos los vértices de $G$:
        $$
        v_1,v_2,\ldots,v_r.
        $$
        Renombremos los vértices de $G$ con números naturales de tal forma que el primer vértice de la caminata sea $1$, el segundo $2$ y cada vez que aparece un vértice que no ha sido renombrado se le asigna el número siguiente.
        
        \vskip .4cm

        Luego la caminata comienza en $1$ y termina en $n$, donde $n = |V|$.  
        
        \vskip .4cm

        Observar: si $i$ tal que $1 < i \le n$ tenemos que la caminata tiene la forma
        $$
        1,\ldots,j_i,i,\ldots,j_n,n
        $$ 
        donde $j_i < i$, luego es claro  que 
        $$
        \{j_{2},2\}, \{j_{3},3\}, \ldots, \{j_{n},n\}
        $$
        forman un conjunto de $n-1$ aristas distintas en $G$. \qed
        \end{proof}    

\end{frame}


\begin{frame}
    \frametitle{}

    El siguiente teorema nos da 3 nociones equivalente a la definición de árbol. 

    \vskip .6cm

\begin{teorema}\label{t5.5} Si $T=(V,E)$ es un grafo conexo con al menos dos vértices, entonces son equivalentes las siguientes propiedades
    \begin{enumerate}
    \item[\textbf{T1)}] T es un árbol.
    \item[\textbf{T2)}] \label{T2} Para cada par $x$, $y$ de vértices existe un único camino en $T$ de $x$ a
    $y$.
    \item[\textbf{T3)}] \label{T4} $|E|=|V|-1$.
    \end{enumerate}
    \end{teorema}
    

\end{frame}


\begin{frame}
    \frametitle{}

    Este teorema se demuestra haciendo las pruebas:

    $$
    \textbf{T1} \quad \Rightarrow  \quad \textbf{T2} \quad \Rightarrow  \quad \textbf{T3} \quad \Rightarrow  \quad \textbf{T1}.
    $$


    Luego, toda equivalencia se deduce de estas implicaciones. 
    \vskip .4cm
    Por ejemplo,

    $$
    \textbf{T1} \quad \Leftrightarrow \quad \textbf{T3}\qquad\text{pues} \qquad
    \left\{ 
    \begin{matrix*}[l]
        \textbf{T1} \quad \Rightarrow  \quad \textbf{T2} \quad \Rightarrow  \quad  \textbf{T3} \\
        \textbf{T3}  \quad \Rightarrow  \quad  \textbf{T1}. 
    \end{matrix*}    
    \right. 
    $$

\end{frame}


\begin{frame}
    \frametitle{}

    (T1 $\Rightarrow$ T2) Si hubiera dos caminos podríamos formar un ciclo. 
        \vskip .4cm
    (T2 $\Rightarrow$ T3) Se hará por inducción en $|V|$. 
    
    Sea $uv$  arista de $T$ y sea  $F = T -uv$.  Como hay un único camino de $u$  a $v$,  $F$ tiene dos componentes conexas: $T_1$  la componente conexa de $u$ y $T_2$ la componente conexa de $v$.
        \vskip .2cm
    En cada componente conexa $T_i = (V_i, E_i)$ hay un único camino de un vértice a otro, pues sino esa propiedad no se cumpliría en $T$. 
    \vskip .2cm
    Por HI, $|E_1|=|V_1|-1$ y $|E_2|=|V_2|-1$. Luego 
    \vskip -.6cm
    \begin{align*}
        |E| &= |E_1 \cup E_2| + 1  = |E_1|+|E_2| + 1 \stackrel{\text{(HI)}}{=} (|V_1|-1)+(|V_2|-1) +1 \\
            &= |V_1 \cup V_2|-1 = |V| -1. 
    \end{align*}
    \vskip .0cm
    (T3 $\Rightarrow$ T1)  $|E|=|V|-1$ y supongamos que $T$ no es árbol $\Rightarrow$ hay un ciclo $\Rightarrow$ podemos sacar una arista $uv$ y sigue siendo conexo   $\Rightarrow$ $|E -uv|=|V|-2$ y  conexo. Absurdo por el lema. \qed
        

\end{frame}

\begin{frame}
    \frametitle{}

    \begin{corolario}
        Sea $T$ árbol con al menos dós vértices,  entonces:
        \vskip .4cm
        El grafo obtenido de $T$ removiendo alguna arista tiene dos
        componentes, cada una de las cuales es un árbol.
    \end{corolario}

    \begin{proof}
        Es parte de la demostración de  T2 $\Rightarrow$ T3: el grafo que se obtiene de quitar una arista,  es un grafo con dos componentes conexas cada una de ellas sin ciclos (pues sino los habría en $T$), luego cada una de ellas es árbol.

        \qed
    \end{proof}

     \vskip 2cm

\end{frame}

\begin{frame}
    \frametitle{Coloreo de los vértices de un grafo}

    \begin{block}[Problema]
        ¿Cómo hacer un horario de actividades sin interferencias?.
    \end{block}




    \vskip .8cm

    \begin{ejemplo}\label{ejemplo-coloracion} Supongamos que deseamos hacer un horario con seis cursos de una hora, $v_1,v_2,v_3,v_4,v_5,v_6$. Entre la audiencia potencial hay gente que desea asistir simultáneamente  a
        $$ 
        \{v_1, v_2 \}, \quad\{v_1, v_4 \}, \quad\{v_3, v_5 \}, \quad\{v_2, v_6 \}, \quad\{v_4, v_5 \}, \quad \{v_5, v_6 \}, \quad\{v_1, v_6 \}.
        $$
        
    ¿Cuántas horas son necesarias para poder confeccionar un horario en el cual no haya interferencias?

    \end{ejemplo}    

\end{frame}


\begin{frame}
    \frametitle{}

    \begin{solucion}
        Podemos representar la situación con el grafo:
        \vskip .4cm
            \begin{center}
            \begin{tikzpicture}[scale=0.55]
                %\SetVertexSimple[Shape=circle,FillColor=white]
                %                
                \Vertex[x=3.00, y=0.00]{$v_3$}
                \Vertex[x=1.50, y=2.60]{$v_2$}
                \Vertex[x=-1.50, y=2.60]{$v_1$}
                \Vertex[x=-3.00, y=0.00]{$v_6$}
                \Vertex[x=-1.50, y=-2.60]{$v_5$}
                \Vertex[x=1.50, y=-2.60]{$v_4$}
                \Edges($v_2$,$v_1$,$v_6$, $v_5$,$v_4$,$v_1$)
                \Edges($v_2$,$v_6$)
                \Edges($v_3$,$v_5$)
            \end{tikzpicture}
            \end{center}
            \vskip .4cm
            Los vértices corresponden a las seis clases, y las aristas indican las interferencias potenciales.

    
        \end{solucion}
    


\end{frame}


\begin{frame}
    \frametitle{}

Un horario el cual cumple con la condición de evitar interferencias es el siguiente:
$$
\begin{matrix}
\text{Hora 1} & \text{Hora 2} &\text{ Hora 3}& \text{Hora 4} \\
v_1 \text{ y } v_3 & v_2 \text{ y } v_4 & v_5 & v_6
\end{matrix}
$$
Es una partición del conjuntos de vértices en cuatro partes, con la propiedad que ninguna parte contiene un par de vértices adyacentes del grafo. 


Claramente,  le corresponde una función:
$$
c: \{ v_1,v_2,v_3,v_4,v_5,v_6\} \to  \{1,2,3,4\},
$$
donde
\begin{align*}
    c(v_1) &= c(v_3) = 1 \\
    c(v_2) &= c(v_4) = 2 \\
    c(v_5) &= 3 \\
    c(v_6) &= 4. 
\end{align*}



\end{frame}


\begin{frame}
    \frametitle{}

    También podemos representar esta función como un \textit{coloreo de vértices} donde dos vértices adyacentes tienen distintos colores:
    \vskip .4cm
\begin{center}
    \begin{tikzpicture}[scale=0.55]
        %\SetVertexSimple[Shape=circle,FillColor=white]
        %
        \tikzset{VertexStyle/.append style={fill= red!50}}
        \Vertex[x=-1.50, y=2.60]{$v_1$}              
        \Vertex[x=3.00, y=0.00]{$v_3$}
        \tikzset{VertexStyle/.append style={fill= blue!50}}
        \Vertex[x=1.50, y=2.60]{$v_2$}
        \Vertex[x=1.50, y=-2.60]{$v_4$}
        \tikzset{VertexStyle/.append style={fill= green!50}}
        \Vertex[x=-3.00, y=0.00]{$v_6$}
        \tikzset{VertexStyle/.append style={fill= yellow!50}}
        \Vertex[x=-1.50, y=-2.60]{$v_5$}
        \Edges($v_2$,$v_1$,$v_6$, $v_5$,$v_4$,$v_1$)
        \Edges($v_2$,$v_6$)
        \Edges($v_3$,$v_5$)
    \end{tikzpicture}
    \end{center}
    \vskip .4cm
    Cualquiera de las tres formas de presentar el resultado nos daría una solución (quizás no la mejor). 

    \qed

\end{frame}


\begin{frame}
    \frametitle{}

    \begin{definicion} Una \textit{coloración de vértices} de un grafo $G=(V,E)$ es una función $c:V \to  \mathbb N$ con la siguiente propiedad:
        $$
        c(x)\not= c(y) \quad \text{ si } \quad \{x,y\} \in E.
        $$
        El \textit{número cromático} de $G$, denotado $\chi(G)$, se define \index{número cromático} como el mínimo entero $k$ para el cual existe una coloración de vértices de $G$ usando $k$-colores. 
        
        \vskip .4cm
        
        En otra palabras, $\chi(G)=k$ si  y sólo si existe una coloración de vértices $c$ la cual es una función de $V$ a $\mathbb N_k$, y $k$ es el mínimo entero con esta propiedad. 
        \end{definicion}

\end{frame}


\begin{frame}
    \frametitle{}

    Volviendo al ejemplo de los horarios,  nuestro primer intento fue de 4 colores.
    \vskip .4cm
    Un rápido intento con tres  colores nos da la solución de este problema: 

    $$
    \begin{matrix}
    \text{Color 1}\quad &\text{Color 2}\quad&\text{Color 3} \\
    v_1 &v_2 \text{ y } v_5 \quad & v_3,v_4 \text{ y } v_6 .
    \end{matrix}
    $$

    Más aún, hacen falta por lo menos tres colores, puesto que $v_1$, $v_2$, y $v_6$ son mutuamente adyacentes y por lo tanto deben tener diferentes colores. 
    
    \vskip .4cm

    Luego concluimos que el número cromático del grafo es $3$.

\end{frame}


\begin{frame}
    \frametitle{}

    Podemos representar en el  grafo la coloración: 

    \vskip .8 cM

    \begin{center}
        \begin{tikzpicture}[scale=0.55]
            %\SetVertexSimple[Shape=circle,FillColor=white]
            %
            \tikzset{VertexStyle/.append style={fill= red!50}}
            \Vertex[x=-1.50, y=2.60]{$v_1$}              
            \tikzset{VertexStyle/.append style={fill= blue!50}}
            \Vertex[x=1.50, y=2.60]{$v_2$}
            \Vertex[x=-1.50, y=-2.60]{$v_5$}
            \tikzset{VertexStyle/.append style={fill= green!50}}
            \Vertex[x=3.00, y=0.00]{$v_3$}
            \Vertex[x=-3.00, y=0.00]{$v_6$}
            \Vertex[x=1.50, y=-2.60]{$v_4$}
            \Edges($v_2$,$v_1$,$v_6$, $v_5$,$v_4$,$v_1$)
            \Edges($v_2$,$v_6$)
            \Edges($v_3$,$v_5$)
        \end{tikzpicture}
        \end{center}

\end{frame}


\begin{frame}
    \frametitle{}

    En general, para probar que el número cromático de un grafo dado es $k$, debemos hacer dos cosas:

    \vskip .4cm

\begin{enumerate}[label=\textit{\alph*)}] 
    \item  encontrar una coloración de vértices usando $k$ colores;
    \vskip .4cm
    \item  probar que ninguna coloración de vértices usa menos de $k$ colores.
\end{enumerate}

\vskip .9cm
¿Existe algún algoritmo general eficiente para encontrar el número cromático?
\vskip .2cm
\textbf{Respuesta:} No.
\vskip .4cm

Veremos la clase que viene un algoritmo  para encontrar una coloración de vértices que aunque no es óptima nos da un resultado satisfactorio. 


\end{frame}


\end{document}

