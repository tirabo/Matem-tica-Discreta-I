%\documentclass[handout]{beamer} % sin pausas
\documentclass{beamer} 
\setbeamertemplate{background}[grid][step=8 ] % cuadriculado

\usetheme{CambridgeUS}


\usepackage{etex}
\usepackage{t1enc}
\usepackage[spanish,es-nodecimaldot]{babel}
\usepackage{latexsym}
\usepackage[utf8]{inputenc}
\usepackage{verbatim}
\usepackage{multicol}
\usepackage{amsgen,amsmath,amstext,amsbsy,amsopn,amsfonts,amssymb}
\usepackage{amsthm}
\usepackage{calc}         % From LaTeX distribution
\usepackage{graphicx}     % From LaTeX distribution
\usepackage{ifthen}
%\usepackage{makeidx}
\input{random.tex}        % From CTAN/macros/generic
\usepackage{subfigure} 
\usepackage{tikz}
\usepackage[customcolors]{hf-tikz}
\usetikzlibrary{arrows}
\usetikzlibrary{matrix}
\tikzset{
    every picture/.append style={
        execute at begin picture={\deactivatequoting},
        execute at end picture={\activatequoting}
    }
}
\usetikzlibrary{decorations.pathreplacing,angles,quotes}
\usetikzlibrary{shapes.geometric}
\usepackage{mathtools}
\usepackage{stackrel}
%\usepackage{enumerate}
\usepackage{enumitem}
\usepackage{tkz-graph}
\usepackage{polynom}
\polyset{%
    style=B,
    delims={(}{)},
    div=:
}
\renewcommand\labelitemi{$\circ$}
\setlist[enumerate]{label={(\arabic*)}}
\setbeamertemplate{itemize item}{$\circ$}
\setbeamertemplate{enumerate items}[default]
\definecolor{links}{HTML}{2A1B81}
\hypersetup{colorlinks,linkcolor=,urlcolor=links}


\newcommand{\Id}{\operatorname{Id}}
\newcommand{\img}{\operatorname{Im}}
\newcommand{\nuc}{\operatorname{Nu}}
\newcommand{\im}{\operatorname{Im}}
\renewcommand\nu{\operatorname{Nu}}
\newcommand{\la}{\langle}
\newcommand{\ra}{\rangle}
\renewcommand{\t}{{\operatorname{t}}}
\renewcommand{\sin}{{\,\operatorname{sen}}}
\newcommand{\Q}{\mathbb Q}
\newcommand{\R}{\mathbb R}
\newcommand{\C}{\mathbb C}
\newcommand{\K}{\mathbb K}
\newcommand{\F}{\mathbb F}
\newcommand{\Z}{\mathbb Z}
\newcommand{\N}{\mathbb N}
\newcommand\sgn{\operatorname{sgn}}
\renewcommand{\t}{{\operatorname{t}}}
\renewcommand{\figurename }{Figura}

%
% Ver http://joshua.smcvt.edu/latex2e/_005cnewenvironment-_0026-_005crenewenvironment.html
%

\renewenvironment{block}[1]% environment name
{% begin code
	\par\vskip .2cm%
	{\color{blue}#1}%
	\vskip .2cm
}%
{%
	\vskip .2cm}% end code


\renewenvironment{alertblock}[1]% environment name
{% begin code
	\par\vskip .2cm%
	{\color{red!80!black}#1}%
	\vskip .2cm
}%
{%
	\vskip .2cm}% end code


\renewenvironment{exampleblock}[1]% environment name
{% begin code
	\par\vskip .2cm%
	{\color{blue}#1}%
	\vskip .2cm
}%
{%
	\vskip .2cm}% end code




\newenvironment{exercise}[1]% environment name
{% begin code
	\par\vspace{\baselineskip}\noindent
	\textbf{Ejercicio (#1)}\begin{itshape}%
		\par\vspace{\baselineskip}\noindent\ignorespaces
	}%
	{% end code
	\end{itshape}\ignorespacesafterend
}


\newenvironment{definicion}[1][]% environment name
{% begin code
	\par\vskip .2cm%
	{\color{blue}Definición #1}%
	\vskip .2cm
}%
{%
	\vskip .2cm}% end code

    \newenvironment{notacion}[1][]% environment name
    {% begin code
        \par\vskip .2cm%
        {\color{blue}Notación #1}%
        \vskip .2cm
    }%
    {%
        \vskip .2cm}% end code

\newenvironment{observacion}[1][]% environment name
{% begin code
	\par\vskip .2cm%
	{\color{blue}Observación #1}%
	\vskip .2cm
}%
{%
	\vskip .2cm}% end code

\newenvironment{ejemplo}[1][]% environment name
{% begin code
	\par\vskip .2cm%
	{\color{blue}Ejemplo #1}%
	\vskip .2cm
}%
{%
	\vskip .2cm}% end code


\newenvironment{preguntas}[1][]% environment name
{% begin code
    \par\vskip .2cm%
    {\color{blue}Preguntas #1}%
    \vskip .2cm
}%
{%
    \vskip .2cm}% end code

\newenvironment{ejercicio}[1][]% environment name
{% begin code
	\par\vskip .2cm%
	{\color{blue}Ejercicio #1}%
	\vskip .2cm
}%
{%
	\vskip .2cm}% end code


\renewenvironment{proof}% environment name
{% begin code
	\par\vskip .2cm%
	{\color{blue}Demostración}%
	\vskip .2cm
}%
{%
	\vskip .2cm}% end code



\newenvironment{demostracion}% environment name
{% begin code
	\par\vskip .2cm%
	{\color{blue}Demostración}%
	\vskip .2cm
}%
{%
	\vskip .2cm}% end code

\newenvironment{idea}% environment name
{% begin code
	\par\vskip .2cm%
	{\color{blue}Idea de la demostración}%
	\vskip .2cm
}%
{%
	\vskip .2cm}% end code

\newenvironment{solucion}% environment name
{% begin code
	\par\vskip .2cm%
	{\color{blue}Solución}%
	\vskip .2cm
}%
{%
	\vskip .2cm}% end code



\newenvironment{lema}[1][]% environment name
{% begin code
	\par\vskip .2cm%
	{\color{blue}Lema #1}\begin{itshape}%
		\par\vskip .2cm
	}%
	{% end code
	\end{itshape}\vskip .2cm\ignorespacesafterend
}

\newenvironment{proposicion}[1][]% environment name
{% begin code
	\par\vskip .2cm%
	{\color{blue}Proposición #1}\begin{itshape}%
		\par\vskip .2cm
	}%
	{% end code
	\end{itshape}\vskip .2cm\ignorespacesafterend
}

\newenvironment{teorema}[1][]% environment name
{% begin code
	\par\vskip .2cm%
	{\color{blue}Teorema #1}\begin{itshape}%
		\par\vskip .2cm
	}%
	{% end code
	\end{itshape}\vskip .2cm\ignorespacesafterend
}


\newenvironment{corolario}[1][]% environment name
{% begin code
	\par\vskip .2cm%
	{\color{blue}Corolario #1}\begin{itshape}%
		\par\vskip .2cm
	}%
	{% end code
	\end{itshape}\vskip .2cm\ignorespacesafterend
}

\newenvironment{propiedad}% environment name
{% begin code
	\par\vskip .2cm%
	{\color{blue}Propiedad}\begin{itshape}%
		\par\vskip .2cm
	}%
	{% end code
	\end{itshape}\vskip .2cm\ignorespacesafterend
}

\newenvironment{conclusion}% environment name
{% begin code
	\par\vskip .2cm%
	{\color{blue}Conclusión}\begin{itshape}%
		\par\vskip .2cm
	}%
	{% end code
	\end{itshape}\vskip .2cm\ignorespacesafterend
}


\newenvironment{definicion*}% environment name
{% begin code
	\par\vskip .2cm%
	{\color{blue}Definición}%
	\vskip .2cm
}%
{%
	\vskip .2cm}% end code

\newenvironment{observacion*}% environment name
{% begin code
	\par\vskip .2cm%
	{\color{blue}Observación}%
	\vskip .2cm
}%
{%
	\vskip .2cm}% end code


\newenvironment{obs*}% environment name
	{% begin code
		\par\vskip .2cm%
		{\color{blue}Observación}%
		\vskip .2cm
	}%
	{%
		\vskip .2cm}% end code

\newenvironment{ejemplo*}% environment name
{% begin code
	\par\vskip .2cm%
	{\color{blue}Ejemplo}%
	\vskip .2cm
}%
{%
	\vskip .2cm}% end code

\newenvironment{ejercicio*}% environment name
{% begin code
	\par\vskip .2cm%
	{\color{blue}Ejercicio}%
	\vskip .2cm
}%
{%
	\vskip .2cm}% end code

\newenvironment{propiedad*}% environment name
{% begin code
	\par\vskip .2cm%
	{\color{blue}Propiedad}\begin{itshape}%
		\par\vskip .2cm
	}%
	{% end code
	\end{itshape}\vskip .2cm\ignorespacesafterend
}

\newenvironment{conclusion*}% environment name
{% begin code
	\par\vskip .2cm%
	{\color{blue}Conclusión}\begin{itshape}%
		\par\vskip .2cm
	}%
	{% end code
	\end{itshape}\vskip .2cm\ignorespacesafterend
}






\newcommand{\nc}{\newcommand}

%%%%%%%%%%%%%%%%%%%%%%%%%LETRAS

\nc{\FF}{{\mathbb F}} \nc{\NN}{{\mathbb N}} \nc{\QQ}{{\mathbb Q}}
\nc{\PP}{{\mathbb P}} \nc{\DD}{{\mathbb D}} \nc{\Sn}{{\mathbb S}}
\nc{\uno}{\mathbb{1}} \nc{\BB}{{\mathbb B}} \nc{\An}{{\mathbb A}}

\nc{\ba}{\mathbf{a}} \nc{\bb}{\mathbf{b}} \nc{\bt}{\mathbf{t}}
\nc{\bB}{\mathbf{B}}

\nc{\cP}{\mathcal{P}} \nc{\cU}{\mathcal{U}} \nc{\cX}{\mathcal{X}}
\nc{\cE}{\mathcal{E}} \nc{\cS}{\mathcal{S}} \nc{\cA}{\mathcal{A}}
\nc{\cC}{\mathcal{C}} \nc{\cO}{\mathcal{O}} \nc{\cQ}{\mathcal{Q}}
\nc{\cB}{\mathcal{B}} \nc{\cJ}{\mathcal{J}} \nc{\cI}{\mathcal{I}}
\nc{\cM}{\mathcal{M}} \nc{\cK}{\mathcal{K}}

\nc{\fD}{\mathfrak{D}} \nc{\fI}{\mathfrak{I}} \nc{\fJ}{\mathfrak{J}}
\nc{\fS}{\mathfrak{S}} \nc{\gA}{\mathfrak{A}}
%%%%%%%%%%%%%%%%%%%%%%%%%LETRAS



\title[Clase 14 - Factorización prima]{Matemática Discreta I \\ Clase 14 - Factorización prima 1}
%\author[C. Olmos / A. Tiraboschi]{Carlos Olmos / Alejandro Tiraboschi}
\institute[]{\normalsize FAMAF / UNC
    \\[\baselineskip] ${}^{}$
    \\[\baselineskip]
}
\date[10/05/2022]{10 de mayo de 2022}




\begin{document}
    
    \frame{\titlepage} 
    
    
    \begin{frame}
        \begin{definicion} Se dice que un entero positivo $p$ es {\em primo}\index{número primo} si $p\ge 2$ y
            los únicos enteros positivos que dividen $p$ son 1 y $p$ mismo.
        \end{definicion}\pause
        \vskip .4cm
        Los primeros primos (los  menores que $100$) son
        \vskip .4cm
        2, 3, 5, 7, 11, 13, 17, 19, 23, 29, 31, 37, 41, 43, 47, 53, 59, 61, 67, 71, 73, 79, 83, 89 y 97.
        
        \vskip .4cm\pause
        Enfaticemos que de acuerdo a la definición, 1 {\it no} es primo.
        
        \vskip .4cm
        
        
        
    \end{frame}
    
    
    \begin{frame}
        
        
        \begin{itemize}
            \item[$\circ$]  $p$ es primo si y solo si $P = m_1m_2$ $\Rightarrow$ $m_1 = 1, m_2=p$ o  $m_1 = p, m_2=1$. 
            \vskip .4cm\pause
            \item[$\circ$]   $m\ge 2$ no es un primo si y sólo si existen $1 < m_1, m_2 < m$, tales que $m=m_1m_2$.
        \end{itemize}
        \pause
        \vskip .6cm
        
        No son primos:
        \begin{alignat*}4
            4 &= 2 \cdot 2,&\qquad 6&= 2\cdot3, &\qquad 8&=2 \cdot 4,&\qquad 9&= 3\cdot 3,\quad \\ 
            10&= 2\cdot5 ,& 12&= 3\cdot 4,& 14&=2 \cdot 7,& 15&=3 \cdot 5, \\
            16&= 2\cdot 8,&  18&= 2\cdot 9,& 20&= 4 \cdot 5,& 21 &= 3 \cdot 7.
        \end{alignat*}
        
        
    \end{frame}
    
    
    \begin{frame}
        Veremos que todo número entero positivo puede expresarse como producto de primos. Por ejemplo
        $$
        825=3\cdot 5\cdot 5\cdot 11.
        $$ \pause 
        Si el entero es negativo, es $-1$ por un producto de primos
        $$
        -825=-1 \cdot 3\cdot 5\cdot 5\cdot 11.
        $$ \pause 
        
        Como consecuencia del axioma del buen orden obtenemos que todo entero positivo es producto de primos (si $p$ es primo, también lo consideramos un {producto de primos.})
        
    \end{frame}
    
    
    \begin{frame}
        \begin{teorema}
            Todo  entero  mayor que $1$ es producto de números primos. 
        \end{teorema}  
        \pause
        
        \begin{idea}\pause
            \vskip -.4cm
            $$
            B = \{n >0: \text{$n$ no es  producto de primos} \}.
            $$\pause
            Si $B \ne \emptyset$, por BO existe $m$ mínimo de $B$. \pause
            \vskip .3cm
            $m$ no es primo $\Rightarrow$ $m= m_1m_2$ con $1 < m_1, m_2 < m$. \pause
            \vskip .3cm
            Como $ m_1, m_2 < m$, ambos son producto de  primos.\pause
            \vskip .3cm
            Luego $m_1m_2 = m$ es producto de primos. Absurdo.\pause
            \vskip .3cm
            El absurdo vino de suponer que $B \ne \emptyset$.
            \qed
            
        \end{idea}  



    \end{frame}
    
    
    \begin{frame}

        \begin{observacion}
			\begin{itemize}
				\item[$\circ$] Por el teorema,  decimos que todo número  \textit{admite una factorización con factores primos}.
			\end{itemize}
			
		\end{observacion}
        \pause
        \begin{ejemplo} 
            Encontremos la factorización en números primos de $201\,000$. Esto se hace di\-vi\-dien\-do  sucesivamente los números hasta llegar a factores primos:\pause
            \begin{align*}
                201\,000 &= 201\cdot 1000 = 3\cdot 67\cdot 10\cdot 10\cdot 10\\ &=  3\cdot 67\cdot 2\cdot 5 \cdot 2\cdot 5 \cdot 2\cdot 5 \\&= 2^3\cdot 3\cdot 5^3\cdot 67.
            \end{align*}
            Como vimos más arriba $2, 3, 5$ y $67$ son  números primos y por lo tanto hemos obtenido la descomposición prima de $201\,000$.
        \end{ejemplo}
    \end{frame}
    
    
    \begin{frame}
        
        \begin{observacion}
            Sea $a \in \mathbb Z$ y $p$ primo. Entonces 
            \begin{enumerate}
                \item[a)]  Si $p{\not|}a$, entonces $\operatorname{mcd}(a,p) = 1$.\pause
                \item[b)]  Si $p$ y $p'$ son primos y $p|p'$ entonces $p=p'$.
            \end{enumerate}\pause
        \end{observacion}
        \begin{proof}
            \pause
            
            \noindent (a) Como los únicos divisores de $p$ son $p$ y $1$, y $p{\not|}a$, el único  divisor común de $p$ y $a$ es $1$.
            \vskip .3cm\pause
            \noindent (b) $p'$ es primo, por lo tanto tiene sólo dos divisores positivos $1$ y $p'$. Como $p$ no es $1$, tenemos que  $p=p'$.  \qed
        \end{proof}
    \end{frame}
    
    
    \begin{frame}
        
        \begin{lema} Si $n>0$ no es primo, entonces existe $m>0$ tal que $m|n$ y $m \le \sqrt{n}$.  
        \end{lema}\pause
        \begin{demostracion}\pause
            Sea $n > 1$ que no es primo. Entonces existe  $m_1, m_2$ tal que 
            $$
            1 < m_1, m_2 < n \quad \text{y} \quad n = m_1 m_2.$$\pause
            
            Si $m_1 > \sqrt{n}$ y $m_2 >\sqrt{n}$ entonces 
            
            $$
            n = m_1m_2 > \sqrt{n}\sqrt{n} = \sqrt{n}^2 = n.
            $$\pause
            \vskip .2cm
            Absurdo. Vino de suponer que $m_1 > \sqrt{n}$ y $m_2 >\sqrt{n}$. 
            \vskip .3cm
            Luego $m_1 \le \sqrt{n}$ o $m_2 \le \sqrt{n}$\qed
            
            
        \end{demostracion}
        
    \end{frame}
    
    
    \begin{frame}
        \vskip .4cm
        El contrarrecíproco del  lema anterior, es el \textit{criterio de la raíz}:\pause
        
        \begin{proposicion}Sea $n\ge 2$. Si para todo $m$ tal que $1<m \le \sqrt{n}$ se cumple que $m{\not|}n$, entonces $n$ es primo.
        \end{proposicion}\pause
        \vskip .3cm
        \begin{corolario}
            Sea $n\ge 2$. Si para todo $p$ primo tal que $1<p \le \sqrt{n}$ se cumple que $p{\not|}n$, entonces $n$ es primo.
        \end{corolario}
    \end{frame}
    
    
    \begin{frame}
        \begin{ejemplo} Verifiquemos si $467$ es primo o no.
        \end{ejemplo}\pause
        \begin{solucion}
            Si no utilizamos el criterio de la raíz deberíamos hacer 465 divisiones: deberíamos comprobar si $m|467$ con  $1<m <467$. \pause
            \vskip .2cm
            Como $\sqrt{467} = 21.61...$, sólo debemos comprobar si $m|467$ para $2\le m \le 21$. \pause
            \vskip .2cm
            Un sencilla comprobación (dividiendo) muestra que los números $2,3,\cdots,20,21$ no  dividen a $467$ y por  lo tanto $467$ es primo. 
        \end{solucion}
        \pause
        \begin{observacion}
            En  el ejemplo anterior podríamos haber solo comprobado si los primos 2, 3, 5, 7, 11, 13, 17, 19 dividen a 467. 
        \end{observacion}
        
    \end{frame}
    
    
    \begin{frame}
        
        \begin{teorema} Sea $p$  un número  primo.
            
            \begin{enumerate} 
                \item[a)] Si $p|xy$ entonces $p|x$ o $p|y$.
                \item[b)] $x_1,x_2,\ldots,x_n$ son enteros tales que
                $$
                p|x_1x_2\ldots x_n
                $$
                entonces $p|x_i$ para algún $x_i$ ($1\le i \le n$).
            \end{enumerate}
        \end{teorema}
        \vskip .4cm
        \begin{observacion}
            La propiedad a) es \textit{muy importante} y podíamos haber definido número primo como aquel  número que cumple esta propiedad.
        \end{observacion}
    \end{frame}
    
    
    \begin{frame}
        Un error común es asumir que la propiedad a) se mantiene
        verdadera cuando reemplazamos el primo $p$ por un entero
        arbitrario . Pero esto claramente falso: por ejemplo
        $$
        6| 3\cdot 8 \quad \text{ pero } \quad 6{\not|} 3 \quad \text{ y }
        \quad 6{\not|}8.
        $$
        
        \vskip .4cm
        
        \pause
        
        La propiedad a)  juega un papel crucial en la demostración del
        siguiente resultado, que a veces es llamado el {\it Teorema
            Fundamental de la Aritmética}.
        \vskip .4cm
        \begin{teorema}La factorización en primos de un entero
            positivo $n\ge 2$ es única, salvo el orden de los factores primos.
        \end{teorema}
    \end{frame}
    
    
    \begin{frame}
        Luego todo entero positivo $n$ puede escribirse como
        $$
        m = q_{1}\cdot q_{2}\cdot \cdots q_{m-1}\cdot q_{m}
        $$
        donde los $q_i$ son  primos no necesariamente distintos.
        \vskip .8cm
        En la práctica a menudo reunimos los primos iguales en la
        factorización de $n$ y escribimos
        \begin{equation}
            n=p_1^{e_1}p_2^{e_2}\ldots p_r^{e_r},
        \end{equation}
        donde $p_1,p_2,\ldots ,p_r$ son primos distintos y $e_1,e_2,\ldots,e_r$ son enteros positivos.
    \end{frame}
    
    
    \begin{frame}
        
        \begin{ejemplo} Encontrar la descomposición prima de 7000.
        \end{ejemplo}
        \begin{solucion}
            $ 7000 = 1000 \cdot 7 = 10\cdot 10\cdot 10\cdot 7= 2\cdot 5\cdot 2\cdot 5\cdot 2 \cdot 5\cdot 7$, Es decir
            $$
            7000 = 2\cdot 5\cdot 2\cdot 5\cdot 2 \cdot 5\cdot 7.
            $$ 
            
            Pero suele ser mejor escribirlo  como
            $$7000 = 2^3 \cdot 5^3    \cdot 7.$$
        \end{solucion}
        
        
        
        
    \end{frame}
    
    
    
    
\end{document}

