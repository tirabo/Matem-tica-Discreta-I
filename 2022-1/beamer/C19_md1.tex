%\documentclass{beamer} 
\documentclass[handout]{beamer} % sin pausas
\usetheme{CambridgeUS}
%\setbeamertemplate{background}[grid][step=8 ] % cuadriculado

\usepackage{etex}
\usepackage{t1enc}
\usepackage[spanish,es-nodecimaldot]{babel}
\usepackage{latexsym}
\usepackage[utf8]{inputenc}
\usepackage{verbatim}
\usepackage{multicol}
\usepackage{amsgen,amsmath,amstext,amsbsy,amsopn,amsfonts,amssymb}
\usepackage{amsthm}
\usepackage{calc}         % From LaTeX distribution
\usepackage{graphicx}     % From LaTeX distribution
\usepackage{ifthen}
%\usepackage{makeidx}
\input{random.tex}        % From CTAN/macros/generic
\usepackage{subfigure} 
\usepackage{tikz}
\usepackage[customcolors]{hf-tikz}
\usetikzlibrary{arrows}
\usetikzlibrary{matrix}
\tikzset{
    every picture/.append style={
        execute at begin picture={\deactivatequoting},
        execute at end picture={\activatequoting}
    }
}
\usetikzlibrary{decorations.pathreplacing,angles,quotes}
\usetikzlibrary{shapes.geometric}
\usepackage{mathtools}
\usepackage{stackrel}
%\usepackage{enumerate}
\usepackage{enumitem}
\usepackage{tkz-graph}
\usepackage{polynom}
\polyset{%
    style=B,
    delims={(}{)},
    div=:
}
\renewcommand\labelitemi{$\circ$}
\setlist[enumerate]{label={(\arabic*)}}

\setbeamertemplate{itemize item}{$\circ$}
\setbeamertemplate{enumerate items}[default]
\definecolor{links}{HTML}{2A1B81}
\hypersetup{colorlinks,linkcolor=,urlcolor=links}


\newcommand{\Id}{\operatorname{Id}}
\newcommand{\img}{\operatorname{Im}}
\newcommand{\nuc}{\operatorname{Nu}}
\newcommand{\im}{\operatorname{Im}}
\renewcommand\nu{\operatorname{Nu}}
\newcommand{\la}{\langle}
\newcommand{\ra}{\rangle}
\renewcommand{\t}{{\operatorname{t}}}
\renewcommand{\sin}{{\,\operatorname{sen}}}
\newcommand{\Q}{\mathbb Q}
\newcommand{\R}{\mathbb R}
\newcommand{\C}{\mathbb C}
\newcommand{\K}{\mathbb K}
\newcommand{\F}{\mathbb F}
\newcommand{\Z}{\mathbb Z}
\newcommand{\N}{\mathbb N}
\newcommand\sgn{\operatorname{sgn}}
\renewcommand{\t}{{\operatorname{t}}}
\renewcommand{\figurename }{Figura}

%
% Ver http://joshua.smcvt.edu/latex2e/_005cnewenvironment-_0026-_005crenewenvironment.html
%

\renewenvironment{block}[1]% environment name
{% begin code
	\par\vskip .2cm%
	{\color{blue}#1}%
	\vskip .2cm
}%
{%
	\vskip .2cm}% end code


\renewenvironment{alertblock}[1]% environment name
{% begin code
	\par\vskip .2cm%
	{\color{red!80!black}#1}%
	\vskip .2cm
}%
{%
	\vskip .2cm}% end code


\renewenvironment{exampleblock}[1]% environment name
{% begin code
	\par\vskip .2cm%
	{\color{blue}#1}%
	\vskip .2cm
}%
{%
	\vskip .2cm}% end code




\newenvironment{exercise}[1]% environment name
{% begin code
	\par\vspace{\baselineskip}\noindent
	\textbf{Ejercicio (#1)}\begin{itshape}%
		\par\vspace{\baselineskip}\noindent\ignorespaces
	}%
	{% end code
	\end{itshape}\ignorespacesafterend
}


\newenvironment{definicion}[1][]% environment name
{% begin code
	\par\vskip .2cm%
	{\color{blue}Definición #1}%
	\vskip .2cm
}%
{%
	\vskip .2cm}% end code

    \newenvironment{notacion}[1][]% environment name
    {% begin code
        \par\vskip .2cm%
        {\color{blue}Notación #1}%
        \vskip .2cm
    }%
    {%
        \vskip .2cm}% end code

\newenvironment{observacion}[1][]% environment name
{% begin code
	\par\vskip .2cm%
	{\color{blue}Observación #1}%
	\vskip .2cm
}%
{%
	\vskip .2cm}% end code

\newenvironment{ejemplo}[1][]% environment name
{% begin code
	\par\vskip .2cm%
	{\color{blue}Ejemplo #1}%
	\vskip .2cm
}%
{%
	\vskip .2cm}% end code


\newenvironment{preguntas}[1][]% environment name
{% begin code
    \par\vskip .2cm%
    {\color{blue}Preguntas #1}%
    \vskip .2cm
}%
{%
    \vskip .2cm}% end code

\newenvironment{ejercicio}[1][]% environment name
{% begin code
	\par\vskip .2cm%
	{\color{blue}Ejercicio #1}%
	\vskip .2cm
}%
{%
	\vskip .2cm}% end code


\renewenvironment{proof}% environment name
{% begin code
	\par\vskip .2cm%
	{\color{blue}Demostración}%
	\vskip .2cm
}%
{%
	\vskip .2cm}% end code



\newenvironment{demostracion}% environment name
{% begin code
	\par\vskip .2cm%
	{\color{blue}Demostración}%
	\vskip .2cm
}%
{%
	\vskip .2cm}% end code

\newenvironment{idea}% environment name
{% begin code
	\par\vskip .2cm%
	{\color{blue}Idea de la demostración}%
	\vskip .2cm
}%
{%
	\vskip .2cm}% end code

\newenvironment{solucion}% environment name
{% begin code
	\par\vskip .2cm%
	{\color{blue}Solución}%
	\vskip .2cm
}%
{%
	\vskip .2cm}% end code



\newenvironment{lema}[1][]% environment name
{% begin code
	\par\vskip .2cm%
	{\color{blue}Lema #1}\begin{itshape}%
		\par\vskip .2cm
	}%
	{% end code
	\end{itshape}\vskip .2cm\ignorespacesafterend
}

\newenvironment{proposicion}[1][]% environment name
{% begin code
	\par\vskip .2cm%
	{\color{blue}Proposición #1}\begin{itshape}%
		\par\vskip .2cm
	}%
	{% end code
	\end{itshape}\vskip .2cm\ignorespacesafterend
}

\newenvironment{teorema}[1][]% environment name
{% begin code
	\par\vskip .2cm%
	{\color{blue}Teorema #1}\begin{itshape}%
		\par\vskip .2cm
	}%
	{% end code
	\end{itshape}\vskip .2cm\ignorespacesafterend
}


\newenvironment{corolario}[1][]% environment name
{% begin code
	\par\vskip .2cm%
	{\color{blue}Corolario #1}\begin{itshape}%
		\par\vskip .2cm
	}%
	{% end code
	\end{itshape}\vskip .2cm\ignorespacesafterend
}

\newenvironment{propiedad}% environment name
{% begin code
	\par\vskip .2cm%
	{\color{blue}Propiedad}\begin{itshape}%
		\par\vskip .2cm
	}%
	{% end code
	\end{itshape}\vskip .2cm\ignorespacesafterend
}

\newenvironment{conclusion}% environment name
{% begin code
	\par\vskip .2cm%
	{\color{blue}Conclusión}\begin{itshape}%
		\par\vskip .2cm
	}%
	{% end code
	\end{itshape}\vskip .2cm\ignorespacesafterend
}


\newenvironment{definicion*}% environment name
{% begin code
	\par\vskip .2cm%
	{\color{blue}Definición}%
	\vskip .2cm
}%
{%
	\vskip .2cm}% end code

\newenvironment{observacion*}% environment name
{% begin code
	\par\vskip .2cm%
	{\color{blue}Observación}%
	\vskip .2cm
}%
{%
	\vskip .2cm}% end code


\newenvironment{obs*}% environment name
	{% begin code
		\par\vskip .2cm%
		{\color{blue}Observación}%
		\vskip .2cm
	}%
	{%
		\vskip .2cm}% end code

\newenvironment{ejemplo*}% environment name
{% begin code
	\par\vskip .2cm%
	{\color{blue}Ejemplo}%
	\vskip .2cm
}%
{%
	\vskip .2cm}% end code

\newenvironment{ejercicio*}% environment name
{% begin code
	\par\vskip .2cm%
	{\color{blue}Ejercicio}%
	\vskip .2cm
}%
{%
	\vskip .2cm}% end code

\newenvironment{propiedad*}% environment name
{% begin code
	\par\vskip .2cm%
	{\color{blue}Propiedad}\begin{itshape}%
		\par\vskip .2cm
	}%
	{% end code
	\end{itshape}\vskip .2cm\ignorespacesafterend
}

\newenvironment{conclusion*}% environment name
{% begin code
	\par\vskip .2cm%
	{\color{blue}Conclusión}\begin{itshape}%
		\par\vskip .2cm
	}%
	{% end code
	\end{itshape}\vskip .2cm\ignorespacesafterend
}






\newcommand{\nc}{\newcommand}

%%%%%%%%%%%%%%%%%%%%%%%%%LETRAS

\nc{\FF}{{\mathbb F}} \nc{\NN}{{\mathbb N}} \nc{\QQ}{{\mathbb Q}}
\nc{\PP}{{\mathbb P}} \nc{\DD}{{\mathbb D}} \nc{\Sn}{{\mathbb S}}
\nc{\uno}{\mathbb{1}} \nc{\BB}{{\mathbb B}} \nc{\An}{{\mathbb A}}

\nc{\ba}{\mathbf{a}} \nc{\bb}{\mathbf{b}} \nc{\bt}{\mathbf{t}}
\nc{\bB}{\mathbf{B}}

\nc{\cP}{\mathcal{P}} \nc{\cU}{\mathcal{U}} \nc{\cX}{\mathcal{X}}
\nc{\cE}{\mathcal{E}} \nc{\cS}{\mathcal{S}} \nc{\cA}{\mathcal{A}}
\nc{\cC}{\mathcal{C}} \nc{\cO}{\mathcal{O}} \nc{\cQ}{\mathcal{Q}}
\nc{\cB}{\mathcal{B}} \nc{\cJ}{\mathcal{J}} \nc{\cI}{\mathcal{I}}
\nc{\cM}{\mathcal{M}} \nc{\cK}{\mathcal{K}}

\nc{\fD}{\mathfrak{D}} \nc{\fI}{\mathfrak{I}} \nc{\fJ}{\mathfrak{J}}
\nc{\fS}{\mathfrak{S}} \nc{\gA}{\mathfrak{A}}
%%%%%%%%%%%%%%%%%%%%%%%%%LETRAS



\title[Clase 19 - Isomorfismo de grafos]{Matemática Discreta I \\ Clase 19 -  Isomorfismo de grafos / Valencias}
%\author[C. Olmos / A. Tiraboschi]{Carlos Olmos / Alejandro Tiraboschi}
\institute[]{\normalsize FAMAF / UNC
    \\[\baselineskip] ${}^{}$
    \\[\baselineskip]
}
\date[20/05/2021]{20 de mayo de 2021}




\begin{document}
    
    \frame{\titlepage} 
    
    
    \begin{frame}\frametitle{Isomorfismo de grafos}
        ¿Cuándo debemos considerar a dos grafos ``iguales''?
        \vskip .4cm\pause
        \begin{itemize}
            \item[$\circ$] No importa el  nombre de los vértices. 
            \vskip .2cm\pause
            \item[$\circ$] Un grafo se puede dibujar de muchas maneras.
            \vskip .2cm\pause
            \item[$\circ$]  La propiedad característica de un grafo es la manera en que los vértices están conectados por aristas. 
        \end{itemize}
        \vskip .4cm\pause
        Dicho en forma directa: vamos a considerar que dos grafos $G_1$ y $G_2$ son ``iguales'' si  cambiando  el nombre de los vértices de $G_2$  por el de los vértices de $G_1$,  en cierto orden, obtenemos $G_1$.  
        
    \end{frame}
    
    \begin{frame}
        \begin{ejemplo} Tenemos dos grafos:
            \begin{figure}[ht]
                \begin{center}
                    \begin{tabular}{llll}
                        &
                        \begin{tikzpicture}[scale=1]
                            %\SetVertexSimple[Shape=circle,FillColor=white]
                            \Vertex[x=0,y=0]{$a$}
                            \Vertex[x=2,y=0]{$b$}
                            \Vertex[x=2,y=-2]{$c$}
                            \Vertex[x=0,y=-2]{$d$}
                            \Edges($a$, $b$,$c$,$d$,$a$,$b$,$d$)
                            \draw (1,-3) node {$G_1$};
                        \end{tikzpicture}
                        &
                        \qquad
                        & 
                        \begin{tikzpicture}[scale=1]
                            %\SetVertexSimple[Shape=circle,FillColor=white]
                            \Vertex[x=1,y=0]{$t$}
                            \Vertex[x=1,y=-1.3]{$w$}
                            \Vertex[x=2,y=-2]{$v$}
                            \Vertex[x=0,y=-2]{$u$}
                            \Edges($v$, $t$,$u$,$v$,$w$,$u$)
                            \draw (1,-3) node {$G_2$};
                        \end{tikzpicture}
                    \end{tabular}
                \end{center}
            \end{figure}\pause
            Parecen ser distintos. 
        \end{ejemplo}
    \end{frame}
    
    \begin{frame}
        Sin embargo, si dibujamos $G_2$ diferente (``dando vuelta $w$'' y  rotando), obtenemos: \pause
        
        \begin{figure}[ht]
            \begin{center}
                \begin{tabular}{llll}
                    &
                    \begin{tikzpicture}[scale=1]
                        %\SetVertexSimple[Shape=circle,FillColor=white]
                        \Vertex[x=0,y=0]{$a$}
                        \Vertex[x=2,y=0]{$b$}
                        \Vertex[x=2,y=-2]{$c$}
                        \Vertex[x=0,y=-2]{$d$}
                        \Edges($a$, $b$,$c$,$d$,$a$,$b$,$d$)
                        \draw (1,-3) node {$G_1$};
                    \end{tikzpicture}
                    &
                    \qquad
                    & 
                    \begin{tikzpicture}[scale=1]
                        %\SetVertexSimple[Shape=circle,FillColor=white]
                        \Vertex[x=0,y=0]{$t$}
                        \Vertex[x=2,y=0]{$v$}
                        \Vertex[x=2,y=-2]{$w$}
                        \Vertex[x=0,y=-2]{$u$}
                        \Edges($v$, $t$,$u$,$v$,$w$,$u$)
                        \draw (1,-3) node {$G_2$};
                    \end{tikzpicture}
                \end{tabular}
            \end{center}
        \end{figure}
        Los cuales claramente describen el mismo grafo, pero  con los nombres de los vértices cambiados. 
    \end{frame}
    
    \begin{frame}\frametitle{Isomorfismo de grafos: preliminares}
        
        \begin{definicion}
            Dado  dos conjuntos $X,Y$ diremos que una aplicación $f: X \to Y$ es {\em biyectiva} si para cada $y \in Y$ existe un  único $x \in X$ tal que $f(x) =y$. 
        \end{definicion}
        \pause

        \vskip .6cm
        Un propiedad importante, de las funciones biyectivas es:
        \vskip .4cm
        \begin{teorema}
            $f$ es biyectiva si y sólo sí  $f$ tiene {\em inversa}, es decir existe $f^{-1}: Y \to X$, tal que 
            $$f(f^{-1}(y)) = y\;\;\forall \,y \in Y \quad \wedge \quad f^{-1}(f(x)) = x, \;\;\forall \,x \in X.$$
        \end{teorema}
    \end{frame}
    
    \begin{frame}

    
        \begin{ejemplo}
            La función  
            \begin{align*}
                f&: \{1,2,3\}\to\{a,b,c\} \quad \text{definida } f(1) = c, f(2) = b, f(3) = a
            \end{align*}
            es biyectiva y su  inversa es 
            $$
            f^{-1}(a) = 3,\;f^{-1}(b) = 2,\;f^{-1}(c) =1.
            $$
        \end{ejemplo}

        \vskip 2cm
    \end{frame}
    
    \begin{frame}\frametitle{Isomorfismos de grafos: definición}
        \begin{definicion} Dos grafos $G_1$ y $G_2$ se dicen que
            son {\em isomorfos} cuando  existe una biyección $\alpha$ entre el
            conjunto de vértices de $G_1$ y el conjunto de vértices de
            $G_2$ tal que  
            
            \begin{itemize}\pause
                \item[$\circ$] si $\{x,y\}$ es una arista de $G_1$ entonces $\{\alpha(x),\alpha(y)\}$ es una arista de $G_2$ \pause y, recíprocamente,
                \item[$\circ$]si  $\{z,w\}$ es una arista de $G_2$ entonces $\{\alpha^{-1}(z),\alpha^{-1}(w)\}$ es una arista de $G_1$. 
            \end{itemize} \pause
            \vskip .3cm
            Equivalentemente, diremos que $\alpha$    es un {\em isomorfismo} si es una biyección entre el     conjunto de vértices de $G_1$ y el conjunto de vértices de
            $G_2$ tal que por cada $\{z,w\}$  arista de $G_2$,  existe una y solo una $\{x,y\}$  arista de $G_1$ tal que $\{\alpha(x),\alpha(y)\}  = \{z,w\}$.
            
        \end{definicion}
    \end{frame}
    
    \begin{frame}
        \begin{ejemplo}
            \begin{figure}[ht]
                \begin{center}
                    \begin{tabular}{llll}
                        &
                        \begin{tikzpicture}[scale=1]
                            %\SetVertexSimple[Shape=circle,FillColor=white]
                            \Vertex[x=0,y=0]{$a$}
                            \Vertex[x=2,y=0]{$b$}
                            \Vertex[x=2,y=-2]{$c$}
                            \Vertex[x=0,y=-2]{$d$}
                            \Edges($a$, $b$,$c$,$d$,$a$,$b$,$d$)
                            \draw (1,-3) node {$G_1$};
                        \end{tikzpicture}
                        &
                        \qquad
                        & 
                        \begin{tikzpicture}[scale=1]
                            %\SetVertexSimple[Shape=circle,FillColor=white]
                            \Vertex[x=1,y=0]{$t$}
                            \Vertex[x=1,y=-1.3]{$w$}
                            \Vertex[x=2,y=-2]{$v$}
                            \Vertex[x=0,y=-2]{$u$}
                            \Edges($v$, $t$,$u$,$v$,$w$,$u$)
                            \draw (1,-3) node {$G_2$};
                        \end{tikzpicture}
                    \end{tabular}
                \end{center}
                \caption{$G_1$ y $G_2$ son isomorfos} \label{f5.3}
            \end{figure}
            
            
            Una biyección es dada por
            $$
            \alpha(a)=t,\quad \alpha(b)=v,\quad \alpha(c)=w,\quad \alpha(d)=u.
            $$
            Podemos comprobar que a cada arista de $G_1$ le corresponde una
            arista de $G_2$ y vi\-ce\-ver\-sa.
        \end{ejemplo}
    \end{frame}
    
    \begin{frame}
        \begin{itemize}
            \item Para mostrar que dos grafos no son isomorfos, nosotros debemos
            demostrar que no hay una biyección entre el conjunto de vértices
            de uno con el conjunto de vértices de otro, que lleve las aristas
            de uno en las aristas del otro.\pause
            \item     Si dos grafos tienen diferente número de vértices, entonces no es
            posible ninguna biyección, y los grafos no pueden ser isomorfos.\pause
            \item Si los grafos tienen el mismo número de vértices, pero di\-fe\-ren\-te
            número de aristas, entonces hay biyecciones de vértices  pero ninguna de ellas
            puede ser un isomorfismo. 
        \end{itemize}
        
        
        
        
        
    \end{frame}
    
    \begin{frame}
        
        
        
        \begin{definicion} 
            
            Sea $G=(V,E)$ un grafo. Se dice que $G^{\prime}=(V^{\prime},E^{\prime})$ es {\em subgrafo} de $G=(V,E)$ si 
            \vskip .2cm
            (1)  $G^{\prime}$  es un grafo,\quad  y \quad (2)     $V^{\prime} \subset V$, $E^{\prime} \subset E$.
            
        \end{definicion}
        \pause \vskip .4cm
        Algunos subgrafos de 
        \begin{figure}[ht]
            \begin{center}
                \begin{tikzpicture}[scale=0.4]
                    \SetVertexSimple[Shape=circle,FillColor=white,MinSize=8 pt]
                    \Vertex[x=0.00, y=2.00]{$a$}
                    \Vertex[x=1.90, y=0.62]{$b$}
                    \Vertex[x=1.18, y=-1.62]{$c$}
                    \Vertex[x=-1.18, y=-1.62]{$d$}
                    \Vertex[x=-1.90, y=0.62]{$e$}
                    \Edges($c$, $b$,$a$,$e$,$d$,$b$,$a$,$d$)
                    \Edges($e$,$b$)
                \end{tikzpicture}
            \end{center}
        \end{figure}
        son:    
        \begin{figure}[ht]
            \begin{center}
                \begin{tabular}{llllllll}
                    &
                    \begin{tikzpicture}[scale=0.4]
                        \SetVertexSimple[Shape=circle,FillColor=white,MinSize=8 pt]
                        \Vertex[x=0.00, y=2.00]{$a$}
                        \Vertex[x=1.90, y=0.62]{$b$}
                        \Vertex[x=-1.18, y=-1.62]{$d$}
                        \Edges($a$,$b$,$d$,$a$)
                    \end{tikzpicture}
                    &
                    \qquad
                    & 
                    \begin{tikzpicture}[scale=0.4]
                        \SetVertexSimple[Shape=circle,FillColor=white,MinSize=8 pt]
                        \Vertex[x=0.00, y=2.00]{$a$}
                        \Vertex[x=1.90, y=0.62]{$b$}
                        \Vertex[x=-1.18, y=-1.62]{$d$}
                        \Vertex[x=-1.90, y=0.62]{$e$}
                        \Edges($a$,$e$,$b$,$a$)
                    \end{tikzpicture}
                    &
                    \qquad
                    &
                    \begin{tikzpicture}[scale=0.4]
                        \SetVertexSimple[Shape=circle,FillColor=white,MinSize=8 pt]
                        \Vertex[x=0.00, y=2.00]{$a$}
                        \Vertex[x=1.90, y=0.62]{$b$}
                        \Vertex[x=1.18, y=-1.62]{$c$}
                        \Vertex[x=-1.18, y=-1.62]{$d$}
                        \Vertex[x=-1.90, y=0.62]{$e$}
                        \Edges($a$,$e$,$b$)
                    \end{tikzpicture}
                    &
                    \qquad
                    &
                    \begin{tikzpicture}[scale=0.4]
                        \SetVertexSimple[Shape=circle,FillColor=white,MinSize=8 pt]
                        \Vertex[x=0.00, y=2.00]{$a$}
                        \Vertex[x=1.90, y=0.62]{$b$}
                        \Vertex[x=-1.18, y=-1.62]{$d$}
                        \Vertex[x=-1.90, y=0.62]{$e$}
                        \Edges( $b$,$a$,$e$,$d$,$b$,$a$,$d$)
                        \Edges($e$,$b$)
                    \end{tikzpicture}
                \end{tabular}
            \end{center}
        \end{figure}
        
        
    \end{frame}
    
    
    \begin{frame}
        
        Es claro, pero  no lo demostraremos aquí, que un isomorfismo lleva un subgrafo a un subgrafo isomorfo. Este resultado es una herramienta que puede ser útil para ver si dos grafos no son isomorfos. \pause
        \vskip .5cm
        \begin{figure}[ht]
            \begin{center}
                \begin{tabular}{llll}
                    &
                    \begin{tikzpicture}[scale=1]
                        %\SetVertexSimple[Shape=circle,FillColor=white]
                        \Vertex[x=0.00, y=2.00]{$a$}
                        \Vertex[x=1.90, y=0.62]{$b$}
                        \Vertex[x=1.18, y=-1.62]{$c$}
                        \Vertex[x=-1.18, y=-1.62]{$d$}
                        \Vertex[x=-1.90, y=0.62]{$e$}
                        \Edges($c$, $b$,$a$,$e$,$d$,$b$,$a$,$d$)
                        \Edges($e$,$b$)
                        \draw (0,-2.2) node {$G_1$};
                    \end{tikzpicture}
                    &
                    \qquad
                    & 
                    \begin{tikzpicture}[scale=1]
                        %\SetVertexSimple[Shape=circle,FillColor=white]
                        \Vertex[x=0.00, y=2.00]{1}
                        \Vertex[x=1.90, y=0.62]{2}
                        \Vertex[x=1.18, y=-1.62]{3}
                        \Vertex[x=-1.18, y=-1.62]{4}
                        \Vertex[x=-1.90, y=0.62]{5}
                        \Edges(1,2,3,4,5,1)
                        \Edges(4,2,5)
                        \draw (0,-2.2) node {$G_2$};
                    \end{tikzpicture}
                \end{tabular}
            \end{center}
            \caption{$G_1$ y $G_2$ no son isomorfos pues $K_4 \subset G_1$ y $K_4 \not\subset G_2$}
        \end{figure}
        
    \end{frame}
    
    
    
    \begin{frame}\frametitle{Valencias}
        La {\em {valencia}} de un vértice $v$ en un grafo $G=(V,E)$ es el
        \index{valencia de un vértice} número de aristas de $G$ que contienen a $v$.
        Usaremos la notación $\delta(v)$ para la valencia de $v$,
        formalmente
        $$
        \delta(v)=|D_v|, \quad \text{ donde } \quad D_v=\{e \in E| v\in
        e\}.
        $$
        
        \begin{ejemplo}
            \begin{figure}[ht]
                \begin{center}
                    \begin{tabular}{llll}
                        &\begin{tikzpicture}[scale=0.8]
                            %\SetVertexSimple[Shape=circle,FillColor=white,MinSize=8 pt]
                            \Vertex[x=0.00, y=2.00,L=$3$]{$a$}
                            \Vertex[x=1.90, y=0.62,L=$4$]{$b$}
                            \Vertex[x=1.18, y=-1.62,L=$1$]{$c$}
                            \Vertex[x=-1.18, y=-1.62,L=$3$]{$d$}
                            \Vertex[x=-1.90, y=0.62,L=$3$]{$e$}
                            \Edges($c$, $b$,$a$,$e$,$d$,$b$,$a$,$d$)
                            \Edges($e$,$b$)
                        \end{tikzpicture}
                        &
                        \qquad    \qquad
                        & 
                        \begin{tikzpicture}[scale=0.55]
                            %\SetVertexSimple[Shape=circle,FillColor=white]
                            %                
                            \Vertex[x=3.00, y=0.00, L=$1$]{$v_3$}
                            \Vertex[x=1.50, y=2.60, L=$0$]{$v_2$}
                            \Vertex[x=-1.50, y=2.60, L=$2$]{$v_1$}
                            \Vertex[x=-3.00, y=0.00, L=$2$]{$v_6$}
                            \Vertex[x=-1.50, y=-2.60, L=$3$]{$v_5$}
                            \Vertex[x=1.50, y=-2.60, L=$2$]{$v_4$}
                            \Edges($v_1$,$v_6$, $v_5$,$v_4$,$v_1$)
                            \Edges($v_3$,$v_5$)
                        \end{tikzpicture}
                    \end{tabular}
                \end{center}
            \end{figure}
        \end{ejemplo}
    \end{frame}
    
    \begin{frame}
        En una lista de adyacencia la valencia de un vértice es exactamente la cantidad de elementos que tiene la columna correspondiente al vértice.
        
        \begin{ejemplo}
            Vimos el $G=(V,E)$ es dado por
            \begin{equation*}\label{grafosimple}
                V=\{a,b,c,d,z\}, \qquad\quad
                E=\{\{a,b\},\{a,d\},\{b,z\},\{c,d\},\{d,z\}\}.
            \end{equation*}
            \pause    
            Su lista de adyacencia es 
            
            \begin{center}
                \begin{tabular}{ccccc}
                    $a$&$b$&$c$&$d$&$z$ \\ \hline
                    $b$&$a$&$d$&$a$&$b$ \\
                    $d$&$z$&&$c$&$d$\\
                    &&&$z$&
                \end{tabular}
            \end{center}
            \pause    
            Luego $\delta(a)=2$, $\delta(b)=2$, $\delta(c)=1$, $\delta(d)=3$, $\delta(z)=2$.
            
        \end{ejemplo}
    \end{frame}
    
    \begin{frame}
        
        
        
        \begin{teorema}\label{t5.3} La suma de los valores de las valencias
            $\delta(v)$, tomados sobre todos los vértices $v$ del grafo
            $G=(V,E)$, es igual a dos veces el número de aristas:
            $$
            \sum_{v \in V} \delta(v) = 2|E|.
            $$
        \end{teorema}\vskip -.6cm\pause
        \begin{proof} La valencia de un vértice $v$ indica la cantidad de
            ``extremos'' de aristas que ``tocan'' a $v$. Es claro que hay
            $2|E|$ extremos de aristas, luego la suma total de las valencias
            de los vértices es $2|E|$.

            \qed
        \end{proof}
        
        
    \end{frame}
    
    \begin{frame}
        Diremos que un vértice de $G$ es {\em impar} si su  valencia es impar, y {\em par} si su valencia es par.
        \pause\vskip .3cm
        
        Denotemos $V_i$ y $V_p$ los conjuntos de vértices impares y pares
        respectivamente, luego $V=V_i \cup V_p$ es una partición de $V$.\pause
        Luego
        $$
        \sum_{v \in V_i} \delta(v) + \sum_{v \in V_p} \delta(v)= 2|E|.
        $$ 
        y entonces
        $$
        \sum_{v \in V_i} \delta(v) = \underbrace{2|E|}_{\text{par}} -  \underbrace{\sum_{v \in V_p} \delta(v)}_{\text{par}}.
        $$ 
        \pause
        Luego, la suma de las valencias de los vértices impares  es par \; $\Rightarrow$
        
        \begin{teorema} El número de vértices impares es par.
        \end{teorema}
        
        
    \end{frame}
    
    \begin{frame}
        Este resultado es a veces llamado el ``handshaking lemma'': dado un conjunto de personas, el número de personas
        que le ha dado la mano a un número impar de miembros del conjunto
        es par.
        
        \vskip .4cm
        
        Un grafo en el cual todos los vértices tienen la misma valencia $r$ se llama {\em regular}  (con valencia $r$), o {\em $r$-valente.} Luego, 
        $$
        r|V|=2|E|.
        $$
        
        \begin{itemize}
            \item $K_n$ regular $(n-1)$-valente.
            \item  $C_n$,  el polígono de $n$-lados. Si $n \ge3$, $C_n$  es regular de valencia $2$. 
        \end{itemize}
        
        \vskip .4cm 
        
        $C_n$  es llamado el \textit{grafo cíclico} de $n$ vértices. Formalmente $C_n =(V,E)$, con
        $$
        V = \{ 1,2,\ldots,n\}, \qquad E = \{\{1,2\}, \{2,3\}, \ldots , \{n-1,n\},\{n,1\}  \}.
        $$
        
        
    \end{frame}
    
\end{document}

