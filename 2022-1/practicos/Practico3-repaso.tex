% PDFLaTeX
\documentclass[a4paper,12pt,twoside,spanish,reqno]{amsbook}
%%%---------------------------------------------------

%\renewcommand{\familydefault}{\sfdefault} % la font por default es sans serif
%\usepackage[T1]{fontenc}

% Para hacer el  indice en linea de comando hacer 
% makeindex main
%% En http://www.tug.org/pracjourn/2006-1/hartke/hartke.pdf hay ejemplos de packages de fonts libres, como los siguientes:
%\usepackage{cmbright}
%\usepackage{pxfonts}
%\usepackage[varg]{txfonts}
%\usepackage{ccfonts}
%\usepackage[math]{iwona}
\usepackage[math]{kurier}

\usepackage{etex}
\usepackage{t1enc}
\usepackage{latexsym}
\usepackage[utf8]{inputenc}
\usepackage{verbatim}
\usepackage{multicol}
\usepackage{amsgen,amsmath,amstext,amsbsy,amsopn,amsfonts,amssymb}
\usepackage{amsthm}
\usepackage{calc}         % From LaTeX distribution
\usepackage{graphicx}     % From LaTeX distribution
\usepackage{ifthen}
\input{random.tex}        % From CTAN/macros/generic
\usepackage{subfigure} 
\usepackage{tikz}
\usetikzlibrary{arrows}
\usetikzlibrary{matrix}
\usepackage{mathtools}
\usepackage{stackrel}
\usepackage{enumitem}
\usepackage{tkz-graph}
%\usepackage{makeidx}
\usepackage{hyperref}
\hypersetup{
    colorlinks=true,
    linkcolor=blue,
    filecolor=magenta,      
    urlcolor=cyan,
}
\usepackage{hypcap}
\numberwithin{equation}{section}
% http://www.texnia.com/archive/enumitem.pdf (para las labels de los enumerate)
\renewcommand\labelitemi{$\circ$}
\setlist[enumerate, 1]{label={(\arabic*)}}
\setlist[enumerate, 2]{label=\emph{\alph*)}}


%%% FORMATOS %%%%%%%%%%%%%%%%%%%%%%%%%%%%%%%%%%%%%%%%%%%%%%%%%%%%%%%%%%%%%%%%%%%%%
\tolerance=10000
\renewcommand{\baselinestretch}{1.3}
\usepackage[a4paper, top=3cm, left=3cm, right=2cm, bottom=2.5cm]{geometry}
\usepackage{setspace}
%\setlength{\parindent}{0,7cm}% tamaño de sangria.
\setlength{\parskip}{0,4cm} % separación entre parrafos.
\renewcommand{\baselinestretch}{0.90}% separacion del interlineado
\setlist[1]{topsep=8pt,itemsep=.4cm,partopsep=4pt, parsep=4pt} %espacios nivel 1 listas
\setlist[2]{itemsep=.15cm}  %espacios nivel 2 listas
%%%%%%%%%%%%%%%%%%%%%%%%%%%%%%%%%%%%%%%%%%%%%%%%%%%%%%%%%%%%%%%%%%%%%%%%%%%%%%%%%%%
%\end{comment}
%%% FIN FORMATOS  %%%%%%%%%%%%%%%%%%%%%%%%%%%%%%%%%%%%%%%%%%%%%%%%%%%%%%%%%%%%%%%%%

\newcommand{\rta}{\noindent\textit{Rta: }} 

\begin{document}
    \baselineskip=0.55truecm %original    

{\bf \begin{center} Práctico 3 - Repaso \\ Matemática Discreta I -- Año 2022/1 \\ FAMAF\end{center}}




\begin{enumerate}
\setlength\itemsep{1.1em}

\item Sea $p$ primo positivo. Probar que $(p,(p-1)!)=1$.

\rta Supongamos que  $(p,(p-1)!)>1$,  como el único divisor de $p$ además del $1$ es $p$, esto quiere decir que  $(p,(p-1)!)=p$ y por lo tanto $p|(p-1)!$.

Ahora bien, recodemos que si  $p$ primo, entonces
$$
p | a_1\cdot a_2 \cdot \cdots \cdot a_k \quad \Rightarrow\quad p| a_i \; \text{ para algún $i$ tal que $1 \le i \le k$}.
$$
Luego,
$$
p | (p-1)! = 1\cdot 2 \cdots (p-1) \quad \Rightarrow\quad p| i \; \text{ para algún $i$ tal que $1 \le i \le p-1$}.
$$
Es decir, $p|i$ con $i < p$, absurdo.


\item Demostrar que $\forall n\in{\mathbb Z}$, $n>2$, existe $p$ primo tal que $n<p<n!$. (Ayuda: pensar qué primos dividen a $n! - 1$.)

\rta Si $n! - 1$ es primo,  el ejercicio está demostrado. Si  $n! - 1$ no es primo, entonces existe un primo $p$ tal que $p| n! - 1$. Ahora bien, $p\not| i$ para $1 \le i \le n$, pues si $p|i$ $\Rightarrow$    $p| n!$  $\Rightarrow$  $p| (n! - 1) -n = -1$,  absurdo. 

Por lo tanto, $p$ primo, $p \ne 1,2, \ldots, n$ y  $p| n! - 1$,  esto implica que $p$ primo, $p > n$ y  $p <  n!$, lo cual prueba el resultado.



\item Dado un entero $a>0$ fijo, caracterizar aquellos números que al dividirlos por $a$ tienen cociente igual al resto.

\rta Sea $b$ que cumpla con lo que pide el enunciado del ejercicio, es decir $b = a \cdot r + r$ y $0 \le r < b$. por lo tanto  $b = (a+1) r$ con  $0 \le r < b$. Como $a>0$,  es claro que si $b=(a+1)r$,  entonces $r < b$. 

Concluyendo:  los  números que al dividirlos por $a$ tienen cociente igual al resto son de la forma $(a+1)r$,  con $0 \le r$. 


\item Probar que si $(a,4)=2$ y $(b,4)=2$ entonces $(a+b,4)=4$.

\rta Dividamos $a$ y $b$ por $4$, y enemos $a = 4k+ r$, $b = 4t + s$ con $ 0 \le r,s < 4$. Ahora bien como  $(a,4)=2$ y $(b,4)=2$, $4$ no divide ni $a$, ni a $b$, por  lo tanto $0 < r,s$. Por otro lado, como $2$ divide a $a$ y $b$,  entonces $r,s \not= 1,3$. Todo esto implica que $r=s=2$. Es decir, $a = 4k+ 2$, $b = 4t + 2$. Luego 
$$
a + b =  ( 4k+ 2) + ( 4t + 2) = 4(k+t) +4 = 4(k+t+1). 
$$
Esta ecuación nos dice que $4| a+b$, luego $(a+b,4)=4$.

\item Probar que si $a,b$ son coprimos entonces $(a+b,a-b)=1 \text{
ó } 2 $.

\rta Si $a+b$ y $a-b$ no tienen un primo en común que los divida, entonces $(a+b,a-b)=1 $ y el ejercicio está resuelto. 

En caso contrario, sea $p$ primo tal que $p | a+b$ y $p|a-b$, luego 
\begin{align*}
    p|  (a+b) + (a-b) &= 2a \quad \stackrel{\text{$p$ es primo}}{\Longrightarrow} \quad p|2 \;\vee\; p|a \\
    p|  (a+b) - (a-b) &= 2b \quad \stackrel{\text{$p$ es primo}}{\Longrightarrow} \quad p|2 \;\vee\; p|b. \\
\end{align*} 
Como $a$ y $b$ son coprimos, no tienen un primo en común que los divida, es decir no puede ocurrir  que  $p | b$ y $p|b$. Por lo tanto $p|2$ (por lo de arriba),  es decir $p =2$. Esto nos dice que  $a+b$ y $a-b$ son divisibles por $2$. Tomenos $n = (a+b)/2$ y $m = (a-b)/2$  (son números enteros porque  $a+b$ y $a-b$ son divisibles por $2$). Sea $q$ primo tal que $q|n$ y $q|m$. Entonces, 
\begin{align*}
    q| n+m =  \frac{a+b}{2} + \frac{a-b}{2} &= a  \\
    q| n-m =  \frac{a+b}{2} - \frac{a-b}{2} &= b .\\
\end{align*} 
Es  decir, $q$ primo y $q|a$ y $q|b$, pero esto no puede ocurrir pues $a$ y $b$ coprimos. 


Luego
$$
( \frac{a+b}{2}, \frac{a-b}{2}) =1 \qquad \Rightarrow\qquad (a+b,a-b) =2.
$$


\item Sean $a,b$ enteros no nulos. Completar y demostrar:

a)  $[a,a]=?$

b)  $[a,b]=b$ si y solo si $\ldots$

c) $(a,b)=[a,b]$ si y solo si $\ldots$

\rta 

a)  $[a,a]=|a|$

\textit{Demostración.} Supongamos que $a > 0$. Si $m$  es el mcm de $a$ y $a$,  entonces $m$  es el menor múltiplo positivo de  $a$,  es decir $m=a$. Si $a<0$, entonces $-a>0$ y aplicando el razonamiento anterior  $[-a,-a]=-a = |a|$. Como $[a,a]= [-a,-a]$ obtenemos que  $[a,a]=|a|$. 

\vskip .2cm 

b)  $[a,b]=b$ si y solo si $b>0$ y $a|b$. 

\textit{Demostración.} ($\Rightarrow$) Como $b$  es un mcm, por definición de mcm $b> 0$. Por otro  lado, de nuevo por definición de mcm, $a|b$.

($\Leftarrow$) $b>0$ y $a|b$, $b|b$, luego $b$ es un múltiplo positivo de $a$ y $b$ y como todo múltiplo  de $b$ es mayor o igual a $b$, $b$  es el mcm. 

\vskip .2cm 
c) $(a,b)=[a,b]$ si y solo si $a = \pm b$.

\textit{Demostración.} Supongamos que ambos son positivos (en caso contrario usamos que $(a,b) = (\pm a,\pm b)$ y $[a,b] = [\pm a,\pm b]$). 

($\Rightarrow$) Sea $ k = (a,b)=[a,b]$. Como $k = (a,b)$, $k \ge a, b$. Como  $k = [a,b]$, $k \le a, b$: En consecuencia $a \le k \le a$ y $b \le k \le b$ $\Rightarrow$ $a=k=b$.

($\Leftarrow$) Si $a=b$,  entonces $(a,b) = (a,a) =a$ y $[a,b] =[a,a] =a$, por lo tanto 
$a = (a,b)=[a,b]$. 

\vskip .4cm

\item\label{mcm-abd} Probar que si $d$ es un divisor común de $a$ y $b$, entonces $\dfrac{[a,b]}{d} = \left[\dfrac{a}{d},\dfrac{b}{d}\right]$.

\rta 

$$
\dfrac{[a,b]}{d} = \dfrac{ab/(a,b)}{d} = \dfrac{ab}{d(a,b)}.
$$
Por otro lado, 
$$
\left[\dfrac{a}{d},\dfrac{b}{d}\right] = \frac{(a/d)(b/d)}{(a/d,b/d)} =  
\frac{ab/d^2}{(a,b)/d} = \frac{ab/d}{(a,b)} = \dfrac{ab}{d(a,b)}.
$$
En la última fórmula usamos la propiedad 
$$
\dfrac{(a,b)}{d} = \left(\dfrac{a}{d},\dfrac{b}{d}\right).
$$


\item Probar que $(a+b,[a,b])=(a,b)$. %En particular, si dos números son coprimos, también lo son su suma y su producto.

\rta Primero hagamos el caso $(a,b)=1$. En este caso $[a,b] = ab/(a,b) = ab$, Por lo tanto debemos probar que si $(a,b)=1$,  entonces $(a+b,ab) =1$. 

Supongamos que exista $p$ primo  tal que $p| a+b$ y $p| ab$. Como $p|ab$,  entonces $p|a$ o $p|b$, consideremos que $p|a$ (el otro caso es simétrico),  como $p|a+b$, entonces $p| a+b-a=b$. Es deir,  concluimos que $p|a$ y $p|b$, lo cual es absurdo pues $(a,b)= 1$. El absurdo vino de suponer que existía $p$ primo tal que  $p| a+b$ y $p| ab$. Por lo tanto $(a+b,ab) =1$.

Ahora hagamos el caso en  que  $(a,b)=d > 1$. 

Ahora bien,
\begin{equation*}
    \frac{1}{d}(a+b,[a,b]) = \left(\frac{a}{d}+\frac{b}{d},\frac{[a,b]}{d}\right)  \stackrel{Ej \ref{mcm-abd}}{=}
    \left(\frac{a}{d}+\frac{b}{d},\left[\dfrac{a}{d},\dfrac{b}{d}\right]\right)  =1
\end{equation*}   
Está última igualdad se deduce del caso anterior (hemos visto anteriormente  que $(a/d, b/d)=1$). Por lo tanto,
\begin{equation*}
    \frac{1}{d}(a+b,[a,b]) = 1 \quad \Rightarrow \quad (a+b,[a,b]) = d = (a,b).
\end{equation*}  

\item Probar que si $(a,b)=1$ y $n+2$ es un número primo, entonces $(a+b, a^2 + b^2 - nab) = 1$ ó $n+2$.

\rta Si  $(a+b, a^2 + b^2 - nab) = 1$, listo. En  caso contrario  existe $p$ primo tal que $p|a+b$ y $p| a^2 + b^2 - nab$. 

Como $p|a+b$ $\Rightarrow$ $p|(a+b)^2 = a^2 +2ab +b^2$.

Como $p|(a+b)^2 = a^2 +2ab +b^2$ y  $p| a^2 + b^2 - nab$, entonces  
$$p| ( a^2 +2ab +b^2) - (a^2 + b^2 - nab)= (n+2)ab.$$

Com $p$ es primo y $p|(n+2)ab$ $\Rightarrow$ $p|n+2$ o $p|a$ o $p|b$.

Si $p|a$, como $p|a+b$ $\Rightarrow$ $p | (a+b) -a= b$, luego $(a,b) > 1$,  absurdo. También se llega,  en forma análoga, a un absurdo si $p|b$. 

Luego, $p|n+2$ y por  lo tanto el mcd de $a+b$ y $a^2 + b^2 - nab$ es divisible por $p$. Como $n+2$  es primo, los únicos divisores que tiene son $1$ y el mismo, por lo tanto $p=n+2$.  Cualquier primo que divide a $a+b$ y $a^2 + b^2 - nab$ divide  a $(a+b, a^2 + b^2 - nab)$ y  viceversa. Por lo tanto, hemos probado que $d=p^k$. 

Ahora bien,  razonando como antes podemos ver que $p^k| (n+2)ab$, pero como $p\not|a$ y $p\not|b$ $\Rightarrow$  $p^k| (n+2) = p$ $\Rightarrow$ $k=1$; y por lo tanto $d= n+2$. 


\item Si $a\cdot b$ es un cuadrado y $a$ y $b$ son coprimos, probar que $a$ y $b$ son cuadrados.


\item Probar que $\sqrt 6$ es irracional.

%\medskip

%\item Probar que $2^{3n+4} + 7^{3n+1}$ es divisible por $9$, para todo $n \in {\mathbb N}$, $n$ impar.



\item Hallar el menor múltiplo de 168 que es un cuadrado.


\item Probar que el producto de dos enteros consecutivos no nulos no es un cuadrado. (Ayuda: usar el Teorema Fundamental de la Aritmética).



\item ¿Existen enteros $m$ y $n$ tales que:

a) $m^4=27$? \qquad \qquad b) $m^2 = 12n^2$? \qquad \qquad c) $m^3 = 47n^3$?



\item Sean $a$ y $b$ enteros coprimos. Probar que
\begin{enumerate}
  \item $(a\cdot c, b)=(b,c)$, para todo entero $c$.
  \item $a^m$ y $b^n$ son coprimos, para todo $m,n\in \mathbb N$.
  \item $a+b$ y $a\cdot b$ son coprimos.
\end{enumerate}


\item ¿Cuál es la mayor potencia de $3$ que divide a $100!$? ¿En cuántos ceros termina el de\-sa\-rro\-llo decimal de $100!$?


\item Determinar todos los $p\in\mathbb N$ tales que
\[ p,\, p+2,\, p+6,\, p+8,\, p+12,\, p+14 \]
sean todos primos.


\item  Sea $\{f_n\}_{n\in\mathbb N}$ la sucesión de Fibonacci, definida recursivamente por: $f_1=1,\, f_2=1$, $f_{n+1}=f_{n}+f_{n-1},\, n\geq 2$. Probar que:
\begin{enumerate}
\item\label{fib-a} $f_{3n}$ es par $\forall\, n\in\mathbb N$.
\item\label{fib-b} $f_{3n+1}$ y $f_{3n+2}$ son impares $\forall\, n\in\mathbb N$.
\item\label{fib-c} $f_{n+m}=f_{m}f_{n+1}+f_{m-1}f_{n}\; \forall n,m\in\mathbb N,\, m\geq 2$.
\item\label{fib-d} $f_n\mid f_{nk}\;  \forall k\in\mathbb N$.
\item\label{fib-e} $f_{n+1}f_{n-1}-f_n^2=(-1)^n\; \forall n\geq 2$.
\item\label{fib-f} $(f_{n+1},f_n)=1\; \forall\, n\in\mathbb N$.
\end{enumerate}

\rta 
\begin{enumerate}
    \item[\ref{fib-a}] Demostraremos el resultado por inducción. 

    Caso base $n=1$. En este caso $f_3 = f_2 + f_1 = 2$, es par.
    
    Paso inductivo. Sea $n>1$. Supongamos que $f_{3(n-1)}$ es par (HI) y probemos que $f_{3n}$ es par: 
    \begin{align*}
        f_{3n} & = f_{3n-1} + f_{3n-2} \\
               & = f_{3n-2} + f_{3n-3} + f_{3n-2} \\
               & = 2 f_{3n-2} + f_{3(n-1)}. 
    \end{align*}
    Por (HI), $f_{3(n-1)}$ es par  y claramente $ 2 f_{3n-2}$  es par, luego $f_{3n}$ es par.
    \item[\ref{fib-b}] También demostraremos este caso por inducción. Lo que debemos demostrar es 
    $$
    P(n) : \text{``$f_{3n+1}$ y $f_{3n+2}$ son impares $\forall\, n\in\mathbb N$''}
    $$
    
    Caso base $n=1$. En este caso $f_{3n+1}$ y $f_{3n+2}$ son $f_{4}$ y $f_{5}$ y $f_4 = f_3+f_2 = 2 +1 =3$, $f_5 = f_4 +f_3 =3 +2 =5$, ambos impares. 
    
    Paso  inductivo.  Sea $n>1$. Supongamos que $f_{3(n-1)+1} = f_{3n-2}$ y $f_{3(n-1)+2} = f_{3n-1}$ son impares (HI), probemos que  $f_{3n+1}$ y $f_{3n+2}$ son impares. 
    Ahora bien, 
    $$
    f_{3n+1} = f_{3n+1 -1} + ,f_{3n+1 -2} =f_{3n} + f_{3n-1}.
    $$
    Por el inciso anterior $f_{3n}$  es par y por (HI) $f_{3n-1}$ es impar. Como la suma de un par y un impar es impar, resulta que $f_{3n+1}$ es impar. Con un razonamiento análogo probamos que $f_{3n+2}$  es impar:  
    $$
    f_{3n+2} = f_{3n+2 -1} + ,f_{3n+2 -2} =f_{3n +1} + f_{3n}.
    $$
    Por lo tanto, $f_{3n+2}$  es la suma de un impar y un par, y en consecuencia es impar. 
    
    \item[\ref{fib-c}] También lo hacemos por inducción. El paso inductivo es:
    \begin{align*}
        f_{n+m} &= f_{n+m -1} +f_{n+m-2}&&\text{(Definición recursiva de $f$)} \\
                &= f_{m-1}f_{n+1}+f_{m-2}f_{n}  + f_{m-2}f_{n+1}+f_{m-3}f_{n}&&\text{((HI) dos veces)} \\
                &= (f_{m-1}f_{n+1}  + f_{m-2}f_{n+1})+(f_{m-2}f_{n}+f_{m-3}f_{n})&& \\
                &= (f_{m-1} + f_{m-2})f_{n+1}+(f_{m-2}+f_{m-3})f_{n}&& \\
                &= f_{m}f_{n+1}+f_{m-1}f_{n}.&&\text{(Definición recursiva de $f$)}
    \end{align*}
    
    \item[\ref{fib-d}]
    También lo hacemos por inducción. El paso inductivo es: por el inciso anterior
    $$
    f_{nk} = f_{n(k-1) + n} =  f_{n}f_{n(k-1)+1}+f_{n-1}f_{n(k-1)},
    $$
    Por (HI), $f_{n(k-1)} = hf_n$, luego
    $$
    f_{nk} =  f_{n}f_{n(k-1)+1}+hf_{n-1}f_n =  f_{n}(f_{n(k-1)+1}+hf_{n-1}),
    $$
    y por consiguiente $f_{n}| f_{nk}$. 
    
    \item[\ref{fib-e}] También lo hacemos por inducción. El paso inductivo es:
    \begin{align*}
        f_{n+1}f_{n-1}-f_n^2 &= (f_{n} +f_{n-1})f_{n-1}-f_n^2 \\
        &= f_{n}f_{n-1} +f_{n-1}^2-f_n^2 \\
    \end{align*}
    Por (HI), $f_{n}f_{n-2}-f_{n-1}^2=(-1)^{n-1}$, luego $$f_{n-1}^2= f_{n}f_{n-2}-(-1)^{n-1} = f_{n}f_{n-2}+(-1)^{n} .$$ Por lo tanto, 
    \begin{align*}
        f_{n+1}f_{n-1}-f_n^2 &= f_{n}f_{n-1} +f_{n-1}^2-f_n^2&& \\
        &= f_{n}f_{n-1} +f_{n}f_{n-2}+(-1)^{n} -f_n^2&&\text{(HI)} \\
        &= f_{n}(f_{n-1} +f_{n-2})+(-1)^{n} -f_n^2&& \\
        &= f_{n}f_n+(-1)^{n} -f_n^2 &&\text{(Def. rec. de $f$)}\\
        &= (-1)^{n}
    \end{align*}
    
    \item[\ref{fib-f}] Lo  hacemos por inducción sobre $n$. 
    
    Caso base $n=1$. En este caso $(f_2,f_1) = (1,1) = 1$.
    
    Paso inductivo. Supongamos que el para $n>1$  se cumple 
    \begin{equation*}
        (f_n, f_{n-1}) =1\qquad (HI).
    \end{equation*}
    Probaremos que $(f_{n+1}, f_{n}) =1$. 
    
    Sea $d$  entero positivo tal que $d|f_{n+1}$ y  $d|f_{n}$. Como $d|f_{n+1}$ y $f_{n+1} = f_n + f_{n-1}$, $d| f_n + f_{n-1}$ $\Rightarrow$ (pues $d|f_{n}$), $d| f_n + f_{n-1} -f_n =  f_{n-1}$.
    
    Por lo tanto, $d|f_{n}$. y $d|f_{n-1}$. Por (HI) $\Rightarrow$ $d=1$. Probamos que todo divisor de $f_{n+1}$ y $f_{n}$ es $1$, por lo tanto $(f_{n+1}, f_{n}) =1$. 
\end{enumerate}





\end{enumerate}

\end{document}

