\usepackage{etex}
\usepackage{t1enc}
\usepackage{latexsym}
\usepackage[utf8]{inputenc}
\usepackage{verbatim}
\usepackage{multicol}
\usepackage{amsgen,amsmath,amstext,amsbsy,amsopn,amsfonts,amssymb}
\usepackage{amsthm}
\usepackage{calc}         % From LaTeX distribution
\usepackage{graphicx}     % From LaTeX distribution
\usepackage{ifthen}
\input{random.tex}        
%\usepackage{subfigure} 
\usepackage{tikz}
\usetikzlibrary{arrows}
\usetikzlibrary{matrix}
\usepackage{mathtools}
\usepackage{stackrel}
\usepackage{enumitem}
\usepackage{tkz-graph}
%\usepackage{makeidx}
\usepackage{hyperref}
\hypersetup{
	colorlinks=true,
	linkcolor=blue,
	filecolor=magenta,      
	urlcolor=cyan,
}
\usepackage{hypcap}
\usepackage{nicematrix} %https://ctan.dcc.uchile.cl/macros/latex/contrib/nicematrix/nicematrix.pdf
\tikzset{
	every picture/.append style={
		execute at begin picture={\deactivatequoting},
		execute at end picture={\activatequoting}
	}
}
\usetikzlibrary{decorations.pathreplacing,angles,quotes}
\usetikzlibrary{shapes.geometric}



\usepackage{polynom}

\polyset{%
	style=B,
	delims={(}{)},
	div=:
}
\numberwithin{equation}{section}
\renewcommand\labelitemi{$\circ$}
\setlist[enumerate, 1]{label={(\arabic*)}}
\setlist[enumerate, 2]{label=\emph{\alph*)}}
\setcounter{MaxMatrixCols}{20}


\newtheorem{teorema}{Teorema}[section]
\newtheorem{proposicion}[teorema]{Proposici\'on}
\newtheorem{corolario}[teorema]{Corolario}
\newtheorem{lema}[teorema]{Lema}
\newtheorem{propiedad}[teorema]{Propiedad}
\theoremstyle{definition} %en negrita y letra estándar
\newtheorem{definicion}[teorema]{Definici\'on}

\theoremstyle{remark} % en itálica
\newtheorem{observacion}[teorema]{Observaci\'on}
\newtheorem{obs}[teorema]{Observaci\'on}
\newtheorem{nota}[teorema]{Nota}
\newtheorem{ejemplo}[teorema]{Ejemplo}
\newtheorem{resultado}[teorema]{Resultado}
\newtheorem{problema}[teorema]{Problema}
\newtheorem{ejercicio}[teorema]{Ejercicio}
\newtheorem{ejerciciof}[teorema]{}
% sin numero
\newtheorem*{observacion*}{Observaci\'on}
\newtheorem*{obs*}{Observaci\'on}
\newtheorem*{nota*}{Nota}
\newtheorem*{notacion*}{Notaci\'on}
\newtheorem*{ejemplo*}{Ejemplo}
\newtheorem*{resultado*}{Resultado}
\newtheorem*{problema*}{Problema}
\newtheorem*{ejercicio*}{Ejercicio}
\newtheorem*{ejerciciof*}{}

\newlist{enumelem}{enumerate}{10} 
\setlist[enumelem]{label=\textit{E\arabic*.},ref=\textit{E\arabic*}}

\newlist{enumex}{enumerate}{10} 
\setlist[enumex,1]{label={\arabic*)},ref=\textit{\arabic*}}
\setlist[enumex,2]{label=\emph{\alph*)}}

\usepackage[a4paper, left=3cm, right=4cm, bottom=2.5cm]{geometry}


\renewcommand{\partname }{Parte }
\renewcommand{\indexname}{Indice }
\renewcommand{\figurename }{Figura }
\renewcommand{\tablename }{Tabla }
\renewcommand{\proofname}{Demostraci\'on}
\renewcommand{\appendixname }{}
\renewcommand{\contentsname }{Contenidos }
\renewcommand{\chaptername }{}
\renewcommand{\bibname }{Bibliograf\'\i a }


\setcounter{MaxMatrixCols}{20}

\newcommand{\Id}{\operatorname{Id}}
\newcommand{\img}{\operatorname{Im}}
\newcommand{\nuc}{\operatorname{Nu}}
\newcommand\im{\operatorname{Im}}
\renewcommand\nu{\operatorname{Nu}}
\newcommand{\la}{\langle}
\newcommand{\ra}{\rangle}
\renewcommand{\t}{{\operatorname{t}}}
\renewcommand{\sin}{{\,\operatorname{sen}}}
\newcommand{\N}{\mathbb N}
\newcommand{\Q}{\mathbb Q}
\newcommand{\R}{\mathbb R}
\newcommand{\C}{\mathbb C}
\newcommand{\K}{\mathbb K}
\newcommand{\F}{\mathbb F}
\newcommand{\Z}{\mathbb Z}
\newcommand{\sen}{\operatorname{sen}}
\newcommand{\nodivide}{\operatorname{\not|\,}}
\newcommand \mcd{\operatorname{mcd}}
\newcommand \mcm{\operatorname{mcm}}
\newcommand \circled[1]{\tikz[baseline=(char.base)]{
		\node[shape=circle,draw,inner sep=2pt] (char) {#1};}}

\newcommand\ponertz[3]{\node at (#1/10,#2/10+3) {#3};} %tikz % coloca frase #3 en 8pt en (#1,#2). Coordenadas en mm.
\newcommand\lineatz[4]{\draw (#1/10,#2/10+3)-- (#3/10,#4/10+3);} %tikz % hace una linea de x,y a x',y'
\newcommand{\varx}{0} % variable para cambiar coordenada x
\newcommand{\vary}{0} % variable para cambiar coordenada y
\newcommand{\varc}{1} % variable para hacer homotecias

\newcommand{\modulo}{{\scriptstyle \%}\,}
\newcommand{\cociente}{\,/\hskip-.1cm/\,}
